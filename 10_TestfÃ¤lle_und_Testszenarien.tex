\section{Testfälle und Testszenarien}
Für jede Funktion existiert mindestens ein atomarer Testfall. Neben jedem Testfall wird aufgelistet, welche Funktion dieser abdeckt. Die Testfälle werden, wie die korrespondierenden Anforderungen, in API-Testfälle, Web-Interface-Testfälle und Visualisierung-Testfälle unterschieden.

\subsection{API-Testfälle}


\begin{itemize}

    \item[T1010] \textbf{Einen neuen Nutzer registrieren.} (F1000)
    
    \item[T1011] \textbf{Einen Nutzer mit Administrationsrechten registrieren} (F1000)
    
    \item[T1020] \textbf{Einen Nutzer authentifizieren.} (F1010)
    
    \item[T1021] \textbf{Einen Nutzer mit Administrationsrechten authentifizieren.} (F1010)
    
    \item[T1022] \textbf{Einen Nutzer mit fehlerhaften Zugangsdaten authentifizieren.} (F1010)
    
    \item[T1030] \textbf{Einen Job mit seperater Job-Beschreibungs-Datei einreichen.} (F1020)
    
    \item[T1031] \textbf{Einen Job mit enthaltener Job-Beschreibung einreichen.} (F1020)
    
    \item[T1032] \textbf{Einen Job mit einem Link, der auf eine Job-Beschreibungs-Datei verweist, einreichen.} (F1020)
    
    \item[T1033] \textbf{Einen bereits eingereichten aber abgebrochenen Job mit einer Referenz auf Diesen erneut einreichen.} (F1020)
    
    \item[T1034] \textbf{Einen fehlerhaften Job einreichen.} (F1020)
    
    \item[T1040] \textbf{Einen eingereichten Job abbrechen.} (F1030)
    
    \item[T1060] \textbf{Mallob starten.} (F1050)
    
    \item[T1070] \textbf{Mallob stoppen.} (F1060)
    
    \item[T1080] \textbf{Mallob neustarten.} (F1070)
    
    \item[T1090] \textbf{Den Status eines inkrementellen Jobs zurückgeben.} (F1080)
    
    \item[T1100] \textbf{Information über einen Job abfragen.} (F1090)
    
    \item[T1101] \textbf{Informationen über mehrere Jobs abfragen.} (F1090)
    
    \item[T1110] \textbf{Ergebnis von einem bearbeiteten Job ausgeben.} (F1100)
    
    \item[T1111] \textbf{Ergebnisse von mehreren bearbeiteten Jobs ausgeben.} (F1100)
    
    \item[T1120] \textbf{Job-Beschreibung von einem eingereichten Job ausgeben.} (F1110)
    
    \item[T1121] \textbf{Job-Beschreibungen von mehreren eingereichten Job ausgeben.} (F1110)
    
    \item[T1130] \textbf{Informationen über Mallob durch einen Admin abfragen.} (F1120)
    
    \item[T1140] \textbf{Einen Ereignis-Stream von Mallob ausgeben.} (F1130)
    
    \item[T1150] \textbf{Priorität eines Nutzers durch einen Admin ändern.} (F1140)
    
    \item[T1160] \textbf{Einstellungen abrufen.} (F1160)
    
    \item[T1170] \textbf{Einstellungen aktualisieren.} (F1150)
    
    \item[T1180] \textbf{Plugin neu einlesen.} (F1150)
    

\end{itemize}

\subsection{Web-Interface Testfälle}

\begin{itemize}
    \item[T2010] \textbf{Nutzer anmelden.} (F2000)
    
    \item[T2011] \textbf{Admin anmelden.} (F2000)
    
    \item[T2020] \textbf{Nutzer registrieren.} (F2010)
    
    \item[T2030] \textbf{Job mit zugehöriger Job-Beschreibung über das Eingabefeld einreichen.} (F2020)
    
    \item[T2031] \textbf{Job mit Upload der Job-Beschreibungs-Datei einreichen.} (F2020)
    
    \item[T2032] \textbf{Job mit Angabe einer URL einreichen.} (F2020)
    
    \item[T2040] \textbf{Einen Job über die Job-Tabelle abbrechen.} (F2030)
    
    \item[T2041] \textbf{Einen Job über die Job-Seite abbrechen.} (F2040)
    
    \item[T2050] \textbf{Mehrere Jobs abbrechen.} (F2040)
    
    \item[T2060] \textbf{Einen abgebrochenen Job über die Job-Tabelle neustarten.} (F2050)
    
    \item[T2061] \textbf{Einen abgebrochenen Job über die Job-Seite neustarten.} (F2050)
    
    \item[T2070] \textbf{Ergebnis eines Jobs über die Job-Tabelle herunterladen.} (F2060)
    
    \item[T2071] \textbf{Ergebnis eines Jobs über die Job-Seite herunterladen.} (F2060)
    
    \item[T2080] \textbf{Ergebnis mehrerer Jobs herunterladen.} (F2070)
    
    \item[T2090] \textbf{Vorläufiges Konto verifizieren.} (F2090)
    
    \item[T2100] \textbf{Mallob starten.} (F2110)
    
    \item[T2101] \textbf{Mallob stoppen.} (F2110)
    
    \item[T2102] \textbf{Mallob neustarten.} (F2110)
    
    \item[T2110] \textbf{Einstellungen ändern.} (F2120)
    
    \item[T2120] \textbf{Plugin einlesen.} (F2120)
    
    \item[T2130] \textbf{Job-Information als eigene Spalte in der Job-Tabelle anzeigen.} (F2130)
    
    \item[T2131] \textbf{Job-Information über aufgeklapptem Fenster anzeigen.} (F2130)
    
    \item[T2132] \textbf{Job-Information über Job-Seite anzeigen.} (F2130)
    
    \item[T2140] \textbf{Spalten in der Job-Tabelle auswählen.} (F2140)
    
    \item[T2150] \textbf{Job-Tabelle nach Attributen sortieren.} (F2150)
    
    \item[T2160] \textbf{Plugins anzeigen.} (F2160)
    
    \item[T2170] \textbf{Nutzer abmelden.} (F2170)
    
\end{itemize}

\subsection{Visualisierung Testfälle}

\begin{itemize}
    \item[T3010] \textbf{Einem Nutzer den Systemzustand von Mallob anzeigen.} (F3000)
    
    \item[T3011] \textbf{Einem Admin den Systemzustand von Mallob anzeigen.} (F3000)
    
    \item[T3020] \textbf{Binärbaum für einen Job anzeigen.} (F3010)
    
    \item[T3030] \textbf{Visualisierung pausieren.} (F3020)
    
    \item[T3040] \textbf{Visualisierung starten.} (F3030)
    
    \item[T3050] \textbf{Geschwindigkeit der Visualisierung einstellen.} (F3040)
    
    \item[T3060] \textbf{Beliebigen Zeitpunkt der Visualisierung auswählen.} (F3050)
    
\end{itemize}


\subsection{Testszenarien}

\subsubsection{Testszenario 1: Einen Job einreichen - API}
Ein neuer Nutzer möchte sich registrieren/authentifizieren und darauf hin einen Job einreichen. Dieser wird aber abgebrochen und anschließend erneut eingereicht. 

\begin{enumerate}
    \item 
    \begin{enumerate}
        \item \textbf{T1010} Einen neuen Nutzer registrieren.
        
        \item \textbf{T1020} Einen Nutzer authentifizieren.
    \end{enumerate}
    
    \item \textbf{T1034} Einen fehlerhaften Job einreichen.
    
    \item 2. Punkt beliebig oft wiederholen.
    
    \item 
    \begin{enumerate}
        \item \textbf{T1030} Einen Job mit seperater Job-Beschreibungs-Datei einreichen. 
        
        \item \textbf{T1031} Einen Job mit enthaltener Job-Beschreibung einreichen.
        
        \item \textbf{T1032} Einen Job mit einem Link, der auf eine Job-Beschreibungs-Datei verweist, einreichen.
    \end{enumerate}
    
    \item \textbf{T1100} Information über einen Job abfragen.
    
    \item \textbf{T1120} Job-Beschreibung von einem eingereichten Job ausgeben.
    
    \item \textbf{T1040} Einen eingereichten Job abbrechen.
    
    \item \textbf{T1033} Einen bereits eingereichten aber abgebrochenen Job mit einer Referenz auf Diesen erneut einreichen.
    
    \item Job wurde fertig bearbeitet.
    
    \item \textbf{T1110} Ergebnis von einem bearbeiteten Job ausgeben. 
    
\end{enumerate}

\subsubsection{Testszenario 2: Mehr als einen Job einreichen - API}
Ein neuer Nutzer möchte sich registrieren/authentifizieren und mehr als einen Job einreichen.

\begin{enumerate}
    \item 
    \begin{enumerate}
        \item \textbf{T1010} Einen neuen Nutzer registrieren.
        
        \item \textbf{T1020} Einen Nutzer authentifizieren.
    \end{enumerate}
    
    \item \textbf{T1034} Einen fehlerhaften Job einreichen.
    
    \item 2. Punkt beliebig oft wiederholen.
    
    \item 
    \begin{enumerate}
        \item \textbf{T1030} Einen Job mit seperater Job-Beschreibungs-Datei einreichen. 
        
        \item \textbf{T1031} Einen Job mit enthaltener Job-Beschreibung einreichen.
        
        \item \textbf{T1032} Einen Job mit einem Link, der auf eine Job-Beschreibungs-Datei verweist, einreichen.
    \end{enumerate}
    
    \item 4.Punkt beliebig oft bis zur Obergrenze einzureichender Jobs pro Nutzer wiederholen
    
    \item \textbf{T1101} Informationen über mehrere Jobs abfragen. 
    
    \item \textbf{T1121} Job-Beschreibungen von mehreren eingereichten Job ausgeben.
    
    \item Mindestens 2 Jobs wurden fertig bearbeitet.
    
    \item \textbf{T1111} Ergebnisse von mehreren bearbeiteten Jobs ausgeben. 
\end{enumerate}

\subsubsection{Testszenario 3: Zugriff eines Admins - API}
Ein bereits im System verifizierter Admin möchte sich authentifizieren bzw. ein neuer Admin soll registriert werden und darauf hin einige Funktionen ausführen, die nur für einen Admin vorgesehen sind.

\begin{enumerate}
    \item
    \begin{enumerate}
        \item \textbf{T1011} Einen Nutzer mit Administrationsrechten registrieren.
        
        \item \textbf{T1021} Einen Nutzer mit Administrationsrechten authentifizieren. 
    \end{enumerate}
    
    \item \textbf{T1060} Mallob starten.
    
    \item \textbf{T1080} Mallob neustarten.
    
    \item \textbf{T1130} Informationen über Mallob durch einen Admin abfragen. 
    
    \item \textbf{T1150} Priorität eines Nutzers durch einen Admin ändern. 
    
    \item \textbf{T1160} Einstellungen abrufen.
    
    \item \textbf{T1170} Einstellungen aktualisieren. 
    
    \item \textbf{T1180} Plugin neu einlesen.
    
    \item \textbf{T1070} Mallob stoppen. 
\end{enumerate}

\subsubsection{Testszenario 4: Aufrufen des Web-Interface und Einreichen eines Jobs - Web-Interface}
Ein Nutzer gelangt über die Anmelde-Maske auf das Web-Interface, reicht darüber einen Job ein und greift auf verschiedene Funktionen, die das Web-Interface in Bezug auf Jobs anbietet, zu.

\begin{enumerate}
    \item 
    \begin{enumerate}
        \item \textbf{T2010} Nutzer anmelden.
        
        \item \textbf{T2020} Nutzer registrieren.
    \end{enumerate}
    
    \item 
    \begin{enumerate}
        \item \textbf{T2030} Job mit zugehöriger Job-Beschreibung über das Eingabefeld einreichen.
        
        \item \textbf{T2031} Job mit Upload der Job-Beschreibungs-Datei einreichen.
        
        \item \textbf{T2032} Job mit Angabe einer URL einreichen.
    \end{enumerate}
    
    \item 
    \begin{enumerate}
        \item \textbf{T2040} Einen Job über die Job-Tabelle abbrechen. 
        
        \item \textbf{T2041} Einen Job über die Job-Seite abbrechen.
    \end{enumerate}
    
    \item
    \begin{enumerate}
        \item \textbf{T2060} Einen abgebrochenen Job über die Job-Tabelle neustarten.
        
        \item \textbf{T2061} Einen abgebrochenen Job über die Job-Seite neustarten. 
    \end{enumerate}
    

    \item \textbf{T2160} Spalten in der Job-Tabelle auswählen. 
    
    \item
    \begin{enumerate}
        \item \textbf{T2150} Job-Information als eigene Spalte in der Job-Tabelle anzeigen.
        
        \item \textbf{T2151} Job-Information über aufgeklapptem Fenster anzeigen.
         
        \item \textbf{T2152} Job-Information über Job-Seite anzeigen. 
    \end{enumerate}
    
    \item \textbf{T2170} Job-Tabelle nach Attributen sortieren.
    
    \item
    \begin{enumerate}
        \item \textbf{T2070} Ergebnis eines Jobs über die Job-Tabelle herunterladen.
        
        \item \textbf{T2071} Ergebnis eines Jobs über die Job-Seite herunterladen. 
    \end{enumerate}
    
    \item \textbf{T2190} Nutzer abmelden.

\end{enumerate}

\subsubsection{Testszenario 5: Aufrufen des Web-Interface und Einreichen mehrerer Jobs - Web-Interface}
Ein Nutzer gelangt über die Anmelde-Maske auf das Web-Interface und reicht mehr als einen Job ein.

\begin{enumerate}
     \item 
     \begin{enumerate}
        \item \textbf{T2010} Nutzer anmelden.
        
        \item \textbf{T2020} Nutzer registrieren.
     \end{enumerate}
     
     \item 
     \begin{enumerate}
        \item \textbf{T2030} Job mit zugehöriger Job-Beschreibung über das Eingabefeld einreichen.
        
        \item \textbf{T2031} Job mit Upload der Job-Beschreibungs-Datei einreichen.
        
        \item \textbf{T2032} Job mit Angabe einer URL einreichen.
     \end{enumerate}
     
     \item 2.Punkt mindestens einmal wiederholen.
     
     \item Mindestens 2 Jobs sind fertig bearbeitet
     
     \item \textbf{T2080} Ergebnis mehrerer Jobs herunterladen.
     
     \item \textbf{T2190} Nutzer abmelden.
\end{enumerate}

\subsubsection{Testszenario 6: Einreichen und Abbrechen mehrerer Jobs - Web-Interface}
Ein Nutzer reicht über das Web-Interface mehr als einen Job ein und bricht diese wieder ab.

\begin{enumerate}
     \item 
     \begin{enumerate}
        \item \textbf{T2010} Nutzer anmelden.
        
        \item \textbf{T2020} Nutzer registrieren.
     \end{enumerate}
     
     \item 
     \begin{enumerate}
        \item \textbf{T2030} Job mit zugehöriger Job-Beschreibung über das Eingabefeld einreichen.
        
        \item \textbf{T2031} Job mit Upload der Job-Beschreibungs-Datei einreichen.
        
        \item \textbf{T2032} Job mit Angabe einer URL einreichen.
     \end{enumerate}
     
     \item 2.Punkt mindestens einmal wiederholen.
     
     \item \textbf{T2050} Mehrere Jobs abbrechen.
     
     \item \textbf{T2190} Nutzer abmelden.
\end{enumerate}

\subsubsection{Testszenario 7: Aufrufen des Web-Interface durch einen Admin - Web-Interface}
Ein Admin ruft das Web-Interface auf, meldet sich an, und greift auf einige Funktionen zu, die nur für einen Admin bestimmt sind.

\begin{enumerate}
    \item \textbf{T2011} Admin anmelden.
    
    \item \textbf{T2100} Mallob starten.
    
    \item \textbf{T2102} Mallob neustarten.
    
    \item \textbf{T2110} Einstellungen ändern.
    
    \item \textbf{T2160} Plugins anzeigen.
    
    \item \textbf{T2120} Plugin einlesen.

    \item \textbf{T2101} Mallob beenden.
    
    \item \textbf{T2170} Nutzer abmelden.
\end{enumerate}

\subsubsection{Testszenario 8: Einsehen der Visualisierung durch einen Nutzer - Visualisierung}
Ein Nutzer greift auf die verschiedenen Visualisierungen zu einem Job und dem Gesamtzustand von Mallob, die das Web-Interface anbietet, zu. Die Vorbedingung für dieses Szenario ist die, dass der Nutzer bereits mindestens einen Job eingereicht hat.

\begin{enumerate}
    \item \textbf{T2010} Nutzer anmelden.
    
    \item \textbf{T3010} Einem Nutzer den Systemzustand von Mallob anzeigen. 
    
    \item \textbf{T3020} Binärbaum für einen Job anzeigen. 
    
    \item \textbf{T3060} Beliebigen Zeitpunkt der Visualisierung auswählen.
    
    \item \textbf{T3050} Geschwindigkeit der Visualisierung einstellen.
    
    \item \textbf{T3040} Visualisierung starten. 
    
    \item \textbf{T3030} Visualisierung pausieren.

    \item \textbf{T3040} Visualisierung starten. 
\end{enumerate}

\subsubsection{Testszenario 9:Einsehen der Visualisierung durch einen Admin - Visualisierung}
Ein Admin greift auf die Visualisierung des Gesamtzustands von Mallob zu. Hierbei wird keine Vorbedingung benötigt, da der Admin jeden im System befindlichen Job mit allen Informationen einsehen kann.

\begin{enumerate}
    \item \textbf{T2011} Admin anmelden.
    
    \item \textbf{T3011} Einem Admin den Systemzustand von Mallob anzeigen.
    
    \item \textbf{T3060} Beliebigen Zeitpunkt der Visualisierung auswählen.
    
    \item \textbf{T3050} Geschwindigkeit der Visualisierung einstellen.
    
    \item \textbf{T3040} Visualisierung starten. 
    
    \item \textbf{T3030} Visualisierung pausieren.

    \item \textbf{T3040} Visualisierung starten.
\end{enumerate}



	