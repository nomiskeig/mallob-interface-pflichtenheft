\section{Testfälle und Testszenarien}
Für jede Funktion existiert mindestens ein atomarer Testfall. Unter jedem Testfall wird aufgelistet, welche Funktionen dieser abdeckt. Die Testfälle werden, wie die korrespondierenden Anforderungen, in API-Testfälle, Web-Interface-Testfälle und Visualisierung-Testfälle unterschieden.

\subsection{API Testfälle}


\begin{itemize}

    \item[T010] \textbf{Einen neuen Benutzer registrieren.}
    
    %\item[T020] \textbf{Eine neue Benutzerregistrierung verifizieren.}
    
    \item[T030] \textbf{Einen registrierten Benutzer anmelden.}
    
    \item[T040] \textbf{Die Benutzerdaten eines registrierten Benutzers ändern.}
    
    \item[T050] \textbf{Einen Benutzer mit Administrationsrechten registrieren}
    
    \item[T060] \textbf{Einen Benutzer mit Administrationsrechten anmelden.}
    
    \item[T080] \textbf{Einen Benutzer abmelden.}
    
    \item[T090] \textbf{Einen Benutzer löschen.}
    
    \item[T100] \textbf{Einen Benutzer mit fehlerhaften Zugangsdaten anmelden.}
    
    \item[T120] \textbf{Priorität eines Benutzers durch einen Admin ändern.}
    
    \item[T130] \textbf{Einen Job durch eine JSON-Datei und eine separate Job-Beschreibungs-Datei einreichen.}
    
    \item[T140] \textbf{Einen Job durch eine JSON-Datei mit enthaltener Job-Beschreibung einreichen.}
    
    \item[T150] \textbf{Einen Job durch eine JSON-Datei mit einem Link, der auf eine Job-Beschreibungs-Datei verweist, einreichen.}
    
    \item[T160] \textbf{Einen fehlerhaften Job einreichen.}
    
    \item[T170] \textbf{Eine Jobbearbeitung abbrechen.}
    
    \item[T180] \textbf{Das Ergebnis eines vollendeten Jobs zurückgeben}
    
    \item[T010] \textbf{Ein Teilergebnis eines Jobs mit erreichter maximaler Berechnungszeit zurückgeben.}
    
    \item[T010] \textbf{Eine Fehlermeldung bei fehlerhafter Bearbeitung eines Jobs zurückgeben.}
    
    \item[T190] \textbf{Eine Antwort auf einen abgebrochenen Job zurückgeben.}
    
    \item[200] \textbf{Den Status eines Jobs zurückgeben}
    
    \item[T190] \textbf{Einen neuen Job erstellen.}
    
    \item[T010] \textbf{Den Status eines inkrementellen Jobs zurückgeben.}
    
    \item[T010] \textbf{Information über mindestens einen Job abfragen.}
    
    \item[T240] \textbf{Ergebnisse von mindestens einem bearbeiteten Job ausgeben.}
    
    \item[T010] \textbf{Job-Beschreibung von mindestens einem eingereichten Job ausgeben.}
    
    \item[T270] \textbf{Aufrufen bereits abgeschlossener Jobs}
    
    \item[T290] \textbf{Mallob beenden}
    
    \item[T300] \textbf{Mallob neustarten}
\end{itemize}

\subsection{Web-Interface Testfälle}

\subsection{Visualisierung Testfälle}


\subsection{Testszenarien}
%emptiness
	