\section{Einleitung}


%Neben der
%-expliziten Benennung des Auftragnehmers und des Auftraggebers 
%sollte an dieser Stellung auch eine 
%-grobe Kurzbeschreibung des Projektes erfolgen. 
%Gehen Sie darauf ein, 
%--was das Projekt beinhaltet und 
%--wie das Endergebnis aussehen soll. 
%Wichtig ist, dass auch eine Person, die das erste Mal von dem Projekt hört, versteht, worum es geht.

%---Vorstellung des Auftraggebers und von uns 
Mallob ist ein dezentrales System zur Lösung von NP-schweren Problemen, hauptsächlich entwickelt von Dominik Schreiber im Rahmen seiner Doktorarbeit. Der Auftraggeber - die Entwickler von Mallob - wünschen sich von 5 randoms eine [bedienerfreundliche Möglichkeit, Schnittstelle, Interface], um mit Mallob von außen kommunizieren zu können.\\

%---Grobe kruzbeschreibung des Projekts
Unser System - \textbf{a friendly face for Mallob} - soll diese Brücke zwischen Mallob und Außenwelt darstellen. 

Es verfügt daher über all diejenigen Funktionen, die es möglich machen mit Mallob zu interagieren. Es ist also möglich direkt Aufträge an Mallob zu stellen und Informationen über diese zu erlangen.

Eine Hauptaufgabe des friendly faces ist die Echtzeit-Visualisierung der Arbeitsweise von Mallob. Hier ist es möglich zu sehen wie Mallob die eigenen Jobs (so werden Probleme genannt, die Mallob lösen soll) bearbeitet. Auch die Gesamtauslastung sowie Arbeitsweise ist einsehbar. 

Die Interaktion mit Mallob soll sowohl über ein Web-Interface, als auch direkt über eine von uns bereitgestellte API möglich sein. %\\

%Des weiteren wird das System eine Nutzer-Verwaltung beinhalten, um %[warum haben wir eine Nutzerverwaltung?]. 
