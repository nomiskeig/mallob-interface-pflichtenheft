\section{Einleitung}


%Neben der
%-expliziten Benennung des Auftragnehmers und des Auftraggebers 
%sollte an dieser Stellung auch eine 
%-grobe Kurzbeschreibung des Projektes erfolgen. 
%Gehen Sie darauf ein, 
%--was das Projekt beinhaltet und 
%--wie das Endergebnis aussehen soll. 
%Wichtig ist, dass auch eine Person, die das erste Mal von dem Projekt hört, versteht, worum es geht.

%---Vorstellung des Auftraggebers und von uns 
\href{https://github.com/domschrei/mallob}{Mallob} ist ein dezentrales System zum Scheduling und Lösen von \glslink{NP-schweres Problem}{NP-schweren Problemen}, hauptsächlich entwickelt von Dominik Schreiber im Rahmen seiner Doktorarbeit. Der Auftraggeber - der Entwickler von Mallob - wünscht sich ein bedienerfreundliches Softwaresystem, um mit Mallob von außen kommunizieren zu können. Das Bauen dieses Systems ist unsere Aufgabe.\\
%---Grobe kruzbeschreibung des Projekts
Unser System \textbf{\textit{Fallob - a Friendly Face for Mallob}} soll diese Brücke zwischen Mallob und Außenwelt darstellen.
Es verfügt daher über all diejenigen Funktionen, die es möglich machen, mit Mallob zu interagieren. Es ist also möglich direkt \hyperref[B:Jobs]{Jobs} an Mallob zu senden und \hyperref[B:Job-Informationen]{Informationen} über diese zu erlangen.\\
Eine Hauptaufgabe von \textit{Fallob} ist die \hyperref[pages:visualization]{Visualisierung} des \hyperref[B:Systemzustand]{Systemzustandes} von Mallob. Hier ist es möglich zu sehen, wie Mallob die eigenen \hyperref[B:Jobs]{Jobs} (so werden Probleme genannt, die Mallob lösen soll) bearbeitet. Auch die Gesamtauslastung sowie die Zuordnung von \hyperref[B:Jobs]{Jobs} auf Prozessoren ist einsehbar. 
Die Interaktion mit Mallob soll sowohl über ein \gls{Web-Interface}, als auch direkt über eine von uns bereitgestellte \gls{API} möglich sein. %\\

%Des weiteren wird das System eine Nutzer-Verwaltung beinhalten, um %[warum haben wir eine Nutzerverwaltung?]. Identifikation - wer welche Job eingereicht hat. Erlaubt Rollenverteilung - auf Admin und Benutzer. Erlaubt Admins leichter in Kontakt mit dem Benutzer zu treten. Sicherheitsgründen - nicht jeder kann Jobs einreichen, Kontrolle darauf, wer Mallob benutzt. Historie speichern - erlaubt dem Nutzer Informationen über bisher eingereichten Jobs zu bekommen (nicht nur vom selben Rechner). "Schutz vor Fremdzugriffen und somit die Vertraulichkeit der Inhalte" - laut https://glossar.hs-augsburg.de/Benutzerverwaltung



