\section{Begrifflichkeiten}
%Erklärung von Begrifflichkeiten, die nicht von uns stammen (von \gls{Mallob})

\subsection{Jobs}
\label{B:Jobs}
i
Jobs sind die Instanzen, die von \gls{Mallob} verarbeitet werden. Ein einzelner Job stellt ein einzelnes zu lösendes Problem dar. Ein Job besteht aus Sicht des \glslink{Nutzer}{Nutzers} aus der Job-Konfiguration und der Job-Beschreibung.

\subsection{Job-Konfiguration}
\label{B:Job-Konfiguration}
Die Job-Konfiguration besteht aus Parametern, welche von \gls{Mallob} bei der Bearbeitung des Jobs beachtet werden. Die notwendigen Parameter müssen immer genannt werden.\\
Die Job-Konfiguration besteht aus folgenden Parametern: \\

\begin{tabular}{lL{0.6\textwidth}l}
        Name & Beschreibung & Notwendig\\
        \\
        name & Der Name des Jobs & Ja\\
        priority & Die Prioriät des Jobs & Ja\\
        application & Das Anwendungsfeld des Jobs & Ja\\ 
        max-demand & Die maximaler Anzahl paralleler \glslink{Prozess}{Prozesse}, die dieser Job nutzen kann & Nein\\
        wallclock-limit & Zeitliches Ausführungsbudget des Jobs & Nein\\
        cpu-limit & Corestunden-Ausführungsbudget des Jobs & Nein\\
        arrival & Frühester Bearbeitungsbeginn in Sekunden seit Start von \gls{Mallob} & Nein\\
        dependencies & Jobs, die abgeschlossen sein müssen, bevor dieser Job beginnt & Nein\\
        incremental & Handelt es sich um einen Job mit mehreren Revisionen oder Inkrementen? &  Nein\\
        precursor & Vorgänger-Job bei inkrementellen Jobs & Nein\\
        % TODO: files hier auch?----
        content-mode & Text- oder Binärformat der Beschreibung & Nein\\
        % TODO: literals und assumptions?
    \end{tabular} 
    
    
\subsection{Job-Beschreibung}
\label{B:Job-Beschreibung}
Die Job-Beschreibung enthält die Daten, die das eigentliche Problem beschreiben, das von \gls{Mallob} gelöst werden soll. Ein Beispiel wäre eine Instanz von \gls{SAT}.

\subsection{Job-Ergebnis}
\label{B:Job-Ergebnis}
Das Job-Ergebnis enthält nur das rohe Ergebnis des Jobs, nicht noch etwaige Meta-Informationen, die \gls{Mallob} mit dem Ergebnis liefert.

\subsection{Job-Informationen}
\label{B:Job-Informationen}
Job-Informationen sind alle Informationen über einen Job, die nicht von der internen Bearbeitung des Jobs durch \gls{Mallob} abhängen und nicht das Ergebnis des Jobs sind, wie etwa die Job-Konfiguration, die Job-Beschreibung, der Einreichezeitpunkt des Jobs, den Job-Status und die Meta-Daten des Ergebnisses, wenn diese vorhanden sind. Diese Meta-Daten werden von \gls{Mallob} mit dem eigentlich Ergebnis geliefert und sie enthalten beispielsweise Informationen darüber, wie viel Zeit der Job zum Lösen in Anspruch genommen hat.

\subsection{Event}
\label{B:Event}
Ein Event ist eine Veränderung im internen Systemzustand von \gls{Mallob}, z.B. ein \gls{Prozess} beginnt mit der Bearbeitung eines Jobs, oder ein \gls{Prozess} beginnt mit der Bearbeitung eines anderen Jobs als zuvor. 

\subsection{Job-Details}
\label{B:Job-Details}
Job-Details werden aus Job-Events abgeleitet und sind somit Informationen, die nur die interne Verarbeitung von \gls{Mallob} betreffen, zum Beispiel wie viele \glslink{Prozess}{Prozesse} gerade an einem Job arbeiten und welche \glslink{Prozess}{Prozesse} gerade an einem Job arbeiten. 

\subsection{Systemzustand}
\label{B:Systemzustand}
Der Systemzustand setzt sich aus den Job-Details aller Jobs im System zusammen.

\subsection{Job-Typ}
\label{B:Job-Typ}
Der Job-Typ beschreibt die Art des Problems, welches der Job lösen soll. Der Standard-Typ für jeden Job ist \gls{SAT}.

\subsection{Job-Status}
\label{B:Job-Status}
Der Status eines Jobs beschreibt seinen aktuellen Zustand. Folgende Zustände sind möglich:
\begin{itemize}[noitemsep]
    \item In Bearbeitung
    \item Abgeschlossen
    \item Abgebrochen
\end{itemize}

\subsection{Job-ID}
\label{B:Job-ID}
Die Job-ID ist eine Referenz auf den Job. Die Job-ID ist immer eindeutig.

