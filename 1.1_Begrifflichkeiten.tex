\section{Begrifflichkeiten}
%Erklärung von Begrifflichkeiten, die nicht von uns stammen (von mallob)
\label{B:Jobs}
\subsection{Jobs}

Jobs sind die Instanzen, die von Mallob verarbeitet werden. Ein einzelner Job stellt ein einzelnes zu lösendes Problem dar. Ein Job besteht aus der Job-Konfiguration und der Job—Beschreibung.
Jeder Job besitzt eigene Parameter, die bei der Bearbeitung durch Mallob beachtet werden. 
Diese Parameter bilden zusammen die Job-Konfiguration.

Die Parameter sind:
\begin{tabular}{lL{0.6\textwidth}l}
        Name & Beschreibung & Notwendig\\
        \\
        name & Der Name des Jobs & Ja\\
        priority & Die Prioriät des Jobs & Ja\\
        application & Das Anwendungsfeld des Jobs & Ja\\ 
        max-demand & Die maximaler Anzahl paralleler Prozesse, die dieser Job nutzen kann & Nein\\
        wallclock-limit & Zeitliches Ausführungsbudget des Jobs & Nein\\
        cpu-limit & Corestudnen-Ausführungsbudget des Jobs & Nein\\
        arrival & Frühester Bearbeitungsbeginn in Sekunden seit Start von Mallob & Nein\\
        dependencies & Jobs, die abgeschlossen sein müssen, bevor dieser Job beginnt & Nein\\
        incremental & Handelt es sich um einen Job mit mehreren Revisionen oder Inkrementen? &  Nein\\
        precursor & Vorgänger-Job bei inkrementellen Jobs & Nein\\
        % TODO: files hier auch?
        content-mode & Text- oder Binärformat der Beschreibung & Nein\\
        % TODO: literals und assumptions?
    \end{tabular} 



\subsection{Inkrementelle Jobs}

\subsection{Inhalt der Konfigurationsdatei}