
% Hier gibts ne Vorlage falls wir die Nutzen wollen https://git.scc.kit.edu/IPDSnelting/pflichtenheft/-/blob/master/pflichtenheft.tex
%Bindet die setup-datei ein. Beim Kompilieren wird alles was in setup.tex steht an diese stelle kopiert
%Das hier ist eine setup-Datei. hier schreiben wir Packages und setup-befehle für das Pflichtenheft rein. 
%Hält das ganze etwas übersichtlicher


%\documentclass[parskip=full,11pt,twoside]{scrartcl} - DIeses two side ist mega der bs
\documentclass[parskip=full,11pt]{scrartcl}
\usepackage[utf8]{inputenc}
\usepackage{hyperref}
\usepackage[nonumberlist]{glossaries}     % provides glossary commands, taken from SWT

\usepackage{svg}
\usepackage{enumitem}
\usepackage{tabularx}
\usepackage{changepage}
\usepackage{float}
% section numbers in margins:
\renewcommand\sectionlinesformat[4]{\makebox[0pt][r]{#3}#4}

% header & footer
\usepackage{scrlayer-scrpage}
\lofoot{\today}
\refoot{\today}
\pagestyle{scrheadings}


\usepackage[sfdefault,light]{roboto}
\usepackage[T1]{fontenc}
\usepackage[german]{babel}
\usepackage[yyyymmdd]{datetime} % must be after babel
\renewcommand{\dateseparator}{-} % ISO8601 date format
%usepackage[]{hyperref}
\usepackage{amsmath} % for $\text{}$
\usepackage[nameinlink]{cleveref}
\crefname{figure}{Abb}{Abb}
\usepackage[section]{placeins}
\usepackage{xcolor}
\usepackage{graphicx}
\hypersetup{
	pdftitle={Pflichtenheft},
	bookmarks=true,
}
\usepackage{csquotes}

\newcommand\urlpart[2]{$\underbrace{\text{\texttt{#1}}}_{\text{#2}}$}
\newenvironment{FA}
    {
    \begin{adjustwidth}{-3.5cm}{}
    \begin{tabular}[t]{rL{0.8\textwidth}}
    }
    {
    \end{tabular}
    \end{adjustwidth}
    \vspace{1em}
    }
    
\newenvironment{FAList}{\begin{itemize}[noitemsep, leftmargin=1cm,align=parleft]}{\end{itemize}}


\title{Pflichtenheft PSE}
\author{Valentin Schenk, Kaloyan Krasimirov Enev, Simon Wilhelm Schübel,\\ Maik Sept, Simon Aaron Giek}

\date{May 2022}

\begin{document}

\maketitle

\section{Introduction}

\section{Einleitung}


%Neben der
%-expliziten Benennung des Auftragnehmers und des Auftraggebers 
%sollte an dieser Stellung auch eine 
%-grobe Kurzbeschreibung des Projektes erfolgen. 
%Gehen Sie darauf ein, 
%--was das Projekt beinhaltet und 
%--wie das Endergebnis aussehen soll. 
%Wichtig ist, dass auch eine Person, die das erste Mal von dem Projekt hört, versteht, worum es geht.

%---Vorstellung des Auftraggebers und von uns 
\href{https://github.com/domschrei/mallob}{Mallob} ist ein dezentrales System zum Scheduling und Lösen von \glslink{NP-schweres Problem}{NP-schweren Problemen}, hauptsächlich entwickelt von Dominik Schreiber im Rahmen seiner Doktorarbeit. Der Auftraggeber - der Entwickler von Mallob - wünscht sich ein bedienerfreundliches Softwaresystem, um mit Mallob von außen kommunizieren zu können. \\
%---Grobe kruzbeschreibung des Projekts
Unser System \textbf{\textit{Fallob - a Friendly Face for Mallob}} soll diese Brücke zwischen Mallob und Außenwelt darstellen.
Es verfügt daher über all diejenigen Funktionen, die es möglich machen, mit Mallob zu interagieren. Es ist also möglich direkt \hyperref[B:Jobs]{Jobs} an Mallob zu senden und \hyperref[B:Job-Informationen]{Informationen} über diese zu erlangen.\\
Eine Hauptaufgabe von \textit{Fallob} ist die \hyperref[pages:visualization]{Visualisierung} des \hyperref[B:Systemzustand]{Systemzustandes} von Mallob. Hier ist es möglich zu sehen, wie Mallob die eigenen \hyperref[B:Jobs]{Jobs} (so werden Probleme genannt, die Mallob lösen soll) bearbeitet. Auch die Gesamtauslastung sowie die Zuordnung von \hyperref[B:Jobs]{Jobs} auf Prozessoren ist einsehbar. 
Die Interaktion mit Mallob soll sowohl über ein \gls{Web-Interface}, als auch direkt über eine von uns bereitgestellte \gls{API} möglich sein. %\\

%Des weiteren wird das System eine Nutzer-Verwaltung beinhalten, um %[warum haben wir eine Nutzerverwaltung?]. Identifikation - wer welche Job eingereicht hat. Erlaubt Rollenverteilung - auf Admin und Benutzer. Erlaubt Admins leichter in Kontakt mit dem Benutzer zu treten. Sicherheitsgründen - nicht jeder kann Jobs einreichen, Kontrolle darauf, wer Mallob benutzt. Historie speichern - erlaubt dem Nutzer Informationen über bisher eingereichten Jobs zu bekommen (nicht nur vom selben Rechner). "Schutz vor Fremdzugriffen und somit die Vertraulichkeit der Inhalte" - laut https://glossar.hs-augsburg.de/Benutzerverwaltung




%\section{Zielbestimmung}
%\subsection{Musskriterien}
%\subsubsection{Web-Interface}
%\begin{itemize}
%    \item Das System verfügt über ein \gls{Web-Interface}.
%    \item Beim ersten Aufruf des \glslink{Web-Interface}{Web-Interfaces} wird eine Anmelde-Maske angezeigt.
%    \item In der Anmelde-Maske der Nutzer auch zu einer Maske für die Registierung gelangen.
%    \item Ein \gls{Auftrag} kann über das \gls{Web-Interface} hinzugefügt werden. Dies geschieht über eine Eingabemaske.
%    \item Das Übergeben der Probleminstanz kann wahlweise direkt über ein Eingabefeld, hochladen der CNF-Datei oder Angabe einer URL erfolgen.
%    \item Das \gls{Web-Interface} besitzt eine Visualierung des Systemzustandes. Hier wird standardmäßig der aktuelle Zustand visualisiert,  es kann aber auch die Historie betrachtet werden.
%    \item Admins haben Zugriff auf eine exklusive Ansicht, mit der die Instanz von Mallob verwaltet werden kann.
%
%\subsection{Wunschkriterien}
%\subsubsection{Web-Interface}
%\begin{itemize}
%    \item Der Benutzer kann die Eingabe des Problems mit einem graphischem Editor vornehmen.
%    \item Ein Admin kann die Instanz von Mallob starten, beenden und neustarten. Hier gibt es auch die Möglichkeit, entsprechende Parameter einzugeben.
%\end{itemize}
%\end{itemize}

\section{Zielbestimmung}
Dieses Programm ermöglicht es Nutzern, einfacher mit \href{https://github.com/domschrei/mallob}{Mallob} zu interagieren. Diese Interaktion kann auf zwei Wege erfolgen:
Einerseits wird eine REST-API angeboten für die Interaktion angeboten. Anderseits gibt es aber auch ein benutzerfreundliches Web-Interface, welches die API nutzt und somit eine graphische Oberfläche für diese darstellt.
\subsection{Musskriterien}
    \subsubsection{API}
        \begin{itemize}
            \item Unterscheiden zwischen Benutzern und Administratoren
            \item Authentifizieren von Benutzern
            \item Registrierung von neuen Benutzern
            \item Verwalten von Jobs
                \begin{itemize}
                    \item Jobs hinzufügen
                    \item Jobs abbrechen
                    \item Aktuellen Status eines Jobs abfragen
                    \item Auflisten der eigenen Jobs
                \end{itemize}
            \item Abfragen von Jobs
            \item Bereitstellung eines Ereignis-Streams von Mallob zum Erhalten von Updates
            \item Erhalten von Informationen zu aufgetretenen Fehlern
        \end{itemize}
    \subsubsection{Web-Interface}
        \begin{itemize}
            \item Authentifizieren von Nutzern
            \item Registrierung von neuen Nutzern
            \item Verwalten von Jobs
                 \begin{itemize}
                    \item Jobs hinzufügen
                    \item Jobs abbrechen
                    \item Aktuellen Status des Jobs betrachten
                    \item Auflisten der eigenen Jobs
                \end{itemize}
            \item Einsehen der eigenen Aufträge und derer Ergebnisse
            \item Herunterladen der Ergebnisse der Jobs
            \item Anzeigen von aufgetretenen Fehlern
            \item Verifizierungsmöglichkeit von neuen Nutzern für Administratoren 
            
                      
        \end{itemize}
    \subsubsection{Visualisierung}    
        \begin{itemize}
            \item Das Web-Interface enthält eine Visualisierung des System-Zustands
            \item Wenn die Live-Ansicht ausgewählt ist, wird die Visualisierung automatisch aktualisiert
            %\item Darstellung der PE als Punkte %TODO: Ranks? oder PEs? oder ganz anderes Word?
            %\item Färbung der Punkte entsprechend dem zugehörigen Job
            %\item Verbinden der Punkte entsprechend der Position im Binärbaum des Jobs
            %\item Dynamische Größe der Punkte und Verbindungen entsprechend der Position im Binärbaum des Jobs.
        \end{itemize}
    
    \subsubsection{System}
        \begin{itemize}
            \item Konfigurationsdatei für globale Einstellungen des Systems
        \end{itemize}
        
        
\subsection{Wunschkriterien}
    \begin{itemize}
        %\item Speichern von Jobdaten für Statistiken -> Irgendwie redundaten, da wir eh schon jobdaten speichern
        \item Möglichkeit für Administratoren, zu starten, beenden oder neustarten
        \item Sortieren der Job-Tabelle im Web—Interface
        \item Graphischer Editor zur Eingabe der Job-Beschreibung
        \item Diagnosemöglichkeit für Administratoren
        \item Nutzer benachrichtigen, falls Mallob abstürzen sollte
        \item Darstellen eines bestimmten vergangenen Zeitpunktes bei der Visualisierung
        \item Ändern der Wiedergabegeschwindigkeit bei der Visualisierung
        \item Einsehen von Details zu einem aktiven Job in der Visualisierung
        \item Darstellung des vollständigen Binärbaums des Jobs
        \item Schnittstelle für Plugins in Web—Interface
    \end{itemize}
    
\subsection{Abgrenzungskriterien}
    \begin{itemize}
        \item Es kann maximal ein Auftrag pro Anfrage hinzugefügt werden
        \item Korrektheit der Daten wird durch Mallob geprüft
        \item Es gibt keine Möglichkeit, falsche Daten in der API zu korrigieren, in diesem Fall muss die gleiche Anfrage mit korrekten Daten erneut erfolgen
    \end{itemize}

\section{Produkteinsatz}
% TODO: das sollte noch mindestens zwei mal überarbeitet werden, lol

\subsection{Andwendungsbereich}

Das System dient dem Lösen von Problemen. Ein Nutzer kann das Web-Interface im Browser aufrufen oder stattdessen die bereitgestellte API nutzen.
\subsection{Zielgruppe}

Insgesamt richtet sich das System an Personen, die komplexe Probleme lösen möchten, deren Laufzeit zu groß ist. Hier kann dieses System helfen, indem es das Problem schneller löst. 
Das Web-Interface bietet eine moderne, einfach und intuitive Umgebung zur Nutzung dieses Systems. Hier sind abgesehen von der Bereitstellung der Job-Beschreibung im korrekten Format keine weiteren Kenntnisse notwendig. \\
Die API dagegen richtet sich an Personen, die dieses System in ihr eigenes integrieren möchten. Hier wird Wissen über den Umgang einer solchen API vorausgesetzt. 
\subsection{Betriebsbedingungen}

\begin{itemize}
    \item Zur Nutzung ist eine stabile Internetverbindung notwendig. Insbesondere kann eine instabile Verbindung dazu führen, das Ereignisse nicht in Echtzeit angezeigt werden.
    \item Zur Verwendung des Web-Interfaces muss ein aktuelle Version des Browsers genutzt werden.
    \item Das System muss Zugriff auf die API der Instanz von Mallob haben.
    \item Eine Instanz von Mallob muss existieren, sodass diese von diesem System genutzt werden kann. 
\end{itemize}
\section{Produktumgebung}

\subsection{Client-Seite}
Auf der Seite des Clients wird zur Nutzung des \glslink{Web-Interface}{Web-Interfaces} ein Browser mit aktuellster Version benötigt.\\
Der Browser kann dabei auf folgenden Betriebssystemen laufen:

\begin{itemize}
    \item Unix-Basierte Betriebssysteme
    \item Windows 10 und neuer
    \item Android
    \item iOS
\end{itemize}

Die \glslink{Betriebssystem}{Betriebssysteme} können auf folgender Hardware laufen:

\begin{itemize}
    \item Desktop-Rechner / Laptop 
    \item Mobiles Endgerät; Smartphone, Tablet, ...
\end{itemize}

    
\subsection{Server-Seite}
\begin{itemize}
    \item Da das Backend Java-basiert ist, ist es notwendig, dass die Hardware, auf der \textit{Fallob} läuft, Java 17 und älter unterstützt.
    \item Es müssen mindestens 300 Gigabyte Speicher zur Verfügung stehen, damit die anfallenden \hyperref[PD]{Produktdaten} gespeichert werden können.
\end{itemize}
\section{Funktionale Anforderungen}
% Sollten wir noch nutzen
%https://de.overleaf.com/learn/latex/Cross_referencing_sections%2C_equations_and_floats


\subsection{API}

\begin{itemize}
    \item[FA10] \textbf{Registrierung von Nutzern} \\
    Es ist möglich neue Nutzer über die API zu registrieren, sodass diese sich mit ihren Zugangsdaten authentifizieren können.
    
    \item[FA20] \textbf{Authentifizierung von Nutzern} \\
    Nach dem Registrieren ist es dem Nutzer möglich sich über die API zu authentifizieren und Zugriff auf die anderen Funktionen der API zu erlangen. Dies kann er mit seinem Bearer-Token tun. Diesen Token erhält ein Nutzer bei Registrierung.
    
    \item[FA30] \textbf{Einreichen von Jobs} \\
    Die API ermöglicht es Jobs, die durch eine JSON-Datei und eine Job-Beschreibung spezifiziert werden, zu übergeben. Es existieren drei verschiedene Möglichkeiten um die Job-Beschreibung zu übergeben. Die Jobs werden von Mallob bearbeitet und das Ergebnis wird an den Nutzer zurückgegeben.
    %genaue Spezifizierung der Jobs (Priorität,Laufzeit, ...) auch hier oder an anderer Stelle?
    
    \begin{itemize}
        \item[FA31] \textbf{Einreichen der Job-Beschreibung separat von der JSON-Datei} \\
        Es ist möglich die Job-Beschreibung in einer eigenen Datei zu spezifizieren. Dabei muss es sich um eine Datei handeln, die dem DIMACS CNF Standard entspricht. Diese Datei wird zusammen mit der JSON-Datei an die API übergeben.
        % soll das Dateiformat überhaupt hier schon spezifiziert werden?
        
        \item[FA32] \textbf{Job-Beschreibung innerhalb der JSON-Datei} \\
        Es ist möglich die Job-Beschreibung direkt in der JSON-Datei zu spezifizieren. Die Beschreibung muss auch in diesem Fall dem DIMACS CNF Format entsprechen. Bei dieser Möglichkeit wird nur die JSON-Datei an die API übergeben
        
        \item[FA33] \textbf{Übergeben der Job-Beschreibung über einen Link} \\
        Es kann ein Link an die API übergeben werden, der auf eine Datei verweist, in der die Job-Beschreibung enthalten ist. Die referenzierte Datei muss ebenfalls dem DIMACS CNF Format entsprechen. Zusätzlich zu dem Link muss die JSON-Datei mit den weiteren Job-Spezifikationen an die API übergeben werden
        
        \item[FA34] \textbf{Bereits eingereichte Job-Beschreibung verwenden}\\
        Ein Nutzer kann Job-Beschreibungen, welche er Bereits eingereicht hat über eine ID referenzieren und wiederverwenden.
        
    \end{itemize}
    
    \item[FA40] \textbf{Abbrechen von eingereichten Jobs} \\
    Der Nutzer kann einen eingereichten Job wieder abbrechen. In diesem Fall wird eine Statistik über die bereits verrichtete Arbeit zurückgegeben.
    
    \item[FA50] \textbf{Zurückgeben von Ergebnissen} \\
    Für jeden eingereichten Job gibt die API immer eine Antwort an den Nutzer zurück. 
    
    \begin{itemize}
        \item[F51] \textbf{Zurückgeben des Ergebnisses bei erfolgreicher Berechnung} \\
        Wurde der eingereichte Job erfolgreich gelöst, wird das Ergebnis des Jobs an den Nutzer zurückgegeben.
        
        \item[F52] \textbf{Zurückgeben des Ergebnisses nach Erreichen der maximalen Bearbeitungszeit} \\
        Wenn die maximale Bearbeitungszeit des eingereichten Jobs erreicht wurde und kein Ergebnis gefunden wurde, wird eine Statistik über die bereits verrichtete Arbeit an den Nutzer zurückgegeben
        
        \item[F53] \textbf{Zurückgeben des Ergebnisses nach einem Fehler} \\
        Wenn während der Bearbeitung des eingereichten Jobs ein Fehler auftritt, wird eine aussagekräftige Fehlermeldung an den Nutzer zurückgegeben
        
        \item[FA54] \textbf{Ergebnisabfrage von Jobs} \\
        Es ist möglich für jeden Job (auch nach Beendigung) seinen aktuellen Status (in Bearbeitung, Bearbeitet, Fehler) abzufragen. Die Antwort  enthält dabei alle Informationen zum Status des Jobs, wie das Ergebnis oder eventuelle Fehlermeldung.
    \end{itemize}
    

    \item[FA60] \textbf{Zurückgeben des Systemzustands von Mallob} \\
    
    

\end{itemize}


%------------------------------------------------------------WEB-Interface
\subsection{Web-Interface}



% Web-Interface nochmal anders formuliert, eher an Beispiel von Betreuern orrientiert
\begin{itemize}
     \item[FA010] \textbf{Web-Interface aufrufen} \\
        Nach dem Aufrufen des Web-Interface über die URL gelangt man zur Anmelde-Maske.

     \item[FA020] \textbf{Anmelden} \\
        Der Nutzer kann sich über das Web-Interface anmelden. Dies geschieht über die Anmelde-Maske. Die Anmeldung geschieht mit Nutzernamen und Passwort. Nach erfolgreicher Registierung wird der Nutzer zur Auftrag-Seite gebracht.
        
     \item[FA030] \textbf{Registrieren} \\
        Der Nutzer kann sich über das Web-Interface registrieren. Dies geschieht über die Registrieren-Maske. Diese kann von der Anmelde-Maske mit der entsprechenden Schaltfläche erreicht werden. 
        
    \begin{itemize}
        \item[FA031] \textbf{Daten zur Registrierung} \\
        Für die Registrierung wird ein Nutzername, ein Passwort und die wiederholte Eingabe des Passworts benötigt. Nach erfolgreicher Registierung wird der Nutzer zur Auftrag-Seite gebracht.
    \end{itemize}
        

        
    \item[FA040] \textbf{Ergebnisse einsehen} \\
        Befindet sich in der Liste der Aufträge ein abgeschlossener Auftrag, so kann über die zu diesem Auftrag gehörige Schaltfläche "get results" das Ergebnis angezeigt werden. Hier kann es auch in die Zwischenablage kopiert werden oder heruntergeladen werden.
        
   \item[FA050] \textbf{Auftrag hinzufügen} \\ 
        Mittels einer Schaltfläche über der Liste der eigenen Aufträge gelangt der Nutzer zu einer Eingabe-Maske, über welche er einen neuen Auftrag hinzufügen kann. 
   \item[FA00] \textbf{Auftrag abbrechen} \\
   
   \item[FA070] \textbf{Anzeigen von Fehlern} \\
        Tritt bei Mallob ein Fehler auf, so wird der Nutzer umgehend mittels einer Fehlermeldung darauf aufmerksam gemacht. Diese Fehlermeldung wird immer angezeigt, unabhängig davon auf welcher Seite der Nutzer sich momentan befindet. 
    \item[FA080] \textbf{Visualisierung} \\
        Das Web-Interface besitzt eine Visualisierung des System-Zustandes. Diese kann über den entsprechenden Reiter erreicht werden.
        
        \begin{itemize}
            \item[FA081] \textbf{Anzeigen des aktuellen Zustandes} \\
                Standardmäßig wird immer der aktuelle Zustand angezeigt. Dieser Zustand wird, solange kein anderer Zeitpunkt ausgewählt wurde, steht dynamisch aktuell gehalten. 
            \item[FA082] \textbf{Zeitachse} \\
                Die Visualisierung verfügt über eine Zeitachse, mit derer ein entsprechender Zeitpunkt der letzten [...] Minuten/Stunden ausgewählt werden kann. Nach der Auswahl wird der System-Zustand zum entsprechenden Zeitpunkt angezeigt.
            \item[FA083] \textbf{Zurückspringen zu aktueller Ansicht} \\
                Wird mittels FA71 [TODO: REF] der angezeigte Zeitpunkt geändert, so wird eine Schaltfläche angezeigt, mit der der Nutzer jederzeit wieder zur aktuellen Zeit zurückspringen kann.
            \item[FA084] \textbf{Einsehen der Verteilung eines Auftrages} \\
                 
        \end{itemize}
    \item[FA090] \textbf{Ändern von Benutzer-Daten} \\
    
    \item[FA100] \textbf{Verwalten von Nutzern}
    \item[FA110] \textbf{Verwalten von Nutzern}
        \begin{itemize}
            \item Nutzer löschen
            \item Nutzer verifizieren [TODO: besseres Wort für freigeben, ich meine damit das das der Admin den Nutzer eben bestätigen muss, bevor er Aufträge hinzufügen kann...j]
            \item 
        \end{itemize} 
    \item[FA120] \textbf{Abmelden} \\
        Der Nutzer kann sich jederzeit über das entsprechende Menü in der Navigationsleiste abmelden. In diesem Falle wird wieder die Anmelde-Maske angezeigt.
\end{itemize}




\subsection{Plugins}
\section{Nichtfunktionale Anforderungen}
\label{PD}
\section{Produktdaten}


\label{PD:Registrierungsdaten}
\subsection{Registrierungsdaten}
Die Registrierungsdaten sind diejenigen Daten, die ein Nutzer angeben muss, um sich registrieren zu können. Diese Daten werden gespeichert, bis das Konto durch einen Administrator verifiziert wurde.
\begin{itemize}[noitemsep]
    \item Nutzername
    \item verschlüsseltes Passwort
    \item E-Mail
\end{itemize}

\label{PD:Nutzerdaten}
\subsection{Nutzerdaten}
Nutzerdaten sind alle nutzerspezifischen Daten, die über den Nutzer dauerhaft oder temporär gespeichert werden.
\begin{itemize}[noitemsep]
    \item Nutzername
    \item verschlüsseltes Passwort
    \item E-Mail
    \item API-Requests %[TODO: warum]
    \item Eingereichte Jobs (so lange, wie in der Konfigurationsdatei definiert)
    \item Spalten, die der Nutzer in der \hyperref[pages:job-table]{Job-Tabelle} ausgewählt hat
\end{itemize}



\subsection{Job-Daten}
\begin{itemize}[noitemsep]
    \item Job-Konfiguration
    \item Job-Beschreibung
    \item Ergebnis des Jobs
    \item Job-ID
    \item Zughöriger Nutzer
    \item Status des Jobs
\end{itemize}

\subsection{Mallob-Daten}
\begin{itemize}
    \item Relevante Inhalte des Mallob-Outputstream (so lange, wie in der Konfigurationsdatei definiert.)
\end{itemize}

\section{Systemmodell}

\subsection{Anwendungsfalldiagramme}

\begin{center}
    \includegraphics[scale=0.4]{images-interface/UsecaseDiagram_Login-Register.jpg}
    Anwendungsfalldiagramm 1 : Benutzer-Login und Registrierung
\end{center}
\section{Produktleistungen}

%%
%ProduktleistungenSofern 
%an einzelne Funktionen des Programms besondere Anforderungen in Bezug auf die Zeit oder die Genauigkeit gestellt werden, sollten diese in diesem Kapitel dargestellt werden. Dabei sollten Sie prüfen, ob die zu erbringenden Leistungen mit den in Punkt 5 genannten Angaben  realisierbar sind.
%

\begin{itemize}[noitemsep]
    \item[P100] Die maximale Anzahl der \hyperref[B:Jobs]{Jobs} ist begrenzt.
    
    \item[P110] Die maximale Anzahl \hyperref[B:Jobs]{Jobs}, die ein Nutzer gleichzeitig in Bearbeitung haben kann, ist beschränkt.
    
    \item[P120] Die Zeit, die benötigt wird, um einen beliebigen Zeitpunkt in der Visualisierung darzustellen, muss linear in der Anzahl der zu ladenden \hyperref[B:Event]{Events} sein.
    
    \item[P130] (Wunschbedingung)  Die Zeit, die benötigt wird, um einen beliebigen Zeitpunkt in der Visualisierung darzustellen, muss konstant sein, unabhängig vom gewählten Zeitpunkt.
    
    \item[P140] Das \gls{Web-Interface} ist auch auf kleineren Bildschirmen, wie etwa einem Handy-Bildschirm, nutzbar.
    
    \item[P150] Die \gls{Konfigurationsdatei} wird immer nur beim Systemstart eingelesen, etwaige Änderungen werden also erst mit einem Neustart des Systems wirksam.
    
    \item[P160] Beim Einreichen eines \hyperref[B:Jobs]{Jobs} im Interface erfolgt schon im Frontend eine Kontrolle der Syntax, welche den Nutzer momentan über Fehler in der Eingabe informiert.
    
    \item[P170] Nutzernamen sind eindeutig und bestehen aus 4 bis 25 Zeichen.

    \item[P180] Passwörter müssen mindestens 8-stellig sein.
    
    \item[P190] Die gespeicherten \hyperref[B:Jobs]{Jobs} werden automatisch nach einem spezifizierten Zeitraum gelöscht.

    \item[P200] Die Größe der \hyperref[B:Job-Beschreibung]{Job-Beschreibung}, die man im Web-Interface eingeben kann, ist beschränkt.

    
    \item[P210] Jedes \gls{Nutzerkonto} besitzt nach Registrierung die gleiche Priorität. Diese kann vom kann vom \gls{System-Administrator} geändert werden.
    
    
    \item[P220] Der \gls{Nutzer} erhält auf jede \gls{API}-Anfrage außer \hyperref[FA:API:Andauernde Abfrage des Ergebnisses eines Jobs]{F1110} unmittelbar eine  Antwort.
    
    \item[P230] Muss in der \hyperref[pages:visualization]{Visualisierung} zu viel angezeigt werden, so wird die Qualität herabgestuft, um weiterhin eine performante Darstellung zu ermöglichen. Dies geschieht beispielsweise durch das weglassen von Verbindungen zwischen den Prozessen.
    
    \item[P240] Daten über Jobs, die nicht dem angemeldetem Nutzer gehören, werden stehts pseudomynisiert ausgegeben und dargestellt. Ist ein Administrator  angemeldet, so werden die Daten nicht pseudomynisiert.
    
    \item[P250] Es ist möglich, die Job-Seite direkt über eine passende \gls{URL} aufzurufen.
    
    \item[P260] Die maximale Geschwindigkeit der Visualisierung ist das zweihundertfache.

    
\end{itemize}
\section{Benutzeroberfläche}
%[TODO: remove this link] https://tex.stackexchange.com/questions/442077/is-it-possible-to-use-svg-images-with-overleaf

%[TODO: Arrange the images correctly, probably see here https://de.overleaf.com/learn/latex/Inserting_Images]
\subsection{Bilder des GUI-Entwurfs}
\begin{figure}[H]
    \centering

        \includegraphics[width=\textwidth]{images-interface/Loginv1.png}
        \caption{Anmelde-Maske}
        \label{fig:login}
   
        \includegraphics[width=\textwidth]{images-interface/Job-Viewv1.png}
        \caption{Auftrags-Übersicht}
        \label{fig:viewjobs}
  
\end{figure}


\begin{figure}[H]
    \centering
    \includegraphics[width=\textwidth]{images-interface/Submit-Filev1.png}
    \caption{Maske zum Hinzufügen neuer Aufträge}
    \label{fig:addjobs}
     \includegraphics[width=\textwidth]{images-interface/overviewv1.png}
    \caption{Visualisierung des Systems}
    \label{fig:visualsn}
\end{figure}


\subsection{Funktionen}
\begin{itemize}
    \item /B010/ Es sind zwei Sichten zu unterscheiden: die des Admins, die des Benutzers. 
    \item /B020/ Benutzer können Funktionen F10, F20, F30 jederzeit nach dem Einloggen aufrufen.
    \item /B030/ Jeder Benutzer fängt auf der Login/Register-Seite an.
    \item /B040/ Sobald der Benutzer eingeloggt wird, sieht er die Startseite.
    \item /B050/ Jeder Nutzer kann seine Jobs anhand der Aufwand auf das Kern unterscheiden (z.B Farbe, Größe usw.)
    \item /B060/ Admins können alle Funktionen, die die Benutzer können.
    \item /B070/ Unterschiedliche Benutzer und ihre Befugnisse sollen entsprechend behandelt werden (nicht-funktionale Anforderung?).
    \item /B080/ Die Bedienungsoberfläche ist auf Mausbedienung auszulegen; eine Bedienung ohne Maus muss aber auch möglich sein. ([TODO] Bediengung ohne Maus auf jeden Fall wunschkriterium oder vlt sogar gar nicth, auf jeden Fall mal fragen, hört sich kompliziert und nicht wirklich notwendig an)
\end{itemize}
\section{Testszenarien}

Testfälle:
•	Aufrufen des Web-Interface
•	Registrieren eines neuen Benutzers
•	Verifizierung einer neuen Benutzerregistrierung durch einen Admin
•	Anmelden eines registrierten Benutzers
•	Ändern der Benutzerdaten eines registrierten Benutzers
•	Anmelden eines Admins
•	Ändern der Benutzerdaten eines Admins
•	Abmelden eines Benutzers
•	Abmelden eines Admins
•	Anmelden eines Benutzers mit fehlerhaften Zugangsdaten
•	Anmelden eines Admins mit fehlerhaften Zugangsdaten
•	Priorität eines Benutzers durch einen Admin ändern
•	Einreichen eines Jobs durch eine JSON-Datei und eine separate Job-Beschreibungs-Datei
•	Einreichen eines Jobs durch eine JSON-Datei mit enthaltener Job-Beschreibung
•	Einreichen eines Jobs durch eine JSON-Datei mit einem Link, der auf eine Job-Beschreibungs-Datei verweist
•	Einreichen eines Fehlerhaften Jobs
•	Abbrechen einer Jobbearbeitung
•	Erhalten einer Antwort auf einen vollendeten Job/abgebrochenen Job
•	Erstellen eines neuen korrekten Jobs
•	Erstellen eines fehlerhaften Jobs
•	Aufrufen des derzeitigen Systemzustands von mallob
•	Aufrufen eines hierarchischen Baumes aller an einem Job arbeitenden Prozesseinheiten 
•	Aufrufen eines vergangenen Systemzustands von mallob über die Zeitachse
•	Aufrufen der laufenden Jobs eines Benutzers
•	Aufrufen aller der Zeit auf dem System laufenden Jobs durch einen Admin
•	Aufrufen der Statistik eines vollendeten/abgebrochenen Jobs
•	Aufrufen bereits abgeschlossener Jobs 

Erweiterte Testfälle
•	Aufrufen einer Visualisierung der Lösung eines Jobs
•	Mallob beenden
•	Mallob neustarten
•	


%---------------------------Kommunikation mit Mallob------    
\section{Kommunikation mit Mallob}

\begin{itemize}
    \item Annahme für uns : Mallob läuft auf selben Dateisystem wie Backend 
    \item JSON und Jobbeschreibung werden in einem Verzeichnis im Dateisystem abgelegt und von dort aus von Mallob verarbeitet
    \item Der Output wird von Mallob ebenfalls las Datei in einem Verzeichnis ausgegeben
    \item JSON und Jobbschreibung können unterschiedliche Dateien sein (siehe Einreichen der Jobbeschreibung, API)
\end{itemize}



%--------------------------------Abgrenzungskritierien-----
\section{Abgrenzungskriterien}
    \begin{itemize}
        \item Nur eine Jobbeschreibung pro Anfrage
        \item Korrektheit der Daten wird durch Mallob geprüft
        \item Es wird keine Korrektur von falschen Daten durch die API geben
    \end{itemize}









%------------------------Glossar

\section{Glossar}
\end{document}
