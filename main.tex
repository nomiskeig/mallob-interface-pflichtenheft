
% Hier gibts ne Vorlage falls wir die Nutzen wollen https://git.scc.kit.edu/IPDSnelting/pflichtenheft/-/blob/master/pflichtenheft.tex
%Bindet die setup-datei ein. Beim Kompilieren wird alles was in setup.tex steht an diese stelle kopiert
%Das hier ist eine setup-Datei. hier schreiben wir Packages und setup-befehle für das Pflichtenheft rein. 
%Hält das ganze etwas übersichtlicher


%\documentclass[parskip=full,11pt,twoside]{scrartcl} - DIeses two side ist mega der bs
\documentclass[parskip=full,11pt]{scrartcl}
\usepackage[utf8]{inputenc}
\usepackage[linkcolor=blue, colorlinks=true]{hyperref}
\usepackage{glossaries}     % provides glossary commands, taken from SWT

\usepackage{svg}
\usepackage{enumitem}
\usepackage{tabularx}
\usepackage{changepage}
\usepackage{float}
\usepackage{xcolor}
% section numbers in margins:
\renewcommand\sectionlinesformat[4]{\makebox[0pt][r]{#3}#4}

% header & footer
\usepackage{scrlayer-scrpage}
\lofoot{\today}
\refoot{\today}
\pagestyle{scrheadings}


\usepackage[sfdefault,light]{roboto}
\usepackage[T1]{fontenc}
\usepackage[german]{babel}
\usepackage[yyyymmdd]{datetime} % must be after babel
\renewcommand{\dateseparator}{-} % ISO8601 date format
%usepackage[]{hyperref}
\usepackage{amsmath} % for $\text{}$
\usepackage[nameinlink]{cleveref}
\crefname{figure}{Abb}{Abb}
\usepackage[section]{placeins}
\usepackage{xcolor}
\usepackage{graphicx}
\hypersetup{
	pdftitle={Pflichtenheft},
	bookmarks=true,
}
\usepackage{csquotes}

\newcommand\urlpart[2]{$\underbrace{\text{\texttt{#1}}}_{\text{#2}}$}
\newenvironment{FA}
    {
    \begin{adjustwidth}{-3.5cm}{}
    \begin{tabular}[t]{rL{0.8\textwidth}}
    }
    {
    \end{tabular}
    \end{adjustwidth}
    \vspace{1em}
    }
    
\newenvironment{FAList}{\begin{itemize}[noitemsep, leftmargin=1cm,align=parleft]}{\end{itemize}}


\title{Pflichtenheft PSE}
\author{Valentin Schenk, Kaloyan Krasimirov Enev, Simon Wilhelm Schübel,\\ Maik Sept, Simon Aaron Giek}

\date{May 2022}
% https://en.wikibooks.org/wiki/LaTeX/Glossary

%\makenoidxglossaries
\makeglossaries

%\newglossaryentry{Job}{
%    name=Job,
%    plural=Jobs,
%    description={Jobs sind die Instanzen, die von Mallob verarbeitet werden. Ein einzelner Job stellt ein einzelnes zu lösendes Problem dar. Ein Job besteht aus Sicht des Nutzers aus der \gls{Job-Konfiguration} und der \gls{Job-Beschreibung}}
%}
%
%\newglossaryentry{Job-Konfiguration}{
%    name=Job-Konfiguration,
%    plural=Job-Konfigurationen,
%    description={
%    Die Job-Konfiguration beinhaltet alle Parameter des Jobs, welche bei der Bearbeitung berücksichtigt werden. Sie sind in den \hyperref[B:Job]{Begrifflichkeiten} näher erläutert
%    }
%}
%
%\newglossaryentry{Job-Beschreibung}{
%    name=Job-Beschreibung,
%    plural=Job-Beschreibungen,
%    description={Die Job-Beschreibung ist der Teil des Jobs, der das eigentliche Problem darstellt}
%}
%
%\newglossaryentry{Job-Informationen}{
%    name=Job-Informationen,
%    description={Die Job-Informationen enthalten die \gls{Job-Konfiguration} und den Einreiche-Zeitpunkt, den Zustand des Jobs, die Informationen die Mallob über den Job bereitstellt und die Job-ID. Sie enthält nicht die Job-Beschreibung und nicht das rohe Ergebnis des Jobs}
%}
%
%\newglossaryentry{Job-Updates}{
%    name=Job-Updates,
%    description={Job-Updates sind jene Informationen eines Jobs, die für die Visualisierung des Systems benötigt werden. Dazu gehören die Informationen, wie Mallob den Job gerade auf die Kerne verteilt hat und wie sich der \gls{Binaerbaum} des Jobs verändert}
%}


\newglossaryentry{Nutzer}{
    name=Nutzer,
    plural=Nutzer,
    description={Ein Nutzer ist eine Person, welche sich registriert hat und die mit dem System interagiert}
}


\newglossaryentry{Nutzerkonto}{
    name=Nutzerkonto,
    plural=Nutzerkonten,
    description={Ein Nutzerkonto ist ein im System durch einen Nutzer registriertes und durch einen Administrator verifiziertes Konto}
}

\newglossaryentry{Administrator}{
    name=Administrator,
    plural=Administratoren,
    description={Administratoren sind Nutzer mit mehr Rechten zur Verwaltung des Systems. Sie haben dennoch alle Möglichkeiten, die auch ein Nutzer hat, welcher kein Administrator ist}
}

\newglossaryentry{Web-Interface}{
    name=Web-Interface,
    description={Mit Web-Interface wird die Webseite von \textit{Fallob} referenziert, die der Nutzer im Internet aufrufen kann}
}
\newglossaryentry{API}{
    name=API,
    plural=APIs,
    description={Application Programming Interface. Wird im Kontext dieses Systems genutzt, um die Dienste in einer Art und Weise bereitszustellen, dass sie in andere Anwendungen integriert werden kann}
}

\newglossaryentry{Konfigurationsdatei}{
    name=Konfigurationsdatei,
    description={Eine Datei in einem spezifischen Format, die vom System eingelesen wird. Darin werden bestimmte Werte definiert, die dann vom System verwendet werden}
}
\newglossaryentry{System-Administrator}{
    name=System-Administrator,
    description={Eine Person, die die Ausführungsumgebung des Systems verwaltet. Verantwortlich für das Starten und Beenden des Systems}
}

\newglossaryentry{Authentifizierungstoken}{
    name=Authentifizierungstoken,
    description={Ein Token (sog. Bearer-Token), kann benutzt werden, um sich gegenüber einer API zu authentifizieren. Der Token verweist auf nur genau einen Nutzer. Dieser Token kann von jedem benutzt werden, der ihn besitzt (deswegen Bearer-Token). Der Token wird für jeden Nutzer bei der Registrierung generiert, sodass niemals zwei Nutzer denselben Token haben. Für den Nutzer ist es wichtig den Token, wie seine Anmeldedaten, geheim zu halten, bzw. nur authorisierten Personen mitzuteilen}
}

%\newglossaryentry{Anfrage}{
%    name=Anfrage
%    plural=Anfragen
%    description={Ein Nutzer kann eine Anfrage an eine API oder eine Website stellen.}
%}

\newglossaryentry{Datenbank}{
    name=Datenbank,
    plural=Datenbanken,
    description={Eine Datenbank ist ein System, welches zur Datenspeicherung und Verwaltung genutzt wird. Die Hauptaufgabe einer Datenbank besteht darin, vordefinierte Daten zu speichern und schnellen Zugriff auf die Daten zu erlangen}
}

\newglossaryentry{Output-Log}{
    name=Output-Log,
    description={}
    }

\newglossaryentry{Stream}{
    name=Stream,
    description={Ein Stream von Daten ist eine andauernde eingehende oder ausgehende Menge von Daten}
}

\newglossaryentry{Log-Datei}{
    name=Log-Datei,
    description={Eine Log-Datei ist eine Datei, welche Log-Daten speichert. Log-Daten sind Meta-Daten, welche gewisse Ereignisse festhalten sollen. Im Kontext von Mallob beispielsweise das Eingehen eines Jobs}
}

\newglossaryentry{Vorlaeufiges Nutzerkonto}{
    name={Vorläufiges Konto},
    description={Ein vorläufiges Konto sind Konten, die neu erstellt wurden und noch nicht durch einen \gls{Administrator} verifiziert wurden. Ein solches Konto eingeschränkt, es könnten noch keine Jobs in Auftrag gegeben werden}
}

    
\newglossaryentry{Bearbeitungszeit}{
name=Bearbeitungszeit,
description= {tt}%[todo]
}


\newglossaryentry{Dropdown-Menue}{
    name=Dropdown-Menü,
    plural=Dropdown-Menüs,
    description={Das Dropdown-Menü ist eine spezielle Form eines Auswahlmenüs. Nach dem Klick auf einen entsprechenden Button oder durch die Berührung mit dem Mauszeiger erscheint eine Auswahlliste auf dem Bildschirm. Durch einen weiteren Klick auf den gewünschten Menüpunkt wird dieser aufgerufen}
}

\newglossaryentry{URL}{
    name=URL,
    plural=URLs,
    description={Die URL (Uniform Resource Locator) ist die Adresse einer einzelnen Webseite}
}


%--------------neue glossareinträge ab 26.05.2022

\newglossaryentry{Checkbox}{
    name=Checkbox,
    plural={Checkboxen},
    description={Eine Checkbox ist ein Kästchen, welches 'gecheckt' werden kann, also betätigt oder nicht betätigt. Im falle des clickens einer Checkbox wird ein Haken in die Checkbox gesetzt (die Checkbox ist bestätigt), welcher bestehen bleibt, bis die checkbox ein weiteres mal geklickt wird (Checkbox ist nicht mehr bestätigt)}
}

\newglossaryentry{Nutzername}{
    name=Nutzername,
    plural=Nutzernamen,
    description={Der Nutzername ist derjenige Name, mit dem sich ein \gls{Nutzer} im System registriert und mit dem er auch referenziert wird.}
}


\newglossaryentry{Model-View-Controller}{
    name=Model-View-Controller,
    description={Model-View-Controller beschreibt ein Konzept aus der Softwaretechnik, nachdem eine Software in drei Komponenten aufgeteilt wird}
    }
    
\newglossaryentry{API-Anfrage}{
    name= API-Anfrage,
    description={Eine API-Anfrage ist die Benutzung der API durch einen Nutzer}%TODO : was??
}

\newglossaryentry{Testueberdeckung}{
    name=Testüberdeckung,
    plural=Testüberdeckungen,
    description={Die Testüberdeckung eines Programmes beschreibt den prozentualen Anteil der Codezeilen, die durch mindestens einen Testfall überprüft werden}
}


\newglossaryentry{Betriebssystem}{
    name=Betriebssystem,
    plural=Betriebssysteme,
    description={Ein Betriebssystem ist eine Software, die eine Schnittstelle zwischen der Hard- und Software des Computers herstellt und das Ausführen von Programmen ermöglicht}
}

\newglossaryentry{k-Means}{
    name={k-Means},
    description={Ein k-Means Algorithmus ist ein verfahren zur Vektorquantisierung, das auch zur Clusteranalyse verwendet wird. Dabei wird aus einer Menge von ähnlichen Objekten eine vorher bekannte Anzahl von k Gruppen gebildet}
}

\newglossaryentry{SAT}{
    name=SAT,
    description={SAT steht für Statisfyability und beschreibt ein NP-Schweres Problem. Eine SAT-Probleminstanz besteht aus einer Klauselmenge $C$ und einer Variablenmenge $V$. Eine Erfüllende Lösung für eine SAT-Instanz weißt jeder Variable $v \in V$ eine Belegung $b \in \{0,1\}$ zu, sodass alle Klauseln erfüllt sind}
}

\newglossaryentry{NP-schweres Problem}{
    name={NP-schweres Problem},
    plural={NP-schwere Probleme},
    description={NP-Schwere bezeichnet die Eigenschaft eines algorithmischen Problems, mindestens so schwer lösbar zu sein, wie die Probleme der Klasse NP. Die Klasse NP beinhaltet Probleme, für deren Lösung keine Algorithmen in Polynomialzeit bekannt sind} 
}


\newglossaryentry{Versionsverwaltung}{
    name=Versionsverwaltung,
    plural=Versionsverwaltungen,
    description={Eine Versionsverwaltung ist ein Programm, das zur Erfassung von Änderungen an Dokumenten oder Dateien verwendet wird. Es ist somit möglich, Änderungen rückgängig zu machen und alte Versionen wiederherzustellen}
}

\newglossaryentry{Systemzustand}{
    name=Systemzustand,
    description={Der Systemzustand beschreibt die Aktuelle Einstellung des Systems. Jede Einstellung, jede gespiecherte Variable}
}

\newglossaryentry{Prozess}{
    name=Prozess,
    plural=Prozesse,
    description={Ein Prozess, auch Programminstanz genannt, ist ein Computerprogramm zur Laufzeit. Genauer ist ein Prozess die konkrete Instanziierung eines Programms}
}

\newglossaryentry{Binaerbaum}{
    name={Binärbaum},
    plural={Binärbäume},
    description={Ein Binärbaum ist, im Sinne der Graphentheorie, ein zusammenhängender, kreisfreier Graph, welcher für jeden Knoten höchstens Ausgansgrad 2 aufweist. In unserem Fall sprechen wir speziell von balancierten Binär-Bäumen, dass heißt, dass die Höhe des Baumes (die maximal mögliche Länge eines Weges, der in der Wurzel endet), durch $c*log(n)$ beschränkt ist (dabei ist $c$ eine Konstante und $n$ die Anzahl der Elemente im Baum)}
}

\newglossaryentry{Teilbaum}{
    name={Teilbaum},
    plural={Binär-Bäume},
    description={Ein Teilbaum ist, im Sinne der Graphentheorie, ein Baum, dessen Wurzel ein Konten eines anderen Baumes ist}
}


\newglossaryentry{Plugin}{
    name=Plugin, 
    plural=Plugins,
    description={Ein Plug-In ist ein Softwareprogramm, auf das von anderen Softwareanwendungen zugegriffen werden kann, um deren Funktionalität zu erweitern}
}

\begin{document}

\maketitle

\newpage
\tableofcontents 
\newpage
\section{Einleitung}


%Neben der
%-expliziten Benennung des Auftragnehmers und des Auftraggebers 
%sollte an dieser Stellung auch eine 
%-grobe Kurzbeschreibung des Projektes erfolgen. 
%Gehen Sie darauf ein, 
%--was das Projekt beinhaltet und 
%--wie das Endergebnis aussehen soll. 
%Wichtig ist, dass auch eine Person, die das erste Mal von dem Projekt hört, versteht, worum es geht.

%---Vorstellung des Auftraggebers und von uns 
Mallob ist ein dezentrales System zur Lösung von NP-schweren Problemen, hauptsächlich entwickelt von Dominik Schreiber im Rahmen seiner Doktorarbeit. Der Auftraggeber - die Entwickler von Mallob - wünschen sich von 5 randoms eine [bedienerfreundliche Möglichkeit, Schnittstelle, Interface], um mit Mallob von außen kommunizieren zu können.\\

%---Grobe kruzbeschreibung des Projekts
Unser System - \textbf{a friendly face for Mallob} - soll diese Brücke zwischen Mallob und Außenwelt darstellen. 

Es verfügt daher über all diejenigen Funktionen, die es möglich machen mit Mallob zu interagieren. Es ist also möglich direkt Aufträge an Mallob zu stellen und Informationen über diese zu erlangen.

Eine Hauptaufgabe des friendly faces ist die Echtzeit-Visualisierung der Arbeitsweise von Mallob. Hier ist es möglich zu sehen wie Mallob die eigenen Jobs (so werden Probleme genannt, die Mallob lösen soll) bearbeitet. Auch die Gesamtauslastung sowie Arbeitsweise ist einsehbar. 

Die Interaktion mit Mallob soll sowohl über ein Web-Interface, als auch direkt über eine von uns bereitgestellte API möglich sein. %\\

%Des weiteren wird das System eine Nutzer-Verwaltung beinhalten, um %[warum haben wir eine Nutzerverwaltung?]. 

\newpage
%\section{Zielbestimmung}
%\subsection{Musskriterien}
%\subsubsection{Web-Interface}
%\begin{itemize}
%    \item Das System verfügt über ein \gls{Web-Interface}.
%    \item Beim ersten Aufruf des \glslink{Web-Interface}{Web-Interfaces} wird eine Anmelde-Maske angezeigt.
%    \item In der Anmelde-Maske der Nutzer auch zu einer Maske für die Registierung gelangen.
%    \item Ein \gls{Auftrag} kann über das \gls{Web-Interface} hinzugefügt werden. Dies geschieht über eine Eingabemaske.
%    \item Das Übergeben der Probleminstanz kann wahlweise direkt über ein Eingabefeld, hochladen der CNF-Datei oder Angabe einer URL erfolgen.
%    \item Das \gls{Web-Interface} besitzt eine Visualierung des Systemzustandes. Hier wird standardmäßig der aktuelle Zustand visualisiert,  es kann aber auch die Historie betrachtet werden.
%    \item Admins haben Zugriff auf eine exklusive Ansicht, mit der die Instanz von Mallob verwaltet werden kann.
%
%\subsection{Wunschkriterien}
%\subsubsection{Web-Interface}
%\begin{itemize}
%    \item Der Benutzer kann die Eingabe des Problems mit einem graphischem Editor vornehmen.
%    \item Ein Admin kann die Instanz von Mallob starten, beenden und neustarten. Hier gibt es auch die Möglichkeit, entsprechende Parameter einzugeben.
%\end{itemize}
%\end{itemize}

\section{Zielbestimmung}
\textit{Fallob} ermöglicht es Nutzern, einfacher mit \href{https://github.com/domschrei/mallob}{Mallob} zu interagieren. Diese Interaktion kann auf zwei Wege erfolgen:
Einerseits wird eine REST-API für die Interaktion angeboten. Anderseits gibt es aber auch ein benutzerfreundliches Web-Interface, welches die API nutzt und somit eine graphische Oberfläche für diese darstellt. Auf beide Wege können eigene Jobs verwaltet werden. Darüber hinaus bietet das Web-Interface auch eine Visualisierung des Systems.


\subsection{Musskriterien}
    \subsubsection{API}
        \begin{itemize}[noitemsep]
            \item Authentifizieren von Nutzern
            %\item Registrieren von neuen Nutzer
            \item Job einreichen
            \item Job abbrechen
            \item Informationen zu Jobs abfragen
            \item Ergebnis von Jobs abfragen
            \item Beschreibung von Jobs abfragen
            \item Bereitstellung eines Ereignis-Streams von Mallob zum Erhalten von Updates von Jobs
            \item Abfrage des Status der Mallob-Instanz
            %\item Erhalten von Informationen zu aufgetretenen Fehlern
        \end{itemize}
    \subsubsection{Web-Interface}
        \begin{itemize}[noitemsep]
            \item Authentifizieren von Nutzern
            %\item Registrierung von neuen Nutzern
            \item Tabellarische Übersicht  der Jobs
                \begin{itemize}[noitemsep]
                    \item Abbrechen mehrerer Jobs auf einmal
                    \item Herunterladen von mehreren Ergebnissen auf einmal
                \end{itemize}
            \item Job über Eingabemaske einreichen
                \begin{itemize}[noitemsep]
                    \item Einreichen der Job-Beschreibung über ein Eingabe-Feld
                    \item Einreichen der Job-Beschreibung durch Hochladen einer entsprechenden Datei
                \end{itemize}
            \item Job abbrechen
            \item Aktuellen Status des Jobs betrachten
            \item Job-Seite um Details von Jobs anzusehen
            \item Herunterladen der Ergebnisse der Jobs
            \item Anzeigen von aufgetretenen Fehlern bei Nutzern
            \item Anzeigen von Warnungen von Mallob im Administratoren-Bereich 
            %\item Verifizierungsmöglichkeit von neuen Nutzern für Administratoren 
            
                      
        \end{itemize}
    \subsubsection{Visualisierung}    
        \begin{itemize}[noitemsep]
            \item Das Web-Interface enthält eine Visualisierung des System-Zustands
                \begin{itemize}[noitemsep]
                    \item Darstellung der Prozesse als Punkte, ein Punkt je Prozess %TODO: Ranks? oder PEs? oder ganz anderes Word?
                    \item Färbung der Punkte entsprechend dem zugehörigen Job
                    \item Verbinden der Punkte entsprechend der Position im Binärbaum des Jobs
                    \item Dynamische Größe der Punkte und Verbindungen entsprechend der Position im Binärbaum des Jobs
                \end{itemize}
            \item Automatische Aktualisierung der Visualisierung, wenn die Live-Ansicht ausgewählt ist
            %\item Wenn die Live-Ansicht ausgewählt ist, wird die Visualisierung automatisch aktualisiert
            \item Darstellen eines bestimmten vergangenen Zeitpunktes
            \item Ändern der Wiedergabegeschwindigkeit
            \item Einsehen von Details zu einem aktiven Job in einem separatem Feld 
            %\item Darstellung der PE als Punkte %TODO: Ranks? oder PEs? oder ganz anderes Word?
            %\item Färbung der Punkte entsprechend dem zugehörigen Job
            %\item Verbinden der Punkte entsprechend der Position im Binärbaum des Jobs
            %\item Dynamische Größe der Punkte und Verbindungen entsprechend der Position im Binärbaum des Jobs.
        \end{itemize}
    
    \subsubsection{System}
        \begin{itemize}[noitemsep]
            \item Konfigurationsdatei für globale Einstellungen des Systems
        \end{itemize}
        
        
\subsection{Wunschkriterien}
    \begin{itemize}[noitemsep]
        %\item Speichern von Jobdaten für Statistiken -> Irgendwie redundaten, da wir eh schon jobdaten speichern
        \item Möglichkeit für Administratoren, Mallob zu starten, beenden oder neuzustarten
        \item Registrierung von Nutzern über die API
        \item Registrierung von Nutzern über das Web-Interface
        \item Sortieren der Job-Tabelle im Web-Interface
        \item Möglichkeit zum Neustart von abgebrochenen oder abgeschlossenen Jobs
        \item Eingabemaske für die Eingabe der Job-Beschreibung im Web-Interface
        \item Einreichen der Job-Beschreibung über eine URL, die auf die Job-Beschreibung zeigt
        \item Diagnosemöglichkeit für Administratoren im Web-Interface
        \item Nutzer benachrichtigen, falls Mallob abstürzen sollte
        \item Darstellung des vollständigen Binärbaums des Jobs in der Visualisierung
        \item Schnittstelle für Plugins in Web-Interface
        \item Entwicklung eines beispielhaften Plugins
    \end{itemize}
    
\subsection{Abgrenzungskriterien}
    \begin{itemize}[noitemsep]
        \item Es kann maximal ein Auftrag pro Anfrage hinzugefügt werden
        \item Korrektheit der Daten wird durch Mallob geprüft
        \item Es gibt keine Möglichkeit, falsche Daten in der API zu korrigieren, in diesem Fall muss die gleiche Anfrage mit korrekten Daten erneut erfolgen
    \end{itemize}

\newpage
\section{Produkteinsatz}
% TODO: das sollte noch mindestens zwei mal überarbeitet werden, lol

\subsection{Andwendungsbereich}

Das System dient dem Lösen von Problemen. Ein Nutzer kann das Web-Interface im Browser aufrufen oder stattdessen die bereitgestellte API nutzen.
\subsection{Zielgruppe}

Insgesamt richtet sich das System an Personen, die komplexe Probleme lösen möchten, deren Laufzeit zu groß ist. Hier kann dieses System helfen, indem es das Problem schneller löst. 
Das Web-Interface bietet eine moderne, einfach und intuitive Umgebung zur Nutzung dieses Systems. Hier sind abgesehen von der Bereitstellung der Job-Beschreibung im korrekten Format keine weiteren Kenntnisse notwendig. \\
Die API dagegen richtet sich an Personen, die dieses System in ihr eigenes integrieren möchten. Hier wird Wissen über den Umgang einer solchen API vorausgesetzt. 
\subsection{Betriebsbedingungen}

\begin{itemize}
    \item Zur Nutzung ist eine stabile Internetverbindung notwendig. Insbesondere kann eine instabile Verbindung dazu führen, das Ereignisse nicht in Echtzeit angezeigt werden.
    \item Zur Verwendung des Web-Interfaces muss ein aktuelle Version des Browsers genutzt werden.
    \item Das System muss Zugriff auf die API der Instanz von Mallob haben.
    \item Eine Instanz von Mallob muss existieren, sodass diese von diesem System genutzt werden kann. 
\end{itemize}
\newpage
\section{Produktumgebung}
\newpage
\section{Funktionale Anforderungen}
% Sollten wir noch nutzen
%https://de.overleaf.com/learn/latex/Cross_referencing_sections%2C_equations_and_floats


\subsection{API}

\begin{itemize}
    \item[FA10] \textbf{Registrierung von Nutzern} \\
    Es ist möglich neue Nutzer über die API zu registrieren, sodass diese sich mit ihren Zugangsdaten authentifizieren können.
    
    \item[FA20] \textbf{Authentifizierung von Nutzern} \\
    Nach dem Registrieren ist es dem Nutzer möglich sich über die API zu authentifizieren und Zugriff auf die anderen Funktionen der API zu erlangen. Dies kann er mit seinem Bearer-Token tun. Diesen Token erhält ein Nutzer bei Registrierung.
    
    \item[FA30] \textbf{Einreichen von Jobs} \\
    Die API ermöglicht es Jobs, die durch eine JSON-Datei und eine Job-Beschreibung spezifiziert werden, zu übergeben. Es existieren drei verschiedene Möglichkeiten um die Job-Beschreibung zu übergeben. Die Jobs werden von Mallob bearbeitet und das Ergebnis wird an den Nutzer zurückgegeben.
    %genaue Spezifizierung der Jobs (Priorität,Laufzeit, ...) auch hier oder an anderer Stelle?
    
    \begin{itemize}
        \item[FA31] \textbf{Einreichen der Job-Beschreibung separat von der JSON-Datei} \\
        Es ist möglich die Job-Beschreibung in einer eigenen Datei zu spezifizieren. Dabei muss es sich um eine Datei handeln, die dem DIMACS CNF Standard entspricht. Diese Datei wird zusammen mit der JSON-Datei an die API übergeben.
        % soll das Dateiformat überhaupt hier schon spezifiziert werden?
        
        \item[FA32] \textbf{Job-Beschreibung innerhalb der JSON-Datei} \\
        Es ist möglich die Job-Beschreibung direkt in der JSON-Datei zu spezifizieren. Die Beschreibung muss auch in diesem Fall dem DIMACS CNF Format entsprechen. Bei dieser Möglichkeit wird nur die JSON-Datei an die API übergeben
        
        \item[FA33] \textbf{Übergeben der Job-Beschreibung über einen Link} \\
        Es kann ein Link an die API übergeben werden, der auf eine Datei verweist, in der die Job-Beschreibung enthalten ist. Die referenzierte Datei muss ebenfalls dem DIMACS CNF Format entsprechen. Zusätzlich zu dem Link muss die JSON-Datei mit den weiteren Job-Spezifikationen an die API übergeben werden
        
        \item[FA34] \textbf{Bereits eingereichte Job-Beschreibung verwenden}\\
        Ein Nutzer kann Job-Beschreibungen, welche er Bereits eingereicht hat über eine ID referenzieren und wiederverwenden.
        
    \end{itemize}
    
    \item[FA40] \textbf{Abbrechen von eingereichten Jobs} \\
    Der Nutzer kann einen eingereichten Job wieder abbrechen. In diesem Fall wird eine Statistik über die bereits verrichtete Arbeit zurückgegeben.
    
    \item[FA50] \textbf{Zurückgeben von Ergebnissen} \\
    Für jeden eingereichten Job gibt die API immer eine Antwort an den Nutzer zurück. 
    
    \begin{itemize}
        \item[F51] \textbf{Zurückgeben des Ergebnisses bei erfolgreicher Berechnung} \\
        Wurde der eingereichte Job erfolgreich gelöst, wird das Ergebnis des Jobs an den Nutzer zurückgegeben.
        
        \item[F52] \textbf{Zurückgeben des Ergebnisses nach Erreichen der maximalen Bearbeitungszeit} \\
        Wenn die maximale Bearbeitungszeit des eingereichten Jobs erreicht wurde und kein Ergebnis gefunden wurde, wird eine Statistik über die bereits verrichtete Arbeit an den Nutzer zurückgegeben
        
        \item[F53] \textbf{Zurückgeben des Ergebnisses nach einem Fehler} \\
        Wenn während der Bearbeitung des eingereichten Jobs ein Fehler auftritt, wird eine aussagekräftige Fehlermeldung an den Nutzer zurückgegeben
        
        \item[FA54] \textbf{Ergebnisabfrage von Jobs} \\
        Es ist möglich für jeden Job (auch nach Beendigung) seinen aktuellen Status (in Bearbeitung, Bearbeitet, Fehler) abzufragen. Die Antwort  enthält dabei alle Informationen zum Status des Jobs, wie das Ergebnis oder eventuelle Fehlermeldung.
    \end{itemize}
    

    \item[FA60] \textbf{Zurückgeben des Systemzustands von Mallob} \\
    
    

\end{itemize}


%------------------------------------------------------------WEB-Interface
\subsection{Web-Interface}



% Web-Interface nochmal anders formuliert, eher an Beispiel von Betreuern orrientiert
\begin{itemize}
     \item[FA010] \textbf{Web-Interface aufrufen} \\
        Nach dem Aufrufen des Web-Interface über die URL gelangt man zur Anmelde-Maske.

     \item[FA020] \textbf{Anmelden} \\
        Der Nutzer kann sich über das Web-Interface anmelden. Dies geschieht über die Anmelde-Maske. Die Anmeldung geschieht mit Nutzernamen und Passwort. Nach erfolgreicher Registierung wird der Nutzer zur Auftrag-Seite gebracht.
        
     \item[FA030] \textbf{Registrieren} \\
        Der Nutzer kann sich über das Web-Interface registrieren. Dies geschieht über die Registrieren-Maske. Diese kann von der Anmelde-Maske mit der entsprechenden Schaltfläche erreicht werden. 
        
    \begin{itemize}
        \item[FA031] \textbf{Daten zur Registrierung} \\
        Für die Registrierung wird ein Nutzername, ein Passwort und die wiederholte Eingabe des Passworts benötigt. Nach erfolgreicher Registierung wird der Nutzer zur Auftrag-Seite gebracht.
    \end{itemize}
        

        
    \item[FA040] \textbf{Ergebnisse einsehen} \\
        Befindet sich in der Liste der Aufträge ein abgeschlossener Auftrag, so kann über die zu diesem Auftrag gehörige Schaltfläche "get results" das Ergebnis angezeigt werden. Hier kann es auch in die Zwischenablage kopiert werden oder heruntergeladen werden.
        
   \item[FA050] \textbf{Auftrag hinzufügen} \\ 
        Mittels einer Schaltfläche über der Liste der eigenen Aufträge gelangt der Nutzer zu einer Eingabe-Maske, über welche er einen neuen Auftrag hinzufügen kann. 
   \item[FA00] \textbf{Auftrag abbrechen} \\
   
   \item[FA070] \textbf{Anzeigen von Fehlern} \\
        Tritt bei Mallob ein Fehler auf, so wird der Nutzer umgehend mittels einer Fehlermeldung darauf aufmerksam gemacht. Diese Fehlermeldung wird immer angezeigt, unabhängig davon auf welcher Seite der Nutzer sich momentan befindet. 
    \item[FA080] \textbf{Visualisierung} \\
        Das Web-Interface besitzt eine Visualisierung des System-Zustandes. Diese kann über den entsprechenden Reiter erreicht werden.
        
        \begin{itemize}
            \item[FA081] \textbf{Anzeigen des aktuellen Zustandes} \\
                Standardmäßig wird immer der aktuelle Zustand angezeigt. Dieser Zustand wird, solange kein anderer Zeitpunkt ausgewählt wurde, steht dynamisch aktuell gehalten. 
            \item[FA082] \textbf{Zeitachse} \\
                Die Visualisierung verfügt über eine Zeitachse, mit derer ein entsprechender Zeitpunkt der letzten [...] Minuten/Stunden ausgewählt werden kann. Nach der Auswahl wird der System-Zustand zum entsprechenden Zeitpunkt angezeigt.
            \item[FA083] \textbf{Zurückspringen zu aktueller Ansicht} \\
                Wird mittels FA71 [TODO: REF] der angezeigte Zeitpunkt geändert, so wird eine Schaltfläche angezeigt, mit der der Nutzer jederzeit wieder zur aktuellen Zeit zurückspringen kann.
            \item[FA084] \textbf{Einsehen der Verteilung eines Auftrages} \\
                 
        \end{itemize}
    \item[FA090] \textbf{Ändern von Benutzer-Daten} \\
    
    \item[FA100] \textbf{Verwalten von Nutzern}
    \item[FA110] \textbf{Verwalten von Nutzern}
        \begin{itemize}
            \item Nutzer löschen
            \item Nutzer verifizieren [TODO: besseres Wort für freigeben, ich meine damit das das der Admin den Nutzer eben bestätigen muss, bevor er Aufträge hinzufügen kann...j]
            \item 
        \end{itemize} 
    \item[FA120] \textbf{Abmelden} \\
        Der Nutzer kann sich jederzeit über das entsprechende Menü in der Navigationsleiste abmelden. In diesem Falle wird wieder die Anmelde-Maske angezeigt.
\end{itemize}




\subsection{Plugins}
\section{Nichtfunktionale Anforderungen}
\newpage
\section{Produktdaten}
\begin{itemize}
    \item Job-Beschreibungen werden gespeichert
    \item Daten zur Authentifizierung der Nutzer werden gespeichert 
    \item Vergangene Log-Dateien
\end{itemize}
\newpage
\section{Systemmodell}
\newpage
\section{Produktleistungen}

%%
%ProduktleistungenSofern 
%an einzelne Funktionen des Programms besondere Anforderungen in Bezug auf die Zeit oder die Genauigkeit gestellt werden, sollten diese in diesem Kapitel dargestellt werden. Dabei sollten Sie prüfen, ob die zu erbringenden Leistungen mit den in Punkt 5 genannten Angaben  realisierbar sind.
%

\begin{itemize}[noitemsep]
    \item[P100] Die maximale Anzahl der \hyperref[B:Jobs]{Jobs} ist begrenzt.
    
    \item[P110] Die maximale Anzahl \hyperref[B:Jobs]{Jobs}, die ein \gls{Nutzer} gleichzeitig in Bearbeitung haben kann, ist beschränkt.
    
    \item[P120] Die Zeit, die benötigt wird, um einen beliebigen Zeitpunkt in der \hyperref[pages:visualization]{Visualisierung} darzustellen, muss linear in der Anzahl der zu ladenden \hyperref[B:Event]{Events} sein.
    
    \item[P130] (Wunschbedingung)  Die Zeit, die benötigt wird, um einen beliebigen Zeitpunkt in der \hyperref[pages:visualization]{Visualisierung} darzustellen, muss konstant sein, unabhängig vom gewählten Zeitpunkt.
    
    \item[P140] Das \gls{Web-Interface} ist auch auf kleineren Bildschirmen, wie etwa einem Handy-Bildschirm, nutzbar.
    
    \item[P150] Die \gls{Konfigurationsdatei} wird immer nur beim Systemstart eingelesen, etwaige Änderungen werden also erst mit einem Neustart des Systems wirksam.
    
    \item[P160] Beim Einreichen eines \hyperref[B:Jobs]{Jobs} im Interface erfolgt schon im Frontend eine Kontrolle der Syntax, welche den \gls{Nutzer} momentan über Fehler in der Eingabe informiert.
    
    \item[P170] Nutzernamen sind eindeutig und bestehen aus 4 bis 25 Zeichen.

    \item[P180] Passwörter müssen mindestens 8-stellig sein.
    
    \item[P190] Die gespeicherten \hyperref[B:Jobs]{Jobs} werden automatisch nach einem spezifizierten Zeitraum gelöscht.

    \item[P200] Die Größe der \hyperref[B:Job-Beschreibung]{Job-Beschreibung}, die man im \gls{Web-Interface} eingeben kann, ist beschränkt.

    
    \item[P210] Jedes \gls{Nutzerkonto} besitzt nach Registrierung die gleiche Priorität. Diese kann vom kann vom \gls{System-Administrator} geändert werden.
    
    
    \item[P220] Der \gls{Nutzer} erhält auf jede \gls{API}-Anfrage außer \hyperref[FA:API:Andauernde Abfrage des Ergebnisses eines Jobs]{F1110} unmittelbar eine  Antwort.
    
    \item[P230] Die \hyperref[pages:visualization]{Visualiseriung} ist skalierbar, sodass auch mehrere Tausend \glslink{Prozess}{Prozesse} angezeigt werden können. Um dies zu ermöglichen, wird die Qualität bei vielen \glslink{Prozess}{Prozessen} entsprechend reduziert.%, beispielsweise durch das Weglassen von Verbindungen zwischen den Prozessen.
    
    \item[P240] Daten über \hyperref[B:Jobs]{Jobs}, die nicht dem angemeldetem \gls{Nutzer} gehören, werden stets pseudonymisiert ausgegeben und dargestellt. Ist ein \gls{Administrator} angemeldet, so werden die Daten nicht pseudonymisiert.
    
    %\item[P250] Es ist möglich, die \hyperref[pages:job-page]{Job-Seite} direkt über eine passende \gls{URL} aufzurufen.
    
    \item[P260] Die maximale Geschwindigkeit der \hyperref[pages:visualization]{Visualisierung} ist das zweihundertfache.
    
    \item[P270] Die über \gls{Nutzer} gespeicherte Daten können von einem \glspl{System-Administrator} geändert werden, um beispielsweise die Priorität des \gls{Nutzer}s zu ändern oder ein \glslink{Nutzerkonto}{Konto} zu verifizieren, aber auch andere \gls{Nutzer} zum \gls{Administrator} zu machen.
    
    \item[P280] Das Backend wird in Java implementiert.

    \item[P290] Globale Einstellungen des Systems werden in einer \gls{Konfigurationsdatei} gespeichert und mit jedem Neustart aktualisiert.

\end{itemize}
\newpage
\input{9_Benutzeroberfläche}
\newpage
\section{Testszenarien}

Testfälle:
•	Aufrufen des Web-Interface
•	Registrieren eines neuen Benutzers
•	Verifizierung einer neuen Benutzerregistrierung durch einen Admin
•	Anmelden eines registrierten Benutzers
•	Ändern der Benutzerdaten eines registrierten Benutzers
•	Anmelden eines Admins
•	Ändern der Benutzerdaten eines Admins
•	Abmelden eines Benutzers
•	Abmelden eines Admins
•	Anmelden eines Benutzers mit fehlerhaften Zugangsdaten
•	Anmelden eines Admins mit fehlerhaften Zugangsdaten
•	Priorität eines Benutzers durch einen Admin ändern
•	Einreichen eines Jobs durch eine JSON-Datei und eine separate Job-Beschreibungs-Datei
•	Einreichen eines Jobs durch eine JSON-Datei mit enthaltener Job-Beschreibung
•	Einreichen eines Jobs durch eine JSON-Datei mit einem Link, der auf eine Job-Beschreibungs-Datei verweist
•	Einreichen eines Fehlerhaften Jobs
•	Abbrechen einer Jobbearbeitung
•	Erhalten einer Antwort auf einen vollendeten Job/abgebrochenen Job
•	Erstellen eines neuen korrekten Jobs
•	Erstellen eines fehlerhaften Jobs
•	Aufrufen des derzeitigen Systemzustands von mallob
•	Aufrufen eines hierarchischen Baumes aller an einem Job arbeitenden Prozesseinheiten 
•	Aufrufen eines vergangenen Systemzustands von mallob über die Zeitachse
•	Aufrufen der laufenden Jobs eines Benutzers
•	Aufrufen aller der Zeit auf dem System laufenden Jobs durch einen Admin
•	Aufrufen der Statistik eines vollendeten/abgebrochenen Jobs
•	Aufrufen bereits abgeschlossener Jobs 

Erweiterte Testfälle
•	Aufrufen einer Visualisierung der Lösung eines Jobs
•	Mallob beenden
•	Mallob neustarten
•	



%---------------------------Kommunikation mit Mallob------    
\section{Kommunikation mit Mallob}

\begin{itemize}
    \item Annahme für uns : Mallob läuft auf selben Dateisystem wie Backend 
    \item JSON und Jobbeschreibung werden in einem Verzeichnis im Dateisystem abgelegt und von dort aus von Mallob verarbeitet
    \item Der Output wird von Mallob ebenfalls las Datei in einem Verzeichnis ausgegeben
    \item JSON und Jobbschreibung können unterschiedliche Dateien sein (siehe Einreichen der Jobbeschreibung, API)
\end{itemize}



%--------------------------------Abgrenzungskritierien-----
\section{Abgrenzungskriterien}
    \begin{itemize}
        \item Nur eine Jobbeschreibung pro Anfrage
        \item Korrektheit der Daten wird durch Mallob geprüft
        \item Es wird keine Korrektur von falschen Daten durch die API geben
    \end{itemize}









%------------------------Glossar

\printnoidxglossaries
\end{document}
