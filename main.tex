
% Hier gibts ne Vorlage falls wir die Nutzen wollen https://git.scc.kit.edu/IPDSnelting/pflichtenheft/-/blob/master/pflichtenheft.tex
%Bindet die setup-datei ein. Beim Kompilieren wird alles was in setup.tex steht an diese stelle kopiert
%Das hier ist eine setup-Datei. hier schreiben wir Packages und setup-befehle für das Pflichtenheft rein. 
%Hält das ganze etwas übersichtlicher


%\documentclass[parskip=full,11pt,twoside]{scrartcl} - DIeses two side ist mega der bs
\documentclass[parskip=full,11pt]{scrartcl}
\usepackage[utf8]{inputenc}
\usepackage{hyperref}
\usepackage[nonumberlist]{glossaries}     % provides glossary commands, taken from SWT

\usepackage{svg}
\usepackage{enumitem}
\usepackage{tabularx}
\usepackage{changepage}
\usepackage{float}
% section numbers in margins:
\renewcommand\sectionlinesformat[4]{\makebox[0pt][r]{#3}#4}

% header & footer
\usepackage{scrlayer-scrpage}
\lofoot{\today}
\refoot{\today}
\pagestyle{scrheadings}


\usepackage[sfdefault,light]{roboto}
\usepackage[T1]{fontenc}
\usepackage[german]{babel}
\usepackage[yyyymmdd]{datetime} % must be after babel
\renewcommand{\dateseparator}{-} % ISO8601 date format
%usepackage[]{hyperref}
\usepackage{amsmath} % for $\text{}$
\usepackage[nameinlink]{cleveref}
\crefname{figure}{Abb}{Abb}
\usepackage[section]{placeins}
\usepackage{xcolor}
\usepackage{graphicx}
\hypersetup{
	pdftitle={Pflichtenheft},
	bookmarks=true,
}
\usepackage{csquotes}

\newcommand\urlpart[2]{$\underbrace{\text{\texttt{#1}}}_{\text{#2}}$}
\newenvironment{FA}
    {
    \begin{adjustwidth}{-3.5cm}{}
    \begin{tabular}[t]{rL{0.8\textwidth}}
    }
    {
    \end{tabular}
    \end{adjustwidth}
    \vspace{1em}
    }
    
\newenvironment{FAList}{\begin{itemize}[noitemsep, leftmargin=1cm,align=parleft]}{\end{itemize}}


\title{Pflichtenheft PSE}
\author{Valentin Schenk, Kaloyan Krasimirov Enev, Simon Wilhelm Schübel,\\ Maik Sept, Simon Aaron Giek}

\date{05.05.2022 - 29.05.2022}
% https://en.wikibooks.org/wiki/LaTeX/Glossary

\makenoidxglossaries

\newglossaryentry{Job}{
    name=Job,
    plural=Jobs,
    description={Jobs sind die Instanzen, die von Mallob verarbeitet werden. Ein einzelner Job stellt ein einzelnes zu lösendes Problem dar. Ein Job besteht aus der \gls{Job-Konfiguration} und der \gls{Job-Beschreibung}}
}

\newglossaryentry{Job-Konfiguration}{
    name=Job-Konfiguration,
    plural=Job-Konfigurationen,
    description={
    Die Job-Konfiguration beinhaltet alle Parameter des Jobs, welche bei der Bearbeitung berücksichtigt werden. Sie sind in den \hyperref[B:Job]{Begrifflichkeiten} näher erläutert
    }
}

\newglossaryentry{Job-Beschreibung}{
    name=Job-Beschreibung,
    plural=Job-Beschreibungen,
    description={Die Job-Beschreibung ist der Teil des Jobs, der das eigentliche Problem darstellt.}
}

\newglossaryentry{Job-Informationen}{
    name=Job-Informationen,
    description={Die Job-Informationen enthalten die \gls{Job-Konfiguration} und den Einreiche-Zeitpunkt, den Zustand des Jobs, die Informationen die Mallob über den Job bereitstellt und die Job-ID. Sie enthält nicht die Job-Beschreibung und nicht das Ergebnis des Jobs.}
}


\newglossaryentry{Nutzer}{
    name=Nutzer,
    plural=Nutzer,
    description={Ein Nutzer ist eine Person, welche sich registriert hat und die mit dem System interagiert.}
}


\newglossaryentry{Nutzerkonto}{
    name=Nutzerkonto,
    plural=Nutzerkonten,
    description={Ein Nutzerkonto ist ein im System durch einen Nutzer registriertes und durch einen Administrator verifiziertes Konto.}
}

\newglossaryentry{Administrator}{
    name=Administrator,
    plural=Administratoren,
    description={Administratoren sind Nutzer mit mehr Rechten zur Verwaltung des Systems. Sie haben dennoch alle Möglichkeiten, die auch ein Nutzer hat, welcher kein Administrator ist.}
}

\newglossaryentry{Web-Interface}{
    name=Web-Interface,
    description={Mit Web-Interface wird die Webseite referenziert, die der Nutzer im Internet aufrufen kann}
}
\newglossaryentry{API}{
    name=API,
    plural=APIs,
    description={Application Programming Interface. Wird im Kontext dieses Systems genutzt, um die Dienste in einer Art und Weise bereitszustellen, dass sie in andere Anwendungen integriert werden kann}
}

\newglossaryentry{Konfigurationsdatei}{
    name=Konfigurationsdatei,
    description={Eine Datei in einem spezifischen Format, die vom System eingelesen wird. Darin werden bestimmte Werte definiert, die dann vom System verwendet werden, wie etwa die maximale Anzahl der parallelen Prozesse pro Job.}
}
\newglossaryentry{System-Administrator}{
    name=System-Administrator,
    description={Eine Person, die die Ausführungsumgebung des Systems verwaltet. Verantwortlich für das Starten und Beenden des Systems.}
}

\newglossaryentry{Token}{
    name=Token,
    description={Ein Token (sog. Bearer-Token), kann benutzt werden, um sich gegenüber einer API zu authentifizieren. Der Token verweist auf nur genau einen Nutzer. Dieser Token kann von jedem benutzt werden, der ihn besitzt (deswegen Bearer-Token). Der Token wird für jeden Nutzer bei der Registrierung generiert, sodass niemals zwei Nutzer denselben Token haben. Für den Nutzer ist es wichtig den Token, wie seine Anmeldedaten, geheim zu halten, bzw. nur authorisierten Personen mitzuteilen.}
}

%\newglossaryentry{Anfrage}{
%    name=Anfrage
%    plural=Anfragen
%    description={Ein Nutzer kann eine Anfrage an eine API oder eine Website stellen.}
%}

\newglossaryentry{Datenbank}{
    name=Datenbank,
    plural=Datenbanken,
    description={Eine Datenbank ist ein System, welches zur Datenspeicherung und Verwaltung genutzt wird. Die Hauptaufgabe einer Datenbank besteht darin, vordefinierte Daten zu speichern und schnellen Zugriff auf die Daten zu erlangen.}
}

\newglossaryentry{Output-Log}{
    name=Datenbank,
    plural=Datenbanken,
    description={Eine Datenbank ist ein System, welches zur Datenspeicherung und Verwaltung genutzt wird. Die Hauptaufgabe einer Datenbank besteht darin, vordefinierte Daten zu speichern und schnellen Zugriff auf die Daten zu erlangen.}
}

\newglossaryentry{Stream}{
    name=Stream,
    description={Ein Stream von Daten ist eine andauernde eingehende oder ausgehende Menge von Daten.}
}

\newglossaryentry{Log-Datei}{
    name=Log-Datei,
    description={Eine Log-Datei ist eine Datei, welche Log-Daten speichert. Log-Daten sind Meta-Daten, welche gewisse Ereignisse festhalten sollen. Im Kontext von Mallob beispielsweise das Eingehen eines Jobs.}
}

\newglossaryentry{Vorlaeufiges Konto}{
    name={Vorläufiges Konto},
    description={Ein vorläufiges Konto sind Konten, die neu erstellt wurden und noch nicht durch einen \gls{Administrator} verifiziert wurden. Ein solches Konto eingeschränkt, es könnten noch keine Jobs in Auftrag gegeben werden.}
}

    
\newglossaryentry{Bearbeitungszeit}{
name=Bearbeitungszeit,
description= {tt}%[todo]
}

\newglossaryentry{Binaerbaum}{
    name=Binärbaum,
    plural=Binärbäume,
    description={...}
}


\begin{document}

\newcolumntype{L}[1]{>{\raggedright\arraybackslash}p{#1}}
\newcolumntype{C}[1]{>{\centering\arraybackslash}p{#1}}
\newcolumntype{R}[1]{>{\raggedleft\arraybackslash}p{#1}}


\maketitle
\newpage
\tableofcontents 
\newpage
\section{Einleitung}


%Neben der
%-expliziten Benennung des Auftragnehmers und des Auftraggebers 
%sollte an dieser Stellung auch eine 
%-grobe Kurzbeschreibung des Projektes erfolgen. 
%Gehen Sie darauf ein, 
%--was das Projekt beinhaltet und 
%--wie das Endergebnis aussehen soll. 
%Wichtig ist, dass auch eine Person, die das erste Mal von dem Projekt hört, versteht, worum es geht.

%---Vorstellung des Auftraggebers und von uns 
\href{https://github.com/domschrei/mallob}{Mallob} ist ein dezentrales System zum Scheduling und Lösen von \glslink{NP-schweres Problem}{NP-schweren Problemen}, hauptsächlich entwickelt von Dominik Schreiber im Rahmen seiner Doktorarbeit. Der Auftraggeber - der Entwickler von Mallob - wünscht sich ein bedienerfreundliches Softwaresystem, um mit Mallob von außen kommunizieren zu können. \\
%---Grobe kruzbeschreibung des Projekts
Unser System \textbf{\textit{Fallob - a Friendly Face for Mallob}} soll diese Brücke zwischen Mallob und Außenwelt darstellen.
Es verfügt daher über all diejenigen Funktionen, die es möglich machen, mit Mallob zu interagieren. Es ist also möglich direkt \hyperref[B:Jobs]{Jobs} an Mallob zu senden und \hyperref[B:Job-Informationen]{Informationen} über diese zu erlangen.\\
Eine Hauptaufgabe von \textit{Fallob} ist die \hyperref[pages:visualization]{Visualisierung} des \hyperref[B:Systemzustand]{Systemzustandes} von Mallob. Hier ist es möglich zu sehen, wie Mallob die eigenen \hyperref[B:Jobs]{Jobs} (so werden Probleme genannt, die Mallob lösen soll) bearbeitet. Auch die Gesamtauslastung sowie die Zuordnung von \hyperref[B:Jobs]{Jobs} auf Prozessoren ist einsehbar. 
Die Interaktion mit Mallob soll sowohl über ein \gls{Web-Interface}, als auch direkt über eine von uns bereitgestellte \gls{API} möglich sein. %\\

%Des weiteren wird das System eine Nutzer-Verwaltung beinhalten, um %[warum haben wir eine Nutzerverwaltung?]. Identifikation - wer welche Job eingereicht hat. Erlaubt Rollenverteilung - auf Admin und Benutzer. Erlaubt Admins leichter in Kontakt mit dem Benutzer zu treten. Sicherheitsgründen - nicht jeder kann Jobs einreichen, Kontrolle darauf, wer Mallob benutzt. Historie speichern - erlaubt dem Nutzer Informationen über bisher eingereichten Jobs zu bekommen (nicht nur vom selben Rechner). "Schutz vor Fremdzugriffen und somit die Vertraulichkeit der Inhalte" - laut https://glossar.hs-augsburg.de/Benutzerverwaltung




\newpage
\section{Begrifflichkeiten}
%Erklärung von Begrifflichkeiten, die nicht von uns stammen (von mallob)
\label{B:Jobs}
\subsection{Jobs}

Jobs sind die Instanzen, die von Mallob verarbeitet werden. Ein einzelner Job stellt ein einzelnes zu lösendes Problem dar. Ein Job besteht aus Sicht des Nutzers aus der Job-Konfiguration und der Job—Beschreibung.
Jeder Job besitzt eigene Parameter, die bei der Bearbeitung durch Mallob beachtet werden. 
Diese Parameter bilden zusammen die Job-Konfiguration.

Die Parameter sind:

\subsection{Job-Konfiguration}


\begin{tabular}{lL{0.6\textwidth}l}
        Name & Beschreibung & Notwendig\\
        \\
        name & Der Name des Jobs & Ja\\
        priority & Die Prioriät des Jobs & Ja\\
        application & Das Anwendungsfeld des Jobs & Ja\\ 
        max-demand & Die maximaler Anzahl paralleler Prozesse, die dieser Job nutzen kann & Nein\\
        wallclock-limit & Zeitliches Ausführungsbudget des Jobs & Nein\\
        cpu-limit & Corestudnen-Ausführungsbudget des Jobs & Nein\\
        arrival & Frühester Bearbeitungsbeginn in Sekunden seit Start von Mallob & Nein\\
        dependencies & Jobs, die abgeschlossen sein müssen, bevor dieser Job beginnt & Nein\\
        incremental & Handelt es sich um einen Job mit mehreren Revisionen oder Inkrementen? &  Nein\\
        precursor & Vorgänger-Job bei inkrementellen Jobs & Nein\\
        % TODO: files hier auch?
        content-mode & Text- oder Binärformat der Beschreibung & Nein\\
        % TODO: literals und assumptions?
    \end{tabular} 

\subsection{Job-Informationen}
Job-Informationen sind alle Informationen über einen Job, die nicht von der internen Bearbeitung des Jobs durch Mallob abhängen. 


\subsection{Event}
Ein Event ist eine Veränderung im internen Systemzustand von Mallob, z.B. Ein Prozess beginnt mit der Bearbeitung eines Jobs, oder ein Prozess beginnt mit der Bearbeitung eines anderen Jobs als zuvor.


\subsection{Job-Details}
Job-Details werden aus Job-Events abgeleitet und sind somit Informationen, die nur die interne Verarbeitung von Mallob betreffen, zum Beispiel wie viele Prozesse arbeiten an einem Job? Welche Prozesse arbeiten an einem Job?


\subsection{Konfigurationsdatei}
Die Konfigurationsdatei beinhaltet globale Einstellungen für \textit{Fallob}.



\newpage
%\section{Zielbestimmung}
%\subsection{Musskriterien}
%\subsubsection{Web-Interface}
%\begin{itemize}
%    \item Das System verfügt über ein \gls{Web-Interface}.
%    \item Beim ersten Aufruf des \glslink{Web-Interface}{Web-Interfaces} wird eine Anmelde-Maske angezeigt.
%    \item In der Anmelde-Maske der Nutzer auch zu einer Maske für die Registierung gelangen.
%    \item Ein \gls{Auftrag} kann über das \gls{Web-Interface} hinzugefügt werden. Dies geschieht über eine Eingabemaske.
%    \item Das Übergeben der Probleminstanz kann wahlweise direkt über ein Eingabefeld, hochladen der CNF-Datei oder Angabe einer URL erfolgen.
%    \item Das \gls{Web-Interface} besitzt eine Visualierung des Systemzustandes. Hier wird standardmäßig der aktuelle Zustand visualisiert,  es kann aber auch die Historie betrachtet werden.
%    \item Admins haben Zugriff auf eine exklusive Ansicht, mit der die Instanz von Mallob verwaltet werden kann.
%
%\subsection{Wunschkriterien}
%\subsubsection{Web-Interface}
%\begin{itemize}
%    \item Der Benutzer kann die Eingabe des Problems mit einem graphischem Editor vornehmen.
%    \item Ein Admin kann die Instanz von Mallob starten, beenden und neustarten. Hier gibt es auch die Möglichkeit, entsprechende Parameter einzugeben.
%\end{itemize}
%\end{itemize}

\section{Zielbestimmung}
Dieses Programm ermöglicht es Nutzern, einfacher mit \href{https://github.com/domschrei/mallob}{Mallob} zu interagieren. Diese Interaktion kann auf zwei Wege erfolgen:
Einerseits wird eine REST-API angeboten für die Interaktion angeboten. Anderseits gibt es aber auch ein benutzerfreundliches Web-Interface, welches die API nutzt und somit eine graphische Oberfläche für diese darstellt.
\subsection{Musskriterien}
    \subsubsection{API}
        \begin{itemize}
            \item Unterscheiden zwischen Benutzern und Administratoren
            \item Authentifizieren von Benutzern
            \item Registrierung von neuen Benutzern
            \item Verwalten von Jobs
                \begin{itemize}
                    \item Jobs hinzufügen
                    \item Jobs abbrechen
                    \item Aktuellen Status eines Jobs abfragen
                    \item Auflisten der eigenen Jobs
                \end{itemize}
            \item Abfragen von Jobs
            \item Bereitstellung eines Ereignis-Streams von Mallob zum Erhalten von Updates
            \item Erhalten von Informationen zu aufgetretenen Fehlern
        \end{itemize}
    \subsubsection{Web-Interface}
        \begin{itemize}
            \item Authentifizieren von Nutzern
            \item Registrierung von neuen Nutzern
            \item Verwalten von Jobs
                 \begin{itemize}
                    \item Jobs hinzufügen
                    \item Jobs abbrechen
                    \item Aktuellen Status des Jobs betrachten
                    \item Auflisten der eigenen Jobs
                \end{itemize}
            \item Einsehen der eigenen Aufträge und derer Ergebnisse
            \item Herunterladen der Ergebnisse der Jobs
            \item Anzeigen von aufgetretenen Fehlern
            \item Verifizierungsmöglichkeit von neuen Nutzern für Administratoren 
            
                      
        \end{itemize}
    \subsubsection{Visualisierung}    
        \begin{itemize}
            \item Das Web-Interface enthält eine Visualisierung des System-Zustands
            \item Wenn die Live-Ansicht ausgewählt ist, wird die Visualisierung automatisch aktualisiert
            %\item Darstellung der PE als Punkte %TODO: Ranks? oder PEs? oder ganz anderes Word?
            %\item Färbung der Punkte entsprechend dem zugehörigen Job
            %\item Verbinden der Punkte entsprechend der Position im Binärbaum des Jobs
            %\item Dynamische Größe der Punkte und Verbindungen entsprechend der Position im Binärbaum des Jobs.
        \end{itemize}
    
    \subsubsection{System}
        \begin{itemize}
            \item Konfigurationsdatei für globale Einstellungen des Systems
        \end{itemize}
        
        
\subsection{Wunschkriterien}
    \begin{itemize}
        %\item Speichern von Jobdaten für Statistiken -> Irgendwie redundaten, da wir eh schon jobdaten speichern
        \item Möglichkeit für Administratoren, zu starten, beenden oder neustarten
        \item Sortieren der Job-Tabelle im Web—Interface
        \item Graphischer Editor zur Eingabe der Job-Beschreibung
        \item Diagnosemöglichkeit für Administratoren
        \item Nutzer benachrichtigen, falls Mallob abstürzen sollte
        \item Darstellen eines bestimmten vergangenen Zeitpunktes bei der Visualisierung
        \item Ändern der Wiedergabegeschwindigkeit bei der Visualisierung
        \item Einsehen von Details zu einem aktiven Job in der Visualisierung
        \item Darstellung des vollständigen Binärbaums des Jobs
        \item Schnittstelle für Plugins in Web—Interface
    \end{itemize}
    
\subsection{Abgrenzungskriterien}
    \begin{itemize}
        \item Es kann maximal ein Auftrag pro Anfrage hinzugefügt werden
        \item Korrektheit der Daten wird durch Mallob geprüft
        \item Es gibt keine Möglichkeit, falsche Daten in der API zu korrigieren, in diesem Fall muss die gleiche Anfrage mit korrekten Daten erneut erfolgen
    \end{itemize}

\newpage
\section{Produkteinsatz}
% TODO: das sollte noch mindestens zwei mal überarbeitet werden, lol

\subsection{Andwendungsbereich}

Das System dient dem Lösen von Problemen. Ein Nutzer kann das Web-Interface im Browser aufrufen oder stattdessen die bereitgestellte API nutzen.
\subsection{Zielgruppe}

Insgesamt richtet sich das System an Personen, die komplexe Probleme lösen möchten, deren Laufzeit zu groß ist. Hier kann dieses System helfen, indem es das Problem schneller löst. 
Das Web-Interface bietet eine moderne, einfach und intuitive Umgebung zur Nutzung dieses Systems. Hier sind abgesehen von der Bereitstellung der Job-Beschreibung im korrekten Format keine weiteren Kenntnisse notwendig. \\
Die API dagegen richtet sich an Personen, die dieses System in ihr eigenes integrieren möchten. Hier wird Wissen über den Umgang einer solchen API vorausgesetzt. 
\subsection{Betriebsbedingungen}

\begin{itemize}
    \item Zur Nutzung ist eine stabile Internetverbindung notwendig. Insbesondere kann eine instabile Verbindung dazu führen, das Ereignisse nicht in Echtzeit angezeigt werden.
    \item Zur Verwendung des Web-Interfaces muss ein aktuelle Version des Browsers genutzt werden.
    \item Das System muss Zugriff auf die API der Instanz von Mallob haben.
    \item Eine Instanz von Mallob muss existieren, sodass diese von diesem System genutzt werden kann. 
\end{itemize}
\newpage
\section{Produktumgebung}

\subsection{Client-Seite}
Auf der Seite des Clients wird zur Nutzung des \glslink{Web-Interface}{Web-Interfaces} ein Browser mit aktuellster Version benötigt.\\
Der Browser kann dabei auf folgenden Betriebssystemen laufen:

\begin{itemize}
    \item Unix-Basierte Betriebssysteme
    \item Windows 10 und neuer
    \item Android
    \item iOS
\end{itemize}

Die \glslink{Betriebssystem}{Betriebssysteme} können auf folgender Hardware laufen:

\begin{itemize}
    \item Desktop-Rechner / Laptop 
    \item Mobiles Endgerät; Smartphone, Tablet, ...
\end{itemize}

    
\subsection{Server-Seite}
\begin{itemize}
    \item Da das Backend Java-basiert ist, ist es notwendig, dass die Hardware, auf der \textit{Fallob} läuft, Java 17 und älter unterstützt.
    \item Es müssen mindestens 300 Gigabyte Speicher zur Verfügung stehen, damit die anfallenden \hyperref[PD]{Produktdaten} gespeichert werden können.
\end{itemize}
\newpage
%\section{Funktionale Anforderungen}
% Sollten wir noch nutzen
%https://de.overleaf.com/learn/latex/Cross_referencing_sections%2C_equations_and_floats


\subsection{API}

\begin{itemize}
    \item[FA10] \textbf{Registrierung von Nutzern} \\
    Es ist möglich neue Nutzer über die API zu registrieren, sodass diese sich mit ihren Zugangsdaten authentifizieren können.
    
    \item[FA20] \textbf{Authentifizierung von Nutzern} \\
    Nach dem Registrieren ist es dem Nutzer möglich sich über die API zu authentifizieren und Zugriff auf die anderen Funktionen der API zu erlangen. Dies kann er mit seinem Bearer-Token tun. Diesen Token erhält ein Nutzer bei Registrierung.
    
    \item[FA30] \textbf{Einreichen von Jobs} \\
    Die API ermöglicht es Jobs, die durch eine JSON-Datei und eine Job-Beschreibung spezifiziert werden, zu übergeben. Es existieren drei verschiedene Möglichkeiten um die Job-Beschreibung zu übergeben. Die Jobs werden von Mallob bearbeitet und das Ergebnis wird an den Nutzer zurückgegeben.
    %genaue Spezifizierung der Jobs (Priorität,Laufzeit, ...) auch hier oder an anderer Stelle?
    
    \begin{itemize}
        \item[FA31] \textbf{Einreichen der Job-Beschreibung separat von der JSON-Datei} \\
        Es ist möglich die Job-Beschreibung in einer eigenen Datei zu spezifizieren. Dabei muss es sich um eine Datei handeln, die dem DIMACS CNF Standard entspricht. Diese Datei wird zusammen mit der JSON-Datei an die API übergeben.
        % soll das Dateiformat überhaupt hier schon spezifiziert werden?
        
        \item[FA32] \textbf{Job-Beschreibung innerhalb der JSON-Datei} \\
        Es ist möglich die Job-Beschreibung direkt in der JSON-Datei zu spezifizieren. Die Beschreibung muss auch in diesem Fall dem DIMACS CNF Format entsprechen. Bei dieser Möglichkeit wird nur die JSON-Datei an die API übergeben
        
        \item[FA33] \textbf{Übergeben der Job-Beschreibung über einen Link} \\
        Es kann ein Link an die API übergeben werden, der auf eine Datei verweist, in der die Job-Beschreibung enthalten ist. Die referenzierte Datei muss ebenfalls dem DIMACS CNF Format entsprechen. Zusätzlich zu dem Link muss die JSON-Datei mit den weiteren Job-Spezifikationen an die API übergeben werden
        
        \item[FA34] \textbf{Bereits eingereichte Job-Beschreibung verwenden}\\
        Ein Nutzer kann Job-Beschreibungen, welche er Bereits eingereicht hat über eine ID referenzieren und wiederverwenden.
        
    \end{itemize}
    
    \item[FA40] \textbf{Abbrechen von eingereichten Jobs} \\
    Der Nutzer kann einen eingereichten Job wieder abbrechen. In diesem Fall wird eine Statistik über die bereits verrichtete Arbeit zurückgegeben.
    
    \item[FA50] \textbf{Zurückgeben von Ergebnissen} \\
    Für jeden eingereichten Job gibt die API immer eine Antwort an den Nutzer zurück. 
    
    \begin{itemize}
        \item[F51] \textbf{Zurückgeben des Ergebnisses bei erfolgreicher Berechnung} \\
        Wurde der eingereichte Job erfolgreich gelöst, wird das Ergebnis des Jobs an den Nutzer zurückgegeben.
        
        \item[F52] \textbf{Zurückgeben des Ergebnisses nach Erreichen der maximalen Bearbeitungszeit} \\
        Wenn die maximale Bearbeitungszeit des eingereichten Jobs erreicht wurde und kein Ergebnis gefunden wurde, wird eine Statistik über die bereits verrichtete Arbeit an den Nutzer zurückgegeben
        
        \item[F53] \textbf{Zurückgeben des Ergebnisses nach einem Fehler} \\
        Wenn während der Bearbeitung des eingereichten Jobs ein Fehler auftritt, wird eine aussagekräftige Fehlermeldung an den Nutzer zurückgegeben
        
        \item[FA54] \textbf{Ergebnisabfrage von Jobs} \\
        Es ist möglich für jeden Job (auch nach Beendigung) seinen aktuellen Status (in Bearbeitung, Bearbeitet, Fehler) abzufragen. Die Antwort  enthält dabei alle Informationen zum Status des Jobs, wie das Ergebnis oder eventuelle Fehlermeldung.
    \end{itemize}
    

    \item[FA60] \textbf{Zurückgeben des Systemzustands von Mallob} \\
    
    

\end{itemize}


%------------------------------------------------------------WEB-Interface
\subsection{Web-Interface}



% Web-Interface nochmal anders formuliert, eher an Beispiel von Betreuern orrientiert
\begin{itemize}
     \item[FA010] \textbf{Web-Interface aufrufen} \\
        Nach dem Aufrufen des Web-Interface über die URL gelangt man zur Anmelde-Maske.

     \item[FA020] \textbf{Anmelden} \\
        Der Nutzer kann sich über das Web-Interface anmelden. Dies geschieht über die Anmelde-Maske. Die Anmeldung geschieht mit Nutzernamen und Passwort. Nach erfolgreicher Registierung wird der Nutzer zur Auftrag-Seite gebracht.
        
     \item[FA030] \textbf{Registrieren} \\
        Der Nutzer kann sich über das Web-Interface registrieren. Dies geschieht über die Registrieren-Maske. Diese kann von der Anmelde-Maske mit der entsprechenden Schaltfläche erreicht werden. 
        
    \begin{itemize}
        \item[FA031] \textbf{Daten zur Registrierung} \\
        Für die Registrierung wird ein Nutzername, ein Passwort und die wiederholte Eingabe des Passworts benötigt. Nach erfolgreicher Registierung wird der Nutzer zur Auftrag-Seite gebracht.
    \end{itemize}
        

        
    \item[FA040] \textbf{Ergebnisse einsehen} \\
        Befindet sich in der Liste der Aufträge ein abgeschlossener Auftrag, so kann über die zu diesem Auftrag gehörige Schaltfläche "get results" das Ergebnis angezeigt werden. Hier kann es auch in die Zwischenablage kopiert werden oder heruntergeladen werden.
        
   \item[FA050] \textbf{Auftrag hinzufügen} \\ 
        Mittels einer Schaltfläche über der Liste der eigenen Aufträge gelangt der Nutzer zu einer Eingabe-Maske, über welche er einen neuen Auftrag hinzufügen kann. 
   \item[FA00] \textbf{Auftrag abbrechen} \\
   
   \item[FA070] \textbf{Anzeigen von Fehlern} \\
        Tritt bei Mallob ein Fehler auf, so wird der Nutzer umgehend mittels einer Fehlermeldung darauf aufmerksam gemacht. Diese Fehlermeldung wird immer angezeigt, unabhängig davon auf welcher Seite der Nutzer sich momentan befindet. 
    \item[FA080] \textbf{Visualisierung} \\
        Das Web-Interface besitzt eine Visualisierung des System-Zustandes. Diese kann über den entsprechenden Reiter erreicht werden.
        
        \begin{itemize}
            \item[FA081] \textbf{Anzeigen des aktuellen Zustandes} \\
                Standardmäßig wird immer der aktuelle Zustand angezeigt. Dieser Zustand wird, solange kein anderer Zeitpunkt ausgewählt wurde, steht dynamisch aktuell gehalten. 
            \item[FA082] \textbf{Zeitachse} \\
                Die Visualisierung verfügt über eine Zeitachse, mit derer ein entsprechender Zeitpunkt der letzten [...] Minuten/Stunden ausgewählt werden kann. Nach der Auswahl wird der System-Zustand zum entsprechenden Zeitpunkt angezeigt.
            \item[FA083] \textbf{Zurückspringen zu aktueller Ansicht} \\
                Wird mittels FA71 [TODO: REF] der angezeigte Zeitpunkt geändert, so wird eine Schaltfläche angezeigt, mit der der Nutzer jederzeit wieder zur aktuellen Zeit zurückspringen kann.
            \item[FA084] \textbf{Einsehen der Verteilung eines Auftrages} \\
                 
        \end{itemize}
    \item[FA090] \textbf{Ändern von Benutzer-Daten} \\
    
    \item[FA100] \textbf{Verwalten von Nutzern}
    \item[FA110] \textbf{Verwalten von Nutzern}
        \begin{itemize}
            \item Nutzer löschen
            \item Nutzer verifizieren [TODO: besseres Wort für freigeben, ich meine damit das das der Admin den Nutzer eben bestätigen muss, bevor er Aufträge hinzufügen kann...j]
            \item 
        \end{itemize} 
    \item[FA120] \textbf{Abmelden} \\
        Der Nutzer kann sich jederzeit über das entsprechende Menü in der Navigationsleiste abmelden. In diesem Falle wird wieder die Anmelde-Maske angezeigt.
\end{itemize}




\subsection{Plugins}
\section{Nichtfunktionale Anforderungen}

\section{Funktionale Anforderungen}


\subsection{API}


\begin{itemize}[nosep]
    \setlength\itemsep{4em}
    
    
    
  
    %-----------------API - Authentifizierung von Nutzern
    \phantomsection
    \label{FA:API:Authentifizieren von Nutzern}
    \item[F1000] \textbf{Authentifizieren von Nutzern}\\
    
    \begin{FA}
        \textbf{Ziel:} & Ein \gls{Nutzer} kann sich mit seinem Nutzernamen und Passwort authentifizieren \\
        \textbf{Vorbedingung:} & Der \gls{Nutzer} ist registriert \\
        \textbf{Nachbedingung (Erfolg):} & Der \gls{Nutzer} hat einen \gls{Authentifizierungstoken}, der ihm Zugriff auf die Funktionen der \gls{API} ermöglicht \\
        \textbf{Nachbedingung (Fehlschlag):} & Der \gls{Nutzer} hat keinen \gls{Authentifizierungstoken} und hat eine Fehlermeldung erhalten \\
         \textbf{Akteure:} & \gls{Nutzer} \\
        \textbf{Auslösendes Ereignis:} & \gls{Nutzer} möchte Funktionen der \gls{API} verwenden \\
    \end{FA}
    \textbf{Beschreibung:}
    \begin{FAList}
        \item[1.] Der \gls{Nutzer} schickt eine Anfrage zur Authentifizierung an die \gls{API}, die seinen Nutzernamen und sein Passwort enthält.
        \item[2.] Die Anfrage wird von der \gls{API} verarbeitet und der \gls{Nutzer} wird authentifiziert.
    \end{FAList}
    
    

    
    %-----------------API - Einreichen von Jobs
    \phantomsection
    \label{FA:API:Einreichen von Jobs} 
    \item[F1010] \textbf{Einreichen von Jobs} \\
    \begin{FA}
        \textbf{Ziel:} & Ein \gls{Nutzer} kann einen Job an die \gls{API} übergeben, der von Mallob bearbeitet wird \\
        \textbf{Vorbedingung:} & Der \gls{Nutzer} hat sich mittels \hyperref[FA:API:Authentifizieren von Nutzern]{F1000} authentifiziert \\
        \textbf{Nachbedingung (Erfolg):} & Der \gls{Nutzer} erhält eine Bestätigung \\
        \textbf{Nachbedingung (Fehlschlag):} & Der \gls{Nutzer} hat eine Fehlermeldung \\
        \textbf{Akteure:} & \gls{Nutzer} \\
        \textbf{Auslösendes Ereignis:} & Der \gls{Nutzer} möchte  einen Job einreichen \\
    \end{FA}
    \textbf{Beschreibung:}
    \begin{FAList}
            \item[1.a.] Der \gls{Nutzer} schickt eine Anfrage an die \gls{API}, die sowohl die Job-Konfiguration als auch die Job-Beschreibung enthält, wobei die Job-Beschreibung in einer separaten Datei spezifiziert ist.
            \item[1.b.] Der \gls{Nutzer} schickt eine Anfrage an die \gls{API}, die ebenfalls die Job-Konfiguration des Jobs als auch die Job-Beschreibung enthält, wobei in diesem Fall beides in  einer einzelnen Datei spezifiziert wird.
            \item[1.c.] (gewünscht) Der \gls{Nutzer} schickt eine Anfrage an die \gls{API}, die ebenfalls die Job-Konfiguration und einen Link  zu einer Datei enthält, in der die Job-Beschreibung spezifiziert ist.
            \item[1.d.] (gewünscht) Der \gls{Nutzer} schickt eine Anfrage an die \gls{API}, die eine Referenz auf einen bereits eingereichten Job enthält, der nochmal ausgeführt werden soll.
            \item[2.] Der eingereichte Job wird von Mallob bearbeitet.
    \end{FAList}
    

    
    %-----------------API - Abbrechen von Jobs
   \phantomsection
    \label{FA:API:Abbrechen von eingereichten Jobs}  
    \item[F1020] \textbf{Abbrechen von eingereichten Jobs} \\
    \begin{FA}
        \textbf{Ziel:} & Ein \gls{Nutzer} kann einen oder mehrere eingereichte Jobs wieder abbrechen \\
        \textbf{Vorbedingung:} & Der \gls{Nutzer} hat sich mittels \hyperref[FA:API:Authentifizieren von Nutzern]{F1000} authentifiziert und hat mindestens einen laufenden Job \\
        \textbf{Nachbedingung (Erfolg):} & Die Jobs wurde abgebrochen und der \gls{Nutzer} hat das Teil-Ergebnis des eingereichten Jobs \\
        \textbf{Nachbedingung (Fehlschlag):} & Die Jobs sind nicht abgebrochen und der \gls{Nutzer} hat eine Fehlermeldung \\
        \textbf{Akteure:} & \gls{Nutzer} \\
        \textbf{Auslösendes Ereignis:} & Der \gls{Nutzer} möchte den Job abbrechen \\
    \end{FA}
    \textbf{Beschreibung:}
    \begin{FAList}
            \item[1.] Der \gls{Nutzer} schickt eine Anfrage an die \gls{API}, in welcher spezifiziert ist, welche Jobs abgebrochen werden sollen. Es ist auch möglich alle eigenen, noch laufenden Jobs abzubrechen.
            \item[2.] Die Jobs werden abgebrochen.
    \end{FAList}
    

  

    
    
    %--------------------------API - Abfrage für inkrementelle Jobs
    %\phantomsection
    %\label{FA:API:Abfragen der Fertigstellung eines inkrementellen Jobs}  
    %\item[F1070] \textbf{Abfragen der Fertigstellung eines inkrementellen Jobs} \\
    %\begin{FA}
    %    \textbf{Ziel:} & Es wird abgefragt, ob ein inkrementeller Job abgeschlossen ist oder ob Teil-Jobs existieren, die noch bearbeitet werden müssen \\
    %    \textbf{Vorbedingung:} & Der \gls{Nutzer} ist authentifiziert (siehe F20) es wurde ein inkrementeller Job eingereicht \\
    %    \textbf{Nachbedingung (Erfolg):} & Der \gls{Nutzer} hat eine Antwort erhalten, ob der Job abgeschlossen ist oder nicht \\
    %    \textbf{Nachbedingung (Fehlschlag):} & Der \gls{Nutzer} hat eine Fehlermeldung erhalten \\
    %    \textbf{Akteure:} & \gls{Nutzer} \\
    %    \textbf{Auslösendes Ereignis:} & Der \gls{Nutzer} möchte herausfinden, ob ein inkrementeller Job abgeschlossen ist \\
    %\end{FA}
    %\textbf{Beschreibung:}
    %\begin{FAList} 
    %    \item[1.] Der \gls{Nutzer} schickt eine Anfrage an die API mit der Job-ID des inkrementellen Jobs, der abgefragt werden soll
    %    \item[2.] Der Bearbeitungsstand des inkrementellen Jobs wird abgefragt
    %    \item[3.a] Wenn die Abfrage erfolgreich war, wird das Ergebnis an den \gls{Nutzer} zurückgegeben
    %    \item[3.b.] Wenn die Abfrage nicht erfolgreich war, wird eine Fehlermeldung an den \gls{Nutzer} zurückgegeben
    %\end{FAList}
    
    
    %--------------------------API - Abfragen der Daten eines einzelnen Jobs
    \phantomsection
    \label{FA:API:Abfragen der Informationenen von Jobs}  
    \item[F1030] \textbf{Ausgeben der Informationen von Jobs} \\
    \begin{FA}
        \textbf{Ziel:} & Es können die Informationen von Jobs ausgegeben ohne die Job-Beschreibung werden\\
        \textbf{Vorbedingung:} & Der \gls{Nutzer} ist gemäß \hyperref[FA:API:Authentifizieren von Nutzern]{F1000} authentifiziert und die gewünschten Jobs wurden bereits eingereicht \\
        \textbf{Nachbedingung (Erfolg):} & Der \gls{Nutzer} hat die Informationen zu den angefragten Jobs erhalten \\
        \textbf{Nachbedingung (Fehlschlag):} &  Der \gls{Nutzer} hat eine Fehlermeldung erhalten \\
        \textbf{Akteure:} & \gls{Nutzer} \\
        \textbf{Auslösendes Ereignis:} & Der \gls{Nutzer} möchte Informationen über eingereichte Jobs erhalten \\
    \end{FA}
    \textbf{Beschreibung:}
    \begin{FAList} 
        \item[1.] Der \gls{Nutzer} stellt eine Anfrage an die \gls{API}. In dieser dieser ist spezifiziert, über welche Jobs Informationen ausgegeben werden sollen. Es steht dem Nutzer frei, ob er Informationen zu bestimmten Jobs, allen eigenen Jobs oder allen Jobs im System erhalten möchte.
        \item[2.] Die Informationen der Jobs werden abgefragt.
        \item[3.a.] Wenn die Abfrage erfolgreich war, werden die Informationen an den \gls{Nutzer} zurückgegeben.
        \item[3.b.] Wenn die Abfrage nicht erfolgreich war, wird eine Fehlermeldung an den \gls{Nutzer} zurückgegeben.
    \end{FAList}
    
    
    \phantomsection
    \label{FA:API:Ausgeben eines Systemzustandes}
    \item[F1040] \textbf{Ausgeben eines Systemzustandes} \\
    \begin{FA}
        \textbf{Ziel:} & Es kann ein beliebiger vergangener Systemzustand abgefragt werden \\
        \textbf{Vorbedingung:} & Der \gls{Nutzer} ist gemäß \hyperref[FA:API:Authentifizieren von Nutzern]{F1010} authentifiziert \\
        \textbf{Nachbedingung (Erfolg):} & Der \gls{Nutzer} hat den gewünschten Systemzustand erhalten \\
        \textbf{Nachbedingung (Fehlschlag):} &  Der \gls{Nutzer} hat eine Fehlermeldung erhalten \\
        \textbf{Akteure:} & \gls{Nutzer} \\
        \textbf{Auslösendes Ereignis:} & Einen vergangenen Systemzustand einsehen \\
    \end{FA}
     \textbf{Beschreibung:}
    \begin{FAList} 
        \item[1.] Der \gls{Nutzer} stellt eine Anfrage an die \gls{API} mit dem gewünschten Zeitpunkt.
        \item[2.] Der Systemzustand wird abgefragt
    \end{FAList} 
    
    
    
    
    \phantomsection
    \label{FA:API:Ausgeben von vergangenen Events}
    \item[F1050] \textbf{Ausgeben von vergangenen Events} \\
    \begin{FA}
        \textbf{Ziel:} & Es können vergangene Events ausgegeben werden\\
        \textbf{Vorbedingung:} & Der \gls{Nutzer} ist gemäß \hyperref[FA:API:Authentifizieren von Nutzern]{F1010} authentifiziert und die gewünschten Jobs wurden bereits eingereicht \\
        \textbf{Nachbedingung (Erfolg):} & Der \gls{Nutzer} hat die gewünschten Events zu den angefragten Jobs erhalten \\
        \textbf{Nachbedingung (Fehlschlag):} &  Der \gls{Nutzer} hat eine Fehlermeldung erhalten \\
        \textbf{Akteure:} & \gls{Nutzer} \\
        \textbf{Auslösendes Ereignis:} & Der \gls{Nutzer} möchte Events von eingereichte Jobs erhalten \\
    \end{FA}
     \textbf{Beschreibung:}
    \begin{FAList} 
        \item[1.] Der \gls{Nutzer} stellt eine Anfrage an die \gls{API}. In dieser ist spezifiziert, von welchen Jobs Events ausgegeben werden sollen und aus welcher Zeit diese Events stammen sollen.  Es steht dem Nutzer frei, ob er Events zu bestimmten Jobs, allen eigenen Jobs oder allen Jobs im System erhalten möchte.
        \item[2.] Die Informationen der Jobs werden abgefragt.
        \item[2.] Die entsprechenden Events werden abgefragt.
    \end{FAList} 
    
    %---------------API - Ergebnisdatei anfordern
    \phantomsection
    \label{FA:API:Ausgeben des Ergebnisses für eine oder mehrere Jobs}  
    \item[F1060] \textbf{Ausgeben des Ergebnisses für eine oder mehrere Jobs} \\
    \begin{FA}
        \textbf{Ziel:} & Der Nutzer hat die Möglichkeit, Ergebnisse seiner Jobs abzufragen \\
        \textbf{Vorbedingung:} & Der \gls{Nutzer} ist authentifiziert und die angeforderte Jobs wurden eingereicht und bearbeitet \\
        \textbf{Nachbedingung (Erfolg):} & Der \gls{Nutzer} hat die Ergebnisse der spezifizierten Jobs \\
        \textbf{Nachbedingung (Fehlschlag):} & Der \gls{Nutzer} hat eine Fehlermeldung erhalten  \\
        \textbf{Akteure:} & \gls{Nutzer} \\
        \textbf{Auslösendes Ereignis:} & Der \gls{Nutzer} möchte die Ergebnissevon einem oder mehreren Jobs haben \\
    \end{FA}
    \textbf{Beschreibung:}
    \begin{FAList} 
        \item[1.] Der \gls{Nutzer} schickt eine Anfrage an die \gls{API}, in welcher spezifiziert ist, von welchen Jobs das Ergebnis zurückgegeben werden soll. Es ist auch möglich, die Ergebnisse aller eigenen Jobs abzufragen.
        \item[2.a.] Wenn die Anfrage erfolgreich war, werden die Ergebnisse der Jobs an den \gls{Nutzer} zurückgegeben. 
        \item[2.b.] Wenn die Anfrage nicht erfolgreich war, wird eine Fehlermeldung an den \gls{Nutzer} zurückgegeben.
    \end{FAList}
    
    
    %---------------API - Job-Beschreibung ausgeben
    \phantomsection
    \label{FA:API:Ausgeben der Job-Beschreibung}  
    \item[F1070] \textbf{Ausgeben der Job-Beschreibung für einen oder mehrere Jobs} \\
    \begin{FA}
        \textbf{Ziel:} & Der Nutzer hat die Möglichkeit, Beschreibungen seiner Jobs abzufragen \\
        \textbf{Vorbedingung:} & Der \gls{Nutzer} ist authentifiziert und die angegebenen Jobs wurden bereits eingereicht \\
        \textbf{Nachbedingung (Erfolg):} & Der \gls{Nutzer} hat die Job-Beschreibungen der spezifizierten Jobs \\
        \textbf{Nachbedingung (Fehlschlag):} & Der \gls{Nutzer} hat eine Fehlermeldung erhalten \\
        \textbf{Akteure:} & \gls{Nutzer} \\
        \textbf{Auslösendes Ereignis:} & Der \gls{Nutzer} möchte die Job-Beschreibungen von einem oder mehreren Jobs haben \\
    \end{FA}
    \textbf{Beschreibung:}
    \begin{FAList} 
        \item[1.] Der \gls{Nutzer} schickt eine Anfrage an die \gls{API}, in der er spezifiziert, von welchen Jobs die Beschreibung zurückgegeben werden soll. Es ist auch möglich, die Beschreibungen aller eigenen Jobs abzufragen.
        \item[2.a.] Wenn die Anfrage erfolgreich war, werden die Job-Beschreibungen an den \gls{Nutzer} zurückgegeben. 
        \item[2.b.] Wenn die Anfrage nicht erfolgreich war, wird eine Fehlermeldung an den \gls{Nutzer} zurückgegeben. 
    \end{FAList}
    
    
    %--------------API - Abfragen der Informationen von Mallob
    \phantomsection
    \label{FA:API:Abfragen der Informationen von Mallob}  
    \item[F1080] \textbf{Abfragen der Informationen von Mallob} \\
    \begin{FA}
        \textbf{Ziel:} & Der \gls{Administrator} kann den Status und weitere Informationen, unter anderem Warnungen, zu Mallob abrufen \\
        \textbf{Vorbedingung:} & Der \gls{Administrator} muss authentifiziert sein \\
        \textbf{Nachbedingung (Erfolg):} & Der \gls{Administrator} hat die Informationen zu Mallob erhalten \\
        \textbf{Nachbedingung (Fehlschlag):} & Der \gls{Administrator} hat eine Fehlermeldung erhalten \\
        \textbf{Akteure:} & \gls{Administrator} \\
        \textbf{Auslösendes Ereignis:} & Der \gls{Administrator} möchte genauere Informationen zu Mallob haben \\
    \end{FA}
    \textbf{Beschreibung:}
    \begin{FAList} 
        \item[1.] Der \gls{Administrator} schickt eine entsprechende Anfrage an die \gls{API}.
        \item[2.a.] Wenn die Anfrage erfolgreich war, werden die Daten an den Nutzer zurückgegeben.
        \item[2.b.] Wenn die Anfrage nicht erfolgreich war, wird eine Fehlermeldung an den Nutzer zurückgegeben.
    \end{FAList}
    
    
    %-----------API - Ausgeben eines Event-Streams von Mallob
    \phantomsection

    \label{FA:API:Ausgeben eines Event-Streams von Mallob}
    \item[F1090] \textbf{Ausgeben eines Event-\gls{Stream}s von Mallob} \\
    \begin{FA}
        \textbf{Ziel:} & Der \gls{Nutzer} hat Zugriff auf einen Event-\gls{Stream}, über den kontinuierlich die Events der Jobs im System übertragen werden \\
        \textbf{Vorbedingung:} & Der \gls{Nutzer} muss authentifiziert sein \\
        \textbf{Nachbedingung (Erfolg):} & Der \gls{Nutzer} hat Zugriff auf den Event-\gls{Stream} \\
        \textbf{Nachbedingung (Fehlschlag):} & Der \gls{Nutzer} hat keinen Zugriff und hat eine Fehlermeldung erhalten \\
        \textbf{Akteure:} & \gls{Nutzer} \\
        \textbf{Auslösendes Ereignis:} & Der \gls{Nutzer} möchte die Events von Mallob einsehen \\
    \end{FA}
    \textbf{Beschreibung:}
    \begin{FAList} 
        \item[1.] Der \gls{Nutzer} schickt eine Anfrage an die \gls{API}.
        \item[2.a.] Wenn die Anfrage erfolgreich war, wird der Event-\gls{Stream} an den \gls{Nutzer} zurückgegeben. 
        \item[2.b.] Wenn die Anfrage nicht erfolgreich war, wird eine Fehlermeldung an den \gls{Nutzer} zurückgegeben. 
    \end{FAList}

    
    
    
    \phantomsection
    \label{FA:API:Abrufen von Einstellungen}  
    \item[F1100] \textbf{Abrufen von durch die \gls{Konfigurationsdatei} festgelegte Einstellungen} \\
    \begin{FA}
        \textbf{Ziel:} & Die Einstellungen der \gls{Konfigurationsdatei} können über die \gls{API} abgerufen werden\\
        \textbf{Vorbedingung:} & - \\
        \textbf{Nachbedingung (Erfolg):}  & Die Einstellungen werden zurückgegeben\\
        \textbf{Nachbedingung (Fehlschlag):} & Es wird eine Fehlermeldung zurückgegeben \\
        \textbf{Akteure:} & \gls{Nutzer} \\
        \textbf{Auslösendes Ereignis:} & Der \gls{Nutzer} möchte die aktuellen Einstellungen abfragen \\
    \end{FA}
    \textbf{Beschreibung:}
    \begin{FAList} 
        \item[1.] Schicken der entsprechenden \gls{API}-Anfrage.
        \item[2.a.] Wenn die Anfrage erfolgreich war, werden die Einstellungen der \gls{Konfigurationsdatei} an den \gls{Nutzer} zurückgegeben. 
        \item[2.b.] Wenn die Anfrage nicht erfolgreich war, wird eine Fehlermeldung an den \gls{Nutzer} zurückgegeben. 
    \end{FAList} 
    
    \phantomsection
    \label{FA:API:Andauernde Abfrage des Ergebnises eines Jobs}
    \item[F1110] \textbf{Andauernde Abfrage des \hyperref[B:Job-Ergebnis]{Ergebnisses} eines Jobs}
    \begin{FA}
        \textbf{Ziel:} & Das Fertigstellen eines Jobs kann über die \gls{API} abgewartet werden \\
        \textbf{Vorbedingung:} & Der \gls{Nutzer} muss authentifiziert sein und mindestens einen Job eingereicht haben \\ 
        \textbf{Nachbedingung (Erfolg):} & Der \gls{Nutzer} hat die Meta-Daten des Ergebnisses des Jobs erhalten \\
        \textbf{Nachbedingung (Fehlschlag): } & Die Anfrage an die \gls{API} bricht ab \ \\
        \textbf{Akteure:} & \gls{Nutzer} \\
        \textbf{Auslösendes Ereignis:} & Der \gls{Nutzer} möchte in Echtzeit über den Status eines Jobs informiert werden \\
    \end{FA}
    \textbf{Beschreibung:}
    \begin{FAList}
        \item[1.] Der \gls{Nutzer} schickt eine Anfrage an die \gls{API}.
        \item[2.] Die Anfrage wird nicht beantwortet.
        \item[3.] Der Job wurde erfolgreich bearbeitet.
        \item[4.] Die Anfrage beantwortet und die Meta-Daten des Ergebnisses werden zurück gegeben.
    \end{FAList}

    % --- ab hier wunschkriterium
        
    %---------------------API - Starten von Mallob
    \phantomsection
    \label{FA:API:Starten von Mallob}  
    \item[F1120] (Wunschkriterium) \textbf{Starten von Mallob} \\
    \begin{FA}
        \textbf{Ziel:} & Die Mallob Instanz kann von einem \gls{Administrator} über die \gls{API} gestartet werden\\
        \textbf{Vorbedingung:} & Mallob läuft noch nicht \\
        \textbf{Nachbedingung (Erfolg):} & Mallob ist gestartet und der \gls{Administrator} hat eine Bestätigung erhalten \\
        \textbf{Nachbedingung (Fehlschlag):} & Mallob ist nicht gestartet und der \gls{Administrator} hat eine Fehlermeldung erhalten \\
        \textbf{Akteure:} & \gls{Administrator} \\
        \textbf{Auslösendes Ereignis:} & Der \gls{Administrator} will die Mallob-Instanz starten \\
    \end{FA}
    \textbf{Beschreibung:}
    \begin{FAList}
        \item[1.] Der \gls{Administrator} schickt eine Anfrage an die \gls{API}
        \item[2.] Die Mallob-Instanz wird gestartet
    \end{FAList}
    
    
    %---------------------API - Stoppen von Mallob
    \phantomsection
    \label{FA:API:Stoppen von Mallob}  
    \item[F1130] (Wunschkriterium) \textbf{Stoppen von Mallob} \\
    \begin{FA}
        \textbf{Ziel:} & Die Mallob Instanz kann von einem \gls{Administrator} über die \gls{API} gestoppt werden \\
        \textbf{Vorbedingung:} & Die Mallob-Instanz läuft bereits \\
        \textbf{Nachbedingung (Erfolg):} & Die Mallob-Instanz läuft nicht mehr und der \gls{Administrator} hat eine Bestätigung erhalten \\
        \textbf{Nachbedingung (Fehlschlag:} & Die Mallob-Instanz läuft noch und der \gls{Administrator} hat eine Fehlermeldung erhalten \\
        \textbf{Akteure:} & \gls{Administrator} \\
        \textbf{Auslösendes Ereignis:} & Der \gls{Administrator} will die Mallob-Instanz stoppen \\
    \end{FA}
    \textbf{Beschreibung:}
    \begin{FAList}
        \item[1.] Der Administator schickt eine Anfrage an die \gls{API}
        \item[2.] Die Mallob-Instanz wird gestoppt
    \end{FAList}
    
    
    %-------------------------------API - Neustart von Mallob
    % das könnte man eigentlich aus rausmachen, kann durch stoppen und starten erreicht werden, aber idk
    \phantomsection
    \label{FA:API:Neustart von Mallob}  
    \item[F1140] (Wunschkriterium) \textbf{Neustart von Mallob} \\
    \begin{FA}
        \textbf{Ziel:} & Die Mallob Instanz kann von einem \gls{Administrator} über die \gls{API} neu gestartet werden\\
        \textbf{Vorbedingung:} & Die Mallob-Instanz läuft bereits \\
        \textbf{Nachbedingung (Erfolg):} & Die Mallob-Instanz wurde neu gestartet und der \gls{Administrator} hat eine Bestätigung erhalten \\
        \textbf{Nachbedingung (Fehlschlag):} & Der \gls{Administrator} hat eine Fehlermeldung erhalten \\
        \textbf{Akteure:} & \gls{Administrator} \\
        \textbf{Auslösendes Ereignis:} & Der \gls{Administrator} möchte die Mallob-Instanz neu starten \\
    \end{FA}
    \textbf{Beschreibung:}
    \begin{FAList}
        \item[1.] Der \gls{Administrator} schickt eine Anfrage an die \gls{API}
        \item[2.] Die Mallob-Instanz wird neu gestartet
    \end{FAList}
    
    %\phantomsection
    %\label{FA:API:Einreichen von Jobs per URL} 
    %\item[F1020] (Wunschkriterium) \textbf{Einreichen von Jobs mit Beschreibung per URL} \\
    %\begin{FA}
    %    \textbf{Ziel:} & Ein \gls{Nutzer} kann einen Job über die API einreichen, derer Beschreibung über eine URL verfügbar ist. \\
    %    \textbf{Vorbedingung:} & Der \gls{Nutzer} hat sich mittels \hyperref[FA:API:Authentifizieren von Nutzern]{F1010} authentifiziert \\
    %    \textbf{Nachbedingung (Erfolg):} & Der \gls{Nutzer} erhält eine Bestätigung. \\
    %    \textbf{Nachbedingung (Fehlschlag):} & Der \gls{Nutzer} hat eine Fehlermeldung. \\
    %    \textbf{Akteure:} & \gls{Nutzer} \\
    %    \textbf{Auslösendes Ergebnis:} & Der \gls{Nutzer} möchte  einen Job einreichen. \\
    %\end{FA}
    %\textbf{Beschreibung:}
    %\begin{FAList}
    %        \item[1.a.] Der \gls{Nutzer} schickt eine Anfrage an die API, die sowohl die Job-Konfiguration als auch die URL mit der Job-Beschreibung enthält
    %        \item[2.] Der eingereichte Job wird von Mallob bearbeitet
    %\end{FAList}

    \phantomsection
    \label{FA:API:Registrierung von Nutzern}
    \item[F1150] (Wunschkriterium) \textbf{Registrierung von Nutzern} \\ 
    
    \begin{FA}
        \textbf{Ziel: } & Registrierung über die \gls{API} ist möglich \\
        \textbf{Vorbedingung:} &  -keine- \\
        \textbf{Nachbedingung (Erfolg):} &  Ein \gls{Vorlaeufiges Nutzerkonto} wurde für den \gls{Nutzer} erstellt und der \gls{Nutzer} hat eine Bestätigung erhalten \\
        \textbf{Nachbedingung (Fehlschlag):} &  Der \gls{Nutzer} ist nicht registriert und hat eine Fehlermeldung erhalten \\
        \textbf{Akteure:} & \gls{Nutzer}\\
        \textbf{Auslösendes Ereignis:} & \gls{Nutzer} möchte das System verwenden \\
    \end{FA}
    \textbf{Beschreibung:}
    \begin{FAList} 
        \item[1.] Der \gls{Nutzer} schickt eine Anfrage an die \gls{API}, die seine \hyperref[PD:Registrierungsdaten]{Registrierungsdaten} enthält
        \item[2.] \glslink{Vorlaeufiges Nutzerkonto}{Vorläufiges Konto} wird erstellt
        \item[2.1.a] Wenn die Registrierung erfolgreich war, wird eine Bestätigung an den \gls{Nutzer} zurückgegeben
        \item[2.1.b] Wenn die Registrierung nicht erfolgreich war, wird eine Fehlermeldung an den \gls{Nutzer} zurückgegeben
        \item[3.] Verifizierung des Kontos durch \gls{System-Administrator}
    \end{FAList}


\end{itemize}
    
%-------------------------------------------------------------------
%--------------------WEB INTERFACE----------------------------------
%-------------------------------------------------------------------
\pagebreak

\subsection{Web-Interface}
Die folgenden funktionalen Anforderungen beziehen sich alle auf das \gls{Web-Interface} und sind auch in diesem Kontext zu verstehen.


\begin{itemize}
    \setlength\itemsep{4em}

    \phantomsection
    \label{FA:Web-Interface:Anmelden} 
    \item[F2000] \textbf{Anmelden} \\
    \begin{FA}
        \textbf{Ziel:} & Ein \gls{Nutzer} ist in der Lage, sich im \gls{Web-Interface} zu authentifizieren. \\
        \textbf{Vorbedingung:} & Der \gls{Nutzer} besitzt bereits ein \gls{Nutzerkonto} und ist noch nicht angemeldet. \\
        \textbf{Nachbedingung (Erfolg):}  &  Der \gls{Nutzer} wird angemeldet und zur \hyperref[pages:job-table]{Job-Tabelle} weitergeleitet.\\
        \textbf{Nachbedingung (Fehlschlag):} & Die Anmeldung findet nicht statt und es  wird eine Fehlermeldung angezeigt. \\
        \textbf{Akteure:} & \gls{Nutzer} \\
        \textbf{Auslösendes Ereignis:} &  Der \gls{Nutzer} möchte sich im \gls{Web-Interface} anmelden. \\
    \end{FA}
    \textbf{Beschreibung:}
    \begin{FAList} 
        \item[1.] Aufrufen des Web-Interfaces.
        \item[2.] Eingabe des Nutzernamens.
        \item[3.] Eingabe des Passwortes
        \item[4.] Bestätigung durch Betätigung der Schaltfläche \enquote{Log in}
    \end{FAList}

   
    \phantomsection
    \label{FA:Web-Interface:Job einreichen} 
    \item[F2010] \textbf{Job einreichen} \\
    \begin{FA}
        \textbf{Ziel:} & Einreichen eines neuen Jobs über das \gls{Web-Interface}.\\
        \textbf{Vorbedingung:} & Der \gls{Nutzer} ist angemeldet.  \\
        \textbf{Nachbedingung (Erfolg):} & Der \gls{Nutzer} wird auf die \hyperref[pages:job-page]{Seite des gerade eingereichten Job} weitergeleitet.  \\
        \textbf{Nachbedingung (Fehlschlag):} & Im \gls{Web-Interface} wird eine aussagekräftige Fehlermeldung angezeigt. \\
        \textbf{Akteure:} & \gls{Nutzer} \\
        \textbf{Auslösendes Ereignis:} & Der \gls{Nutzer} möchte einen Job in Auftrag geben. \\
    \end{FA}
    \textbf{Beschreibung:}
    \begin{FAList} 
        \item[1.] Auswahl entsprechenden Schaltfläche in der Navigationsleiste
        \item[2.] Weiterleitung zur \hyperref[pages:submit-job]{Seite zum Einreichen von Jobs}
        \item[3.] Eingabe der notwendigen Optionen des Jobs
        \item[4.] Eingabe der Job-Beschreibung über ein Eingabe-Feld direkt im \gls{Web-Interface}
        \item[5.] Bestätigung der Eingaben
    \end{FAList}
    \textbf{Erweiterungen}
    \begin{FAList}
        \item[3a.] Hinzufügen von gewünschten optionalen Optionen mithilfe des entsprechenden \glslink{Dropdown-Menue}{Dropdown-Menüs}.
        \item[3b.] Eingabe der optionalen Optionen über die entsprechenden Felder.
    \end{FAList}
    \textbf{Alternative 1 zu Schritt 4:}
    \begin{FAList}
        \item[4a] \gls{Dropdown-Menue} nutzen, um Upload der Job-Beschreibung auszuwählen.
        \item[4b] Entsprechende Schaltfläche nutzen, um die entsprechende Job-Beschreibung zum Hochladen auszuwählen.
    \end{FAList}
      \textbf{Alternative 2 zu Schritt 4: (gewünscht)}
    \begin{FAList}
        \item[4a] \gls{Dropdown-Menue} nutzen, um Angabe einer \gls{URL} der Job-Beschreibung auszuwählen.
        \item[4b] Eingabe der \gls{URL} zur Job-Beschreibung im entsprechenden Feld.
    \end{FAList}
    \pagebreak[3]
    
    \phantomsection
    \label{FA:Web-Interface:Abbruch eines einzelnen Jobs} 
    \item[F2020] \textbf{Abbruch eines einzelnen Jobs} \\
    \begin{FA}
        \textbf{Ziel:} & Ein bereits eingereichter Job wird wieder abgebrochen. \\
        \textbf{Vorbedingung:} & Es gibt einen bereits eingereichten, noch nicht fertiggestellten Job. \\
        \textbf{Nachbedingung (Erfolg):}  & Der Job wurde abgebrochen. \\
        \textbf{Nachbedingung (Fehlschlag):} &  Der Job läuft weiter und dem \gls{Nutzer} wird eine entsprechende Fehlermeldung angezeigt. \\
        \textbf{Akteure:} & \gls{Nutzer} \\
        \textbf{Auslösendes Ereignis:} & Der \gls{Nutzer} möchte einen laufenden Job abbrechen. \\
    \end{FA}
    \textbf{Beschreibung:}
    \begin{FAList} 
        \item[1.] Navigation zur Job-Tabelle.
        \item[2.] Anklicken des entsprechenden Jobs in der Tabelle.
        \item[3.] Auswahl der entsprechenden Schaltfläche im nebenstehenden Fenster.
        \item[4a.] Bestätigung des Abbrechens.
        \item[4b.] Keine Bestätigung des Abbrechens, die Aktion wird abgebrochen und der Job läuft weiter.
    \end{FAList}
    \textbf{Alternative zu den Schritten 1 bis 3:}
    \begin{FAList}
        \item[1.] Navigation zur Job-Seite des abzubrechenden Jobs.
        \item[2.] Auswahl der entsprechenden Schaltfläche.
    \end{FAList}
    
    
    \phantomsection
    \label{FA:Web-Interface:Abbruch mehrerer Jobs auf einmal} 
    \item[F2030] \textbf{Abbruch mehrerer Jobs auf einmal} \\
    \begin{FA}
        \textbf{Ziel:} & Mehrere bereits laufende werden auf einmal abgebrochen werden. Mit \enquote{auf einmal} ist hier gemeint, das nicht jeder Job einzeln abgebrochen wird, sondern alle abzubrechenden Jobs mit einem Klick zur selben Zeit abgebrochen werden können. \\
        \textbf{Vorbedingung:} & Es gibt mehrere, bereits eingereichte und noch nicht fertiggestellte Jobs. \\
        \textbf{Nachbedingung (Erfolg):}  & Die Jobs wurden alle abgebrochen. \\
        \textbf{Nachbedingung (Fehlschlag):} & Ein oder mehrere Jobs konnten nicht abgebrochen werden und der \gls{Nutzer} erhält eine entsprechende Fehlermeldung. Die Jobs, bei denen der Abbruch erfolgreich ist, werden auch abgebrochen, falls  dies bei anderen Jobs fehlschlägt.\\
        % [TODO: Formulierung überarbeiten]
        \textbf{Akteure:} & \gls{Nutzer} \\
        \textbf{Auslösendes Ereignis:} & Der \gls{Nutzer} möchte mehrere laufende Jobs auf einmal abbrechen. \\
    \end{FA}
    \textbf{Beschreibung:}
    \begin{FAList} 
        \item[1.] Navigation zur Job-Tabelle
        \item[2.] Setzen eines Kreuzes in der \gls{Checkbox} bei allen Jobs, die Abgebrochen werden sollen.
        \item[3.] Auswahl der entsprechenden Aktion im entsprechenden \gls{Dropdown-Menue}.
        \item[4.] Bestätigung des Abbruchs
    \end{FAList}
    
    
    \phantomsection
    \label{FA:Web-Interface:Herunterladen eines einzelnen Ergebnisses} 
    \item[F2040] \textbf{Herunterladen eines einzelnen Ergebnisses} \\
    \begin{FA}
        \textbf{Ziel:} & Ein einzelnes Ergebnis eines abgeschlossen Jobs kann heruntergeladen werden. \\
        \textbf{Vorbedingung:} & Es gibt einen bereits abgeschlossenen Job. \\
        \textbf{Nachbedingung (Erfolg):}  & Das Ergebnis wurde heruntergeladen. \\
        \textbf{Nachbedingung (Fehlschlag):} &  Das Ergebnis wurde nicht heruntergeladen und eine Fehlermeldung wird angezeigt. \\
        \textbf{Akteure:} & \gls{Nutzer} \\
        \textbf{Auslösendes Ereignis:} & Der \gls{Nutzer} möchte ein einzelnes Ergebnis herunterladen. \\
    \end{FA}
    \textbf{Beschreibung:}
    \begin{FAList} 
        \item[1.] Navigation zur Job-Tabelle.
        \item[2.] Anklicken des entsprechenden abgeschlossenen Jobs in der Tabelle.
        \item[3.] Auswahl der entsprechenden Schaltfläche im nebenstehenden Fenster.
        \item[4.] Das Ergebnis wird heruntergeladen.
    \end{FAList}
    \textbf{Alternative zu den Schritten 1 bis 3:}
    \begin{FAList}
        \item[1.] Navigation zur Job-Seite des Jobs mit dem gewünschten Ergebnis.
        \item[2.] Auswahl der entsprechenden Schaltfläche.
    \end{FAList}
    
    
    \phantomsection
    \label{FA:Web-Interface:herunterladen mehrerer Ergebnisse auf einmal} 
    \item[F2050] \textbf{Herunterladen mehrerer Ergebnisse auf einmal} \\
    \begin{FA}
        \textbf{Ziel:} & Mehrere Ergebnisse können auf einmal heruntergeladen werden. Mit \enquote{auf einmal} ist hier gemeint, das nicht alle Ergebnisse einzeln heruntergeladen werden, sondern alle gewünschten Ergebnisse mit einem Klick zur selben Zeit heruntergeladen werden können. \\
        \textbf{Vorbedingung:} & Es gibt mindestens einen bereits fertigen Job. \\
        \textbf{Nachbedingung (Erfolg):}  & Die Ergebnisse wurde alle heruntergeladen. \\
        \textbf{Nachbedingung (Fehlschlag):} & Ein oder mehrere Ergebnisse konnten nicht heruntergeladen werden und der \gls{Nutzer} erhält eine entsprechende Fehlermeldung. Die restlichen gewünschten Ergebnisse werden dennoch heruntergeladen.\\
        \textbf{FAkteure:} & \gls{Nutzer} \\
        \textbf{Auslösendes Ereignis:} & Der \gls{Nutzer} möchte mehrere Ergebnisse auf einmal herunterladen. \\
    \end{FA}
    \textbf{Beschreibung:}
    \begin{FAList} 
        \item[1.] Navigation zur Job-Liste.
        \item[2.] Setzen eines Kreuzes in der \gls{Checkbox} bei allen Jobs, die heruntergeladen werden sollen.
        \item[3.] Auswahl der entsprechenden Action im entsprechenden \gls{Dropdown-Menue}.
        \item[4.] Bestätigung des Herunterladens.
    \end{FAList}
    
    
    \phantomsection
    \label{FA:Web-Interface:Anzeigen von Fehlern} 
     \item[F2060] \textbf{Anzeigen von Fehlern} \\
    \begin{FA}
        \textbf{Ziel:} & Es gibt eine Möglichkeit, den \gls{Nutzer} über aufgetretene Fehler zu informieren. \\
        \textbf{Vorbedingung:} & - \\
        \textbf{Nachbedingung (Erfolg):}  & Eine Fehlermeldung wird angezeigt. \\
        \textbf{Nachbedingung (Fehlschlag):} & Es wird keine Fehlermeldung angezeigt. \\
        \textbf{Akteure:} & System \\
        \textbf{Auslösendes Ereignis:} & Ein Fehler ist aufgetreten \\
    \end{FA}
    \textbf{Beschreibung:}
    \begin{FAList} 
        \item[1.] Ein Fehler tritt im System auf.
        \item[2.] Dem \gls{Nutzer} wird ein Fenster mit der Fehlermeldung angezeigt. Dieses Fenster befindet sich im Vordergrund und blockiert das restliche \gls{Web-Interface}, bis der \gls{Nutzer} die Bestätigungs-Schaltfläche betätigt oder er neben das Fenster klickt, um dieses zu schließen.
    \end{FAList}
    
    
   
    
    
    \phantomsection
    \label{FA:Web-Interface:Anzeigen von Warnungen und Fehlermeldungen}
    \item[F2070] \textbf{Anzeigen von Mallob Warnungen} \\
    \begin{FA}
        \textbf{Ziel:} & Der \gls{Administrator} kann die Fehlermeldungen und Warnungen einsehen, die von Mallob ausgegeben werden \\
        \textbf{Vorbedingung:} & Der \gls{Administrator} muss im \gls{Web-Interface} angemeldet sein \\
        \textbf{Nachbedingung (Erfolg):} & Es werden die Fehlermeldungen und Warnungen von Mallob angezeigt \\
        \textbf{Nachbedingung (Fehlschlag):} & Es werde die Fehlermeldungen und Warnungen von Mallob nicht angezeigt \\
        \textbf{Akteure:} & \gls{Administrator} \\
        \textbf{Auslösendes Ereignis:} & Der \gls{Administrator} möchte die Fehlermeldungen und Warnungen von Mallob sehen \\
    \end{FA}
    \textbf{Beschreibung:}
    \begin{FAList} 
        \item[1.] Navigation zur \hyperref[pages:admin]{Administratoren-Seite}.
        \item[2.] Die Fehlermeldungen und Warnungen werden im \gls{Web-Interface} angezeigt
    \end{FAList}
    
    
    \phantomsection
    \label{FA:Web-Interface:Einsehen von Job-Informationen}
    \item[F2080] \textbf{Einsehen von Job-Information} \\
    \begin{FA}
        \textbf{Ziel:} & Verschiedene Wege, Informationen zu einem Job anzuzeigen. \\
        \textbf{Vorbedingung:} &  Der \gls{Nutzer} ist angemeldet und besitzt mindestens einen Job. \\
        \textbf{Nachbedingung (Erfolg):}  &  Der \gls{Nutzer} kann Informationen zum gewünschten Job einsehen. \\
        \textbf{Nachbedingung (Fehlschlag):} &  Eine Fehlermeldung wird angezeigt. \\
        \textbf{Akteure:} & \gls{Nutzer} \\
        \textbf{Auslösendes Ereignis:} & \gls{Nutzer} möchte Informationen zu einem Job einsehen. \\
    \end{FA}
    \textbf{Beschreibung:}
    \begin{FAList} 
        \item[1.] Navigation zur Job-Tabelle.
        \item[2.] Auswahl der gewünschten Attribute im \gls{Dropdown-Menue} über der Liste.
        \item[3.] Die ausgewählten Attribute werden jeweils als eigene Spalte in der Tabelle angezeigt. 
    \end{FAList}
    \textbf{Alternative 1:}
    \begin{FAList}
        \item[1.] Navigation zur Job-Tabelle.
        \item[2.] Anklicken des entsprechenden Jobs in der Tabelle.
        \item[3.] Die Job-Informationen werden im nebenstehenden Fenster angezeigt.
    \end{FAList}
    \textbf{Alternative 2:}
    \begin{FAList}
        \item[1.] Navigation zur Job-Tabelle des Jobs. 
        \item[2.] Die Job-Informationen werden auf der Job-Seite angezeigt.
    \end{FAList}
    

 
   % \phantomsection
   % \label{FA:Web-Interface:Einsehen von Job-Informationen}
   % \item[F2080] \textbf{Einsehen von Job-Information} \\
   % \begin{FA}
   %     \textbf{Ziel:} & Verschiedene Wege, Informationen zu einem Job anzuzeigen. \\
   %     \textbf{Vorbedingung:} &  Der \gls{Nutzer} ist angemeldet und besitzt mindestens einen Job. \\
   %     \textbf{Nachbedingung (Erfolg):}  &  Der \gls{Nutzer} kann Informationen zum gewünschten Job einsehen. \\
   %     \textbf{Nachbedingung (Fehlschlag):} &  Eine Fehlermeldung wird angezeigt. \\
   %     \textbf{Akteure:} & \gls{Nutzer} \\
   %     \textbf{Auslösendes Ereignis:} & \gls{Nutzer} möchte Informationen zu einem Job einsehen. \\
   % \end{FA}
   % \textbf{Beschreibung:}
   % \begin{FAList} 
   %     \item[1.] Navigation zur Job-Tabelle.
   %     \item[2.] Auswahl der gewünschten Attribute im \gls{Dropdown-Menue} über der Liste.
   %     \item[3.] Die ausgewählten Attribute werden jeweils als eigene Spalte in der Tabelle angezeigt. 
   % \end{FAList}
   % \textbf{Alternative 1:}
   % \begin{FAList}
   %     \item[1.] Navigation zur Job-Tabelle.
   %     \item[2.] Anklicken des entsprechenden Jobs in der Tabelle.
   %     \item[3.] Die Job-Informationen werden im nebenstehenden Fenster angezeigt.
   % \end{FAList}
   % \textbf{Erweiterung der Alternative 1:}
   % \begin{FAList}
   %     \item[4.] Anklicken der Schaltfläche \enquote{open job page}.
   %     \item[5.] Weiterleitung zur Job-Seite.
   %     \item[6.] Die Job-Informationen werden auf der Job-Seite angezeigt.
   % \end{FAList}
   % \textbf{Alternative 2:}
   % \begin{FAList}
   %     \item[1.] Direkter Aufruf der Job-Seite über die entsprechende \gls{URL}.
   %     \item[2.] Die Job-Informationen werden auf der Job-Seite angezeigt.
   % \end{FAList}
    
    
    \phantomsection
    \label{FA:Web-Interface:Hinzufügen von Spalten}
    \item[F2090] \textbf{Hinzufügen von Spalten in der Job-Tabelle} \\
    \begin{FA}
        \textbf{Ziel:} & Es können Spalten zur Job-Tabelle hinzugefügt werden, die jeweils ein Attribut der Job-Informationen beinhalten \\
        \textbf{Vorbedingung:} & Der \gls{Nutzer} muss im \gls{Web-Interface} angemeldet sein \\
        \textbf{Nachbedingung (Erfolg):} & Die gewünschte Spalte wurde zur Job-Tabelle hinzugefügt  \\
        \textbf{Nachbedingung (Fehlschlag):} & Die gewünschte Spalte wurde nicht hinzugefügt \\
        \textbf{Akteure:} & \gls{Nutzer} \\
        \textbf{Auslösendes Ereignis:} & Der \gls{Nutzer} möchte eine Spalte zur Job-Tabelle hinzufügen \\
    \end{FA}
    \textbf{Beschreibung:}
    \begin{FAList} 
        \item[1.] Navigation zur Job-Tabelle
        \item[2.] Anklicken des \glslink{Dropdown-Menue}{Dropdown-Menüs}
        \item[3.] Durch Anklicken auswählen, welches Attribut als Spalte hinzugefügt werden soll. Im \glslink{Dropdown-Menue}{Dropdown-Menü} werden immer nur Attribute angezeigt, die noch nicht als Spalte in der Tabelle aufgeführt sind.
    \end{FAList}
    
    
    \phantomsection
    \label{FA:Web-Interface:Entfernen von Spalten}
    \item[F2100] \textbf{Entfernen von Spalten in der Job-Tabelle} \\
    \begin{FA}
        \textbf{Ziel:} & Es können Spalten aus der Job-Tabelle entfernt werden \\
        \textbf{Vorbedingung:} & Der \gls{Nutzer} muss im \gls{Web-Interface} angemeldet sein und die es wurde mindestens eine Spalte mittels \hyperref[FA:Web-Interface:Hinzufügen von Spalten]{F2090} hinzugefügt \\
        \textbf{Nachbedingung (Erfolg):} & Die gewünschte Spalte wurde aus der Job-Tabelle entfernt \\
        \textbf{Nachbedingung (Fehlschlag):} & Die gewünschte Spalte wurde nicht aus der Job-Tabelle entfernt \\
        \textbf{Akteure:} & \gls{Nutzer} \\
        \textbf{Auslösendes Ereignis:} & Der \gls{Nutzer} möchte eine angezeigte Spalte aus der Tabelle löschen\\
    \end{FA}
    \textbf{Beschreibung:}
    \begin{FAList} 
        \item[1.] Navigation zur Job-Tabelle
        \item[2.] Anklicken des \enquote{x}-Symbols in der entsprechenden Spalte
    \end{FAList}
    
    \phantomsection
    \label{FA:Web-Interface:Akutalisieren}
    \item[F2105] \textbf{Aktualisieren der Job-Tabelle} \\
    \begin{FA}
        \textbf{Ziel:} & Die Job-Tabelle kann aktualisiert werden, ohne dass die Seite neu geladen werden muss \\
        \textbf{Vorbedingung:} & Der \gls{Nutzer} muss im \gls{Web-Interface} angemeldet sein \\
        \textbf{Nachbedingung (Erfolg):} & Die Job-Tabelle  und eventuelle im nebenstehenden Fenster angezeigte Informationen wurde aktualisiert\\
        \textbf{Nachbedingung (Fehlschlag):} & Die Job-Tabelle wurde nicht aktualisiert und eine Fehlermeldung wird angezeigt \\
        \textbf{Akteure:} & \gls{Nutzer} \\
        \textbf{Auslösendes Ereignis:} & Der \gls{Nutzer} möchte die Job-Tabelle aktualisieren\\
    \end{FA}
    \textbf{Beschreibung:}
    \begin{FAList} 
        \item[1.] Navigation zur Job-Tabelle
        \item[2.] Anklicken der entsprechenden Schaltfläche.
    \end{FAList}
    \textbf{Alternative (gewünscht)}
    \begin{FAList}
        \item[1.] Die Tabelle wird automatische aktuell gehalten, die  entsprechende Schaltfläche entfällt.
    \end{FAList}
    
 


    \phantomsection
    \label{FA:Web-Interface:Abmelden} 
    \item[F2110] \textbf{Abmelden} \\
    \begin{FA}
        \textbf{Ziel:} & Der \gls{Nutzer} kann sich wieder abmelden. \\
        \textbf{Vorbedingung:} & Der \gls{Nutzer} ist angemeldet. \\
        \textbf{Nachbedingung (Erfolg):}  & Der \gls{Nutzer} ist abgemeldet und wird zur Login-Seite weitergeleitet. \\
        \textbf{Nachbedingung (Fehlschlag):} & Der \gls{Nutzer} ist weiterhin angemeldet und eine Fehlermeldung wird angezeigt. \\
        \textbf{Akteure:} & \gls{Nutzer}\\
        \textbf{Auslösendes Ereignis:} & \gls{Nutzer} möchte sich Abmelden \\
    \end{FA}
    \textbf{Beschreibung:}
    \begin{FAList} 
        \item[1.] Betätigung der entsprechenden Schaltfläche im Navigations-Menü.
    \end{FAList}




\phantomsection
    \label{FA:Web-Interface:Registrierung von Nutzern} 
    \item[F2120] (Wunschkriterium) \textbf{Registrierung von Nutzern} \\
    \begin{FA}
        \textbf{Ziel:} & Ein \gls{Nutzer} ist in der Lage, ein neues \gls{Nutzerkonto} zu erstellen.\\
        \textbf{Vorbedingung:} &  Der \gls{Nutzer} ist nicht angemeldet. \\
        \textbf{Nachbedingung (Erfolg):}  &  Ein \gls{Vorlaeufiges Nutzerkonto} wird für den \gls{Nutzer} erstellt und er wird zur \hyperref[pages:job-table]{Job-Tabelle} weitergeleitet. \\
        \textbf{Nachbedingung (Fehlschlag):} &  Das \gls{Nutzerkonto} kann nicht erstellt werden und es wird eine Fehlermeldung angezeigt. \\
        \textbf{Akteure:} & Person, welche ein neues \gls{Nutzerkonto} erstellen möchte. \\
        \textbf{Auslösendes Ereignis:} &  Die Person möchte ein neues \gls{Nutzerkonto} erstellen. \\
    \end{FA}
    \textbf{Beschreibung:}
    \begin{FAList}
        \item[1.] Aufrufen des Web-Interfaces
        \item[2.] Auswählen der Schaltfläche \enquote{register}
        \item[3.] Weiterleitung zur \hyperref[pages:register]{Registerung}
        \item[2.] Eingabe des gewünschten Nutzernames
        \item[3.] Eingabe des gewünschten Passwortes
        \item[4.] Eingabe der Wiederholung des Passwortes
        \item[5.] Bestätigung der Eingabe mittels der Schaltfläche \enquote{register}
        \item[6.] \glslink{Vorlaeufiges Nutzerkonto}{Vorläufiges Konto} wird erstellt
        \item[7.] \glslink{Vorlaeufiges Nutzerkonto}{Vorläufiges Nutzerkonto} wird durch \gls{System-Administrator} verifiziert.
    \end{FAList}
    
    
    
    \phantomsection
    \label{FA:Web-Interface:Neustart} 
    \item[F2130] (Wunschkriterium) \textbf{Neustart eines abgebrochenen Jobs} \\
    \begin{FA}
        \textbf{Ziel:} & Ein \gls{Nutzer} kann einen abgebrochenen Job neustarten\\
        \textbf{Vorbedingung:} & Der \gls{Nutzer} hat einen abgebrochenen Job \\
        \textbf{Nachbedingung (Erfolg):}  &  Der abgebrochene Job wird bearbeitet \\
        \textbf{Nachbedingung (Fehlschlag):} &  Der abgebrochene Job wird nicht bearbeiten \\
        \textbf{Akteure:} & \gls{Nutzer}\\
        \textbf{Auslösendes Ereignis:} & Der \gls{Nutzer} möchte einen abgebrochenen Job wieder starten \\
    \end{FA}
    \textbf{Beschreibung:}
    \begin{FAList} 
        \item[1.] Navigation zur Job-Tabelle.
        \item[2.] Anklicken des neuzustartenden, abgebrochenen Jobs.
        \item[3.] Auswählen der entsprechenden Schaltfläche im der nebenstehenden Fenster.
        \item[4.] Weiterleitung zur \hyperref[pages:submit-job]{Seite zum Einreichen von Jobs}, wobei hier die entsprechenden Optionen bereits wieder ausgefüllt sind.
        \item[5.] Verändern der Job-Beschreibung oder der Job-Konfiguration.
        \item[6.] Einreichen des Jobs.
    \end{FAList} 
    \textbf{Alternative der Schritte 1-3:}
    \begin{FAList}
        \item[1.] Navigation zur Job-Seite des neuzustartenden Jobs.
        \item[2.] Auswählen der entsprechenden Schaltfläche.
    \end{FAList}
    \textbf{Alternative von Schritt 5.}
    \begin{FAList}
        \item[5.] Nichts verändern.
    \end{FAList}
    
    
    \phantomsection
    \label{FA:Web-Interface:Verwalten von Malllob}
    \item[F2140] (Wunschkriterium) \textbf{Verwalten von Mallob} \\
    \begin{FA}
        \textbf{Ziel:} & Mallob kann vom \gls{Administrator} gestartet, gestoppt und neu gestartet werden \\
        \textbf{Vorbedingung:} & Der \gls{Administrator} muss im \gls{Web-Interface} angemeldet sein \\
        \textbf{Nachbedingung (Erfolg):} & Mallob ist gestartet, gestoppt oder neu gestartet \\
        \textbf{Nachbedingung (Fehlschlag):} & Mallob ist nicht gestartet, gestoppt oder neu gestartet \\
        \textbf{Akteure:} & \gls{Administrator} \\
        \textbf{Auslösendes Ereignis:} & Der \gls{Administrator} will Mallob starten, stoppen oder neu starten \\
    \end{FA}
    \textbf{Beschreibung:}
    \begin{FAList} 
        \item[1.] Navigation zur \hyperref[pages:admin]{Administratoren-Seite}
        \item[2.] Anklicken des Knopfes \enquote{start mallob} 
        \item[3.] Mallob wird gestartet
    \end{FAList}
    \textbf{Alternative 1 zu den Schritten 2 bis 3:}
    \begin{FAList}
        \item[2.] Anklicken des Knopfes \enquote{stop mallob}
        \item[3.] Mallob wird gestoppt
    \end{FAList}
    \textbf{Alternative 2 zu den Schritten 2 bis 3:}
    \begin{FAList}
        \item[2.] Anklicken des Knopfes \enquote{restart mallob}
        \item[3.] Mallob wird neu gestartet
    \end{FAList}
    
    
    \phantomsection
    \label{FA:Web-Interface:Sortieren der Tabelle}
    \item[F2150] (Wunschkriterium) \textbf{Sortieren der Job-Tabelle nach Attributen} \\
    \begin{FA}
        \textbf{Ziel:} & Die Einträge der Job-Tabelle können nach den Attributen der verschiedenen Spalten sortiert werden \\
        \textbf{Vorbedingung:} & Der \gls{Nutzer} muss im \gls{Web-Interface} angemeldet sein \\
        \textbf{Nachbedingung (Erfolg):} & Die Einträge der Job-Tabelle sind nach dem Wunsch des \glslink{Nutzer}{Nutzers} sortiert \\
        \textbf{Nachbedingung (Fehlschlag):} & Die Einträge der Job-Tabelle sind nicht nach dem Wunsch des \glslink{Nutzer}{Nutzers} sortiert \\
        \textbf{Akteure:} & \gls{Nutzer} \\
        \textbf{Auslösendes Ereignis:} & Der \gls{Nutzer} möchte die Einträge der Job-Tabelle sortieren \\
    \end{FA}
    \textbf{Beschreibung:}
    \begin{FAList} 
        \item[1.] Navigation zur Job-Tabelle
        \item[2.] Anklicken der Spalte, nach deren Attribut die Tabelle sortiert werden soll
        \item[3.a.] Zeigt der Pfeil neben dem Attribut nach unten, werden die Einträge absteigend, bzw. alphabetisch sortiert
        \item[3.b.] Zeigt der Pfeil neben dem Attribut nach oben, werden die Einträge aufsteigend, bzw. umgekehrt alphabetisch sortiert
    \end{FAList}
    
       \phantomsection
    \label{FA:Web-Interface:Filtern für Admins}
    \item[F2160] (Wunschkriterium) \textbf{Filtern der Job-Tabelle für Administratoren} \\
    \begin{FA}
        \textbf{Ziel:} & Administratoren können wahlweise die Jobs aller \gls{Nutzer} oder nur die eigenen in der Tabelle sehen \\
        \textbf{Vorbedingung:} & Ein \gls{Administrator} ist angemeldet \\
        \textbf{Nachbedingung (Erfolg):} & Der Filter wurde korrekt angewandt, es sind nur die gewünschten Jobs zu sehen \\
        \textbf{Nachbedingung (Fehlschlag):} & Der Filter wurde nicht angewandt, eine Fehlermeldung wird ausgegeben \\
        \textbf{Akteure:} & \gls{Nutzer} \\
        \textbf{Auslösendes Ereignis:} & Ein \gls{Administrator} möchte ändern, welche Jobs er in der Job-Tabelle sieht.\\
    \end{FA}
    \textbf{Beschreibung:}
    \begin{FAList} 
        \item[1.] Navigation zur Job-Tabelle
        \item[2.] Setzen der \gls{Checkbox} \enquote{see all jobs}
    \end{FAList}
    
    
    \phantomsection
    \label{FA:Web-Interface:Anzeigen von Plugins}
    \item[F2170] (Wunschkriterium) \textbf{Anzeigen von Plugins} \\
    \begin{FA}
        \textbf{Ziel:} & Das \gls{Web-Interface} kann extern erstelle Plugins anzeigen. \\
        \textbf{Vorbedingung:} & - \\
        \textbf{Nachbedingung (Erfolg):}  &  Plugins werden als Dropdown-Menü in der Navigations-Leiste angezeigt und können ausgewählt werden.\\
        \textbf{Nachbedingung (Fehlschlag):} & Plugins werden nicht angezeigt. \\
        \textbf{Akteure:} & \gls{Nutzer} \\
        \textbf{Auslösendes Ereignis:} &  Mindestens Plugin wurde eingelesen\\
    \end{FA}
    \textbf{Beschreibung:}
    \begin{FAList} 
        \item[1.] Einlesen von Plugins durch \hyperref[FA:System:Einstellungen festlegen]{F4000}
        \item[2.] Anzeigen eines Eintrags für Plugins in der Navigations-Leiste
    \end{FAList}
    
    
  \end{itemize}
%-------------------------------------------------------------------
%--------------------Visualisierung---------------------------------
%-------------------------------------------------------------------
\pagebreak

\subsection{Visualisierung}
Die Visualisierung findet \gls{Web-Interface} statt und ist nur dort einsehbar.


\begin{itemize}
    \setlength\itemsep{4em}



    %----------------------Visualisierung - Anzeigen des Systemzustandsja lles 
    
    \phantomsection
    \label{FA:Visualisierung:Anzeigen des Systemzustandes}
    \item[F3000] \textbf{Anzeigen des aktuellen Systemzustands} \\
    \begin{FA}
        \textbf{Ziel:} & Der \gls{Nutzer} kann eine Visualisierung des aktuellen Systemzustandes von Mallob sehen \\
        \textbf{Vorbedingung:} & Der \gls{Nutzer} muss im \gls{Web-Interface} angemeldet sein \\
        \textbf{Nachbedingung (Erfolg):} & Der Systemzustand von Mallob wird im \gls{Web-Interface} angezeigt \\
        \textbf{Nachbedingung (Fehlschlag):} &  Es wird eine Fehlermeldung im \gls{Web-Interface} angezeigt \\
        \textbf{Akteure:} & \gls{Nutzer} \\
        \textbf{Auslösendes Ereignis:} & Der \gls{Nutzer} möchte die Visualisierung des Systemzustands sehen \\
    \end{FA}
    \textbf{Beschreibung:}
    \begin{FAList} 
        \item[1.] Navigation zur Visualisierung im \gls{Web-Interface}
        \item[2.a.] Wenn die Visualisierung erfolgreich geladen wurde, wird diese im \gls{Web-Interface} angezeigt  
        \item[3.b.] Wenn die Visualisierung nicht erfolgreich geladen wurde, wird eine Fehlermeldung im \gls{Web-Interface} angezeigt 
    \end{FAList}
    
    
    
     %-----------------Visualisierung - Ansehen von Details
    \phantomsection
    \label{FA:Visualisierung:Anzeigen von Details} 
    \item[F3010] \textbf{Anzeigen von Job-Details} \\
    \begin{FA}
        \textbf{Ziel:} & Durch Anklicken eines Jobs in der Visualisierung können Details zum Job eingesehen werden \\
        \textbf{Vorbedingung:} & Der \gls{Nutzer} ist im \gls{Web-Interface} angemeldet \\
        \textbf{Nachbedingung (Erfolg):} & Es werden Details zum ausgewählten Jobs angezeigt \\
        \textbf{Nachbedingung (Fehlschlag):} & Es wird eine Fehlermeldung angezeigt und es werden keine Details angezeigt \\
        \textbf{Akteure:} & \gls{Nutzer} \\
        \textbf{Auslösendes Ereignis:} & Der \gls{Nutzer} möchte die Details eines Job in der Visualisierung einsehen. \\
    \end{FA}
    \textbf{Beschreibung:}
    \begin{FAList} 
        \item[1.] Navigation zur Visualisierung im \gls{Web-Interface}
        \item[2.] Anklicken eines Jobs im linken Fenster der Visualisierung
        \item[3] Die Details werden im rechten Fenster angezeigt. Ist der Job keiner der eigenen und der \gls{Nutzer} kein \gls{Administrator}, so werden die Datails pseudonymisiert dargestellt. Ebenso werden beiden Schaltflächen nicht angezeigt.
    \end{FAList}

    
    
    %-----------------Visualisierung - Pausieren der Visualisierung
    \phantomsection
    \label{FA:Visualisierung:Pausieren der Visualisierung} 
    \item[F3020] \textbf{Pausieren der Visualisierung} \\
    \begin{FA}
        \textbf{Ziel:} & Die Visualisierung kann pausiert werden \\
        \textbf{Vorbedingung:} & Der \gls{Nutzer} muss im \gls{Web-Interface} angemeldet sein und die ist  die Visualisierung läuft bereits \\
        \textbf{Nachbedingung (Erfolg):} & Der \gls{Nutzer} muss im \gls{Web-Interface} angemeldet sein und die Visualisierung ist pausiert und aktualisiert sich nicht mehr \\
        \textbf{Nachbedingung (Fehlschlag):} & Die Visualisierung ist nicht pausiert und läuft weiter \\
        \textbf{Akteure:} & \gls{Nutzer} \\
        \textbf{Auslösendes Ereignis:} & Der \gls{Nutzer} möchte die Visualisierung pausieren \\
    \end{FA}
    \textbf{Beschreibung:}
    \begin{FAList} 
        \item[1.] Navigation zur Visualisierung im \gls{Web-Interface}
        \item[2.] Anklicken der Pause-Taste
        \item[3.] Die Visualisierung pausiert und wird nicht mehr aktualisiert.
    \end{FAList}
    
    
    %--------------------Visualisierung - Starten der Visualisierung
    \phantomsection
    \label{FA:Visualisierung:Starten der Visualisierung} 
    \item[F3030] \textbf{Starten der Visualisierung} \\
    \begin{FA}
        \textbf{Ziel:} & Die Visualisierung kann nach dem pausieren wieder gestartet werden \\
        \textbf{Vorbedingung:} & Der \gls{Nutzer} ist im \gls{Web-Interface} angemeldet und die Visualisierung ist bereits pausiert \\
        \textbf{Nachbedingung (Erfolg):} & Die Visualisierung läuft wieder und wird aktualisiert \\
        \textbf{Nachbedingung (Fehlschlag):} & Die Visualisierung ist weiterhin pausiert \\
        \textbf{Akteure:} & \gls{Nutzer} \\
        \textbf{Auslösendes Ereignis:} & Der \gls{Nutzer} möchte die Visualisierung wieder starten \\
    \end{FA}
    \textbf{Beschreibung:}
    \begin{FAList} 
        \item[1.] Navigation zur Visualisierung im \gls{Web-Interface}
        \item[2.] Anklicken der Wiedergabe-Taste
        \item[3.] Die Visualisierung startet und wird wieder aktualisiert
    \end{FAList}
    
    
    
    
    
    %-------------Visualisierung - Springen zu einem bestimmten Zeitpunkt
    \phantomsection
    \label{FA:Visualisierung:Springen} 
    \item[F3040] \textbf{Springen zu einem bestimmten Zeitpunkt der Visualisierung} \\
    \begin{FA}
        \textbf{Ziel:} & Der \gls{Nutzer} kann zu einem beliebigen Zeitpunkt in der Visualisierung vor- oder zurückspringen \\
        \textbf{Vorbedingung:} & Der \gls{Nutzer} muss im \gls{Web-Interface} angemeldet sein und Mallob muss gestartet sein \\
        \textbf{Nachbedingung (Erfolg):} & Die Visualisierung wird ab dem gewählten Zeitpunkt abgespielt \\
        \textbf{Nachbedingung (Fehlschlag):} & Die Visualisierung wird nicht ab dem gewählten Zeitpunkt abgespielt  \\
        \textbf{Akteure:} & \gls{Nutzer} \\
        \textbf{Auslösendes Ereignis:} & Der \gls{Nutzer} möchte die Visualisierung an einem bestimmten Zeitpunkt ansehen \\
    \end{FA}
    \textbf{Beschreibung:}
    \begin{FAList} 
        \item[1.] Navigation zur Visualisierung
        \item[2.] Wählen des gewünschte Zeitpunktes durch Anklicken oder Ziehen des Sliders zur gewünschten Position
        \item[3.] Die Visualisierung wird ab dem gewählten Zeitpunkt abgespielt
    \end{FAList}
    
    
    
    
    %----------------Visualisierung - Ändern der Wiedergabegeschwindigkeit
    \phantomsection
    \label{FA:Visualisierung:Aendern der Wiedergabegeschwindigkeit} 
    \item[F3050] (Wunschkriterium) \textbf{Ändern der Wiedergabegeschwindigkeit} \\
    \begin{FA}
        \textbf{Ziel:} & Die Geschwindigkeit, mit der die Visualisierung abgespielt wird kann geändert werden \\
        \textbf{Vorbedingung:} & Der \gls{Nutzer} ist im \gls{Web-Interface} angemeldet \\
        \textbf{Nachbedingung (Erfolg):} & Die Visualisierung wird in der gewünschten Geschwindigkeit abgespielt \\
        \textbf{Nachbedingung (Fehlschlag):} & Die Visualisierung wird weiterhin in der alten Geschwindigkeit abgespielt \\
        \textbf{Akteure:} & \gls{Nutzer} \\
        \textbf{Auslösendes Ereignis:} & Der \gls{Nutzer} möchte die Wiedergabegeschwindigkeit der Visualisierung ändern \\
    \end{FA}
    \textbf{Beschreibung:}
    \begin{FAList} 
        \item[1.] Navigation zur Visualisierung im \gls{Web-Interface}
        \item[2.] Eingabe der gewünschten Wiedergabegeschwindigkeit in das Feld \enquote{replay speed}
        \item[3.a.] Die Visualisierung wird in der gewünschten Wiedergabegeschwindigkeit abgespielt
        \item[3.b.] Wenn die gewählte Wiedergabegeschwindigkeit schneller als Echtzeit ist und sich die Visualisierung am aktuellsten Punkt befindet, wird die Wiedergabegeschwindigkeit auf Echtzeit gesetzt
        \item[3.c.] Wenn die gewählte Wiedergabegeschwindigkeit negativ ist, wird die Visualisierung rückwärts in der gewünschten Wiedergabegeschwindigkeit abgespielt
    \end{FAList}
    
    
      %---------------Visualisierung - Anzeigen des Binärbaumes für einen Job
    
    \phantomsection
    \label{FA:Visualisierung:Anzeigen des Binaerbaumes für einen Job}
    \item[F3060] (Wunschkriterium) \textbf{Anzeigen des \glslink{Binaerbaum}{Binärbaumes} für einen Job} \\
    \begin{FA}
        \textbf{Ziel:} & Visualisierung des zu einem Job gehörenden \glslink{Binaerbaum}{Binärbaumes} \\
        \textbf{Vorbedingung:} & Der \gls{Nutzer} muss im \gls{Web-Interface} angemeldet sein und muss einen Job eingereicht oder abgeschlossen haben \\
        \textbf{Nachbedingung (Erfolg):} & Der \gls{Binaerbaum} zu dem gewünschten Job wird im \gls{Web-Interface} angezeigt \\
        \textbf{Nachbedingung (Fehlschlag):} & Der \gls{Binaerbaum} wird nicht im \gls{Web-Interface} angezeigt  \\
        \textbf{Akteure:} & \gls{Nutzer} \\
        \textbf{Auslösendes Ereignis:} & Der \gls{Nutzer} klickt einen Job in der Visualisierung an \\
    \end{FA}
    \textbf{Beschreibung:}
    \begin{FAList} 
        \item[1.] Navigation zur Visualisierung im \gls{Web-Interface}
        \item[2.] Anklicken eines Jobs im linken Panel der Visualisierung
        \item[3.] Der \gls{Binaerbaum} zu dem ausgewählten Job wird im \gls{Web-Interface} angezeigt
    \end{FAList}
    
\end{itemize}


%-------------------------------------------------------------------
%----------------------------System---------------------------------
%-------------------------------------------------------------------
\pagebreak

\subsection{System}
    \setlength\itemsep{4em}




\begin{itemize}
    \phantomsection
    \label{FA:System:Einstellungen festlegen}
    \item[F4000] \textbf{Festlegen von Einstellungen mittels einer \gls{Konfigurationsdatei} bei Systemstart} \\
    \begin{FA}
        \textbf{Ziel:} & Bestimmte Einstellungen können mit einer \gls{Konfigurationsdatei} konfiguriert werden, welche bei Systemstart eingelesen werden. \\
        \textbf{Vorbedingung:} & Eine korrekt formatierte \gls{Konfigurationsdatei} existiert an einem fest vorgeschriebenen Ort im Dateisystem.\\
        \textbf{Nachbedingung (Erfolg):}  & Die Einstellungen der \gls{Konfigurationsdatei} sind angewandt \\
        \textbf{Nachbedingung (Fehlschlag):} & Die \gls{Konfigurationsdatei} kann nicht geladen werden, das Programm beendet sich mit einer entsprechenden Fehlermeldung.\\
        
        \textbf{Akteure:} & System\\
        \textbf{Auslösendes Ereignis:} & Starten des Systems
    \end{FA}
    \textbf{Beschreibung:}
    \begin{FAList} 
        \item[1.] Starten des Systems.
        \item[2.] \gls{Konfigurationsdatei} wird beim Start automatisch eingelesen.
    \end{FAList} 

  
    \phantomsection
    \label{FA:System:Einlesen von Plugins bei Systemstart}
    \item[F4010] (Wunschkriterium) \textbf{Einlesen von Plugins bei Systemstart} \\
    \begin{FA}
        \textbf{Ziel:} & Das System kann extern erstellte Plugins einlesen. \\
        \textbf{Vorbedingung:} & Es liegen ein einzulesende Plugins vor \\
        \textbf{Nachbedingung (Erfolg):}  & Die Plugins wurden eingelesen. \\
        \textbf{Nachbedingung (Fehlschlag):} & Ein oder mehrere Plugins konnten nicht eingelesen werden und auf der \hyperref[pages:admin]{Administratoren-Seite} werden mittels \hyperref[A:Web-Interface:Anzeigen von Warungen und Fehlermeldungen]{F2110} \\
        \textbf{Akteure:} & System \\
        \textbf{Auslösendes Ereignis:} & Starten des Systems \\
    \end{FA}
    \textbf{Beschreibung:}
    \begin{FAList} 
        \item[1.] Starten des Systems.
        \item[2.] Plugins werden beim Start eingelesen.
    \end{FAList} 


\end{itemize}


\newpage
\label{PD}
\section{Produktdaten}


\label{PD:Registrierungsdaten}
\subsection{Registrierungsdaten}
Die Registrierungsdaten sind diejenigen Daten, die ein Nutzer angeben muss, um sich registrieren zu können. Diese Daten werden gespeichert, bis das Konto durch einen Administrator verifiziert wurde.
\begin{itemize}[noitemsep]
    \item Nutzername
    \item verschlüsseltes Passwort
    \item E-Mail
\end{itemize}

\label{PD:Nutzerdaten}
\subsection{Nutzerdaten}
Nutzerdaten sind alle nutzerspezifischen Daten, die über den Nutzer dauerhaft oder temporär gespeichert werden.
\begin{itemize}[noitemsep]
    \item Nutzername
    \item verschlüsseltes Passwort
    \item E-Mail
    \item API-Requests %[TODO: warum]
    \item Eingereichte Jobs (so lange, wie in der Konfigurationsdatei definiert)
    \item Spalten, die der Nutzer in der \hyperref[pages:job-table]{Job-Tabelle} ausgewählt hat
\end{itemize}



\subsection{Job-Daten}
\begin{itemize}[noitemsep]
    \item Job-Konfiguration
    \item Job-Beschreibung
    \item Ergebnis des Jobs
    \item Job-ID
    \item Zughöriger Nutzer
    \item Status des Jobs
\end{itemize}

\subsection{Mallob-Daten}
\begin{itemize}
    \item Relevante Inhalte des Mallob-Outputstream (so lange, wie in der Konfigurationsdatei definiert.)
\end{itemize}

\newpage
\section{Systemmodell}

\subsection{Systemdiagramm}

\begin{figure}[H]
    \centering
    \includegraphics[width=\textwidth]{images-interface/Diagramme/Systemdiagramm3.jpg} \\
    \caption{Model-View-Controller}
\end{figure}
Das System kann semantisch in 3 Komponenten gegliedert werden; View, Controller sowie Model. \\
Die Aufgabe des \textbf{Views}, bzw. des Web-Interfaces ist es dabei eine interaktive, grafische Benutzeroberfläche für die Interaktion mit Mallob bereitzustellen. \\
Der \textbf{Controller}, bzw. API sowie Backend (Anfragen gehen über API an das Backend) sind für die Kommunikation zwischen Datenbank, Mallob und Web-Interface, bzw. Nutzer zuständig. Das Mallob-Interface ist im Backend implementiert und für die Kommunikation zwischen der Backend-Logik und einer laufenden Mallob-Instanz zuständig.\\
Im \textbf{Model} hält eine Datenbank die \hyperref[PD]{Daten des Systems} und Mallob ist Mallob. \\
\textbf{Extern} liegt die laufende Mallob-Instanz. Mallob ist nicht Teil unseres Systems. Unser System kommuniziert lediglich mit Mallob.

\pagebreak

\subsection{Anwendungsfalldiagramme}

\begin{figure}[H]
    \centering    
    \includegraphics[width=\textwidth]{images-interface/Diagramme/Login_register_3_screenshot.jpg} 
    \caption{Registrierung von Nutzern}
\end{figure}


Registriert sich ein Nutzer, so wird seine Registrierunganfrage von einem Administrator verifiziert. Die Datenbank speichert die \hyperref[PD:Nutzerdaten]{Nutzerdaten}.\\
Möchte sich ein registrierter Nutzer im Web-Interface anmelden, so werden die Nutzerdaten aus der Datenbank geholt und mit den eingegebenen Daten verglichen. \\
Stellt ein registrierter Nutzer eine Anfrage über API mit seinem Token, wird auch hier der Token aus der Datenbank geholt und verglichen.
\begin{figure}[H]
    \centering
    \includegraphics[width=\textwidth]{images-interface/Diagramme/Request_authntification_screenshot.jpg}
    \caption{Authentifizierung von API-Anfragen}
\end{figure}
Jede Anfrage an die API muss authentifiziert werden. Dies geschieht, indem der Token, der mit der Anfrage geschickt wurde, mit dem in der Datenbank gespeichertem Token abgeglichen wird. Wurde eine Anfrage verifiziert, also sichergestellt, das der Nutzer diese auch ausführen darf, so wird sie weiterverarbeitet.\\
Jede Anfrage an die API wird für eine gewisse Zeit gespeichert, um sie später noch einmal referenzieren oder einsehen zu können.

\pagebreak

https://www.overleaf.com/project
\begin{figure}[H]
    \centering
    \includegraphics[width=\textwidth]{images-interface/Diagramme/Submit-abort-view-screenshot.jpg}
    \caption{Einreichen, Abbrechen und Einsehen von Jobs}
\end{figure}

Reicht ein Nutzer ein Job ein, so wird dieser (je nach Konfiguration) von Mallob bearbeitet. Mallob gibt dabei Rückmeldung über den Job-Status, also entweder das Ergebnis des Jobs, oder ob ein Fehler aufgetreten ist. Dieser Status wird in der Datenbank gespeichert. \\
Ein Nutzer kann die Bearbeitung eines Jobs, der gerade von Mallob berechnet wird, abbrechen. Die berechneten Zwischenergebnisse werden dann gespeichert und können später abgerufen werden. \\
Möchte ein Nutzer einen Job einsehen, so ist es wichtig das dieser bereits eingereicht wurde. Um ein Job einsehen zu können werden zunächst die angeforderten Job-Daten aus der Datenbank geholt und dann vorbereitet an den Nutzer geschickt.



%\begin{center}
%    \includegraphics[scale=0.6]{images-interface/Request_authntification_screenshot.jpg} \\
%    Anwendungsfalldiagramm 2: Authentifizierung von API-Requests
%\end{center}


\begin{figure}[H]
    \centering
    \includegraphics[width=\textwidth]{images-interface/Diagramme/visualisierungsdaten_anwendungsfaelle.jpg}
    \caption{Mallob-Visualisierungsdaten abrufen}
\end{figure}
Möchte ein Nutzer Mallob-Visualisierungsdaten abrufen, so holt das System bereits vergangene Events aus der Datenbank um hieraus die Visualisierung vorzubereiten. Des weiteren wird live der Outputstream der Logdatei von Mallob ausgelesen, um Echtzeit-Visualisierung für den Nutzer anzubieten. 


\pagebreak

\subsection{Aktivitätsdiagramme}
\begin{figure}[H]
    \centering
    \includegraphics[width=\textwidth]{images-interface/Anmelden_Aktivitaetsdiagramm.pdf}
    \caption{Anmeldung im System über das Web-Interface}
    \label{fig:login_activity}
    
\end{figure}

\begin{figure}[H]
    \centering
    \includegraphics[width=\textwidth]{images-interface/Job_einreichen_Aktivitaetsdiagramm.pdf}
    \caption{Einreichen eines Jobs über das Web-Interface}
\end{figure}

\begin{figure}[H]
    \centering
    \includegraphics[width=\textwidth]{images-interface/get_infos.pdf}
    \caption{Einsehen von Job-Informationen}
\end{figure}
\newpage
\section{Produktleistungen}

%%
%ProduktleistungenSofern 
%an einzelne Funktionen des Programms besondere Anforderungen in Bezug auf die Zeit oder die Genauigkeit gestellt werden, sollten diese in diesem Kapitel dargestellt werden. Dabei sollten Sie prüfen, ob die zu erbringenden Leistungen mit den in Punkt 5 genannten Angaben  realisierbar sind.
%

\begin{itemize}[noitemsep]
    \item[P100] Die maximale Anzahl der \hyperref[B:Jobs]{Jobs} ist begrenzt.
    
    \item[P110] Die maximale Anzahl \hyperref[B:Jobs]{Jobs}, die ein Nutzer gleichzeitig in Bearbeitung haben kann, ist beschränkt.
    
    \item[P120] Die Zeit, die benötigt wird, um einen beliebigen Zeitpunkt in der Visualisierung darzustellen, muss linear in der Anzahl der zu ladenden \hyperref[B:Event]{Events} sein.
    
    \item[P130] (Wunschbedingung)  Die Zeit, die benötigt wird, um einen beliebigen Zeitpunkt in der Visualisierung darzustellen, muss konstant sein, unabhängig vom gewählten Zeitpunkt.
    
    \item[P140] Das \gls{Web-Interface} ist auch auf kleineren Bildschirmen, wie etwa einem Handy-Bildschirm, nutzbar.
    
    \item[P150] Die \gls{Konfigurationsdatei} wird immer nur beim Systemstart eingelesen, etwaige Änderungen werden also erst mit einem Neustart des Systems wirksam.
    
    \item[P160] Beim Einreichen eines \hyperref[B:Jobs]{Jobs} im Interface erfolgt schon im Frontend eine Kontrolle der Syntax, welche den Nutzer momentan über Fehler in der Eingabe informiert.
    
    \item[P170] Nutzernamen sind eindeutig und bestehen aus 4 bis 25 Zeichen.

    \item[P180] Passwörter müssen mindestens 8-stellig sein.
    
    \item[P190] Die gespeicherten \hyperref[B:Jobs]{Jobs} werden automatisch nach einem spezifizierten Zeitraum gelöscht.

    \item[P200] Die Größe der \hyperref[B:Job-Beschreibung]{Job-Beschreibung}, die man im Web-Interface eingeben kann, ist beschränkt.

    
    \item[P210] Jedes \gls{Nutzerkonto} besitzt nach Registrierung die gleiche Priorität. Diese kann vom kann vom \gls{System-Administrator} geändert werden.
    
    
    \item[P220] Der \gls{Nutzer} erhält auf jede \gls{API}-Anfrage außer \hyperref[FA:API:Andauernde Abfrage des Ergebnisses eines Jobs]{F1110} unmittelbar eine  Antwort.
    
    \item[P230] Muss in der \hyperref[pages:visualization]{Visualisierung} zu viel angezeigt werden, so wird die Qualität herabgestuft, um weiterhin eine performante Darstellung zu ermöglichen. Dies geschieht beispielsweise durch das weglassen von Verbindungen zwischen den Prozessen.
    
    \item[P240] Daten über Jobs, die nicht dem angemeldetem Nutzer gehören, werden stehts pseudomynisiert ausgegeben und dargestellt. Ist ein Administrator  angemeldet, so werden die Daten nicht pseudomynisiert.
    
    \item[P250] Es ist möglich, die Job-Seite direkt über eine passende \gls{URL} aufzurufen.
    
    \item[P260] Die maximale Geschwindigkeit der Visualisierung ist das zweihundertfache.

    
\end{itemize}
\newpage
\section{Benutzeroberfläche}
%[TODO: remove this link] https://tex.stackexchange.com/questions/442077/is-it-possible-to-use-svg-images-with-overleaf

%[TODO: Arrange the images correctly, probably see here https://de.overleaf.com/learn/latex/Inserting_Images]
\subsection{Bilder des GUI-Entwurfs}
\begin{figure}[H]
    \centering

        \includegraphics[width=\textwidth]{images-interface/Loginv1.png}
        \caption{Anmelde-Maske}
        \label{fig:login}
   
        \includegraphics[width=\textwidth]{images-interface/Job-Viewv1.png}
        \caption{Auftrags-Übersicht}
        \label{fig:viewjobs}
  
\end{figure}


\begin{figure}[H]
    \centering
    \includegraphics[width=\textwidth]{images-interface/Submit-Filev1.png}
    \caption{Maske zum Hinzufügen neuer Aufträge}
    \label{fig:addjobs}
     \includegraphics[width=\textwidth]{images-interface/overviewv1.png}
    \caption{Visualisierung des Systems}
    \label{fig:visualsn}
\end{figure}


\subsection{Funktionen}
\begin{itemize}
    \item /B010/ Es sind zwei Sichten zu unterscheiden: die des Admins, die des Benutzers. 
    \item /B020/ Benutzer können Funktionen F10, F20, F30 jederzeit nach dem Einloggen aufrufen.
    \item /B030/ Jeder Benutzer fängt auf der Login/Register-Seite an.
    \item /B040/ Sobald der Benutzer eingeloggt wird, sieht er die Startseite.
    \item /B050/ Jeder Nutzer kann seine Jobs anhand der Aufwand auf das Kern unterscheiden (z.B Farbe, Größe usw.)
    \item /B060/ Admins können alle Funktionen, die die Benutzer können.
    \item /B070/ Unterschiedliche Benutzer und ihre Befugnisse sollen entsprechend behandelt werden (nicht-funktionale Anforderung?).
    \item /B080/ Die Bedienungsoberfläche ist auf Mausbedienung auszulegen; eine Bedienung ohne Maus muss aber auch möglich sein. ([TODO] Bediengung ohne Maus auf jeden Fall wunschkriterium oder vlt sogar gar nicth, auf jeden Fall mal fragen, hört sich kompliziert und nicht wirklich notwendig an)
\end{itemize}
\newpage
\section{Testfälle und Testszenarien}
Für jede Funktion existiert mindestens ein atomarer Testfall. Neben jedem Testfall wird aufgelistet, welche Funktion dieser abdeckt. Die Testfälle werden, wie die korrespondierenden Anforderungen, in API-Testfälle, Web-Interface-Testfälle und Visualisierung-Testfälle unterschieden.

\subsection{API-Testfälle}


\begin{itemize}

    \item[T1010] \textbf{Einen neuen Nutzer registrieren.} (F1000)
    
    \item[T1011] \textbf{Einen Nutzer mit Administrationsrechten registrieren} (F1000)
    
    \item[T1020] \textbf{Einen Nutzer authentifizieren.} (F1010)
    
    \item[T1021] \textbf{Einen Nutzer mit Administrationsrechten authentifizieren.} (F1010)
    
    \item[T1022] \textbf{Einen Nutzer mit fehlerhaften Zugangsdaten authentifizieren.} (F1010)
    
    \item[T1030] \textbf{Einen Job mit seperater Job-Beschreibungs-Datei einreichen.} (F1020)
    
    \item[T1031] \textbf{Einen Job mit enthaltener Job-Beschreibung einreichen.} (F1020)
    
    \item[T1032] \textbf{Einen Job mit einem Link, der auf eine Job-Beschreibungs-Datei verweist, einreichen.} (F1020)
    
    \item[T1033] \textbf{Einen bereits eingereichten aber abgebrochenen Job mit einer Referenz auf Diesen erneut einreichen.} (F1020)
    
    \item[T1034] \textbf{Einen fehlerhaften Job einreichen.} (F1020)
    
    \item[T1040] \textbf{Einen eingereichten Job abbrechen.} (F1030)
    
    \item[T1060] \textbf{Mallob starten.} (F1050)
    
    \item[T1070] \textbf{Mallob stoppen.} (F1060)
    
    \item[T1080] \textbf{Mallob neustarten.} (F1070)
    
    \item[T1090] \textbf{Den Status eines inkrementellen Jobs zurückgeben.} (F1080)
    
    \item[T1100] \textbf{Information über einen Job abfragen.} (F1090)
    
    \item[T1101] \textbf{Informationen über mehrere Jobs abfragen.} (F1090)
    
    \item[T1110] \textbf{Ergebnis von einem bearbeiteten Job ausgeben.} (F1100)
    
    \item[T1111] \textbf{Ergebnisse von mehreren bearbeiteten Jobs ausgeben.} (F1100)
    
    \item[T1120] \textbf{Job-Beschreibung von einem eingereichten Job ausgeben.} (F1110)
    
    \item[T1121] \textbf{Job-Beschreibungen von mehreren eingereichten Job ausgeben.} (F1110)
    
    \item[T1130] \textbf{Informationen über Mallob durch einen Admin abfragen.} (F1120)
    
    \item[T1140] \textbf{Einen Ereignis-Stream von Mallob ausgeben.} (F1130)
    
    \item[T1150] \textbf{Priorität eines Nutzers durch einen Admin ändern.} (F1140)
    
    \item[T1160] \textbf{Einstellungen abrufen.} (F1160)
    
    \item[T1170] \textbf{Einstellungen aktualisieren.} (F1150)
    
    \item[T1180] \textbf{Plugin neu einlesen.} (F1150)
    

\end{itemize}

\subsection{Web-Interface Testfälle}

\begin{itemize}
    \item[T2010] \textbf{Nutzer anmelden.} (F2000)
    
    \item[T2011] \textbf{Admin anmelden.} (F2000)
    
    \item[T2020] \textbf{Nutzer registrieren.} (F2010)
    
    \item[T2030] \textbf{Job mit zugehöriger Job-Beschreibung über das Eingabefeld einreichen.} (F2020)
    
    \item[T2031] \textbf{Job mit Upload der Job-Beschreibungs-Datei einreichen.} (F2020)
    
    \item[T2032] \textbf{Job mit Angabe einer URL einreichen.} (F2020)
    
    \item[T2040] \textbf{Einen Job über die Job-Tabelle abbrechen.} (F2030)
    
    \item[T2041] \textbf{Einen Job über die Job-Seite abbrechen.} (F2040)
    
    \item[T2050] \textbf{Mehrere Jobs abbrechen.} (F2040)
    
    \item[T2060] \textbf{Einen abgebrochenen Job über die Job-Tabelle neustarten.} (F2050)
    
    \item[T2061] \textbf{Einen abgebrochenen Job über die Job-Seite neustarten.} (F2050)
    
    \item[T2070] \textbf{Ergebnis eines Jobs über die Job-Tabelle herunterladen.} (F2060)
    
    \item[T2071] \textbf{Ergebnis eines Jobs über die Job-Seite herunterladen.} (F2060)
    
    \item[T2080] \textbf{Ergebnis mehrerer Jobs herunterladen.} (F2070)
    
    \item[T2090] \textbf{Vorläufiges Konto verifizieren.} (F2090)
    
    \item[T2100] \textbf{Mallob starten.} (F2110)
    
    \item[T2101] \textbf{Mallob stoppen.} (F2110)
    
    \item[T2102] \textbf{Mallob neustarten.} (F2110)
    
    \item[T2110] \textbf{Einstellungen ändern.} (F2120)
    
    \item[T2120] \textbf{Plugin einlesen.} (F2120)
    
    \item[T2130] \textbf{Job-Information als eigene Spalte in der Job-Tabelle anzeigen.} (F2130)
    
    \item[T2131] \textbf{Job-Information über aufgeklapptem Fenster anzeigen.} (F2130)
    
    \item[T2132] \textbf{Job-Information über Job-Seite anzeigen.} (F2130)
    
    \item[T2140] \textbf{Spalten in der Job-Tabelle auswählen.} (F2140)
    
    \item[T2150] \textbf{Job-Tabelle nach Attributen sortieren.} (F2150)
    
    \item[T2160] \textbf{Plugins anzeigen.} (F2160)
    
    \item[T2170] \textbf{Nutzer abmelden.} (F2170)
    
\end{itemize}

\subsection{Visualisierung Testfälle}

\begin{itemize}
    \item[T3010] \textbf{Einem Nutzer den Systemzustand von Mallob anzeigen.} (F3000)
    
    \item[T3011] \textbf{Einem Admin den Systemzustand von Mallob anzeigen.} (F3000)
    
    \item[T3020] \textbf{Binärbaum für einen Job anzeigen.} (F3010)
    
    \item[T3030] \textbf{Visualisierung pausieren.} (F3020)
    
    \item[T3040] \textbf{Visualisierung starten.} (F3030)
    
    \item[T3050] \textbf{Geschwindigkeit der Visualisierung einstellen.} (F3040)
    
    \item[T3060] \textbf{Beliebigen Zeitpunkt der Visualisierung auswählen.} (F3050)
    
\end{itemize}


\subsection{Testszenarien}

\subsubsection{Testszenario 1: Einen Job einreichen - API}
Ein neuer Nutzer möchte sich registrieren/authentifizieren und darauf hin einen Job einreichen. Dieser wird aber abgebrochen und anschließend erneut eingereicht. 

\begin{enumerate}
    \item 
    \begin{enumerate}
        \item \textbf{T1010} Einen neuen Nutzer registrieren.
        
        \item \textbf{T1020} Einen Nutzer authentifizieren.
    \end{enumerate}
    
    \item \textbf{T1034} Einen fehlerhaften Job einreichen.
    
    \item 2. Punkt beliebig oft wiederholen.
    
    \item 
    \begin{enumerate}
        \item \textbf{T1030} Einen Job mit seperater Job-Beschreibungs-Datei einreichen. 
        
        \item \textbf{T1031} Einen Job mit enthaltener Job-Beschreibung einreichen.
        
        \item \textbf{T1032} Einen Job mit einem Link, der auf eine Job-Beschreibungs-Datei verweist, einreichen.
    \end{enumerate}
    
    \item \textbf{T1100} Information über einen Job abfragen.
    
    \item \textbf{T1120} Job-Beschreibung von einem eingereichten Job ausgeben.
    
    \item \textbf{T1040} Einen eingereichten Job abbrechen.
    
    \item \textbf{T1033} Einen bereits eingereichten aber abgebrochenen Job mit einer Referenz auf Diesen erneut einreichen.
    
    \item Job wurde fertig bearbeitet.
    
    \item \textbf{T1110} Ergebnis von einem bearbeiteten Job ausgeben. 
    
\end{enumerate}

\subsubsection{Testszenario 2: Mehr als einen Job einreichen - API}
Ein neuer Nutzer möchte sich registrieren/authentifizieren und mehr als einen Job einreichen.

\begin{enumerate}
    \item 
    \begin{enumerate}
        \item \textbf{T1010} Einen neuen Nutzer registrieren.
        
        \item \textbf{T1020} Einen Nutzer authentifizieren.
    \end{enumerate}
    
    \item \textbf{T1034} Einen fehlerhaften Job einreichen.
    
    \item 2. Punkt beliebig oft wiederholen.
    
    \item 
    \begin{enumerate}
        \item \textbf{T1030} Einen Job mit seperater Job-Beschreibungs-Datei einreichen. 
        
        \item \textbf{T1031} Einen Job mit enthaltener Job-Beschreibung einreichen.
        
        \item \textbf{T1032} Einen Job mit einem Link, der auf eine Job-Beschreibungs-Datei verweist, einreichen.
    \end{enumerate}
    
    \item 4.Punkt beliebig oft bis zur Obergrenze einzureichender Jobs pro Nutzer wiederholen
    
    \item \textbf{T1101} Informationen über mehrere Jobs abfragen. 
    
    \item \textbf{T1121} Job-Beschreibungen von mehreren eingereichten Job ausgeben.
    
    \item Mindestens 2 Jobs wurden fertig bearbeitet.
    
    \item \textbf{T1111} Ergebnisse von mehreren bearbeiteten Jobs ausgeben. 
\end{enumerate}

\subsubsection{Testszenario 3: Zugriff eines Admins - API}
Ein bereits im System verifizierter Admin möchte sich authentifizieren bzw. ein neuer Admin soll registriert werden und darauf hin einige Funktionen ausführen, die nur für einen Admin vorgesehen sind.

\begin{enumerate}
    \item
    \begin{enumerate}
        \item \textbf{T1011} Einen Nutzer mit Administrationsrechten registrieren.
        
        \item \textbf{T1021} Einen Nutzer mit Administrationsrechten authentifizieren. 
    \end{enumerate}
    
    \item \textbf{T1060} Mallob starten.
    
    \item \textbf{T1080} Mallob neustarten.
    
    \item \textbf{T1130} Informationen über Mallob durch einen Admin abfragen. 
    
    \item \textbf{T1150} Priorität eines Nutzers durch einen Admin ändern. 
    
    \item \textbf{T1160} Einstellungen abrufen.
    
    \item \textbf{T1170} Einstellungen aktualisieren. 
    
    \item \textbf{T1180} Plugin neu einlesen.
    
    \item \textbf{T1070} Mallob stoppen. 
\end{enumerate}

\subsubsection{Testszenario 4: Aufrufen des Web-Interface und Einreichen eines Jobs - Web-Interface}
Ein Nutzer gelangt über die Anmelde-Maske auf das Web-Interface, reicht darüber einen Job ein und greift auf verschiedene Funktionen, die das Web-Interface in Bezug auf Jobs anbietet, zu.

\begin{enumerate}
    \item 
    \begin{enumerate}
        \item \textbf{T2010} Nutzer anmelden.
        
        \item \textbf{T2020} Nutzer registrieren.
    \end{enumerate}
    
    \item 
    \begin{enumerate}
        \item \textbf{T2030} Job mit zugehöriger Job-Beschreibung über das Eingabefeld einreichen.
        
        \item \textbf{T2031} Job mit Upload der Job-Beschreibungs-Datei einreichen.
        
        \item \textbf{T2032} Job mit Angabe einer URL einreichen.
    \end{enumerate}
    
    \item 
    \begin{enumerate}
        \item \textbf{T2040} Einen Job über die Job-Tabelle abbrechen. 
        
        \item \textbf{T2041} Einen Job über die Job-Seite abbrechen.
    \end{enumerate}
    
    \item
    \begin{enumerate}
        \item \textbf{T2060} Einen abgebrochenen Job über die Job-Tabelle neustarten.
        
        \item \textbf{T2061} Einen abgebrochenen Job über die Job-Seite neustarten. 
    \end{enumerate}
    

    \item \textbf{T2160} Spalten in der Job-Tabelle auswählen. 
    
    \item
    \begin{enumerate}
        \item \textbf{T2150} Job-Information als eigene Spalte in der Job-Tabelle anzeigen.
        
        \item \textbf{T2151} Job-Information über aufgeklapptem Fenster anzeigen.
         
        \item \textbf{T2152} Job-Information über Job-Seite anzeigen. 
    \end{enumerate}
    
    \item \textbf{T2170} Job-Tabelle nach Attributen sortieren.
    
    \item
    \begin{enumerate}
        \item \textbf{T2070} Ergebnis eines Jobs über die Job-Tabelle herunterladen.
        
        \item \textbf{T2071} Ergebnis eines Jobs über die Job-Seite herunterladen. 
    \end{enumerate}
    
    \item \textbf{T2190} Nutzer abmelden.

\end{enumerate}

\subsubsection{Testszenario 5: Aufrufen des Web-Interface und Einreichen mehrerer Jobs - Web-Interface}
Ein Nutzer gelangt über die Anmelde-Maske auf das Web-Interface und reicht mehr als einen Job ein.

\begin{enumerate}
     \item 
     \begin{enumerate}
        \item \textbf{T2010} Nutzer anmelden.
        
        \item \textbf{T2020} Nutzer registrieren.
     \end{enumerate}
     
     \item 
     \begin{enumerate}
        \item \textbf{T2030} Job mit zugehöriger Job-Beschreibung über das Eingabefeld einreichen.
        
        \item \textbf{T2031} Job mit Upload der Job-Beschreibungs-Datei einreichen.
        
        \item \textbf{T2032} Job mit Angabe einer URL einreichen.
     \end{enumerate}
     
     \item 2.Punkt mindestens einmal wiederholen.
     
     \item Mindestens 2 Jobs sind fertig bearbeitet
     
     \item \textbf{T2080} Ergebnis mehrerer Jobs herunterladen.
     
     \item \textbf{T2190} Nutzer abmelden.
\end{enumerate}

\subsubsection{Testszenario 6: Einreichen und Abbrechen mehrerer Jobs - Web-Interface}
Ein Nutzer reicht über das Web-Interface mehr als einen Job ein und bricht diese wieder ab.

\begin{enumerate}
     \item 
     \begin{enumerate}
        \item \textbf{T2010} Nutzer anmelden.
        
        \item \textbf{T2020} Nutzer registrieren.
     \end{enumerate}
     
     \item 
     \begin{enumerate}
        \item \textbf{T2030} Job mit zugehöriger Job-Beschreibung über das Eingabefeld einreichen.
        
        \item \textbf{T2031} Job mit Upload der Job-Beschreibungs-Datei einreichen.
        
        \item \textbf{T2032} Job mit Angabe einer URL einreichen.
     \end{enumerate}
     
     \item 2.Punkt mindestens einmal wiederholen.
     
     \item \textbf{T2050} Mehrere Jobs abbrechen.
     
     \item \textbf{T2190} Nutzer abmelden.
\end{enumerate}

\subsubsection{Testszenario 7: Aufrufen des Web-Interface durch einen Admin - Web-Interface}
Ein Admin ruft das Web-Interface auf, meldet sich an, und greift auf einige Funktionen zu, die nur für einen Admin bestimmt sind.

\begin{enumerate}
    \item \textbf{T2011} Admin anmelden.
    
    \item \textbf{T2100} Mallob starten.
    
    \item \textbf{T2102} Mallob neustarten.
    
    \item \textbf{T2110} Einstellungen ändern.
    
    \item \textbf{T2160} Plugins anzeigen.
    
    \item \textbf{T2120} Plugin einlesen.

    \item \textbf{T2101} Mallob beenden.
    
    \item \textbf{T2170} Nutzer abmelden.
\end{enumerate}

\subsubsection{Testszenario 8: Einsehen der Visualisierung durch einen Nutzer - Visualisierung}
Ein Nutzer greift auf die verschiedenen Visualisierungen zu einem Job und dem Gesamtzustand von Mallob, die das Web-Interface anbietet, zu. Die Vorbedingung für dieses Szenario ist die, dass der Nutzer bereits mindestens einen Job eingereicht hat.

\begin{enumerate}
    \item \textbf{T2010} Nutzer anmelden.
    
    \item \textbf{T3010} Einem Nutzer den Systemzustand von Mallob anzeigen. 
    
    \item \textbf{T3020} Binärbaum für einen Job anzeigen. 
    
    \item \textbf{T3060} Beliebigen Zeitpunkt der Visualisierung auswählen.
    
    \item \textbf{T3050} Geschwindigkeit der Visualisierung einstellen.
    
    \item \textbf{T3040} Visualisierung starten. 
    
    \item \textbf{T3030} Visualisierung pausieren.

    \item \textbf{T3040} Visualisierung starten. 
\end{enumerate}

\subsubsection{Testszenario 9:Einsehen der Visualisierung durch einen Admin - Visualisierung}
Ein Admin greift auf die Visualisierung des Gesamtzustands von Mallob zu. Hierbei wird keine Vorbedingung benötigt, da der Admin jeden im System befindlichen Job mit allen Informationen einsehen kann.

\begin{enumerate}
    \item \textbf{T2011} Admin anmelden.
    
    \item \textbf{T3011} Einem Admin den Systemzustand von Mallob anzeigen.
    
    \item \textbf{T3060} Beliebigen Zeitpunkt der Visualisierung auswählen.
    
    \item \textbf{T3050} Geschwindigkeit der Visualisierung einstellen.
    
    \item \textbf{T3040} Visualisierung starten. 
    
    \item \textbf{T3030} Visualisierung pausieren.

    \item \textbf{T3040} Visualisierung starten.
\end{enumerate}



	
\newpage
\section{Qualitätsbestimmung}
\begin{itemize}    
    \item Das System ist leicht analysierbar/nachvollziehbar und erweiterbar gebildet.
    \item Das Web-Interface passt sich dynamisch der Bildschirmgröße an und bleibt dabei benutzbar.
    \item Die Benutzer-Oberfläche bietet eine möglichst einfache und intuitive Schnittstelle an, sowohl zum Registrieren/Anmelden, als auch zur Interaktion und Visualisierung von Jobs.
    \item Die Verwendung des Model-View-Controller Prinzips sichert Wart- und Wiederverwendbarkeit.
    \item Die Testbarkeit ist von der hohen Testüberdeckung garantiert.
    % vielleicht irgendwie Realisierbarkeit, Vollständigkeit, Eindeutigkeit garantieren?
\end{itemize}
\newpage
\section{Entwicklungs-Umgebung}
        \begin{itemize}[noitemsep]
            \item \glspl{Betriebssystem} 
                \begin{itemize}[noitemsep]
                    \item Windows 10
                    \item Arch Linux
                \end{itemize}
            \item Modellierung %[TODO] 
            \item Entwicklung %[TODO]
            \begin{itemize}[noitemsep]
                \item Notepad ++
                \item IntelliJ
            \end{itemize}
            \item \gls{Versionsverwaltung}
                \begin{itemize}[noitemsep]
                    \item Git
                    \item Github
                \end{itemize}
            \item Sonstige Software
                \begin{itemize}[noitemsep]
                    \item \LaTeX \hspace{0.1cm} (für das Pflichtenheft)
                    \item \href{https://de.overleaf.com}{Overleaf} (Schreiben des Pflichtenheftes)
                    \item \href{https://www.figma.com}{Figma} (GUI-Entwürfe)
                    \item \href{https://online.visual-paradigm.com/}{Infographic Maker} (Aktivitätsdiagramme)
                    \item \href{https://app.diagrams.net/}{} (Anwendungsfalldiagramme)
                \end{itemize}
        \end{itemize}
\newpage




%---------------------------Kommunikation mit Mallob------    
%\section{Kommunikation mit Mallob}
%
%\begin{itemize}
%    \item Annahme für uns : Mallob läuft auf selben Dateisystem wie Backend 
%    \item JSON und Jobbeschreibung werden in einem Verzeichnis im Dateisystem abgelegt und von dort aus von Mallob verarbeitet
%    \item Der Output wird von Mallob ebenfalls las Datei in einem Verzeichnis ausgegeben
%    \item JSON und Jobbschreibung können unterschiedliche Dateien sein (siehe Einreichen der Jobbeschreibung, API)
%\end{itemize}









%------------------------Glossar
\newpage
%\glsaddall
\setglossarystyle{altlist}
%\printnoidxglossaries
\printglossary
\end{document}
