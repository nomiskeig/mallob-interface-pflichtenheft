%\section{Zielbestimmung}
%\subsection{Musskriterien}
%\subsubsection{Web-Interface}
%\begin{itemize}
%    \item Das System verfügt über ein \gls{Web-Interface}.
%    \item Beim ersten Aufruf des \glslink{Web-Interface}{Web-Interfaces} wird eine Anmelde-Maske angezeigt.
%    \item In der Anmelde-Maske der Nutzer auch zu einer Maske für die Registierung gelangen.
%    \item Ein \gls{Auftrag} kann über das \gls{Web-Interface} hinzugefügt werden. Dies geschieht über eine Eingabemaske.
%    \item Das Übergeben der Probleminstanz kann wahlweise direkt über ein Eingabefeld, hochladen der CNF-Datei oder Angabe einer URL erfolgen.
%    \item Das \gls{Web-Interface} besitzt eine Visualierung des Systemzustandes. Hier wird standardmäßig der aktuelle Zustand visualisiert,  es kann aber auch die Historie betrachtet werden.
%    \item Admins haben Zugriff auf eine exklusive Ansicht, mit der die Instanz von Mallob verwaltet werden kann.
%
%\subsection{Wunschkriterien}
%\subsubsection{Web-Interface}
%\begin{itemize}
%    \item Der Benutzer kann die Eingabe des Problems mit einem graphischem Editor vornehmen.
%    \item Ein Admin kann die Instanz von Mallob starten, beenden und neustarten. Hier gibt es auch die Möglichkeit, entsprechende Parameter einzugeben.
%\end{itemize}
%\end{itemize}

\section{Zielbestimmung}
Dieses Programm ermöglicht es Nutzern, einfacher mit \href{https://github.com/domschrei/mallob}{Mallob} zu interagieren. Diese Interaktion kann auf zwei Wege erfolgen:
Einerseits wird eine REST-API angeboten für die Interaktion angeboten. Anderseits gibt es aber auch ein benutzerfreundliches Web-Interface, welches die API nutzt und somit eine graphische Oberfläche für diese darstellt.
\subsection{Musskriterien}
    \subsubsection{API}
        \begin{itemize}
            \item Unterscheiden zwischen Benutzern und Administratoren
            \item Authentifizieren von Benutzern
            \item Registrierung von neuen Benutzern
            \item Verwalten von Jobs
                \begin{itemize}
                    \item Jobs hinzufügen
                    \item Jobs abbrechen
                    \item Aktuellen Status eines Jobs abfragen
                    \item Auflisten der eigenen Jobs
                \end{itemize}
            \item Abfragen von Jobs
            \item Bereitstellung eines Ereignis-Streams von Mallob zum Erhalten von Updates
            \item Erhalten von Informationen zu aufgetretenen Fehlern
        \end{itemize}
    \subsubsection{Web-Interface}
        \begin{itemize}
            \item Authentifizieren von Nutzern
            \item Registrierung von neuen Nutzern
            \item Verwalten von Jobs
                 \begin{itemize}
                    \item Jobs hinzufügen
                    \item Jobs abbrechen
                    \item Aktuellen Status des Jobs betrachten
                    \item Auflisten der eigenen Jobs
                \end{itemize}
            \item Einsehen der eigenen Aufträge und derer Ergebnisse
            \item Herunterladen der Ergebnisse der Jobs
            \item Anzeigen von aufgetretenen Fehlern
            \item Verifizierungsmöglichkeit von neuen Nutzern für Administratoren 
            
                      
        \end{itemize}
    \subsubsection{Visualisierung}    
        \begin{itemize}
            \item Das Web-Interface enthält eine Visualisierung des System-Zustands
            \item Wenn die Live-Ansicht ausgewählt ist, wird die Visualisierung automatisch aktualisiert
            %\item Darstellung der PE als Punkte %TODO: Ranks? oder PEs? oder ganz anderes Word?
            %\item Färbung der Punkte entsprechend dem zugehörigen Job
            %\item Verbinden der Punkte entsprechend der Position im Binärbaum des Jobs
            %\item Dynamische Größe der Punkte und Verbindungen entsprechend der Position im Binärbaum des Jobs.
        \end{itemize}
    
    \subsubsection{System}
        \begin{itemize}
            \item Konfigurationsdatei für globale Einstellungen des Systems
        \end{itemize}
        
        
\subsection{Wunschkriterien}
    \begin{itemize}
        %\item Speichern von Jobdaten für Statistiken -> Irgendwie redundaten, da wir eh schon jobdaten speichern
        \item Möglichkeit für Administratoren, zu starten, beenden oder neustarten
        \item Sortieren der Job-Tabelle im Web—Interface
        \item Graphischer Editor zur Eingabe der Job-Beschreibung
        \item Diagnosemöglichkeit für Administratoren
        \item Nutzer benachrichtigen, falls Mallob abstürzen sollte
        \item Darstellen eines bestimmten vergangenen Zeitpunktes bei der Visualisierung
        \item Ändern der Wiedergabegeschwindigkeit bei der Visualisierung
        \item Einsehen von Details zu einem aktiven Job in der Visualisierung
        \item Darstellung des vollständigen Binärbaums des Jobs
        \item Schnittstelle für Plugins in Web—Interface
    \end{itemize}
    
\subsection{Abgrenzungskriterien}
    \begin{itemize}
        \item Es kann maximal ein Auftrag pro Anfrage hinzugefügt werden
        \item Korrektheit der Daten wird durch Mallob geprüft
        \item Es gibt keine Möglichkeit, falsche Daten in der API zu korrigieren, in diesem Fall muss die gleiche Anfrage mit korrekten Daten erneut erfolgen
    \end{itemize}
