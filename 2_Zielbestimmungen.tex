%\section{Zielbestimmung}
%\subsection{Musskriterien}
%\subsubsection{Web-Interface}
%\begin{itemize}
%    \item Das System verfügt über ein \gls{Web-Interface}.
%    \item Beim ersten Aufruf des \glslink{Web-Interface}{Web-Interfaces} wird eine Anmelde-Maske angezeigt.
%    \item In der Anmelde-Maske der Nutzer auch zu einer Maske für die Registierung gelangen.
%    \item Ein \gls{Auftrag} kann über das \gls{Web-Interface} hinzugefügt werden. Dies geschieht über eine Eingabemaske.
%    \item Das Übergeben der Probleminstanz kann wahlweise direkt über ein Eingabefeld, hochladen der CNF-Datei oder Angabe einer URL erfolgen.
%    \item Das \gls{Web-Interface} besitzt eine Visualierung des Systemzustandes. Hier wird standardmäßig der aktuelle Zustand visualisiert,  es kann aber auch die Historie betrachtet werden.
%    \item Admins haben Zugriff auf eine exklusive Ansicht, mit der die Instanz von Mallob verwaltet werden kann.
%
%\subsection{Wunschkriterien}
%\subsubsection{Web-Interface}
%\begin{itemize}
%    \item Der Benutzer kann die Eingabe des Problems mit einem graphischem Editor vornehmen.
%    \item Ein Admin kann die Instanz von Mallob starten, beenden und neustarten. Hier gibt es auch die Möglichkeit, entsprechende Parameter einzugeben.
%\end{itemize}
%\end{itemize}

\section{Zielbestimmung}
Dieses Programm ermöglicht Nutzern, einfacher mit \href{https://github.com/domschrei/mallob}{Mallob} zu interagieren. Diese Interaktion kann auf zwei Wege erfolgen:
Einerseits wird eine REST-API angeboten für die Interaktion angeboten. Anderseits gibt es aber auch ein benutzerfreundliches Web-Interface, welches die API nutzt und somit eine graphische Oberfläche für diese darstellt.
\subsection{Musskriterien}
    \subsubsection{API}
        \begin{itemize}
            \item Authentifizieren von Benutzern
            \item Registrierung von neuen Benutzern
            \item Verwalten von Aufträgen
                \begin{itemize}
                    \item Auftrag hinzufügen
                    \item Auftrag abbrechen
                    \item Aktueller Status des Auftrags betrachten
                    \item Auflisten der eigenen Aufträge
                \end{itemize}
            \item Abfragen von Aufträgen
            \item Erhalten von Informationen zu Fehlern
        \end{itemize}
    \subsubsection{Web-Interface}
        \begin{itemize}
            \item Authentifizieren von Benutzern
            \item Registrierung von neuen Benutzern
            \item Verwalten von Aufträgen
                 \begin{itemize}
                    \item Auftrag hinzufügen
                    \item Auftrag abbrechen
                    \item Aktueller Status des Auftrags betrachten
                    \item Auflisten der eigenen Aufträge
                \end{itemize}
            \item Einsehen der eigenen Aufträge und derer Ergebnisse
            \item Anzeigen von aufgetretenen Fehlern
            \item Visualisierung des aktuellen Systemzustandes im Web-Interface
        \end{itemize}
        
        
\subsection{Wunschkriterien}
    \begin{itemize}
        \item Speichern von Auftragsdaten für Statistiken
        \item Möglichkeit für Admins, die Instanz von Mallob zu starten, beenden oder neustarten
        \item Sortieren der Auftragsliste im Web—Interface
        \item Graphischer Editor zur Eingabe der Job-Beschreibung
        \item Möglichkeit zur Diagnostik der Aufträge für Admins
        \item Nutzer benachrichtigen, falls Mallob abstürzen sollte
        \item Darstellen eines bestimmten vergangenen Zeitpunktes bei der Visualisierung
        \item Plugins [TODO: ?????]
    \end{itemize}
    
\subsection{Abgrenzungskriterien}
    \begin{itemize}
        \item Das Web-Interface wird nur zur Darstellung auf herkömmlichen Desktop-PCs konzipiert. [TODO: Hoffentlich, noch fragen]
        \item Es kann maximal ein Auftrag pro Anfrage hinzugefügt werden 
        \item Korrektheit der Daten wird durch Mallob geprüft
        \item Es gibt keine Möglichkeit, falsche Daten in der API zu korrigieren, in diesem Fall muss die gleiche Anfrage mit korrekten Daten erneut erfolgen.
    \end{itemize}
