%\section{Zielbestimmung}
%\subsection{Musskriterien}
%\subsubsection{Web-Interface}
%\begin{itemize}
%    \item Das System verfügt über ein \gls{Web-Interface}.
%    \item Beim ersten Aufruf des \glslink{Web-Interface}{Web-Interfaces} wird eine Anmelde-Maske angezeigt.
%    \item In der Anmelde-Maske der Nutzer auch zu einer Maske für die Registierung gelangen.
%    \item Ein \gls{Auftrag} kann über das \gls{Web-Interface} hinzugefügt werden. Dies geschieht über eine Eingabemaske.
%    \item Das Übergeben der Probleminstanz kann wahlweise direkt über ein Eingabefeld, hochladen der CNF-Datei oder Angabe einer URL erfolgen.
%    \item Das \gls{Web-Interface} besitzt eine Visualierung des Systemzustandes. Hier wird standardmäßig der aktuelle Zustand visualisiert,  es kann aber auch die Historie betrachtet werden.
%    \item Admins haben Zugriff auf eine exklusive Ansicht, mit der die Instanz von Mallob verwaltet werden kann.
%
%\subsection{Wunschkriterien}
%\subsubsection{Web-Interface}
%\begin{itemize}
%    \item Der Benutzer kann die Eingabe des Problems mit einem graphischem Editor vornehmen.
%    \item Ein Admin kann die Instanz von Mallob starten, beenden und neustarten. Hier gibt es auch die Möglichkeit, entsprechende Parameter einzugeben.
%\end{itemize}
%\end{itemize}

\section{Zielbestimmung}
\textit{Fallob} ermöglicht es Nutzern, einfacher mit \href{https://github.com/domschrei/mallob}{Mallob} zu interagieren. Diese Interaktion kann auf zwei Wege erfolgen:
Einerseits wird eine REST-API für die Interaktion angeboten. Anderseits gibt es aber auch ein benutzerfreundliches Web-Interface, welches die API nutzt und somit eine graphische Oberfläche für diese darstellt. Auf beide Wege können eigene Jobs verwaltet werden. Darüber hinaus bietet das Web-Interface auch eine Visualisierung des Systems.


\subsection{Musskriterien}
    \subsubsection{API}
        \begin{itemize}[noitemsep]
            \item Authentifizieren von Nutzern
            %\item Registrieren von neuen Nutzer
            \item Job einreichen
            \item Job abbrechen
            \item Informationen zu Jobs abfragen
            \item Ergebnis von Jobs abfragen
            \item Beschreibung von Jobs abfragen
            \item Bereitstellung eines Ereignis-Streams von Mallob zum Erhalten von Updates von Jobs
            \item Abfrage des Status der Mallob-Instanz
            %\item Erhalten von Informationen zu aufgetretenen Fehlern
        \end{itemize}
    \subsubsection{Web-Interface}
        \begin{itemize}[noitemsep]
            \item Authentifizieren von Nutzern
            %\item Registrierung von neuen Nutzern
            \item Tabellarische Übersicht  der Jobs
                \begin{itemize}[noitemsep]
                    \item Abbrechen mehrerer Jobs auf einmal
                    \item Herunterladen von mehreren Ergebnissen auf einmal
                \end{itemize}
            \item Job über Eingabemaske einreichen
                \begin{itemize}[noitemsep]
                    \item Einreichen der Job-Beschreibung über ein Eingabe-Feld
                    \item Einreichen der Job-Beschreibung durch Hochladen einer entsprechenden Datei
                \end{itemize}
            \item Job abbrechen
            \item Aktuellen Status des Jobs betrachten
            \item Job-Seite um Details von Jobs anzusehen
            \item Herunterladen der Ergebnisse der Jobs
            \item Anzeigen von aufgetretenen Fehlern bei Nutzern
            \item Anzeigen von Warnungen von Mallob im Administratoren-Bereich 
            %\item Verifizierungsmöglichkeit von neuen Nutzern für Administratoren 
            
                      
        \end{itemize}
    \subsubsection{Visualisierung}    
        \begin{itemize}[noitemsep]
            \item Das Web-Interface enthält eine Visualisierung des System-Zustands
                \begin{itemize}[noitemsep]
                    \item Darstellung der Prozesse als Punkte, ein Punkt je Prozess %TODO: Ranks? oder PEs? oder ganz anderes Word?
                    \item Färbung der Punkte entsprechend dem zugehörigen Job
                    \item Verbinden der Punkte entsprechend der Position im Binärbaum des Jobs
                    \item Dynamische Größe der Punkte und Verbindungen entsprechend der Position im Binärbaum des Jobs
                \end{itemize}
            \item Automatische Aktualisierung der Visualisierung, wenn die Live-Ansicht ausgewählt ist
            %\item Wenn die Live-Ansicht ausgewählt ist, wird die Visualisierung automatisch aktualisiert
            \item Darstellen eines bestimmten vergangenen Zeitpunktes
            \item Ändern der Wiedergabegeschwindigkeit
            \item Einsehen von Details zu einem aktiven Job in einem separatem Feld 
            %\item Darstellung der PE als Punkte %TODO: Ranks? oder PEs? oder ganz anderes Word?
            %\item Färbung der Punkte entsprechend dem zugehörigen Job
            %\item Verbinden der Punkte entsprechend der Position im Binärbaum des Jobs
            %\item Dynamische Größe der Punkte und Verbindungen entsprechend der Position im Binärbaum des Jobs.
        \end{itemize}
    
    \subsubsection{System}
        \begin{itemize}[noitemsep]
            \item Konfigurationsdatei für globale Einstellungen des Systems
        \end{itemize}
        
        
\subsection{Wunschkriterien}
    \begin{itemize}[noitemsep]
        %\item Speichern von Jobdaten für Statistiken -> Irgendwie redundaten, da wir eh schon jobdaten speichern
        \item Möglichkeit für Administratoren, Mallob zu starten, beenden oder neuzustarten
        \item Registrierung von Nutzern über die API
        \item Registrierung von Nutzern über das Web-Interface
        \item Sortieren der Job-Tabelle im Web-Interface
        \item Möglichkeit zum Neustart von abgebrochenen oder abgeschlossenen Jobs
        \item Eingabemaske für die Eingabe der Job-Beschreibung im Web-Interface
        \item Einreichen der Job-Beschreibung über eine URL, die auf die Job-Beschreibung zeigt
        \item Diagnosemöglichkeit für Administratoren im Web-Interface
        \item Nutzer benachrichtigen, falls Mallob abstürzen sollte
        \item Darstellung des vollständigen Binärbaums des Jobs in der Visualisierung
        \item Schnittstelle für Plugins in Web-Interface
        \item Entwicklung eines beispielhaften Plugins
    \end{itemize}
    
\subsection{Abgrenzungskriterien}
    \begin{itemize}[noitemsep]
        \item Es kann maximal ein Auftrag pro Anfrage hinzugefügt werden
        \item Korrektheit der Daten wird durch Mallob geprüft
        \item Es gibt keine Möglichkeit, falsche Daten in der API zu korrigieren, in diesem Fall muss die gleiche Anfrage mit korrekten Daten erneut erfolgen
    \end{itemize}
