\section{Zielbestimmung}
\subsection{Musskriterien}
\subsubsection{Web-Interface}
\begin{itemize}
    \item Das System verfügt über ein \gls{Web-Interface}.
    \item Beim ersten Aufruf des \glslink{Web-Interface}{Web-Interfaces} wird eine Anmelde-Maske angezeigt.
    \item In der Anmelde-Maske der Nutzer auch zu einer Maske für die Registierung gelangen.
    \item Ein \gls{Auftrag} kann über das \gls{Web-Interface} hinzugefügt werden. Dies geschieht über eine Eingabemaske.
    \item Das Übergeben der Probleminstanz kann wahlweise direkt über ein Eingabefeld, hochladen der CNF-Datei oder Angabe einer URL erfolgen.
    \item Das \gls{Web-Interface} besitzt eine Visualierung des Systemzustandes. Hier wird standardmäßig der aktuelle Zustand visualisiert,  es kann aber auch die Historie betrachtet werden.
    \item Admins haben Zugriff auf eine exklusive Ansicht, mit der die Instanz von Mallob verwaltet werden kann.

\subsection{Wunschkriterien}
\subsubsection{Web-Interface}
\begin{itemize}
    \item Der Benutzer kann die Eingabe des Problems mit einem graphischem Editor vornehmen.
    \item Ein Admin kann die Instanz von Mallob starten, beenden und neustarten. Hier gibt es auch die Möglichkeit, entsprechende Parameter einzugeben.
\end{itemize}
\end{itemize}