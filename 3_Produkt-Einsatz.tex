\section{Produkteinsatz}
% TODO: das sollte noch mindestens zwei mal überarbeitet werden, lol

\subsection{Anwendungsbereich}

\gls{Mallob} kann genutzt werden, um komplexe Probleme zu lösen, deren Laufzeit schwankt und die von einer flexibel parallel-verteilten Verarbeitung profitieren. Hier kann \gls{Mallob} helfen, indem es das Problem schneller löst. Um aber \gls{Mallob} einfacher zu nutzen, gibt es \textit{Fallob}, die nutzerfreundliche Schnittstelle von \gls{Mallob}.

\subsection{Zielgruppe}

Das \gls{Web-Interface} bietet eine moderne, einfache und intuitive Umgebung zur Nutzung dieses Systems. Hier sind abgesehen von der Bereitstellung der \hyperref[B:Job-Beschreibung]{Job-Beschreibung} im korrekten Format keine weiteren Kenntnisse notwendig, sodass dieses von jeder Person genutzt werden kann, die mit einer Webseite interagieren  kann.\\
Die \gls{API} dagegen richtet sich an Personen, die \gls{Mallob} in ihr eigenes System integrieren möchten. Hier wird Wissen über den Umgang mit einer solchen \gls{API} vorausgesetzt.

\subsection{Betriebsbedingungen}
\begin{itemize}
    \item Zur Nutzung ist eine stabile Internetverbindung notwendig. Insbesondere kann eine instabile Verbindung dazu führen, das Ereignisse nicht in Echtzeit angezeigt werden.
    \item Zur Verwendung des \glslink{Web-Interface}{Web-Interfaces} muss die neuste Version eines Browsers genutzt werden.
    \item Fallob muss Zugriff auf die \gls{API} einer laufenden Instanz von \gls{Mallob} haben.
\end{itemize}