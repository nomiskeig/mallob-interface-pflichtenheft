\section{Funktionale Anforderungen}
% Sollten wir noch nutzen
%https://de.overleaf.com/learn/latex/Cross_referencing_sections%2C_equations_and_floats


\subsection{API}

\begin{itemize}
    \item[FA10] \textbf{Registrierung von Nutzern} \\
    Es ist möglich neue Nutzer über die API zu registrieren, sodass diese sich mit ihren Zugangsdaten authentifizieren können -> wie genau?
    
    \item[FA20] \textbf{Authentifizierung von Nutzern} \\
    Nach dem Registrieren ist es dem Nutzer möglich sich über die API zu authentifizieren und Zugriff auf die anderen Funktionen der API zu erlangen. Dies kann er mit seinem Bearer-Token tun. Diesen Token erhält ein Nutzer bei Registrierung.
    
    \item[FA30] \textbf{Einreichen von Jobs} \\
    Die API ermöglicht es Jobs, die durch eine JSON-Datei und eine Job-Beschreibung spezifiziert werden, zu übergeben. Es existieren drei verschiedene Möglichkeiten um die Job-Beschreibung zu übergeben. Die Jobs werden von Mallob bearbeitet und das Ergebnis wird an den Nutzer zurückgegeben.
    %genaue Spezifizierung der Jobs (Priorität,Laufzeit, ...) auch hier oder an anderer Stelle?
    
    \begin{itemize}
        \item[FA31] \textbf{Einreichen der Job-Beschreibung separat von der JSON-Datei} \\
        Es ist möglich die Job-Beschreibung in einer eigenen Datei zu spezifizieren. Dabei muss es sich um eine Datei handeln, die dem DIMACS CNF Standard entspricht. Diese Datei wird zusammen mit der JSON-Datei an die API übergeben
        % soll das Dateiformat überhaupt hier schon spezifiziert werden?
        
        \item[FA32] \textbf{Job-Beschreibung innerhalb der JSON-Datei} \\
        Es ist möglich die Job-Beschreibung direkt in der JSON-Datei zu spezifizieren. Die Beschreibung muss auch in diesem Fall dem DIMACS CNF Format entsprechen. Bei dieser Möglichkeit wird nur die JSON-Datei an die API übergeben
        
        \item[FA33] \textbf{Übergeben der Job-Beschreibung über einen Link} \\
        Es kann ein Link an die API übergeben werden, der auf eine Datei verweist, in der die Job-Beschreibung enthalten ist. Die referenzierte Datei muss ebenfalls dem DIMACS CNF Format entsprechen. Zusätzlich zu dem Link muss die JSON-Datei mit den weiteren Job-Spezifikationen an die API übergeben werden
        
        \item[FA34] \textbf{Bereits eingereichte Job-Beschreibung verwenden}\\
        Ein Nutzer kann Job-Beschreibungen, welche er Bereits eingereicht hat über eine ID referenzieren und wiederverwenden.
        
    \end{itemize}
    
    \item[FA40] \textbf{Abbrechen von eingereichten Jobs} \\
    Der Nutzer kann einen eingereichten Job wieder abbrechen. In diesem Fall wird eine Statistik über die bereits verrichtete Arbeit zurückgegeben
    
    \item[FA50] \textbf{Zurückgeben von Ergebnissen} \\
    Für jeden eingereichten Job gibt die API eine Antwort an den Nutzer zurück. Wurde der Job erfolgreich gelöst, wird die Lösung des Problems zurückgegeben. Tritt bei der Berechnung ein Fehler auf, wird eine aussagekräftige Fehlermeldung zurückgegeben. Ist die maximale Bearbeitungszeit des Jobs erreicht, wird eine Statistik mit der bereits verrichteten Arbeit zurückgegeben.
    
    \begin{itemize}
        \item[FA51] \textbf{Ergebnisabfrage von Jobs} \\
        Es ist möglich für jeden Job (auch nach Beendigung) seinen aktuellen Status (in Bearbeitung, Bearbeitet, Fehler) abzufragen. Die Antwort  enthält dabei alle Informationen zum Status des Jobs, wie das Ergebnis oder eventuelle Fehlermeldung.
    \end{itemize}
    

    \item[FA60] \textbf{Zurückgeben des Systemzustands von Mallob} \\
    
    

\end{itemize}




\begin{itemize}
    \item System-Diagnose über API verfügbar sein$^3$
        \begin{itemize}
            \item Abfrage für Fehler, vermutlich auch ganze Liste an Fehlern
            \item Differenzierung in Nutzer und Admin: Admins sehen alle Fehler, Nutzer nur diejenigen Fehler, die sie auch etwas angehen
        \end{itemize}
\end{itemize}



%------------------------------------------------------------WEB-Interface
\subsection{Web-Interface}



% Web-Interface nochmal anders formuliert, eher an Beispiel von Betreuern orrientiert
\begin{itemize}
     \item[FA10] \textbf{Anmelden} \\
        Der Nutzer kann sich im Web-Interface anmelden. Dies geschieht über die Anmelde-Maske. Die Anmeldung geschieht mit Nutzernamen und Passwort. Nach erfolgreicher Registierung wird der Nutzer zur Auftrag-Seite gebracht.(Email)?
     \item[FA20] \textbf{Registrieren} \\
        Der Nutzer kann sich über das Web-Interface registrieren. Dies geschieht über die Registrieren-Maske. Diese kann von der Anmelde-Maske mit der entsprechenden Schaltfläche erreicht werden. Für die Registrierung wird ein Nutzername, ein Passwort und die wiederholte Eingabe des Passworts benötigt. Nach erfolgreicher Registierung wird der Nutzer zur Auftrag-Seite gebracht.
    \item[FA30] \textbf{Ergebnisse einsehen} \\
        Befindet sich in der Liste der Aufträge ein abgeschlossener Auftrag, so kann über die zu diesem Auftrag gehörige Schaltfläche "get results" das Ergebnis angezeigt werden. Hier kann es auch in die Zwischenablage kopiert werden oder heruntergeladen werden.
   \item[FA40] \textbf{Auftrag hinzufügen} \\ 
        Mittels einer Schaltfläche über der Liste der eigenen Aufträge gelangt der Nutzer zu einer Eingabe-Maske, über welche er einen neuen Auftrag hinzufügen kann. 
   \item[FA50] \textbf{Auftrag abbrechen} \\
   
   \item[FA60] \textbf{Anzeigen von Fehlern} \\
        Tritt bei Mallob ein Fehler auf, so wird der Nutzer umgehend mittels einer Fehlermeldung darauf aufmerksam gemacht. Diese Fehlermeldung wird immer angezeigt, unabhängig davon auf welcher Seite der Nutzer sich momentan befindet. 
    \item[FA70] \textbf{Visualisierung} \\
        Das Web-Interface besitzt eine Visualisierung des System-Zustandes. Diese kann über den entsprechenden Reiter erreicht werden.
        
        \begin{itemize}
            \item[FA72] \textbf{Anzeigen des aktuellen Zustandes} \\
                Standardmäßig wird immer der aktuelle Zustand angezeigt. Dieser Zustand wird, solange kein anderer Zeitpunkt ausgewählt wurde, steht dynamisch aktuell gehalten. 
            \item[FA71] \textbf{Zeitachse} \\
                Die Visualisierung verfügt über eine Zeitachse, mit derer ein entsprechender Zeitpunkt der letzten [...] Minuten/Stunden ausgewählt werden kann. Nach der Auswahl wird der System-Zustand zum entsprechenden Zeitpunkt angezeigt.
            \item[FA72] \textbf{Zurückspringen zu aktueller Ansicht} \\
                Wird mittels FA71 [TODO: REF] der angezeigte Zeitpunkt geändert, so wird eine Schaltfläche angezeigt, mit der der Nutzer jederzeit wieder zur aktuellen Zeit zurückspringen kann.
            \item[FA73] \textbf{Einsehen der Verteilung eines Auftrages} \\
                 
        \end{itemize}
    \item[FA80] \textbf{Ändern von Benutzer-Daten} \\
    
    \item[FA90] \textbf{Abmelden} \\
        Der Nutzer kann sich jederzeit über das entsprechende Menü in der Navigationsleiste abmelden. In diesem Falle wird wieder die Anmelde-Maske angezeigt.
\end{itemize}




\subsection{Plugins}
\section{Nichtfunktionale Anforderungen}