\section{Produktleistungen}

%%
%ProduktleistungenSofern 
%an einzelne Funktionen des Programms besondere Anforderungen in Bezug auf die Zeit oder die Genauigkeit gestellt werden, sollten diese in diesem Kapitel dargestellt werden. Dabei sollten Sie prüfen, ob die zu erbringenden Leistungen mit den in Punkt 5 genannten Angaben  realisierbar sind.
%

\begin{itemize}[noitemsep]
    \item[P100] Die maximale Anzahl der \hyperref[B:Jobs]{Jobs} ist begrenzt.
    
    \item[P110] Die maximale Anzahl \hyperref[B:Jobs]{Jobs}, die ein \gls{Nutzer} gleichzeitig in Bearbeitung haben kann, ist beschränkt.
    
    \item[P120] Die Zeit, die benötigt wird, um einen beliebigen Zeitpunkt in der \hyperref[pages:visualization]{Visualisierung} darzustellen, muss linear in der Anzahl der zu ladenden \hyperref[B:Event]{Events} sein.
    
    \item[P130] (Wunschbedingung)  Die Zeit, die benötigt wird, um einen beliebigen Zeitpunkt in der \hyperref[pages:visualization]{Visualisierung} darzustellen, muss konstant sein, unabhängig vom gewählten Zeitpunkt.
    
    \item[P140] Das \gls{Web-Interface} ist auch auf kleineren Bildschirmen, wie etwa einem Handy-Bildschirm, nutzbar.
    
    \item[P150] Die \gls{Konfigurationsdatei} wird immer nur beim Systemstart eingelesen, etwaige Änderungen werden also erst mit einem Neustart des Systems wirksam.
    
    \item[P160] Beim Einreichen eines \hyperref[B:Jobs]{Jobs} im Interface erfolgt schon im \gls{Web-Interface} eine Kontrolle der Syntax, welche den \gls{Nutzer} momentan über Fehler in der Eingabe informiert.
    
    \item[P170] Nutzernamen sind eindeutig und bestehen aus 4 bis 25 Zeichen.

    \item[P180] Passwörter müssen mindestens 8-stellig sein.
    
    \item[P190] Die gespeicherten \hyperref[B:Jobs]{Jobs} werden automatisch nach einem spezifizierten Zeitraum gelöscht.

    \item[P200] Die Größe der \hyperref[B:Job-Beschreibung]{Job-Beschreibung}, die man im \gls{Web-Interface} eingeben kann, ist beschränkt.

    
    \item[P210] Jedes \gls{Nutzerkonto} besitzt nach Registrierung die gleiche Priorität. Diese kann vom kann vom \gls{System-Administrator} geändert werden.
    
    
    \item[P220] Der \gls{Nutzer} erhält auf jede \gls{API}-Anfrage außer \hyperref[FA:API:Andauernde Abfrage des Ergebnisses eines Jobs]{F1110} unmittelbar eine  Antwort.
    
    \item[P230] Die \hyperref[pages:visualization]{Visualiseriung} ist skalierbar, sodass auch mehrere Tausend \glslink{Prozess}{Prozesse} angezeigt werden können. Um dies zu ermöglichen, wird die Qualität bei vielen \glslink{Prozess}{Prozessen} entsprechend reduziert.%, beispielsweise durch das Weglassen von Verbindungen zwischen den Prozessen.
    
    \item[P240] Daten über \hyperref[B:Jobs]{Jobs}, die nicht dem angemeldetem \gls{Nutzer} gehören, werden stets pseudonymisiert ausgegeben und dargestellt. Ist ein \gls{Administrator} angemeldet, so werden die Daten nicht pseudonymisiert.
    
    %\item[P250] Es ist möglich, die \hyperref[pages:job-page]{Job-Seite} direkt über eine passende \gls{URL} aufzurufen.
    
    \item[P260] Die maximale Geschwindigkeit der \hyperref[pages:visualization]{Visualisierung} ist das zweihundertfache.
    
    \item[P270] Die über \gls{Nutzer} gespeicherte Daten können von einem \glspl{System-Administrator} geändert werden, um beispielsweise die Priorität des \gls{Nutzer}s zu ändern oder ein \glslink{Nutzerkonto}{Konto} zu verifizieren, aber auch andere \gls{Nutzer} zum \gls{Administrator} zu machen.
    
    \item[P280] Das Backend wird in Java implementiert.

    \item[P290] Globale Einstellungen des Systems werden in einer \gls{Konfigurationsdatei} gespeichert und mit jedem Neustart aktualisiert.
    
    \item[P300] In der \hyperref[pages:visualization]{Visualisierung} können nur vergangene Zeitpunkte angezeigt werden, die innerhalb einer festgelegten Zeitspanne liegen.
\end{itemize}