\section{Produktleistungen}
\begin{itemize}
    \item Die maximale Anzahl paralleler Jobs wird mittels der Konfigurationsdatei festgelegt. 
    \item Die maximale Anzahl der Jobs ist begrenzt auf der Anzahl der Prozessen.
    \item Die Visualisierung ist performant, um einen beliebigen Zeitpunkt zu betrachten, wird maximal eine Sekunde benötigt, um diesen zu laden. 
    \item Das Web-Interface funktioniert auch bei kleineren Bildschirmen, wie etwa einem Handy-Bildschirm.
    % \item Dem Nutzer stehen verschiedene Filter zum Sortieren der Jobs zur Verfügung. [Keine Produktleistung]
    \item Es werden verständliche Fehlermeldungen dem Nutzer gezeigt, sowohl beim Registrieren/Anmelden, als auch bei Einreichung und Visualisierung von den Jobs.
    \item Die einzugebenden Optionen im Interface beim Hinzufügen von Jobs sind in der Konfigurationsdatei definiert, wobei eine Änderung dieser keine neue Kompilierung des Programms erfordert, sondern nur ein Neustart.
    \item Bei Eingabe im Interface erfolgt schon im Frontend elementare Kontrolle der Daten, was den Benutzer momentan über Fehler bei der Eingabe informiert.
   % \item Admins können mit den Kunden in Kontakt treten (per Email?). [Das ist eher FA, nicht produktleistung]
     \item Passwörter müssen mindestens 8-stellig sein.

%  ->  ok dann lass das rein machen :)
\end{itemize}