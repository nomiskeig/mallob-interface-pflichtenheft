
\section{Funktionale Anforderungen}


\subsection{API}


\begin{itemize}[nosep]
    \setlength\itemsep{4em}
    
    
    
  
    %-----------------API - Authentifizierung von Nutzern
    \phantomsection
    \label{FA:API:Authentifizieren von Nutzern}
    \item[F1000] \textbf{Authentifizieren von Nutzern}\\
    
    \begin{FA}
        \textbf{Ziel:} & Ein \gls{Nutzer} kann sich mit seinem Nutzernamen und Passwort authentifizieren \\
        \textbf{Vorbedingung:} & Der \gls{Nutzer} ist registriert \\
        \textbf{Nachbedingung (Erfolg):} & Der \gls{Nutzer} hat einen Authentifizierungstoken, der ihm Zugriff auf die Funktionen der API ermöglicht \\
        \textbf{Nachbedingung (Fehlschlag):} & Der \gls{Nutzer} hat keinen Authentifizierungstoken und hat eine Fehlermeldung erhalten \\
         \textbf{Akteure:} & \gls{Nutzer} \\
        \textbf{Auslösendes Ereignis:} & \gls{Nutzer} möchte Funktionen der API verwenden \\
    \end{FA}
    \textbf{Beschreibung:}
    \begin{FAList}
        \item[1.] Der \gls{Nutzer} schickt eine Anfrage zur Authentifizierung an die API, die seinen Nutzernamen und sein Passwort enthält.
        \item[2.] Die Anfrage wird von der API verarbeitet und der \gls{Nutzer} wird authentifiziert.
    \end{FAList}
    
    

    
    %-----------------API - Einreichen von Jobs
    \phantomsection
    \label{FA:API:Einreichen von Jobs} 
    \item[F1010] \textbf{Einreichen von Jobs} \\
    \begin{FA}
        \textbf{Ziel:} & Ein \gls{Nutzer} kann einen Job an die API übergeben, der von Mallob bearbeitet wird \\
        \textbf{Vorbedingung:} & Der \gls{Nutzer} hat sich mittels \hyperref[FA:API:Authentifizieren von Nutzern]{F1000} authentifiziert \\
        \textbf{Nachbedingung (Erfolg):} & Der \gls{Nutzer} erhält eine Bestätigung \\
        \textbf{Nachbedingung (Fehlschlag):} & Der \gls{Nutzer} hat eine Fehlermeldung \\
        \textbf{Akteure:} & \gls{Nutzer} \\
        \textbf{Auslösendes Ergebnis:} & Der \gls{Nutzer} möchte  einen Job einreichen \\
    \end{FA}
    \textbf{Beschreibung:}
    \begin{FAList}
            \item[1.a.] Der \gls{Nutzer} schickt eine Anfrage an die API, die sowohl die Job-Konfiguration als auch die Job-Beschreibung enthält, wobei die Job-Beschreibung in einer separaten Datei spezifiziert ist.
            \item[1.b.] Der \gls{Nutzer} schickt eine Anfrage an die API, die ebenfalls die Job-Konfiguration des Jobs als auch die Job-Beschreibung enthält, wobei in diesem Fall beides in  einer einzelnen Datei spezifiziert wird.
            \item[1.c.] (gewünscht) Der \gls{Nutzer} schickt eine Anfrage an die API, die ebenfalls die Job-Konfiguration und einen Link  zu einer Datei enthält, in der die Job-Beschreibung spezifiziert ist.
            \item[1.d.] (gewünscht) Der \gls{Nutzer} schickt eine Anfrage an die API, die eine Referenz auf einen bereits eingereichten Job enthält, der nochmal ausgeführt werden soll.
            \item[2.] Der eingereichte Job wird von Mallob bearbeitet.
            \item[3.a.] Wenn die Bearbeitung erfolgreich war, wird das Ergebnis an den \gls{Nutzer} zurückgegeben.
            \item[3.b.] Wenn die mittels \hyperref[FA:System:Einstellungen festlegen]{F4000} spezifizierte maximale Bearbeitungszeit erreicht wird, ohne dass ein Ergebnis gefunden wurde, wird eine Statistik über die bereits verrichtete Arbeit an den \gls{Nutzer} zurückgegeben.
            \item[3.c.] Wenn während der Bearbeitung ein Fehler auftritt, wird eine Fehlermeldung an den \gls{Nutzer} zurückgegeben.
    \end{FAList}
    

    
    %-----------------API - Abbrechen von Jobs
   \phantomsection
    \label{FA:API:Abbrechen von eingereichten Jobs}  
    \item[F1020] \textbf{Abbrechen von eingereichten Jobs} \\
    \begin{FA}
        \textbf{Ziel:} & Ein \gls{Nutzer} kann einen oder mehrere eingereichte Jobs wieder abbrechen \\
        \textbf{Vorbedingung:} & Der \gls{Nutzer} hat sich mittels \hyperref[FA:API:Authentifizieren von Nutzern]{F1000} authentifiziert und hat mindestens einen laufenden Job \\
        \textbf{Nachbedingung (Erfolg):} & Die Jobs sind abgebrochen und der \gls{Nutzer} hat das Teil-Ergebnis des eingereichten Jobs \\
        \textbf{Nachbedingung (Fehlschlag):} & Die Jobs sind nicht abgebrochen und der \gls{Nutzer} hat eine Fehlermeldung \\
        \textbf{Akteure:} & \gls{Nutzer} \\
        \textbf{Auslösendes Ergebnis:} & Der \gls{Nutzer} möchte den Job abbrechen \\
    \end{FA}
    \textbf{Beschreibung:}
    \begin{FAList}
            \item[1.a] Der \gls{Nutzer} schickt eine Anfrage an die API, welche eine Liste von Job-IDs enthält.
            \item[1.b] Der \gls{Nutzer} schickt eine Anfrage an die API, welche eine Liste von Job-Namen enthält.
            \item[2.] Die Jobs werden abgebrochen.
    \end{FAList}
    

  

    
    
    %--------------------------API - Abfrage für inkrementelle Jobs
    %\phantomsection
    %\label{FA:API:Abfragen der Fertigstellung eines inkrementellen Jobs}  
    %\item[F1070] \textbf{Abfragen der Fertigstellung eines inkrementellen Jobs} \\
    %\begin{FA}
    %    \textbf{Ziel:} & Es wird abgefragt, ob ein inkrementeller Job abgeschlossen ist oder ob Teil-Jobs existieren, die noch bearbeitet werden müssen \\
    %    \textbf{Vorbedingung:} & Der \gls{Nutzer} ist authentifiziert (siehe F20) es wurde ein inkrementeller Job eingereicht \\
    %    \textbf{Nachbedingung (Erfolg):} & Der \gls{Nutzer} hat eine Antwort erhalten, ob der Job abgeschlossen ist oder nicht \\
    %    \textbf{Nachbedingung (Fehlschlag):} & Der \gls{Nutzer} hat eine Fehlermeldung erhalten \\
    %    \textbf{Akteure:} & \gls{Nutzer} \\
    %    \textbf{Auslösendes Ereignis:} & Der \gls{Nutzer} möchte herausfinden, ob ein inkrementeller Job abgeschlossen ist \\
    %\end{FA}
    %\textbf{Beschreibung:}
    %\begin{FAList} 
    %    \item[1.] Der \gls{Nutzer} schickt eine Anfrage an die API mit der Job-ID des inkrementellen Jobs, der abgefragt werden soll
    %    \item[2.] Der Bearbeitungsstand des inkrementellen Jobs wird abgefragt
    %    \item[3.a] Wenn die Abfrage erfolgreich war, wird das Ergebnis an den \gls{Nutzer} zurückgegeben
    %    \item[3.b.] Wenn die Abfrage nicht erfolgreich war, wird eine Fehlermeldung an den \gls{Nutzer} zurückgegeben
    %\end{FAList}
    
    
    %--------------------------API - Abfragen der Daten eines einzelnen Jobs
    \phantomsection
    \label{FA:API:Abfragen der Informationenen von Jobs}  
    \item[F1030] \textbf{Ausgeben der Informationen von Jobs} \\
    \begin{FA}
        \textbf{Ziel:} & Es können die Informationen von Jobs ausgegeben werden.\\
        \textbf{Vorbedingung:} & Der \gls{Nutzer} ist gemäß \hyperref[FA:API:Authentifizieren von Nutzern]{F1000} authentifiziert und die gewünschten Jobs wurden bereits eingereicht. \\
        \textbf{Nachbedingung (Erfolg):} & Der \gls{Nutzer} hat die Informationen zu den angefragten Jobs erhalten \\
        \textbf{Nachbedingung (Fehlschlag):} &  Der \gls{Nutzer} hat eine Fehlermeldung erhalten \\
        \textbf{Akteure:} & \gls{Nutzer} \\
        \textbf{Auslösendes Ereignis:} & Der \gls{Nutzer} möchte Informationen über eingereichte Jobs erhalten \\
    \end{FA}
    \textbf{Beschreibung:}
    \begin{FAList} 
        \item[1.] Der \gls{Nutzer} stellt eine Anfrage an die API. Diese Anfrage enthält einen relativen Zeitpunkt und einen der folgenden  Parameter:
            \begin{itemize}
                \item[1.a.] Die Anfrage enthält den Parameter \enquote{all}. Entsprechend werden alle Jobs, die sich zum angegebenen Zeitpunkt im System befinden, zurückgegeben. Ist der \gls{Nutzer} kein Admin, so werden alle Jobs, die nicht ihm gehören, pseudonymisiert.
                \item[1.b.] Die Anfrage enthält den Parameter \enquote{user}. Entsprechend werden alle Jobs, die zum angegebenen Zeitpunkt dem \gls{Nutzer} gehören, zurückgegeben.
                \item[1.c.] Die Anfrage enthält den Parameter \enquote{custom} und eine Liste mit Job-IDs, deren Informationen zurückgegeben werden sollen.
            \end{itemize}
        \item[2.] Die Daten der Jobs werden abgefragt.
        \item[3.a.] Wenn die Abfrage erfolgreich war, werden die Informationen an den \gls{Nutzer} zurückgegeben.
        \item[3.b.] Wenn die Abfrage nicht erfolgreich war, wird eine Fehlermeldung an den \gls{Nutzer} zurückgegeben.
    \end{FAList}
    
    
    \phantomsection
    \label{FA:API:Ausgeben eines Systemzustandes}
    \item[F1040] \textbf{Ausgeben eines Systemzustandes} \\
    \begin{FA}
        \textbf{Ziel:} & Es kann ein beliebiger vergangener Systemzustand abgefragt werden \\
        \textbf{Vorbedingung:} & Der \gls{Nutzer} ist gemäß \hyperref[FA:API:Authentifizieren von Nutzern]{F1010} authentifiziert \\
        \textbf{Nachbedingung (Erfolg):} & Der \gls{Nutzer} hat den gewünschten Systemzustand erhalten \\
        \textbf{Nachbedingung (Fehlschlag):} &  Der \gls{Nutzer} hat eine Fehlermeldung erhalten \\
        \textbf{Akteure:} & \gls{Nutzer} \\
        \textbf{Auslösendes Ereignis:} & Einen vergangenen Systemzustand einsehen \\
    \end{FA}
     \textbf{Beschreibung:}
    \begin{FAList} 
        \item[1.] Der \gls{Nutzer} stellt eine Anfrage an die API mit dem gewünschten Zeitpunkt.
        \item[2.] Der Systemzustand wird abgefragt.
    \end{FAList} 
    
    
    
    
    \phantomsection
    \label{FA:API:Ausgeben von vergangenen Events}
    \item[F1050] \textbf{Ausgeben von vergangenen Events} \\
    \begin{FA}
        \textbf{Ziel:} & Es können vergangene Events ausgegeben werden.\\
        \textbf{Vorbedingung:} & Der \gls{Nutzer} ist gemäß \hyperref[FA:API:Authentifizieren von Nutzern]{F1010} authentifiziert und die gewünschten Jobs wurden bereits eingereicht. \\
        \textbf{Nachbedingung (Erfolg):} & Der \gls{Nutzer} hat die gewünschten Events zu den angefragten Jobs erhalten. \\
        \textbf{Nachbedingung (Fehlschlag):} &  Der \gls{Nutzer} hat eine Fehlermeldung erhalten. \\
        \textbf{Akteure:} & \gls{Nutzer} \\
        \textbf{Auslösendes Ereignis:} & Der \gls{Nutzer} möchte Events von eingereichte Jobs erhalten. \\
    \end{FA}
     \textbf{Beschreibung:}
    \begin{FAList} 
        \item[1.] Der \gls{Nutzer} stellt eine Anfrage an die API mit folgenden Parametern: 
            \begin{itemize}
                \item[-] Jobs, von welchen die Events ausgegeben werden sollen. Hier ist es möglich, alle Jobs, die eigenen Jobs oder eine beliebige Menge eigener Jobs anzugeben.
                \item[-] Angabe, aus welcher Zeit die Events stammen sollen.
            \end{itemize}
        \item[2.] Die Daten der Jobs werden abgefragt.
    \end{FAList} 
    
    %---------------API - Ergebnisdatei anfordern
    \phantomsection
    \label{FA:API:Ausgeben des Ergebnisses für eine oder mehrere Jobs}  
    \item[F1060] \textbf{Ausgeben des Ergebnisses für eine oder mehrere Jobs} \\
    \begin{FA}
        \textbf{Ziel:} & Es werden die Ergebnisdateien von einem oder mehreren Jobs von einem \gls{Nutzer} abgefragt \\
        \textbf{Vorbedingung:} & Der \gls{Nutzer} ist authentifiziert und die angeforderte Jobs wurden eingereicht und bearbeitet \\
        \textbf{Nachbedingung (Erfolg):} & Der \gls{Nutzer} hat die Ergebnisdateien der spezifizierten Jobs \\
        \textbf{Nachbedingung (Fehlschlag):} & Der \gls{Nutzer} hat eine Fehlermeldung erhalten  \\
        \textbf{Akteure:} & \gls{Nutzer} \\
        \textbf{Auslösendes Ereignis:} & Der \gls{Nutzer} möchte die Ergebnisdateien von einem oder mehreren Jobs haben \\
    \end{FA}
    \textbf{Beschreibung:}
    \begin{FAList} 
        \item[1.] Der \gls{Nutzer} schickt eine Anfrage an die API mit einer Liste von Job-IDs 
        \item[2.a.] Wenn die Anfrage erfolgreich war, werden die Ergebnisdateien der Jobs an den \gls{Nutzer} zurückgegeben 
        \item[2.b.] Wenn die Anfrage nicht erfolgreich war, wird eine Fehlermeldung an den \gls{Nutzer} zurückgegeben
    \end{FAList}
    
    
    %---------------API - Job-Beschreibung ausgeben
    \phantomsection
    \label{FA:API:Ausgeben der Job-Beschreibung}  
    \item[F1070] \textbf{Ausgeben der Job-Beschreibung für einen oder mehrere Jobs} \\
    \begin{FA}
        \textbf{Ziel:} & Es werden die Job-Beschreibungen von einem oder mehreren Jobs an den \gls{Nutzer} ausgegeben  \\
        \textbf{Vorbedingung:} & Der \gls{Nutzer} ist authentifiziert und die angegebenen Jobs wurden bereits eingereicht \\
        \textbf{Nachbedingung (Erfolg):} & Der \gls{Nutzer} hat die Job-Beschreibungen der spezifizierten Jobs \\
        \textbf{Nachbedingung (Fehlschlag):} & Der \gls{Nutzer} hat eine Fehlermeldung erhalten \\
        \textbf{Akteure:} & \gls{Nutzer} \\
        \textbf{Auslösendes Ereignis:} & Der \gls{Nutzer} möchte die Job-Beschreibungen von einem oder mehreren Jobs haben \\
    \end{FA}
    \textbf{Beschreibung:}
    \begin{FAList} 
        \item[1.] Der \gls{Nutzer} schickt eine Anfrage an die API mit einer Liste von Job-IDs
        \item[2.a.] Wenn die Anfrage erfolgreich war, werden die Job-Beschreibungen an den \gls{Nutzer} zurückgegeben 
        \item[2.b] Wenn die Anfrage nicht erfolgreich war, wird eine Fehlermeldung an den \gls{Nutzer} zurückgegeben 
    \end{FAList}
    
    
    %--------------API - Abfragen der Informationen von Mallob
    \phantomsection
    \label{FA:API:Abfragen der Informationen von Mallob}  
    \item[F1080] \textbf{Abfragen der Informationen von Mallob} \\
    \begin{FA}
        \textbf{Ziel:} & Der Administrator kann den Status und weitere Informationen, unter anderem Warnungen, zu Mallob abrufen \\
        \textbf{Vorbedingung:} & Der Administrator muss authentifiziert sein \\
        \textbf{Nachbedingung (Erfolg):} & Der Administrator hat die Informationen zu Mallob erhalten \\
        \textbf{Nachbedingung (Fehlschlag):} & Der Administrator hat eine Fehlermeldung erhalten \\
        \textbf{Akteure:} & Administrator \\
        \textbf{Auslösendes Ereignis:} & Der Administrator möchte genauere Informationen zu Mallob haben \\
    \end{FA}
    \textbf{Beschreibung:}
    \begin{FAList} 
        \item[1.] Der Administrator schickt eine entsprechende Anfrage an die API
    \end{FAList}
    
    
    %-----------API - Ausgeben eines Event-Streams von Mallob
    \phantomsection

    \label{FA:API:Ausgeben eines Event-Streams von Mallob}  
    \item[F1090] \textbf{Ausgeben eines Event-Streams von Mallob} \\
    \begin{FA}
        \textbf{Ziel:} & Der \gls{Nutzer} hat Zugriff auf einen Event-Stream, über den kontinuierlich die Events der Jobs im System übertragen werden \\
        \textbf{Vorbedingung:} & Der \gls{Nutzer} muss authentifiziert sein \\
        \textbf{Nachbedingung (Erfolg):} & Der \gls{Nutzer} hat Zugriff auf den Event-Stream \\
        \textbf{Nachbedingung (Fehlschlag):} & Der \gls{Nutzer} hat keinen Zugriff und hat eine Fehlermeldung erhalten \\
        \textbf{Akteure:} & \gls{Nutzer} \\
        \textbf{Auslösendes Ereignis:} & Der \gls{Nutzer} möchte die Events von Mallob einsehen \\
    \end{FA}
    \textbf{Beschreibung:}
    \begin{FAList} 
        \item[1.] Der \gls{Nutzer} schickt eine Anfrage an die API
        \item[2.a.] Wenn die Anfrage erfolgreich war, wird der Stream an den \gls{Nutzer} zurückgegeben 
        \item[2.b.] Wenn die Anfrage nicht erfolgreich war, wird eine Fehlermeldung an den \gls{Nutzer} zurückgegeben 
    \end{FAList}

    
    
    
    \phantomsection
    \label{FA:API:Abrufen von Einstellungen}  
    \item[F1100] \textbf{Abrufen von durch Konfigurationsdatei festgelegte Einstellungen} \\
    \begin{FA}
        \textbf{Ziel:} & Die API bietet eine Möglichkeit, die Einstellungen der Konfigurationsdatei abzurufen.\\
        \textbf{Vorbedingung:} & - \\
        \textbf{Nachbedingung (Erfolg):}  & Die Einstellungen werden zurückgegeben.\\
        \textbf{Nachbedingung (Fehlschlag):} & Es wird eine Fehlermeldung zurückgegeben. \\
        \textbf{Akteure:} & \gls{Nutzer} \\
        \textbf{Auslösendes Ereignis:} & Der \gls{Nutzer} möchte die aktuellen Einstellungen abfragen. \\
    \end{FA}
    \textbf{Beschreibung:}
    \begin{FAList} 
        \item[1.] Schicken der entsprechenden API-Anfrage.
    \end{FAList} 
    
    \phantomsection
    \label{FA:API:Ausgeben eines Ereignis-Streams zu einem Job}
    \item[F1110] \textbf{Ausgeben eines Ereignis-Streams zu einem Job}
    \begin{FA}
        \textbf{Ziel:} & Der \gls{Nutzer} bekommt Echtzeit-Informationen zum Status eines Jobs \\
        \textbf{Vorbedingung:} & Der \gls{Nutzer} muss authentifiziert sein und mindestens einen Job eingereicht haben \\ 
        \textbf{Nachbedingung (Erfolg):} & Der \gls{Nutzer} hat Zugriff auf den Ereignis-Stream \\
        \textbf{Nachbedingung (Fehlschlag): } & Der \gls{Nutzer} hat keinen Zugriff auf den Ereignis-Stream und hat eine Fehlermeldung erhalten \\
        \textbf{Akteure:} \gls{Nutzer} \\
        \textbf{Auslösendes Ereignis:} & Der \gls{Nutzer} möchte in Echtzeit über den Status eines Jobs informiert werden. \\
    \end{FA}
    \textbf{Beschreibung:}
    \begin{FAList}
        \item[1.] Der \gls{Nutzer} schickt eine Anfrage an die API.
        \item[2.] Wenn die Anfrage erfolgreich war, wird der Stream an den \gls{Nutzer} zurückgegeben.
        \item[3.] Wenn die Job-Bearbeitung abgeschlossen ist, erhält der \gls{Nutzer} über diesen Stream die Meta-Daten zu diesem Job und der Stream wird geschlossen
    \end{FAList}

    % --- ab hier wunschkriterium
        
    %---------------------API - Starten von Mallob
    \phantomsection
    \label{FA:API:Starten von Mallob}  
    \item[F1120] (Wunschkriterium) \textbf{Starten von Mallob} \\
    \begin{FA}
        \textbf{Ziel:} & Die Mallob Instanz wird von einem Administrator gestartet \\
        \textbf{Vorbedingung:} & Mallob läuft noch nicht \\
        \textbf{Nachbedingung (Erfolg):} & Mallob ist gestartet und der Administrator hat eine Bestätigung erhalten \\
        \textbf{Nachbedingung (Fehlschlag):} & Mallob ist nicht gestartet und der Administrator hat eine Fehlermeldung erhalten \\
        \textbf{Akteure:} & Administrator \\
        \textbf{Auslösendes Ereignis:} & Der Administrator will die Mallob-Instanz starten \\
    \end{FA}
    \textbf{Beschreibung:}
    \begin{FAList}
        \item[1.] Der Administrator schickt eine Anfrage an die API
        \item[2.] Die Mallob-Instanz wird gestartet
    \end{FAList}
    
    
    %---------------------API - Stoppen von Mallob
    \phantomsection
    \label{FA:API:Stoppen von Mallob}  
    \item[F1130] (Wunschkriterium) \textbf{Stoppen von Mallob} \\
    \begin{FA}
        \textbf{Ziel:} & Die Mallob Instanz wird von einem Administrator gestoppt \\
        \textbf{Vorbedingung:} & Die Mallob-Instanz läuft bereits \\
        \textbf{Nachbedingung (Erfolg):} & Die Mallob-Instanz läuft nicht mehr und der Administrator hat eine Bestätigung erhalten \\
        \textbf{Nachbedingung (Fehlschlag:} & Die Mallob-Instanz läuft noch und der Administrator hat eine Fehlermeldung erhalten \\
        \textbf{Akteure:} & Administrator \\
        \textbf{Auslösendes Ereignis:} & Der Administrator will die Mallob-Instanz stoppen \\
    \end{FA}
    \textbf{Beschreibung:}
    \begin{FAList}
        \item[1.] Der Administator schickt eine Anfrage an die API
        \item[2.] Die Mallob-Instanz wird gestoppt
    \end{FAList}
    
    
    %-------------------------------API - Neustart von Mallob
    % das könnte man eigentlich aus rausmachen, kann durch stoppen und starten erreicht werden, aber idk
    \phantomsection
    \label{FA:API:Neustart von Mallob}  
    \item[F1140] (Wunschkriterium) \textbf{Neustart von Mallob} \\
    \begin{FA}
        \textbf{Ziel:} & Die Mallob-Instanz wird von einem Administrator neu gestartet \\
        \textbf{Vorbedingung:} & Die Mallob-Instanz läuft bereits \\
        \textbf{Nachbedingung (Erfolg):} & Die Mallob-Instanz wurde neu gestartet und der Administrator hat eine Bestätigung erhalten \\
        \textbf{Nachbedingung (Fehlschlag):} & Der Administrator hat eine Fehlermeldung erhalten \\
        \textbf{Akteure:} & Administrator \\
        \textbf{Auslösendes Ereignis:} & Der Administrator möchte die Mallob-Instanz neu starten \\
    \end{FA}
    \textbf{Beschreibung:}
    \begin{FAList}
        \item[1.] Der Administrator schickt eine Anfrage an die API
        \item[2.] Die Mallob-Instanz wird neu gestartet
    \end{FAList}
    
    %\phantomsection
    %\label{FA:API:Einreichen von Jobs per URL} 
    %\item[F1020] (Wunschkriterium) \textbf{Einreichen von Jobs mit Beschreibung per URL} \\
    %\begin{FA}
    %    \textbf{Ziel:} & Ein \gls{Nutzer} kann einen Job über die API einreichen, derer Beschreibung über eine URL verfügbar ist. \\
    %    \textbf{Vorbedingung:} & Der \gls{Nutzer} hat sich mittels \hyperref[FA:API:Authentifizieren von Nutzern]{F1010} authentifiziert \\
    %    \textbf{Nachbedingung (Erfolg):} & Der \gls{Nutzer} erhält eine Bestätigung. \\
    %    \textbf{Nachbedingung (Fehlschlag):} & Der \gls{Nutzer} hat eine Fehlermeldung. \\
    %    \textbf{Akteure:} & \gls{Nutzer} \\
    %    \textbf{Auslösendes Ergebnis:} & Der \gls{Nutzer} möchte  einen Job einreichen. \\
    %\end{FA}
    %\textbf{Beschreibung:}
    %\begin{FAList}
    %        \item[1.a.] Der \gls{Nutzer} schickt eine Anfrage an die API, die sowohl die Job-Konfiguration als auch die URL mit der Job-Beschreibung enthält
    %        \item[2.] Der eingereichte Job wird von Mallob bearbeitet
    %\end{FAList}

    \phantomsection
    \label{FA:API:Registrierung von Nutzern}
    \item[F1150] (Wunschkriterium) \textbf{Registrierung von Nutzern} \\ 
    
    \begin{FA}
        \textbf{Ziel: } & Registrierung über die API ist möglich \\
        \textbf{Vorbedingung:} &  -keine- \\
        \textbf{Nachbedingung (Erfolg):} &  Ein vorläufiges Konto wurde für den \gls{Nutzer} erstellt und der \gls{Nutzer} hat eine Bestätigung erhalten \\
        \textbf{Nachbedingung (Fehlschlag):} &  Der \gls{Nutzer} ist nicht registriert und hat eine Fehlermeldung erhalten \\
        \textbf{Akteure:} & \gls{Nutzer}\\
        \textbf{Auslösendes Ereignis:} & \gls{Nutzer} möchte das System verwenden \\
    \end{FA}
    \textbf{Beschreibung:}
    \begin{FAList} 
        \item[1.] Der \gls{Nutzer} schickt eine Anfrage an die API, die seine \hyperref[PD:Registrierungsdaten]{Registrierungsdaten} enthält
        \item[2.] Vorläufiges Konto wird erstellt
        \item[2.1.a] Wenn die Registrierung erfolgreich war, wird eine Bestätigung an den \gls{Nutzer} zurückgegeben
        \item[2.1.b] Wenn die Registrierung nicht erfolgreich war, wird eine Fehlermeldung an den \gls{Nutzer} zurückgegeben
        \item[3.] Verifizierung des Kontos durch Administrator, gemäß \hyperref[FA:API:Verifizierung von Nutzern]{F1020} 
    \end{FAList}

%----------------API- Verifizierung von Nutzerkonten
    \phantomsection
    \label{FA:API:Verifizierung von Nutzern}
    \item[1160] (Wunschkriterium) \textbf{Verifizierung von vorläufigen Nutzerkonten}\\
    
    \begin{FA}
        \textbf{Ziel:} & Ein Administrator kann ein vorläufiges Konto verifizieren. \\
        \textbf{Vorbedingung:} & Ein \gls{Nutzer} hat sich gemäß \hyperref[FA:API:Registrierung von Nutzern]{F1000} registriert und somit \gls{Vorlaeufiges Nutzerkonto} erstellt. Der Administrator hat sich gemäß \hyperref[FA:API:Authentifizieren von]{} \\%Nutzern]{F1010} authentifiziert. \\
        \textbf{Nachbedingung (Erfolg):} & Das Konto des \glslink{Nutzer}{Nutzers} ist verifiziert. \\
        \textbf{Nachbedingung (Fehlschlag):} & Das Konto des \glslink{Nutzer}{Nutzers} ist nicht verifiziert \\
        \textbf{Akteure:} & Administrator \\
        \textbf{Auslösendes Ereignis:} & Der Administrator möchte vorläufige Nutzerkonten verifizieren \\
    \end{FA}
    \textbf{Beschreibung:}
    \begin{FAList}
        \item[1.] Der Administrator schickt eine Anfrage an die API, welche die ID der nichtverifizierte Nutzerkonten enthält
        \item[2.] Die Anfrage wird von der API verarbeitet und das \gls{Nutzerkonto} wird verifiziert
    \end{FAList}
   
    
        
 % -----------------API - Erhalt von nichtverifizierten Nutzerkonten
    \phantomsection
    \label{FA:API:Erhalt von vorlaeufigen Nutzerkonten} % fällt mit der registrierung raus
    \item[F1170] (Wunschkriterium) \textbf{Abfragen von vorläufigen Nutzerkonten}\\
    
    \begin{FA}
        \textbf{Ziel:} & Ein Administrator kann die aktuellen vorläufigen Nutzerkonten abrufen\\
        \textbf{Vorbedingung:} & Keine \\
        \textbf{Nachbedingung (Erfolg):} & Der Administrator hat hat eine Liste von vorläufigen Nutzerkonten erhalten \\
        \textbf{Nachbedingung (Fehlschlag):} & Der Administrator hat eine Fehlermeldung erhalten \\
         \textbf{Akteure:} & Administrator \\
        \textbf{Auslösendes Ereignis:} & Administrator möchte alle vorläufigen Nutzerkonten einsehen \\
    \end{FA}
    \textbf{Beschreibung:}
    \begin{FAList}
        \item[1.] Der Administrator schickt eine Anfrage um alle vorläufigen, also nicht-verifizierten, Nutzerkonten einzusehen.
        \item[2.] Die Anfrage wird von der API verarbeitet und die Liste wird an den Anfragesteller gesendet.
    \end{FAList}
\end{itemize}
    
%-------------------------------------------------------------------
%--------------------WEB INTERFACE----------------------------------
%-------------------------------------------------------------------
\pagebreak

\subsection{Web-Interface}
Die folgenden funktionalen Anforderungen beziehen sich alle auf das Web-Interface und sind auch in diesem Kontext zu verstehen.


\begin{itemize}
    \setlength\itemsep{4em}

    \phantomsection
    \label{FA:Web-Interface:Anmelden} 
    \item[F2000] \textbf{Anmelden} \\
    \begin{FA}
        \textbf{Ziel:} & Ein \gls{Nutzer} ist in der Lage, sich im Web-Interface zu authentifizieren. \\
        \textbf{Vorbedingung:} & Der \gls{Nutzer} besitzt bereits ein Konto und ist noch nicht angemeldet. \\
        \textbf{Nachbedingung (Erfolg):}  &  Der \gls{Nutzer} wird angemeldet und zur \hyperref[pages:job-table]{Job-Tabelle} weitergeleitet.\\
        \textbf{Nachbedingung (Fehlschlag):} & Die Anmeldung findet nicht statt und es  wird eine Fehlermeldung angezeigt. \\
        \textbf{Akteure:} & \gls{Nutzer} \\
        \textbf{Auslösendes Ereignis:} &  Der \gls{Nutzer} möchte sich im Web-Interface anmelden. \\
    \end{FA}
    \textbf{Beschreibung:}
    \begin{FAList} 
        \item[1.] Aufrufen des Web-Interfaces
        \item[2.] Eingabe des Nutzernamens
        \item[3.] Eingabe des Passwortes
        \item[4.] Bestätigung durch Betätigung der Schaltfläche \enquote{Log in}
    \end{FAList}

   
    \phantomsection
    \label{FA:Web-Interface:Job einreichen} 
    \item[F2010] \textbf{Job einreichen} \\
    \begin{FA}
        \textbf{Ziel:} & Einreichen eines neuen Jobs über das Web-Interface.\\
        \textbf{Vorbedingung:} & Der \gls{Nutzer} ist angemeldet.  \\
        \textbf{Nachbedingung (Erfolg):} & Der \gls{Nutzer} wird auf die \hyperref[pages:job-page]{Seite des gerade eingereichten Job} weitergeleitet.  \\
        \textbf{Nachbedingung (Fehlschlag):} & Im Web-Interface wird eine aussagekräftige Fehlermeldung angezeigt. \\
        \textbf{Akteure:} & \gls{Nutzer} \\
        \textbf{Auslösendes Ereignis:} & Der \gls{Nutzer} möchte einen Job in Auftrag geben. \\
    \end{FA}
    \textbf{Beschreibung:}
    \begin{FAList} 
        \item[1.] Auswahl entsprechenden Schaltfläche in der Navigationsleiste
        \item[2.] Weiterleitung zur \hyperref[pages:submit-job]{Seite zum Einreichen von Jobs}
        \item[3.] Eingabe der notwendigen Optionen des Jobs
        \item[4.] Eingabe der Job-Beschreibung über ein Eingabe-Feld direkt im Web-Interface
        \item[5.] Bestätigung der Eingaben
    \end{FAList}
    \textbf{Erweiterungen}
    \begin{FAList}
        \item[3a.] Hinzufügen von gewünschten optionalen Optionen mithilfe des entsprechenden Dropdown-Menüs.
        \item[3b.] Eingabe der optionalen Optionen über die entsprechenden Felder.
    \end{FAList}
    \textbf{Alternative 1 zu Schritt 4:}
    \begin{FAList}
        \item[4a] Dropdown-Menü nutzen, um Upload der Job-Beschreibung auszuwählen.
        \item[4b] Entsprechende Schaltfläche nutzen, um die entsprechende Job-Beschreibung zum Hochladen auszuwählen.
    \end{FAList}
      \textbf{Alternative 2 zu Schritt 4: (gewünscht)}
    \begin{FAList}
        \item[4a] Dropdown-Menü nutzen, um Angabe einer URL der Job-Beschreibung auszuwählen.
        \item[4b] Eingabe der URL zur Job-Beschreibung im entsprechenden Feld.
    \end{FAList}
    \pagebreak[3]
    
    \phantomsection
    \label{FA:Web-Interface:Abbruch eines einzelnen Jobs} 
    \item[F2020] \textbf{Abbruch eines einzelnen Jobs} \\
    \begin{FA}
        \textbf{Ziel:} & Ein bereits eingereichter Job wird wieder abgebrochen. \\
        \textbf{Vorbedingung:} & Es gibt einen bereits eingereichten, noch nicht fertiggestellten Job. \\
        \textbf{Nachbedingung (Erfolg):}  & Der Job wurde abgebrochen. \\
        \textbf{Nachbedingung (Fehlschlag):} &  Der Job läuft weiter und dem \gls{Nutzer} wird eine entsprechende Fehlermeldung angezeigt. \\
        \textbf{Akteure:} & \gls{Nutzer} \\
        \textbf{Auslösendes Ereignis:} & Der \gls{Nutzer} möchte einen laufenden Job abbrechen. \\
    \end{FA}
    \textbf{Beschreibung:}
    \begin{FAList} 
        \item[1.] Navigation zur Job-Tabelle.
        \item[2.] Anklicken des entsprechenden Jobs in der Tabelle.
        \item[3.] Auswahl der entsprechenden Schaltfläche im aufgeklappten Fenster.
        \item[4a.] Bestätigung des Abbrechens im erschienenen Menü.
        \item[4b.] Keine Bestätigung des Abbrechens im erschienenen Menü, die Aktion wird abgebrochen und der Job läuft weiter.
    \end{FAList}
    \textbf{Alternative zu den Schritten 1 bis 3:}
    \begin{FAList}
        \item[1.] Navigation zur Job-Seite des abzubrechenden Jobs.
        \item[2.] Auswahl der entsprechenden Schaltfläche.
    \end{FAList}
    
    
    \phantomsection
    \label{FA:Web-Interface:Abbruch mehrerer Jobs auf einmal} 
    \item[F2030] \textbf{Abbruch mehrerer Jobs auf einmal} \\
    \begin{FA}
        \textbf{Ziel:} & Mehrere bereits laufende werden auf einmal abgebrochen werden. Mit \enquote{auf einmal} ist hier gemeint, das nicht jeder Job einzeln abgebrochen wird, sondern alle abzubrechenden Jobs mit einem Klick zur selben Zeit abgebrochen werden können. \\
        \textbf{Vorbedingung:} & Es gibt mehrere, bereits eingereichte und noch nicht fertiggestellte Jobs. \\
        \textbf{Nachbedingung (Erfolg):}  & Die Jobs wurden alle abgebrochen. \\
        \textbf{Nachbedingung (Fehlschlag):} & Ein oder mehrere Jobs konnten nicht abgebrochen werden und der \gls{Nutzer} erhält eine entsprechende Fehlermeldung. Die Jobs, bei denen der Abbruch erfolgreich ist, werden auch abgebrochen, falls  dies bei anderen Jobs fehlschlägt.\\
        % [TODO: Formulierung überarbeiten]
        \textbf{Akteure:} & \gls{Nutzer} \\
        \textbf{Auslösendes Ereignis:} & Der \gls{Nutzer} möchte mehrere laufende Jobs auf einmal abbrechen. \\
    \end{FA}
    \textbf{Beschreibung:}
    \begin{FAList} 
        \item[1.] Navigation zur Job-Tabelle.
        \item[2.] Setzen eines Kreuzes in der Checkbox bei allen Jobs, die Abgebrochen werden sollen.
        \item[3.] Auswahl der entsprechenden Action im entsprechenden Dropdown-Menü.
        \item[4.] Bestätigung des Abbruchs.
    \end{FAList}
    
    
    \phantomsection
    \label{FA:Web-Interface:Herunterladen eines einzelnen Ergebnisses} 
    \item[F2040] \textbf{Herunterladen eines einzelnen Ergebnisses} \\
    \begin{FA}
        \textbf{Ziel:} & Ein einzelnes Ergebnis eines abgeschlossen Jobs kann heruntergeladen werden. \\
        \textbf{Vorbedingung:} & Es gibt einen bereits abgeschlossenen Job. \\
        \textbf{Nachbedingung (Erfolg):}  & Das Ergebnis wurde heruntergeladen. \\
        \textbf{Nachbedingung (Fehlschlag):} &  Das Ergebnis wurde nicht heruntergeladen und eine Fehlermeldung wird angezeigt. \\
        \textbf{Akteure:} & \gls{Nutzer} \\
        \textbf{Auslösendes Ereignis:} & Der \gls{Nutzer} möchte ein einzelnes Ergebnis herunterladen. \\
    \end{FA}
    \textbf{Beschreibung:}
    \begin{FAList} 
        \item[1.] Navigation zur Job-Tabelle.
        \item[2.] Anklicken des entsprechenden Jobs in der Tabelle.
        \item[3.] Auswahl der entsprechenden Schaltfläche im aufgeklappten Fenster.
        \item[4.] Das Ergebnis wird heruntergeladen.
    \end{FAList}
    \textbf{Alternative zu den Schritten 1 bis 3:}
    \begin{FAList}
        \item[1.] Navigation zur Job-Seite des Jobs mit dem gewünschten Ergebnis.
        \item[2.] Auswahl der entsprechenden Schaltfläche.
    \end{FAList}
    
    
    \phantomsection
    \label{FA:Web-Interface:herunterladen mehrerer Ergebnisse auf einmal} 
    \item[F2050] \textbf{Herunterladen mehrerer Ergebnisse auf einmal} \\
    \begin{FA}
        \textbf{Ziel:} & Mehrere Ergebnisse können auf einmal heruntergeladen werden. Mit \enquote{auf einmal} ist hier gemeint, das nicht alle Ergebnisse einzeln heruntergeladen werden, sondern alle gewünschten Ergebnisse mit einem Klick zur selben Zeit heruntergeladen werden können. \\
        \textbf{Vorbedingung:} & Es gibt mindestens einen bereits fertigen Job. \\
        \textbf{Nachbedingung (Erfolg):}  & Die Ergebnisse wurde alle heruntergeladen. \\
        \textbf{Nachbedingung (Fehlschlag):} & Ein oder mehrere Ergebnisse konnten nicht heruntergeladen werden und der \gls{Nutzer} erhält eine entsprechende Fehlermeldung. Die restlichen gewünschten Ergebnisse werden dennoch heruntergeladen.\\
        \textbf{Akteure:} & \gls{Nutzer} \\
        \textbf{Auslösendes Ereignis:} & Der \gls{Nutzer} möchte mehrere Ergebnisse auf einmal herunterladen. \\
    \end{FA}
    \textbf{Beschreibung:}
    \begin{FAList} 
        \item[1.] Navigation zur Job-Liste.
        \item[2.] Setzen eines Kreuzes in der Checkbox bei allen Jobs, die heruntergeladen werden sollen.
        \item[3.] Auswahl der entsprechenden Action im entsprechenden Dropdown-Menü.
        \item[4.] Bestätigung des Herunterladens.
    \end{FAList}
    
    
    \phantomsection
    \label{FA:Web-Interface:Anzeigen von Fehlern} 
     \item[F2060] \textbf{Anzeigen von Fehlern} \\
    \begin{FA}
        \textbf{Ziel:} & Es gibt eine Möglichkeit, den \gls{Nutzer} über aufgetretene Fehler zu informieren. \\
        \textbf{Vorbedingung:} & - \\
        \textbf{Nachbedingung (Erfolg):}  & Eine Fehlermeldung wird angezeigt. \\
        \textbf{Nachbedingung (Fehlschlag):} & Es wird keine Fehlermeldung angezeigt. \\
        \textbf{Akteure:} & System \\
        \textbf{Auslösendes Ereignis:} & Ein Fehler ist aufgetreten \\
    \end{FA}
    \textbf{Beschreibung:}
    \begin{FAList} 
        \item[1.] Ein Fehler tritt im System auf.
        \item[2.] Dem \gls{Nutzer} wird ein Fenster mit der Fehlermeldung angezeigt. Dieses Fenster befindet sich im Vordergrund und blockiert das restliche Web-Interface, bis der \gls{Nutzer} die Bestätigungs-Schaltfläche betätigt oder er neben das Fenster klickt, um dieses zu schließen.
    \end{FAList}
    
    
   
    
    
    \phantomsection
    \label{FA:Web-Interface:Anzeigen von Warnungen und Fehlermeldungen}
    \item[F2070] \textbf{Anzeigen von Mallob Warnungen und Fehlermeldungen} \\
    \begin{FA}
        \textbf{Ziel:} & Der Administrator kann die Fehlermeldungen und Warnungen einsehen, die von Mallob ausgegeben werden \\
        \textbf{Vorbedingung:} & Der Administrator muss im Web-Interface angemeldet sein \\
        \textbf{Nachbedingung (Erfolg):} & Es werden die Fehlermeldungen und Warnungen von Mallob angezeigt \\
        \textbf{Nachbedingung (Fehlschlag):} & Es werde die Fehlermeldungen und Warnungen von Mallob nicht angezeigt \\
        \textbf{Akteure:} & Administrator \\
        \textbf{Auslösendes Ereignis:} & Der Administrator möchte die Fehlermeldungen und Warnungen von Mallob sehen \\
    \end{FA}
    \textbf{Beschreibung:}
    \begin{FAList} 
        \item[1.] Navigation zur \hyperref[pages:admin]{Administratoren-Seite}.
        \item[2.] Die Fehlermeldungen und Warnungen werden im Web-Interface angezeigt
    \end{FAList}
    
    
    
    

 
    \phantomsection
    \label{FA:Web-Interface:Einsehen von Job-Informationen}
    \item[F2080] \textbf{Einsehen von Job-Information} \\
    \begin{FA}
        \textbf{Ziel:} & Verschiedene Wege, Informationen zu einem Job anzuzeigen. \\
        \textbf{Vorbedingung:} &  Der \gls{Nutzer} ist angemeldet und besitzt mindestens einen Job. \\
        \textbf{Nachbedingung (Erfolg):}  &  Der \gls{Nutzer} kann Informationen zum gewünschten Job einsehen. \\
        \textbf{Nachbedingung (Fehlschlag):} &  Eine Fehlermeldung wird angezeigt. \\
        \textbf{Akteure:} & \gls{Nutzer} \\
        \textbf{Auslösendes Ereignis:} & \gls{Nutzer} möchte Informationen zu einem Job einsehen. \\
    \end{FA}
    \textbf{Beschreibung:}
    \begin{FAList} 
        \item[1.] Navigation zur Job-Tabelle.
        \item[2.] Auswahl der gewünschten Attribute im Dropdown-Menü über der Liste.
        \item[3.] Die ausgewählten Attribute werden jeweils als eigene Spalte in der Tabelle angezeigt. 
    \end{FAList}
    \textbf{Alternative 1:}
    \begin{FAList}
        \item[1.] Navigation zur Job-Tabelle.
        \item[2.] Anklicken des entsprechenden Jobs in der Tabelle.
        \item[3.] Die Job-Informationen werden aufgeklappten Fenster angezeigt.
    \end{FAList}
    \textbf{Erweiterung der Alternative 1:}
    \begin{FAList}
        \item[4.] Erneutes Anklicken des entsprechenden Jobs in der Tabelle.
        \item[5.] Weiterleitung zur Job-Seite.
        \item[6.] Die Job-Informationen werden auf der Job-Seite angezeigt.
    \end{FAList}
    \textbf{Alternative 2:}
    \begin{FAList}
        \item[1.] Direkter Aufruf der Job-Seite über die entsprechende URL.
        \item[2.] Die Job-Informationen werden auf der Job-Seite angezeigt.
    \end{FAList}
    
    
    \phantomsection
    \label{FA:Web-Interface:Hinzufügen von Spalten}
    \item[F2090] \textbf{Hinzufügen von Spalten in der Job-Tabelle} \\
    \begin{FA}
        \textbf{Ziel:} & Es können Spalten zur Job-Tabelle hinzugefügt werden \\
        \textbf{Vorbedingung:} & Der \gls{Nutzer} muss im Web-Interface angemeldet sein \\
        \textbf{Nachbedingung (Erfolg):} & Die gewünschte Spalte wurde zur Job-Tabelle hinzugefügt  \\
        \textbf{Nachbedingung (Fehlschlag):} & Die gewünschte Spalte wurde nicht hinzugefügt \\
        \textbf{Akteure:} & \gls{Nutzer} \\
        \textbf{Auslösendes Ereignis:} & Der \gls{Nutzer} möchte eine Spalte zur Job-Tabelle hinzufügen \\
    \end{FA}
    \textbf{Beschreibung:}
    \begin{FAList} 
        \item[1.] Navigation zur Job-Tabelle
        \item[2.] Anklicken des Dropdown-Menüs
        \item[3.] Durch Anklicken auswählen, welches Attribut als Spalte hinzugefügt werden soll
    \end{FAList}
    
    
    \phantomsection
    \label{FA:Web-Interface:Entfernen von Spalten}
    \item[F2100] \textbf{Entfernen von Spalten in der Job-Tabelle} \\
    \begin{FA}
        \textbf{Ziel:} & Es können Spalten aus der Job-Tabelle entfernt werden \\
        \textbf{Vorbedingung:} & Der \gls{Nutzer} muss im Web-Interface angemeldet sein und die es wurde mindestens eine Spalte mittels \hyperref[FA:Web-Interface:Hinzufügen von Spalten]{F1111} hinzugefügt \\
        \textbf{Nachbedingung (Erfolg):} & Die gewünschte Spalte wurde aus der Job-Tabelle entfernt \\
        \textbf{Nachbedingung (Fehlschlag):} & Die gewünschte Spalte wurde nicht aus der Job-Tabelle entfernt \\
        \textbf{Akteure:} & \gls{Nutzer} \\
        \textbf{Auslösendes Ereignis:} & Der \gls{Nutzer} möchte eine angezeigte Spalte aus der Tabelle löschen\\
    \end{FA}
    \textbf{Beschreibung:}
    \begin{FAList} 
        \item[1.] Navigation zur Job-Tabelle
        \item[2.] Klicken des \enquote{x}-Symbols in der entsprechenden Spalte
    \end{FAList}
    
 


    \phantomsection
    \label{FA:Web-Interface:Abmelden} 
    \item[F2110] \textbf{Abmelden} \\
    \begin{FA}
        \textbf{Ziel:} & Der \gls{Nutzer} kann sich wieder abmelden. \\
        \textbf{Vorbedingung:} & Der \gls{Nutzer} ist angemeldet. \\
        \textbf{Nachbedingung (Erfolg):}  & Der \gls{Nutzer} ist abgemeldet und wird zur Login-Seite weitergeleitet. \\
        \textbf{Nachbedingung (Fehlschlag):} & Der \gls{Nutzer} ist weiterhin angemeldet und eine Fehlermeldung wird angezeigt. \\
        \textbf{Akteure:} & \gls{Nutzer}\\
        \textbf{Auslösendes Ereignis:} & \gls{Nutzer} möchte sich Abmelden \\
    \end{FA}
    \textbf{Beschreibung:}
    \begin{FAList} 
        \item[1.] Betätigung der entsprechenden Schaltfläche im Navigations-Menü.
    \end{FAList}




\phantomsection
    \label{FA:Web-Interface:Registrieren} 
    \item[F2120] (Wunschkriterium) \textbf{Registrieren} \\
    \begin{FA}
        \textbf{Ziel:} & Ein \gls{Nutzer} ist in der Lage, ein neues Konto zu erstellen.\\
        \textbf{Vorbedingung:} &  Der \gls{Nutzer} ist nicht angemeldet. \\
        \textbf{Nachbedingung (Erfolg):}  &  Ein vorläufiges Konto wird für den \gls{Nutzer} erstellt und er wird zur \hyperref[pages:job-table]{Job-Tabelle} weitergeleitet. \\
        \textbf{Nachbedingung (Fehlschlag):} &  Das Konto kann nicht erstellt werden und es wird eine Fehlermeldung angezeigt. \\
        \textbf{Akteure:} & Person, welche ein neues Konto erstellen möchte. \\
        \textbf{Auslösendes Ereignis:} &  Die Person möchte ein neues Konto erstellen. \\
    \end{FA}
    \textbf{Beschreibung:}
    \begin{FAList}
        \item[1.] Aufrufen des Web-Interfaces
        \item[2.] Auswählen der Schaltfläche \enquote{register}
        \item[3.] Weiterleitung zur \hyperref[pages:register]{Registerung}
        \item[2.] Eingabe des gewünschten Nutzernames
        \item[3.] Eingabe des gewünschten Passwortes
        \item[4.] Eingabe der Wiederholung des Passwortes
        \item[5.] Bestätigung der Eingabe mittels der Schaltfläche \enquote{register}
    \end{FAList}
    
    
     \phantomsection
    \label{FA:Web-Interface:Verifizieren eines Kontos} 
   \item[F2130] (Wunschkriterium) \textbf{Verifizieren eines Kontos} \\
    \begin{FA}
        \textbf{Ziel:} & Ein vorläufiges Konto kann verifiziert werden. \\
        \textbf{Vorbedingung:} &  Ein vorläufiges Konto existiert. \\
        \textbf{Nachbedingung (Erfolg):}  &  Das vorläufige Konto wird zum regulären Konto. \\
        \textbf{Nachbedingung (Fehlschlag):} &  Das vorläufige Konto  wird nicht verändert. \\
        \textbf{Akteure:} & Administrator \\
        \textbf{Auslösendes Ereignis:} & Ein Kunde hat sich erfolgreich registriert. \\
    \end{FA}
    \textbf{Beschreibung:}
    \begin{FAList} 
        \item[1.] Navigation zum \hyperref[pages:admin]{Adminstratoren-Seite} 
        \item[2a.] Bestätigung des Kontos in der Liste über die entsprechende Schaltfläche
        \item[2b.] Keine Bestätigung des Kontos in der Liste über die entsprechende Schaltfläche.
    \end{FAList}
    
    
    \phantomsection
    \label{FA:Web-Interface:Neustart} 
    \item[F2130] (Wunschkriterium) \textbf{Neustart eines abgebrochenen Jobs} \\
    \begin{FA}
        \textbf{Ziel:} & Ein \gls{Nutzer} kann einen abgebrochenen Job neustarten\\
        \textbf{Vorbedingung:} & Der \gls{Nutzer} hat einen abgebrochenen Job \\
        \textbf{Nachbedingung (Erfolg):}  &  Der abgebrochene Job wird bearbeitet \\
        \textbf{Nachbedingung (Fehlschlag):} &  Der abgebrochene Job wird nicht bearbeiten \\
        \textbf{Akteure:} & \gls{Nutzer}\\
        \textbf{Auslösendes Ereignis:} & Der \gls{Nutzer} möchte einen abgebrochenen Job wieder starten \\
    \end{FA}
    \textbf{Beschreibung:}
    \begin{FAList} 
        \item[1.] Navigation zur Job-Tabelle.
        \item[2.] Anklicken des neuzustartenden, abgebrochenen Jobs.
        \item[3.] Auswählen der entsprechenden Schaltfläche in der aufgeklappten Ansicht.
        \item[4.] Weiterleitung zur \hyperref[pages:submit-job]{Seite zum Einreichen von Jobs}, wobei hier die entsprechenden Optionen bereits wieder ausgefüllt sind.
        \item[5.] Verändern der Job-Beschreibung oder der Job-Konfiguration.
        \item[6.] Einreichen des Jobs.
    \end{FAList} 
    \textbf{Alternative der Schritte 1-3:}
    \begin{FAList}
        \item[1.] Navigation zur Job-Seite des neuzustartenden Jobs.
        \item[2.] Auswählen der entsprechenden Schaltfläche.
    \end{FAList}
    \textbf{Alternative von Schritt 5.}
    \begin{FAList}
        \item[5.] Nichts verändern.
    \end{FAList}
    
    
    \phantomsection
    \label{FA:Web-Interface:Verwalten von Malllob}
    \item[F2140] (Wunschkriterium) \textbf{Verwalten von Mallob} \\
    \begin{FA}
        \textbf{Ziel:} & Mallob kann vom Administrator gestartet, gestoppt und neu gestartet werden \\
        \textbf{Vorbedingung:} & Der Administrator muss im Web-Interface angemeldet sein \\
        \textbf{Nachbedingung (Erfolg):} & Mallob ist gestartet, gestoppt oder neu gestartet \\
        \textbf{Nachbedingung (Fehlschlag):} & Mallob ist nicht gestartet, gestoppt oder neu gestartet \\
        \textbf{Akteure:} & Administrator \\
        \textbf{Auslösendes Ereignis:} & Der Administrator will Mallob starten, stoppen oder neu starten \\
    \end{FA}
    \textbf{Beschreibung:}
    \begin{FAList} 
        \item[1.] Navigation zur \hyperref[pages:admin]{Administratoren-Seite}
        \item[2.] Anklicken des Knopfes \enquote{start mallob} 
        \item[3.] Mallob wird gestartet
    \end{FAList}
    \textbf{Alternative 1 zu den Schritten 2 bis 3:}
    \begin{FAList}
        \item[2.] Anklicken des Knopfes \enquote{stop mallob}
        \item[3.] Mallob wird gestoppt
    \end{FAList}
    \textbf{Alternative 2 zu den Schritten 2 bis 3:}
    \begin{FAList}
        \item[2.] Anklicken des Knopfes \enquote{restart mallob}
        \item[3.] Mallob wird neu gestartet
    \end{FAList}
    
    
    \phantomsection
    \label{FA:Web-Interface:Sortieren der Tabelle}
    \item[F2150] (Wunschkriterium) \textbf{Sortieren der Job-Tabelle nach Attributen} \\
    \begin{FA}
        \textbf{Ziel:} & Die Einträge der Job-Tabelle können nach den Attributen der verschiedenen Spalten sortiert werden \\
        \textbf{Vorbedingung:} & Der \gls{Nutzer} muss im Web-Interface angemeldet sein \\
        \textbf{Nachbedingung (Erfolg):} & Die Einträge der Job-Tabelle sind nach dem Wunsch des \glslink{Nutzer}{Nutzers} sortiert \\
        \textbf{Nachbedingung (Fehlschlag):} & Die Einträge der Job-Tabelle sind nicht nach dem Wunsch des \glslink{Nutzer}{Nutzers} sortiert \\
        \textbf{Akteure:} & \gls{Nutzer} \\
        \textbf{Auslösendes Ereignis:} & Der \gls{Nutzer} möchte die Einträge der Job-Tabelle sortieren \\
    \end{FA}
    \textbf{Beschreibung:}
    \begin{FAList} 
        \item[1.] Navigation zur Job-Tabelle
        \item[2.] Anklicken der Spalte, nach deren Attribut die Tabelle sortiert werden soll
        \item[3.a.] Zeigt der Pfeil neben dem Attribut nach unten, werden die Einträge absteigend, bzw. alphabetisch sortiert
        \item[3.b.] Zeigt der Pfeil neben dem Attribut nach oben, werden die Einträge aufsteigend, bzw. umgekehrt alphabetisch sortiert
    \end{FAList}
    
       \phantomsection
    \label{FA:Web-Interface:Filtern für Admins}
    \item[F2160] (Wunschkriterium) \textbf{Filtern der Job-Tabelle für Administratoren} \\
    \begin{FA}
        \textbf{Ziel:} & Administratoren können wahlweise die Jobs aller \gls{Nutzer} oder nur die eigenen in der Tabelle sehen \\
        \textbf{Vorbedingung:} & Ein Administrator ist angemeldet \\
        \textbf{Nachbedingung (Erfolg):} & Der Filter wurde korrekt angewandt, es sind nur die gewünschten Jobs zu sehen \\
        \textbf{Nachbedingung (Fehlschlag):} & Der Filter wurde nicht angewandt, eine Fehlermeldung wird ausgegeben \\
        \textbf{Akteure:} & \gls{Nutzer} \\
        \textbf{Auslösendes Ereignis:} & Ein Administrator möchte ändern, welche Jobs er in der Job-Tabelle sieht.\\
    \end{FA}
    \textbf{Beschreibung:}
    \begin{FAList} 
        \item[1.] Navigation zur Job-Tabelle
        \item[2.] Setzen der checkbox \enquote{see all jobs}
    \end{FAList}
    
    
    \phantomsection
    \label{FA:Web-Interface:Anzeigen von Plugins}
    \item[F2170] (Wunschkriterium) \textbf{Anzeigen von Plugins} \\
    \begin{FA}
        \textbf{Ziel:} & Das Web-Interface kann extern erstelle Plugins anzeigen. \\
        \textbf{Vorbedingung:} & - \\
        \textbf{Nachbedingung (Erfolg):}  &  Plugins werden als Dropdown-Menü in der Navigations-Leiste angezeigt und können ausgewählt werden.\\
        \textbf{Nachbedingung (Fehlschlag):} & Plugins werden nicht angezeigt. \\
        \textbf{Akteure:} & \gls{Nutzer} \\
        \textbf{Auslösendes Ereignis:} &  Mindestens Plugin wurde eingelesen\\
    \end{FA}
    \textbf{Beschreibung:}
    \begin{FAList} 
        \item[1.] Einlesen von Plugins durch \hyperref[FA:System:Einstellungen festlegen]{F4000}
        \item[2.] Anzeigen eines Eintrags für Plugins in der Navigations-Leiste
    \end{FAList}
    
    
  \end{itemize}
%-------------------------------------------------------------------
%--------------------Visualisierung---------------------------------
%-------------------------------------------------------------------
\pagebreak

\subsection{Visualisierung}
Die Visualisierung findet Web-Interface statt und ist nur dort einsehbar.


\begin{itemize}
    \setlength\itemsep{4em}



    %----------------------Visualisierung - Anzeigen des Systemzustandsja lles 
    
    \phantomsection
    \label{FA:Visualisierung:Anzeigen des Systemzustandes}
    \item[F3000] \textbf{Anzeigen des aktuellen Systemzustands} \\
    \begin{FA}
        \textbf{Ziel:} & Der \gls{Nutzer} kann eine Visualisierung des aktuellen Systemzustands von Mallob sehen \\
        \textbf{Vorbedingung:} & Der \gls{Nutzer} muss im Web-Interface angemeldet sein \\
        \textbf{Nachbedingung (Erfolg):} & Der Systemzustand von Mallob wird im Web-Interface angezeigt \\
        \textbf{Nachbedingung (Fehlschlag):} &  Es wird eine Fehlermeldung im Web-Interface angezeigt \\
        \textbf{Akteure:} & \gls{Nutzer} \\
        \textbf{Auslösendes Ereignis:} & Der \gls{Nutzer} möchte die Visualisierung des Systemzustands sehen \\
    \end{FA}
    \textbf{Beschreibung:}
    \begin{FAList} 
        \item[1.] Navigation zur Visualisierung im Web-Interface
        \item[2.a.] Wenn die Visualisierung erfolgreich geladen wurde, wird diese im Web-Interface angezeigt  
        \item[3.b.] Wenn die Visualisierung nicht erfolgreich geladen wurde, wird eine Fehlermeldung im Web-Interface angezeigt 
    \end{FAList}
    
    
    
     %-----------------Visualisierung - Ansehen von Details
    \phantomsection
    \label{FA:Visualisierung:Anzeigen von Details} 
    \item[F3010] \textbf{Anzeigen von Details} \\
    \begin{FA}
        \textbf{Ziel:} & Durch Anklicken eines Jobs in der Visualisierung können Details zum Job eingesehen werden \\
        \textbf{Vorbedingung:} & Der \gls{Nutzer} ist im Web-Interface angemeldet \\
        \textbf{Nachbedingung (Erfolg):} & Es werden Details zum ausgewählten Jobs angezeigt \\
        \textbf{Nachbedingung (Fehlschlag):} & Es wird eine Fehlermeldung angezeigt und es werden keine Details angezeigt \\
        \textbf{Akteure:} & \gls{Nutzer} \\
        \textbf{Auslösendes Ereignis:} & Der \gls{Nutzer} möchte die Visualisierung pausieren \\
    \end{FA}
    \textbf{Beschreibung:}
    \begin{FAList} 
        \item[1.] Navigation zur Visualisierung im Web-Interface
        \item[2.] Anklicken eines Jobs im linken Panel der Visualisierung
        \item[3] Die Details werden im rechten Panel angezeigt. Ist Job keiner der eigenen und der \gls{Nutzer} kein Administrator, so werden die Datails pseudonymisiert dargestellt. Ebenso werden beiden Schaltflächen nicht angezeigt.
    \end{FAList}
  
    
    \item[F3020] \textbf{Anzeigen von Job-Details}
    
    
    %-----------------Visualisierung - Pausieren der Visualisierung
    \phantomsection
    \label{FA:Visualisierung:Pausieren der Visualisierung} 
    \item[F3020] \textbf{Pausieren der Visualisierung} \\
    \begin{FA}
        \textbf{Ziel:} & Die Visualisierung kann pausiert werden \\
        \textbf{Vorbedingung:} & Der \gls{Nutzer} muss im Web-Interface angemeldet sein und die ist  die Visualisierung läuft bereits \\
        \textbf{Nachbedingung (Erfolg):} & Der \gls{Nutzer} muss im Web-Interface angemeldet sein und die Visualisierung ist pausiert und aktualisiert sich nicht mehr \\
        \textbf{Nachbedingung (Fehlschlag):} & Die Visualisierung ist nicht pausiert und läuft weiter \\
        \textbf{Akteure:} & \gls{Nutzer} \\
        \textbf{Auslösendes Ereignis:} & Der \gls{Nutzer} möchte die Visualisierung pausieren \\
    \end{FA}
    \textbf{Beschreibung:}
    \begin{FAList} 
        \item[1.] Navigation zur Visualisierung im Web-Interface
        \item[2.] Anklicken der Pause-Taste
        \item[3.] Die Visualisierung pausiert und wird nicht mehr aktualisiert
    \end{FAList}
    
    
    %--------------------Visualisierung - Starten der Visualisierung
    \phantomsection
    \label{FA:Visualisierung:Starten der Visualisierung} 
    \item[F3030] \textbf{Starten der Visualisierung} \\
    \begin{FA}
        \textbf{Ziel:} & Die Visualisierung kann nach dem pausieren wieder gestartet werden \\
        \textbf{Vorbedingung:} & Der \gls{Nutzer} ist im Web-Interface angemeldet und die Visualisierung ist bereits pausiert \\
        \textbf{Nachbedingung (Erfolg):} & Die Visualisierung läuft wieder und wird aktualisiert \\
        \textbf{Nachbedingung (Fehlschlag):} & Die Visualisierung ist weiterhin pausiert \\
        \textbf{Akteure:} & \gls{Nutzer} \\
        \textbf{Auslösendes Ereignis:} & Der \gls{Nutzer} möchte die Visualisierung wieder starten \\
    \end{FA}
    \textbf{Beschreibung:}
    \begin{FAList} 
        \item[1.] Navigation zur Visualisierung im Web-Interface
        \item[2.] Anklicken der Wiedergabe-Taste
        \item[3.] Die Visualisierung startet und wird wieder aktualisiert
    \end{FAList}
    
    
    
    
    
    %-------------Visualisierung - Springen zu einem bestimmten Zeitpunkt
    \phantomsection
    \label{FA:Visualisierung:Springen} 
    \item[F3040] \textbf{Springen zu einem bestimmten Zeitpunkt der Visualisierung} \\
    \begin{FA}
        \textbf{Ziel:} & Der \gls{Nutzer} kann zu einem beliebigen Zeitpunkt in der Visualisierung vor- oder zurückspringen \\
        \textbf{Vorbedingung:} & Der \gls{Nutzer} muss im Web-Interface angemeldet sein und Mallob muss gestartet sein \\
        \textbf{Nachbedingung (Erfolg):} & Die Visualisierung wird ab dem gewählten Zeitpunkt abgespielt \\
        \textbf{Nachbedingung (Fehlschlag):} & Die Visualisierung wird nicht ab dem gewählten Zeitpunkt abgespielt  \\
        \textbf{Akteure:} & \gls{Nutzer} \\
        \textbf{Auslösendes Ereignis:} & Der \gls{Nutzer} möchte die Visualisierung an einem bestimmten Zeitpunkt ansehen \\
    \end{FA}
    \textbf{Beschreibung:}
    \begin{FAList} 
        \item[1.] Navigation zur Visualisierung
        \item[2.] Wählen des gewünschte Zeitpunktes durch Anklicken oder Ziehen des Sliders zur gewünschten Position
        \item[3.] Die Visualisierung wird ab dem gewählten Zeitpunkt abgespielt
    \end{FAList}
    
    
    
    
    %----------------Visualisierung - Ändern der Wiedergabegeschwindigkeit
    \phantomsection
    \label{FA:Visualisierung:Aendern der Wiedergabegeschwindigkeit} 
    \item[F3050] (Wunschkriterium) \textbf{Ändern der Wiedergabegeschwindigkeit} \\
    \begin{FA}
        \textbf{Ziel:} & Die Geschwindigkeit, mit der die Visualisierung abgespielt wird kann geändert werden \\
        \textbf{Vorbedingung:} & Der \gls{Nutzer} ist im Web-Interface angemeldet \\
        \textbf{Nachbedingung (Erfolg):} & Die Visualisierung wird in der gewünschten Geschwindigkeit abgespielt \\
        \textbf{Nachbedingung (Fehlschlag):} & Die Visualisierung wird weiterhin in der alten Geschwindigkeit abgespielt \\
        \textbf{Akteure:} & \gls{Nutzer} \\
        \textbf{Auslösendes Ereignis:} & Der \gls{Nutzer} möchte die Wiedergabegeschwindigkeit der Visualisierung ändern \\
    \end{FA}
    \textbf{Beschreibung:}
    \begin{FAList} 
        \item[1.] Navigation zur Visualisierung im Web-Interface
        \item[2.] Eingabe der gewünschten Wiedergabegeschwindigkeit in das Feld \enquote{replay speed}
        \item[3.a.] Die Visualisierung wird in der gewünschten Wiedergabegeschwindigkeit abgespielt
        \item[3.b.] Wenn die gewählte Wiedergabegeschwindigkeit schneller als Echtzeit ist und sich die Visualisierung am aktuellsten Punkt befindet, wird die Wiedergabegeschwindigkeit auf Echtzeit gesetzt
        \item[3.c.] Wenn die gewählte Wiedergabegeschwindigkeit negativ ist, wird die Visualisierung rückwärts in der gewünschten Wiedergabegeschwindigkeit abgespielt
    \end{FAList}
    
    
      %---------------Visualisierung - Anzeigen des Binärbaumes für einen Job
    
    \phantomsection
    \label{FA:Visualisierung:Anzeigen des Binaerbaumes für einen Job}
    \item[F3060] (Wunschkriterium) \textbf{Anzeigen des Binärbaumes für einen Job} \\
    \begin{FA}
        \textbf{Ziel:} & Visualisierung des zu einem Job gehörenden Binärbaumes \\
        \textbf{Vorbedingung:} & Der \gls{Nutzer} muss im Web-Interface angemeldet sein und muss einen Job eingereicht oder abgeschlossen haben \\
        \textbf{Nachbedingung (Erfolg):} & Der Binärbaum zu dem gewünschten Job wird im Web-Interface angezeigt \\
        \textbf{Nachbedingung (Fehlschlag):} & Der Binärbaum wird nicht im Web-Interface angezeigt  \\
        \textbf{Akteure:} & \gls{Nutzer} \\
        \textbf{Auslösendes Ereignis:} & Der \gls{Nutzer} klickt einen Job in der Visualisierung an \\
    \end{FA}
    \textbf{Beschreibung:}
    \begin{FAList} 
        \item[1.] Navigation zur Visualisierung im Web-Interface
        \item[2.] Anklicken eines Jobs im linken Panel der Visualisierung
        \item[3.] Der Binärbaum zu dem ausgewählten Job wird im Web-Interface angezeigt
    \end{FAList}
    
\end{itemize}


%-------------------------------------------------------------------
%----------------------------System---------------------------------
%-------------------------------------------------------------------
\pagebreak

\subsection{System}
    \setlength\itemsep{4em}




\begin{itemize}
    \phantomsection
    \label{FA:System:Einstellungen festlegen}
    \item[F4000] \textbf{Festlegen von Einstellungen mittels einer Konfigurationsdatei bei Systemstart} \\
    \begin{FA}
        \textbf{Ziel:} & Bestimmte Einstellungen können mit einer Konfigurationsdatei konfiguriert werden, welche bei Systemstart eingelesen werden. \\
        \textbf{Vorbedingung:} & Eine korrekt formatierte Konfigurationsdatei existiert an einem fest vorgeschriebenen Ort im Dateisystem.\\
        \textbf{Nachbedingung (Erfolg):}  & Die Einstellungen der Konfigurationsdatei sind angewandt \\
        \textbf{Nachbedingung (Fehlschlag):} & Die Konfigurationsdatei kann nicht geladen werden, das Programm beendet sich mit einer entsprechenden Fehlermeldung.\\
        
        \textbf{Akteure:} & System\\
        \textbf{Auslösendes Ereignis:} & Starten des Systems
    \end{FA}
    \textbf{Beschreibung:}
    \begin{FAList} 
        \item[1.] Starten des Systems.
        \item[2.] Konfigurationsdatei wird beim Start automatisch eingelesen.
    \end{FAList} 

  
    \phantomsection
    \label{FA:System:Einlesen von Plugins bei Systemstart}
    \item[F4010] (Wunschkriterium) \textbf{Einlesen von Plugins bei Systemstart} \\
    \begin{FA}
        \textbf{Ziel:} & Das System kann extern erstellte Plugins einlesen. \\
        \textbf{Vorbedingung:} & Es liegen ein einzulesende Plugins vor \\
        \textbf{Nachbedingung (Erfolg):}  & Die Plugins wurden eingelesen. \\
        \textbf{Nachbedingung (Fehlschlag):} & Ein oder mehrere Plugins konnten nicht eingelesen werden und auf der \hyperref[pages:admin]{Administratoren-Seite} werden mittels \hyperref[A:Web-Interface:Anzeigen von Warungen und Fehlermeldungen]{F2110} \\
        \textbf{Akteure:} & System \\
        \textbf{Auslösendes Ereignis:} & Starten des Systems \\
    \end{FA}
    \textbf{Beschreibung:}
    \begin{FAList} 
        \item[1.] Starten des Systems.
        \item[2.] Plugins werden beim Start eingelesen.
    \end{FAList} 


\end{itemize}

