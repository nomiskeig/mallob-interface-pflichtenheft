
\section{Funktionale Anforderungen}


\subsection{API}


\begin{itemize}[nosep]
    \setlength\itemsep{4em}
    
    
    
  
    %-----------------API - Authentifizierung von Nutzern
    \phantomsection
    \label{FA:API:Authentifizieren von Nutzern}
    \item[F1000] \textbf{Authentifizieren von Nutzern}\\
    
    \begin{FA}
        \textbf{Ziel:} & Ein \gls{Nutzer} kann sich mit seinem Nutzernamen und Passwort authentifizieren. \\
        \textbf{Vorbedingung:} & Der \gls{Nutzer} ist registriert. \\
        \textbf{Nachbedingung (Erfolg):} & Der \gls{Nutzer} hat einen \gls{Authentifizierungstoken}, der ihm Zugriff auf die Funktionen der \gls{API} ermöglicht, erhalten.\\
        \textbf{Nachbedingung (Fehlschlag):} & Der \gls{Nutzer} hat keinen \gls{Authentifizierungstoken} und erhält eine Fehlermeldung. \\
         \textbf{Akteure:} & \gls{Nutzer} \\
        \textbf{Auslösendes Ereignis:} & \gls{Nutzer} möchte Funktionen der \gls{API} verwenden und muss sich dafür authentifizieren. \\
    \end{FA}
    \textbf{Beschreibung:}
    \begin{FAList}
        \item[1.] Der \gls{Nutzer} schickt eine Anfrage zur Authentifizierung an die \gls{API}, die seinen Nutzernamen und sein Passwort enthält.
        \item[2.] Die Anfrage wird von der \gls{API} verarbeitet und der \gls{Nutzer} wird authentifiziert.
    \end{FAList}
    
    

    
    %-----------------API - Einreichen von Jobs
    \phantomsection
    \label{FA:API:Einreichen von Jobs} 
    \item[F1010] \textbf{Einreichen von Jobs} \\
    \begin{FA}
        \textbf{Ziel:} & Ein \gls{Nutzer} kann einen \hyperref[B:Jobs]{Job} an die \gls{API} übergeben, der von Mallob bearbeitet wird .\\
        \textbf{Vorbedingung:} & Der \gls{Nutzer} hat sich mittels \hyperref[FA:API:Authentifizieren von Nutzern]{F1000} authentifiziert. \\
        \textbf{Nachbedingung (Erfolg):} & Der \gls{Nutzer} erhält eine Bestätigung und der \hyperref[B:Jobs]{Job} wird bearbeitet.\\
        \textbf{Nachbedingung (Fehlschlag):} & Der \gls{Nutzer} erhält eine Fehlermeldung. \\
        \textbf{Akteure:} & \gls{Nutzer} \\
        \textbf{Auslösendes Ereignis:} & Der \gls{Nutzer} möchte  einen \hyperref[B:Jobs]{Job} einreichen. \\
    \end{FA}
    \textbf{Beschreibung:}
    \begin{FAList}
            \item[1.a.] Der \gls{Nutzer} schickt eine Anfrage an die \gls{API}, die sowohl die \hyperref[B:Job-Konfiguration]{Job-Konfiguration} als auch die \hyperref[B:Job-Beschreibung]{Job-Beschreibung} enthält, wobei die \hyperref[B:Job-Beschreibung]{Job-Beschreibung} in einer separaten Datei spezifiziert ist.
            \item[1.b.] Der \gls{Nutzer} schickt eine Anfrage an die \gls{API}, die sowohl die \hyperref[B:Job-Konfiguration]{Job-Konfiguration} des \hyperref[B:Jobs]{Jobs} als auch die \hyperref[B:Job-Beschreibung]{Job-Beschreibung} enthält, wobei in diesem Fall beides in  einer einzelnen Datei spezifiziert wird.
            \item[1.c.] (gewünscht) Der \gls{Nutzer} schickt eine Anfrage an die \gls{API}, die sowohl die \hyperref[B:Job-Konfiguration]{Job-Konfiguration} und einen Link  zu einer Datei enthält, in der die \hyperref[B:Job-Beschreibung]{Job-Beschreibung} spezifiziert ist.
            \item[1.d.] (gewünscht) Der \gls{Nutzer} schickt eine Anfrage an die \gls{API}, die eine Referenz auf einen bereits eingereichten \hyperref[B:Jobs]{Job} enthält, der nochmal ausgeführt werden soll.
            \item[2.] Der eingereichte \hyperref[B:Jobs]{Job} wird von Mallob bearbeitet.
    \end{FAList}
    

    
    %-----------------API - Abbrechen von Jobs
   \phantomsection
    \label{FA:API:Abbrechen von eingereichten Jobs}  
    \item[F1020] \textbf{Abbrechen von eingereichten Jobs} \\
    \begin{FA}
        \textbf{Ziel:} & Ein \gls{Nutzer} kann einen oder mehrere eingereichte \hyperref[B:Jobs]{Jobs} wieder abbrechen. \\
        \textbf{Vorbedingung:} & Der \gls{Nutzer} hat sich mittels \hyperref[FA:API:Authentifizieren von Nutzern]{F1000} authentifiziert und hat mindestens einen laufenden Job.\\
        \textbf{Nachbedingung (Erfolg):} & Die \hyperref[B:Jobs]{Jobs} wurden abgebrochen und der \gls{Nutzer} hat die Teil-\hyperref[B:Job-Ergebnis]{Ergebnisse} der abgebrochenen \hyperref[B:Jobs]{Jobs} erhalten. \\
        \textbf{Nachbedingung (Fehlschlag):} & Die \hyperref[B:Jobs]{Jobs} wurden nicht abgebrochen und der \gls{Nutzer} hat eine Fehlermeldung erhalten.\\
        \textbf{Akteure:} & \gls{Nutzer} \\
        \textbf{Auslösendes Ereignis:} & Der \gls{Nutzer} möchte den \hyperref[B:Jobs]{Job} abbrechen.\\
    \end{FA}
    \textbf{Beschreibung:}
    \begin{FAList}
            \item[1.] Der \gls{Nutzer} schickt eine Anfrage an die \gls{API}, in welcher spezifiziert ist, welche \hyperref[B:Jobs]{Jobs} abgebrochen werden sollen. Es ist auch möglich alle eigenen, noch laufenden \hyperref[B:Jobs]{Jobs} abzubrechen.
            \item[2.] Die \hyperref[B:Jobs]{Jobs} werden abgebrochen.
    \end{FAList}
    

  

    
    
    %--------------------------API - Abfrage für inkrementelle Jobs
    %\phantomsection
    %\label{FA:API:Abfragen der Fertigstellung eines inkrementellen Jobs}  
    %\item[F1070] \textbf{Abfragen der Fertigstellung eines inkrementellen Jobs} \\
    %\begin{FA}
    %    \textbf{Ziel:} & Es wird abgefragt, ob ein inkrementeller \hyperref[B:Jobs]{Job} abgeschlossen ist oder ob Teil-\hyperref[B:Jobs]{Jobs} existieren, die noch bearbeitet werden müssen \\
    %    \textbf{Vorbedingung:} & Der \gls{Nutzer} ist authentifiziert (siehe F20) es wurde ein inkrementeller \hyperref[B:Jobs]{Job} eingereicht \\
    %    \textbf{Nachbedingung (Erfolg):} & Der \gls{Nutzer} hat eine Antwort erhalten, ob der \hyperref[B:Jobs]{Job} abgeschlossen ist oder nicht \\
    %    \textbf{Nachbedingung (Fehlschlag):} & Der \gls{Nutzer} hat eine Fehlermeldung erhalten \\
    %    \textbf{Akteure:} & \gls{Nutzer} \\
    %    \textbf{Auslösendes Ereignis:} & Der \gls{Nutzer} möchte herausfinden, ob ein inkrementeller \hyperref[B:Jobs]{Job} abgeschlossen ist \\
    %\end{FA}
    %\textbf{Beschreibung:}
    %\begin{FAList} 
    %    \item[1.] Der \gls{Nutzer} schickt eine Anfrage an die API mit der Job-ID des inkrementellen Jobs, der abgefragt werden soll
    %    \item[2.] Der Bearbeitungsstand des inkrementellen \hyperref[B:Jobs]{Jobs} wird abgefragt
    %    \item[3.a] Wenn die Abfrage erfolgreich war, wird das Ergebnis an den \gls{Nutzer} zurückgegeben
    %    \item[3.b.] Wenn die Abfrage nicht erfolgreich war, wird eine Fehlermeldung an den \gls{Nutzer} zurückgegeben
    %\end{FAList}
    
    
    %--------------------------API - Abfragen der Daten eines einzelnen Jobs
    \phantomsection
    \label{FA:API:Abfragen der Informationenen von Jobs}  
    \item[F1030] \textbf{Ausgeben der \hyperref[B:Job-Informationen]{Informationen von Jobs}}  \\
    \begin{FA}
        \textbf{Ziel:} & Es können die \hyperref[B:Job-Informationen]{Informationen} von \hyperref[B:Jobs]{Jobs} ohne die \hyperref[B:Job-Beschreibung]{Job-Beschreibung} ausgegeben werden.\\
        \textbf{Vorbedingung:} & Der \gls{Nutzer} ist gemäß \hyperref[FA:API:Authentifizieren von Nutzern]{F1000} authentifiziert und die gewünschten \hyperref[B:Jobs]{Jobs} wurden bereits eingereicht. \\
        \textbf{Nachbedingung (Erfolg):} & Der \gls{Nutzer} hat die \hyperref[B:Job-Informationen]{Informationen} zu den angefragten \hyperref[B:Jobs]{Jobs} erhalten, aber nicht deren \hyperref[B:Job-Beschreibung]{Job-Beschreibung}en. \\
        \textbf{Nachbedingung (Fehlschlag):} &  Der \gls{Nutzer} hat eine Fehlermeldung erhalten. \\
        \textbf{Akteure:} & \gls{Nutzer} \\
        \textbf{Auslösendes Ereignis:} & Der \gls{Nutzer} möchte \hyperref[B:Job-Informationen]{Informationen} über eingereichte \hyperref[B:Jobs]{Jobs} erhalten. \\
    \end{FA}
    \textbf{Beschreibung:}
    \begin{FAList} 
        \item[1.] Der \gls{Nutzer} stellt eine Anfrage an die \gls{API}. In dieser ist spezifiziert, über welche \hyperref[B:Jobs]{Jobs} \hyperref[B:Job-Informationen]{Informationen} ausgegeben werden sollen. Es steht dem Nutzer frei, ob er \hyperref[B:Job-Informationen]{Informationen} zu bestimmten Jobs, allen eigenen \hyperref[B:Jobs]{Jobs} oder allen \hyperref[B:Jobs]{Jobs} im System erhalten möchte.
        \item[2.] Die \hyperref[B:Job-Informationen]{Informationen} der \hyperref[B:Jobs]{Jobs} werden abgefragt.
        \item[3.a.] Wenn die Abfrage erfolgreich war, werden die \hyperref[B:Job-Informationen]{Informationen} an den \gls{Nutzer} zurückgegeben.
        \item[3.b.] Wenn die Abfrage nicht erfolgreich war, wird eine Fehlermeldung an den \gls{Nutzer} zurückgegeben.
    \end{FAList}
    
    
    \phantomsection
    \label{FA:API:Ausgeben eines Systemzustandes}
    \item[F1040] \textbf{Ausgeben eines \hyperref[B:Systemzustand]{Systemzustands}s} \\
    \begin{FA}
        \textbf{Ziel:} & Es kann ein beliebiger vergangener \hyperref[B:Systemzustand]{Systemzustand} abgefragt werden. \\
        \textbf{Vorbedingung:} & Der \gls{Nutzer} ist gemäß \hyperref[FA:API:Authentifizieren von Nutzern]{F1010} authentifiziert. \\
        \textbf{Nachbedingung (Erfolg):} & Der \gls{Nutzer} hat den gewünschten \hyperref[B:Systemzustand]{Systemzustand} erhalten .\\
        \textbf{Nachbedingung (Fehlschlag):} &  Der \gls{Nutzer} hat eine Fehlermeldung erhalten .\\
        \textbf{Akteure:} & \gls{Nutzer} \\
        \textbf{Auslösendes Ereignis:} & Einen vergangenen \hyperref[B:Systemzustand]{Systemzustand} einsehen. \\
    \end{FA}
     \textbf{Beschreibung:}
    \begin{FAList} 
        \item[1.] Der \gls{Nutzer} stellt eine Anfrage an die \gls{API} mit dem gewünschten Zeitpunkt..
        \item[2.] Der \hyperref[B:Systemzustand]{Systemzustand} wird abgefragt.
    \end{FAList} 
    
    
    
    
    \phantomsection
    \label{FA:API:Ausgeben von vergangenen Events}
    \item[F1050] \textbf{Ausgeben von vergangenen \hyperref[B:Event]{Events}} \\
    \begin{FA}
        \textbf{Ziel:} & Es können vergangene \hyperref[B:Event]{Events} ausgegeben werden.\\
        \textbf{Vorbedingung:} & Der \gls{Nutzer} ist gemäß \hyperref[FA:API:Authentifizieren von Nutzern]{F1010} authentifiziert und die gewünschten \hyperref[B:Jobs]{Jobs} wurden bereits eingereicht. \\
        \textbf{Nachbedingung (Erfolg):} & Der \gls{Nutzer} hat die gewünschten \hyperref[B:Event]{Events} zu den angefragten \hyperref[B:Jobs]{Jobs} erhalten. \\
        \textbf{Nachbedingung (Fehlschlag):} &  Der \gls{Nutzer} hat eine Fehlermeldung erhalten. \\
        \textbf{Akteure:} & \gls{Nutzer} \\
        \textbf{Auslösendes Ereignis:} & Der \gls{Nutzer} möchte \hyperref[B:Event]{Events} von eingereichte \hyperref[B:Jobs]{Jobs} erhalten. \\
    \end{FA}
     \textbf{Beschreibung:}
    \begin{FAList} 
        \item[1.] Der \gls{Nutzer} stellt eine Anfrage an die \gls{API}. In dieser ist spezifiziert, von welchen \hyperref[B:Jobs]{Jobs} \hyperref[B:Event]{Events} ausgegeben werden sollen und aus welcher Zeit diese \hyperref[B:Event]{Events} stammen sollen.  Es steht dem Nutzer frei, ob er \hyperref[B:Event]{Events} zu bestimmten Jobs, allen eigenen \hyperref[B:Jobs]{Jobs} oder allen \hyperref[B:Jobs]{Jobs} im System erhalten möchte.
        \item[2.] Die entsprechenden \hyperref[B:Event]{Events} werden abgefragt.
    \end{FAList} 
    
    %---------------API - Ergebnisdatei anfordern
    \phantomsection
    \label{FA:API:Ausgeben des Ergebnisses für eine oder mehrere Jobs}  
    \item[F1060] \textbf{Ausgeben des \hyperref[B:Job-Ergebnis]{Ergebnisses} für eine oder mehrere Jobs} \\
    \begin{FA}
        \textbf{Ziel:} & Der Nutzer hat die Möglichkeit, \hyperref[B:Job-Ergebnis]{Ergebnisse} seiner \hyperref[B:Jobs]{Jobs} abzufragen. \\
        \textbf{Vorbedingung:} & Der \gls{Nutzer} ist authentifiziert und die angeforderte \hyperref[B:Jobs]{Jobs} wurden eingereicht und bearbeitet. \\
        \textbf{Nachbedingung (Erfolg):} & Der \gls{Nutzer} hat die \hyperref[B:Job-Ergebnis]{Ergebnisse} der spezifizierten \hyperref[B:Jobs]{Jobs} erhalten. \\
        \textbf{Nachbedingung (Fehlschlag):} & Der \gls{Nutzer} hat eine Fehlermeldung erhalten.  \\
        \textbf{Akteure:} & \gls{Nutzer} \\
        \textbf{Auslösendes Ereignis:} & Der \gls{Nutzer} möchte die \hyperref[B:Job-Ergebnis]{Ergebnisse} von einem oder mehreren \hyperref[B:Jobs]{Jobs} abfragen. \\
    \end{FA}
    \textbf{Beschreibung:}
    \begin{FAList} 
        \item[1.] Der \gls{Nutzer} schickt eine Anfrage an die \gls{API}, in welcher spezifiziert ist, von welchen \hyperref[B:Jobs]{Jobs} das \hyperref[B:Job-Ergebnis]{Ergebnis} zurückgegeben werden soll. Es ist auch möglich, die \hyperref[B:Job-Ergebnis]{Ergebnisse} aller eigenen \hyperref[B:Jobs]{Jobs} abzufragen.
        \item[2.a.] Wenn die Anfrage erfolgreich war, werden die \hyperref[B:Job-Ergebnis]{Ergebnisse} der \hyperref[B:Jobs]{Jobs} an den \gls{Nutzer} zurückgegeben. 
        \item[2.b.] Wenn die Anfrage nicht erfolgreich war, wird eine Fehlermeldung an den \gls{Nutzer} zurückgegeben.
    \end{FAList}
    
    
    %---------------API - Job-Beschreibung ausgeben
    \phantomsection
    \label{FA:API:Ausgeben der Job-Beschreibung}  
    \item[F1070] \textbf{Ausgeben der \hyperref[B:Job-Beschreibung]{Job-Beschreibung} für einen oder mehrere Jobs} \\
    \begin{FA}
        \textbf{Ziel:} & Der Nutzer hat die Möglichkeit, \hyperref[B:Job-Beschreibung]{Beschreibungen} seiner \hyperref[B:Jobs]{Jobs} abzufragen. \\
        \textbf{Vorbedingung:} & Der \gls{Nutzer} ist authentifiziert und die angegebenen \hyperref[B:Jobs]{Jobs} wurden bereits eingereicht. \\
        \textbf{Nachbedingung (Erfolg):} & Der \gls{Nutzer} hat die \hyperref[B:Job-Beschreibung]{Job-Beschreibung}en der spezifizierten \hyperref[B:Jobs]{Jobs} erhalten. \\
        \textbf{Nachbedingung (Fehlschlag):} & Der \gls{Nutzer} hat eine Fehlermeldung erhalten. \\
        \textbf{Akteure:} & \gls{Nutzer} \\
        \textbf{Auslösendes Ereignis:} & Der \gls{Nutzer} möchte die \hyperref[B:Job-Beschreibung]{Job-Beschreibung}en von einem oder mehreren \hyperref[B:Jobs]{Jobs} abfragen. \\
    \end{FA}
    \textbf{Beschreibung:}
    \begin{FAList} 
        \item[1.] Der \gls{Nutzer} schickt eine Anfrage an die \gls{API}, in der er spezifiziert, von welchen \hyperref[B:Jobs]{Jobs} die Beschreibung zurückgegeben werden soll. Es ist auch möglich, die \hyperref[B:Job-Beschreibung]{Beschreibungen} aller eigenen \hyperref[B:Jobs]{Jobs} abzufragen.
        \item[2.a.] Wenn die Anfrage erfolgreich war, werden die \hyperref[B:Job-Beschreibung]{Job-Beschreibung}en an den \gls{Nutzer} zurückgegeben. 
        \item[2.b.] Wenn die Anfrage nicht erfolgreich war, wird eine Fehlermeldung an den \gls{Nutzer} zurückgegeben. 
    \end{FAList}
    
    
    %--------------API - Abfragen der Informationen von Mallob
    \phantomsection
    \label{FA:API:Abfragen der Informationen von Mallob}  
    \item[F1080] \textbf{Abfragen der Informationen von Mallob} \\
    \begin{FA}
        \textbf{Ziel:} & Der \gls{Administrator} kann den Status und weitere Informationen, unter anderem Warnungen, zu Mallob abrufen. \\
        \textbf{Vorbedingung:} & Der \gls{Administrator} muss authentifiziert sein.\\
        \textbf{Nachbedingung (Erfolg):} & Der \gls{Administrator} hat die Informationen zu Mallob erhalten. \\
        \textbf{Nachbedingung (Fehlschlag):} & Der \gls{Administrator} hat eine Fehlermeldung erhalten. \\
        \textbf{Akteure:} & \gls{Administrator} \\
        \textbf{Auslösendes Ereignis:} & Der \gls{Administrator} möchte genauere Informationen zu Mallob haben. \\
    \end{FA}
    \textbf{Beschreibung:}
    \begin{FAList} 
        \item[1.] Der \gls{Administrator} schickt eine entsprechende Anfrage an die \gls{API}.
        \item[2.a.] Wenn die Anfrage erfolgreich war, werden die Daten an den Nutzer zurückgegeben.
        \item[2.b.] Wenn die Anfrage nicht erfolgreich war, wird eine Fehlermeldung an den Nutzer zurückgegeben.
    \end{FAList}
    
    
    %-----------API - Ausgeben eines Event-Streams von Mallob
    \phantomsection

    \label{FA:API:Ausgeben eines Event-Streams von Mallob}
    \item[F1090] \textbf{Ausgeben eines \hyperref[B:Event]{Event}-\gls{Stream}s von Mallob} \\
    \begin{FA}
        \textbf{Ziel:} & Der \gls{Nutzer} hat Zugriff auf einen \hyperref[B:Event]{Event}-\gls{Stream}, über den kontinuierlich die \hyperref[B:Event]{Event}s aller \hyperref[B:Jobs]{Jobs} im System übertragen werden. \\
        \textbf{Vorbedingung:} & Der \gls{Nutzer} muss authentifiziert sein. \\
        \textbf{Nachbedingung (Erfolg):} & Der \gls{Nutzer} hat Zugriff auf den \hyperref[B:Event]{Event}-\gls{Stream}. \\
        \textbf{Nachbedingung (Fehlschlag):} & Der \gls{Nutzer} hat keinen Zugriff und hat eine Fehlermeldung erhalten. \\
        \textbf{Akteure:} & \gls{Nutzer} \\
        \textbf{Auslösendes Ereignis:} & Der \gls{Nutzer} möchte die \hyperref[B:Event]{Events} von Mallob einsehen. \\
    \end{FA}
    \textbf{Beschreibung:}
    \begin{FAList} 
        \item[1.] Der \gls{Nutzer} schickt eine Anfrage an die \gls{API}.
        \item[2.a.] Wenn die Anfrage erfolgreich war, wird der \hyperref[B:Event]{Event}-\gls{Stream} an den \gls{Nutzer} zurückgegeben. 
        \item[2.b.] Wenn die Anfrage nicht erfolgreich war, wird eine Fehlermeldung an den \gls{Nutzer} zurückgegeben. 
    \end{FAList}

    
    
    
    \phantomsection
    \label{FA:API:Abrufen von Einstellungen}  
    \item[F1100] \textbf{Abrufen von durch die \gls{Konfigurationsdatei} festgelegte Einstellungen} \\
    \begin{FA}
        \textbf{Ziel:} & Die Einstellungen der \gls{Konfigurationsdatei} können über die \gls{API} abgerufen werden.\\
        \textbf{Vorbedingung:} & - \\
        \textbf{Nachbedingung (Erfolg):}  & Der \gls{Nutzer} hat die Einstellungen erhalten.\\
        \textbf{Nachbedingung (Fehlschlag):} & Es wird eine Fehlermeldung zurückgegeben. \\
        \textbf{Akteure:} & \gls{Nutzer} \\
        \textbf{Auslösendes Ereignis:} & Der \gls{Nutzer} möchte die aktuellen Einstellungen abfragen. \\
    \end{FA}
    \textbf{Beschreibung:}
    \begin{FAList} 
        \item[1.] Schicken der entsprechenden \gls{API}-Anfrage.
        \item[2.a.] Wenn die Anfrage erfolgreich war, werden die Einstellungen der \gls{Konfigurationsdatei} an den \gls{Nutzer} zurückgegeben. 
        \item[2.b.] Wenn die Anfrage nicht erfolgreich war, wird eine Fehlermeldung an den \gls{Nutzer} zurückgegeben. 
    \end{FAList} 
    
    \phantomsection
    \label{FA:API:Andauernde Abfrage des Ergebnisses eines Jobs}
    \item[F1110] \textbf{Andauernde Abfrage des \hyperref[B:Job-Ergebnis]{Ergebnisses} eines Jobs}
    \begin{FA}
        \textbf{Ziel:} & Das Fertigstellen eines \hyperref[B:Jobs]{Jobs} kann über die \gls{API} abgewartet werden. \\
        \textbf{Vorbedingung:} & Der \gls{Nutzer} muss authentifiziert sein und mindestens einen \hyperref[B:Jobs]{Job} eingereicht haben. \\ 
        \textbf{Nachbedingung (Erfolg):} & Der \gls{Nutzer} hat die Meta-Daten des \hyperref[B:Job-Ergebnis]{Ergebnisses} des \hyperref[B:Jobs]{Jobs} erhalten.\\
        \textbf{Nachbedingung (Fehlschlag): } & Die Anfrage an die \gls{API} bricht ab. \ \\
        \textbf{Akteure:} & \gls{Nutzer} \\
        \textbf{Auslösendes Ereignis:} & Der \gls{Nutzer} möchte in Echtzeit über das \hyperref[B:Job-Ergebnis]{Ergebnis} eines \hyperref[B:Jobs]{Jobs} informiert werden. \\
    \end{FA}
    \textbf{Beschreibung:}
    \begin{FAList}
        \item[1.] Der \gls{Nutzer} schickt eine Anfrage an die \gls{API}, in der er spezifiziert, welchen \hyperref[B:Jobs]{Job} er Abfragen möchte.
        \item[2.] Die Anfrage wird nicht beantwortet, bis der Job abgeschlossen ist.
        \item[3.] Der \hyperref[B:Jobs]{Job} wurde erfolgreich abgeschlossen.
        \item[4.] Die Anfrage beantwortet und die Meta-Daten des \hyperref[B:Job-Ergebnis]{Ergebnisses} werden zurück gegeben.
    \end{FAList}

    % --- ab hier wunschkriterium
        
    %---------------------API - Starten von Mallob
    \phantomsection
    \label{FA:API:Starten von Mallob}  
    \item[F1120] (Wunschkriterium) \textbf{Starten von Mallob} \\
    \begin{FA}
        \textbf{Ziel:} & Die Mallob Instanz kann von einem \gls{Administrator} über die \gls{API} gestartet werden.\\
        \textbf{Vorbedingung:} & Mallob läuft noch nicht. \\
        \textbf{Nachbedingung (Erfolg):} & Mallob ist gestartet und der \gls{Administrator} hat eine Bestätigung erhalten. \\
        \textbf{Nachbedingung (Fehlschlag):} & Mallob ist nicht gestartet und der \gls{Administrator} hat eine Fehlermeldung erhalten.\\
        \textbf{Akteure:} & \gls{Administrator} \\
        \textbf{Auslösendes Ereignis:} & Der \gls{Administrator} will die Mallob-Instanz starten. \\
    \end{FA}
    \textbf{Beschreibung:}
    \begin{FAList}
        \item[1.] Der \gls{Administrator} schickt eine Anfrage an die \gls{API}.
        \item[2.] Die Mallob-Instanz wird gestartet.
    \end{FAList}
    
    
    %---------------------API - Stoppen von Mallob
    \phantomsection
    \label{FA:API:Stoppen von Mallob}  
    \item[F1130] (Wunschkriterium) \textbf{Stoppen von Mallob} \\
    \begin{FA}
        \textbf{Ziel:} & Die Mallob Instanz kann von einem \gls{Administrator} über die \gls{API} gestoppt werden. \\
        \textbf{Vorbedingung:} & Die Mallob-Instanz läuft bereits \\
        \textbf{Nachbedingung (Erfolg):} & Die Mallob-Instanz läuft nicht mehr und der \gls{Administrator} hat eine Bestätigung erhalten. \\
        \textbf{Nachbedingung (Fehlschlag:} & Die Mallob-Instanz läuft noch und der \gls{Administrator} hat eine Fehlermeldung erhalten. \\
        \textbf{Akteure:} & \gls{Administrator} \\
        \textbf{Auslösendes Ereignis:} & Der \gls{Administrator} will die Mallob-Instanz stoppen. \\
    \end{FA}
    \textbf{Beschreibung:}
    \begin{FAList}
        \item[1.] Der Administator schickt eine Anfrage an die \gls{API}.
        \item[2.] Die Mallob-Instanz wird gestoppt.
    \end{FAList}
    
    
    %-------------------------------API - Neustart von Mallob
    % das könnte man eigentlich aus rausmachen, kann durch stoppen und starten erreicht werden, aber idk
    \phantomsection
    \label{FA:API:Neustart von Mallob}  
    \item[F1140] (Wunschkriterium) \textbf{Neustart von Mallob} \\
    \begin{FA}
        \textbf{Ziel:} & Die Mallob Instanz kann von einem \gls{Administrator} über die \gls{API} neu gestartet werden.\\
        \textbf{Vorbedingung:} & Die Mallob-Instanz läuft bereits. \\
        \textbf{Nachbedingung (Erfolg):} & Die Mallob-Instanz wurde neu gestartet und der \gls{Administrator} hat eine Bestätigung erhalten. \\
        \textbf{Nachbedingung (Fehlschlag):} & Der \gls{Administrator} hat eine Fehlermeldung erhalten. \\
        \textbf{Akteure:} & \gls{Administrator} \\
        \textbf{Auslösendes Ereignis:} & Der \gls{Administrator} möchte die Mallob-Instanz neu starten. \\
    \end{FA}
    \textbf{Beschreibung:}
    \begin{FAList}
        \item[1.] Der \gls{Administrator} schickt eine Anfrage an die \gls{API}.
        \item[2.] Die Mallob-Instanz wird neu gestartet.
    \end{FAList}
    
    %\phantomsection
    %\label{FA:API:Einreichen von \hyperref[B:Jobs]{Jobs} per URL} 
    %\item[F1020] (Wunschkriterium) \textbf{Einreichen von \hyperref[B:Jobs]{Jobs} mit Beschreibung per URL} \\
    %\begin{FA}
    %    \textbf{Ziel:} & Ein \gls{Nutzer} kann einen \hyperref[B:Jobs]{Job} über die API einreichen, derer Beschreibung über eine URL verfügbar ist. \\
    %    \textbf{Vorbedingung:} & Der \gls{Nutzer} hat sich mittels \hyperref[FA:API:Authentifizieren von Nutzern]{F1010} authentifiziert \\
    %    \textbf{Nachbedingung (Erfolg):} & Der \gls{Nutzer} erhält eine Bestätigung. \\
    %    \textbf{Nachbedingung (Fehlschlag):} & Der \gls{Nutzer} hat eine Fehlermeldung. \\
    %    \textbf{Akteure:} & \gls{Nutzer} \\
    %    \textbf{Auslösendes Ergebnis:} & Der \gls{Nutzer} möchte  einen \hyperref[B:Jobs]{Job} einreichen. \\
    %\end{FA}
    %\textbf{Beschreibung:}
    %\begin{FAList}
    %        \item[1.a.] Der \gls{Nutzer} schickt eine Anfrage an die API, die sowohl die Job-Konfiguration als auch die URL mit der Job-Beschreibung enthält
    %        \item[2.] Der eingereichte \hyperref[B:Jobs]{Job} wird von Mallob bearbeitet
    %\end{FAList}

    \phantomsection
    \label{FA:API:Registrierung von Nutzern}
    \item[F1150] (Wunschkriterium) \textbf{Registrierung von Nutzern} \\ 
    
    \begin{FA}
        \textbf{Ziel: } & Registrierung über die \gls{API} ist möglich. \\
        \textbf{Vorbedingung:} &  -keine- \\
        \textbf{Nachbedingung (Erfolg):} &  Ein \gls{Vorlaeufiges Nutzerkonto} wurde für den \gls{Nutzer} erstellt und der \gls{Nutzer} hat eine Bestätigung erhalten. \\
        \textbf{Nachbedingung (Fehlschlag):} &  Der \gls{Nutzer} ist nicht registriert und hat eine Fehlermeldung erhalten.\\
        \textbf{Akteure:} & \gls{Nutzer}\\
        \textbf{Auslösendes Ereignis:} & \gls{Nutzer} möchte das System verwenden.\\
    \end{FA}
    \textbf{Beschreibung:}
    \begin{FAList} 
        \item[1.] Der \gls{Nutzer} schickt eine Anfrage an die \gls{API}, die seine \hyperref[PD:Registrierungsdaten]{Registrierungsdaten} enthält.
        \item[2.] \glslink{Vorlaeufiges Nutzerkonto}{Vorläufiges Konto} wird erstellt.
        \item[2.1.a] Wenn die Registrierung erfolgreich war, wird eine Bestätigung an den \gls{Nutzer} zurückgegeben.
        \item[2.1.b] Wenn die Registrierung nicht erfolgreich war, wird eine Fehlermeldung an den \gls{Nutzer} zurückgegeben.
        \item[3.] Verifizierung des Kontos durch \gls{System-Administrator}.
    \end{FAList}


\end{itemize}
    
%-------------------------------------------------------------------
%--------------------WEB INTERFACE----------------------------------
%-------------------------------------------------------------------
\pagebreak

\subsection{Web-Interface}
Die folgenden funktionalen Anforderungen beziehen sich alle auf das \gls{Web-Interface} und sind auch in diesem Kontext zu verstehen.


\begin{itemize}
    \setlength\itemsep{4em}

    \phantomsection
    \label{FA:Web-Interface:Anmelden} 
    \item[F2000] \textbf{Anmelden} \\
    \begin{FA}
        \textbf{Ziel:} & Ein \gls{Nutzer} ist in der Lage, sich im \gls{Web-Interface} zu authentifizieren. \\
        \textbf{Vorbedingung:} & Der \gls{Nutzer} besitzt bereits ein \gls{Nutzerkonto} und ist noch nicht angemeldet. \\
        \textbf{Nachbedingung (Erfolg):}  &  Der \gls{Nutzer} wird angemeldet und zur \hyperref[pages:job-table]{Job-Tabelle} weitergeleitet.\\
        \textbf{Nachbedingung (Fehlschlag):} & Die Anmeldung findet nicht statt und es  wird eine Fehlermeldung angezeigt. \\
        \textbf{Akteure:} & \gls{Nutzer} \\
        \textbf{Auslösendes Ereignis:} &  Der \gls{Nutzer} möchte sich im \gls{Web-Interface} anmelden. \\
    \end{FA}
    \textbf{Beschreibung:}
    \begin{FAList} 
        \item[1.] Aufrufen des Web-Interfaces.
        \item[2.] Eingabe des Nutzernamens.
        \item[3.] Eingabe des Passwortes.
        \item[4.] Bestätigung durch Betätigung der Schaltfläche \enquote{Log in}.
    \end{FAList}

   
    \phantomsection
    \label{FA:Web-Interface:Job einreichen} 
    \item[F2010] \textbf{\hyperref[B:Jobs]{Job} einreichen} \\
    \begin{FA}
        \textbf{Ziel:} & Ein neuer \hyperref[B:Jobs]{Job} kann über das \gls{Web-Interface} eingereicht werden.\\
        \textbf{Vorbedingung:} & Der \gls{Nutzer} ist angemeldet.  \\
        \textbf{Nachbedingung (Erfolg):} & Der \hyperref[B:Jobs]{Job} wird bearbeitet und der \gls{Nutzer} wird auf die \hyperref[pages:job-page]{Seite des gerade eingereichten Job} weitergeleitet.  \\
        \textbf{Nachbedingung (Fehlschlag):} & Im \gls{Web-Interface} wird eine Fehlermeldung angezeigt. \\
        \textbf{Akteure:} & \gls{Nutzer} \\
        \textbf{Auslösendes Ereignis:} & Der \gls{Nutzer} möchte einen \hyperref[B:Jobs]{Job} in Auftrag geben. \\
    \end{FA}
    \textbf{Beschreibung:}
    \begin{FAList} 
        \item[1.] Auswahl entsprechenden Schaltfläche in der Navigationsleiste.
        \item[2.] Weiterleitung zur \hyperref[pages:submit-job]{Seite zum Einreichen von Jobs}.
        \item[3.] Eingabe der notwendigen Optionen des Jobs.
        \item[4.] Eingabe der \hyperref[B:Job-Beschreibung]{Job-Beschreibung} über ein Eingabe-Feld direkt im \gls{Web-Interface}.
        \item[5.] Bestätigung der Eingaben.
    \end{FAList}
    \textbf{Erweiterungen}
    \begin{FAList}
        \item[3a.] Hinzufügen von gewünschten optionalen Optionen mithilfe des entsprechenden \glslink{Dropdown-Menue}{Dropdown-Menüs}.
        \item[3b.] Eingabe der optionalen Optionen über die entsprechenden Felder.
    \end{FAList}
    \textbf{Alternative 1 zu Schritt 4:}
    \begin{FAList}
        \item[4a] \gls{Dropdown-Menue} nutzen, um Upload der \hyperref[B:Job-Beschreibung]{Job-Beschreibung} auszuwählen.
        \item[4b] Entsprechende Schaltfläche nutzen, um die entsprechende \hyperref[B:Job-Beschreibung]{Job-Beschreibung} zum Hochladen auszuwählen.
    \end{FAList}
      \textbf{Alternative 2 zu Schritt 4: (gewünscht)}
    \begin{FAList}
        \item[4a] \gls{Dropdown-Menue} nutzen, um Angabe einer \gls{URL} der \hyperref[B:Job-Beschreibung]{Job-Beschreibung} auszuwählen.
        \item[4b] Eingabe der \gls{URL} zur \hyperref[B:Job-Beschreibung]{Job-Beschreibung} im entsprechenden Feld.
    \end{FAList}
    \pagebreak[3]
    
    \phantomsection
    \label{FA:Web-Interface:Abbruch eines einzelnen Jobs} 
    \item[F2020] \textbf{Abbruch eines einzelnen Jobs} \\
    \begin{FA}
        \textbf{Ziel:} & Ein einzelner, bereits eingereichter \hyperref[B:Jobs]{Job} kann wieder abgebrochen werden. \\
        \textbf{Vorbedingung:} & Der \gls{Nutzer} ist angemeldet und es gibt einen bereits eingereichten, noch nicht fertiggestellten Job. \\
        \textbf{Nachbedingung (Erfolg):}  & Der \hyperref[B:Jobs]{Job} wurde abgebrochen. \\
        \textbf{Nachbedingung (Fehlschlag):} &  Der \hyperref[B:Jobs]{Job} läuft weiter und dem \gls{Nutzer} wird eine entsprechende Fehlermeldung angezeigt. \\
        \textbf{Akteure:} & \gls{Nutzer} \\
        \textbf{Auslösendes Ereignis:} & Der \gls{Nutzer} möchte einen laufenden \hyperref[B:Jobs]{Job} abbrechen. \\
    \end{FA}
    \textbf{Beschreibung:}
    \begin{FAList} 
        \item[1.] Navigation zur \hyperref[pages:job-table]{Job-Tabelle}.
        \item[2.] Anklicken des entsprechenden \hyperref[B:Jobs]{Jobs} in der Tabelle.
        \item[3.] Auswahl der entsprechenden Schaltfläche im nebenstehenden Fenster.
        \item[4a.] Bestätigung des Abbrechens.
        \item[4b.] Keine Bestätigung des Abbrechens, die Aktion wird abgebrochen und der \hyperref[B:Jobs]{Job} läuft weiter.
    \end{FAList}
    \textbf{Alternative zu den Schritten 1 bis 3:}
    \begin{FAList}
        \item[1.] Navigation zur \hyperref[pages:job-page]{Job-Seite} des abzubrechenden Jobs.
        \item[2.] Auswahl der entsprechenden Schaltfläche.
    \end{FAList}
    
    
    \phantomsection
    \label{FA:Web-Interface:Abbruch mehrerer Jobs auf einmal} 
    \item[F2030] \textbf{Abbruch mehrerer \hyperref[B:Jobs]{Jobs} auf einmal} \\
    \begin{FA}
        \textbf{Ziel:} & Mehrere bereits laufende können auf einmal abgebrochen werden. \\
        \textbf{Vorbedingung:} & Der \gls{Nutzer} ist angemeldet und es gibt mehrere, bereits eingereichte und noch nicht fertiggestellte Jobs. \\
        \textbf{Nachbedingung (Erfolg):}  & Die \hyperref[B:Jobs]{Jobs} wurden alle abgebrochen. \\
        \textbf{Nachbedingung (Fehlschlag):} & Die \hyperref[B:Jobs]{Jobs} wurde nicht abgebrochen und der \gls{Nutzer} erhält eine entsprechende Fehlermeldung. Die Jobs, bei denen der Abbruch erfolgreich ist, werden auch abgebrochen, falls dies bei anderen \hyperref[B:Jobs]{Jobs} fehlschlägt.\\
        % [TODO: Formulierung überarbeiten]
        \textbf{Akteure:} & \gls{Nutzer} \\
        \textbf{Auslösendes Ereignis:} & Der \gls{Nutzer} möchte mehrere laufende \hyperref[B:Jobs]{Jobs} auf einmal abbrechen. \\
    \end{FA}
    \textbf{Beschreibung:}
    \begin{FAList} 
        \item[1.] Navigation zur Job-Tabelle.
        \item[2.] Setzen eines Kreuzes in der \gls{Checkbox} bei allen Jobs, die Abgebrochen werden sollen.
        \item[3.] Auswahl der entsprechenden Aktion im entsprechenden \gls{Dropdown-Menue}.
        \item[4.] Bestätigung des Abbruchs.
        \item[5.] Die entsprechenden \hyperref[B:Jobs]{Jobs} werden abgebrochen.
    \end{FAList}
    
    
    \phantomsection
    \label{FA:Web-Interface:Herunterladen eines einzelnen Ergebnisses} 
    \item[F2040] \textbf{Herunterladen eines einzelnen \hyperref[B:Job-Ergebnis]{Ergebnisses}} \\
    \begin{FA}
        \textbf{Ziel:} & Ein einzelnes \hyperref[B:Job-Ergebnis]{Ergebnis} eines abgeschlossen \hyperref[B:Jobs]{Jobs} kann heruntergeladen werden. \\
        \textbf{Vorbedingung:} & Der \gls{Nutzer} ist angemeldet es gibt einen bereits abgeschlossenen Job. \\
        \textbf{Nachbedingung (Erfolg):}  & Das \hyperref[B:Job-Ergebnis]{Ergebnis} wurde heruntergeladen. \\
        \textbf{Nachbedingung (Fehlschlag):} &  Das \hyperref[B:Job-Ergebnis]{Ergebnis} wurde nicht heruntergeladen und eine Fehlermeldung wird angezeigt. \\
        \textbf{Akteure:} & \gls{Nutzer} \\
        \textbf{Auslösendes Ereignis:} & Der \gls{Nutzer} möchte ein einzelnes \hyperref[B:Job-Ergebnis]{Ergebnis} herunterladen. \\
    \end{FA}
    \textbf{Beschreibung:}
    \begin{FAList} 
        \item[1.] Navigation zur \hyperref[pages:job-table]{Job-Tabelle}.
        \item[2.] Anklicken des entsprechenden abgeschlossenen \hyperref[B:Jobs]{Jobs} in der Tabelle.
        \item[3.] Auswahl der entsprechenden Schaltfläche im nebenstehenden Fenster.
        \item[4.] Das \hyperref[B:Job-Ergebnis]{Ergebnis} wird heruntergeladen.
    \end{FAList}
    \textbf{Alternative zu den Schritten 1 bis 3:}
    \begin{FAList}
        \item[1.] Navigation zur \hyperref[pages:job-page]{Job-Seite} des \hyperref[B:Jobs]{Jobs} mit dem gewünschten \hyperref[B:Job-Ergebnis]{Ergebnis}.
        \item[2.] Auswahl der entsprechenden Schaltfläche.
        \item[3.] Das \hyperref[B:Job-Ergebnis]{Ergebnis} wird heruntergeladen.
        \end{FAList}
    
    
    \phantomsection
    \label{FA:Web-Interface:herunterladen mehrerer Ergebnisse auf einmal} 
    \item[F2050] \textbf{Herunterladen mehrerer \hyperref[B:Job-Ergebnis]{Ergebnisse} auf einmal} \\
    \begin{FA}
        \textbf{Ziel:} & Mehrere \hyperref[B:Job-Ergebnis]{Ergebnisse} können auf einmal heruntergeladen werden. \\
        \textbf{Vorbedingung:} & Der \gls{Nutzer} ist angemeldet und es gibt mindestens einen bereits abgeschlossenen Job. \\
        \textbf{Nachbedingung (Erfolg):}  & Die \hyperref[B:Job-Ergebnis]{Ergebnisse} wurde alle heruntergeladen. \\
        \textbf{Nachbedingung (Fehlschlag):} & Ein oder mehrere \hyperref[B:Job-Ergebnis]{Ergebnisse} konnten nicht heruntergeladen werden und der \gls{Nutzer} erhält eine entsprechende Fehlermeldung. Die restlichen gewünschten \hyperref[B:Job-Ergebnis]{Ergebnisse} werden dennoch heruntergeladen.\\
        \textbf{FAkteure:} & \gls{Nutzer} \\
        \textbf{Auslösendes Ereignis:} & Der \gls{Nutzer} möchte mehrere \hyperref[B:Job-Ergebnis]{Ergebnisse} auf einmal herunterladen. \\
    \end{FA}
    \textbf{Beschreibung:}
    \begin{FAList} 
        \item[1.] Navigation zur Job-Liste.
        \item[2.] Setzen eines Kreuzes in der \gls{Checkbox} bei allen Jobs, die heruntergeladen werden sollen.
        \item[3.] Auswahl der entsprechenden Action im entsprechenden \gls{Dropdown-Menue}.
        \item[4.] Bestätigung des Herunterladens.
        \item[5.] Die \hyperref[B:Job-Ergebnis]{Ergebnisse} werden heruntergeladen.
    \end{FAList}
    
    
    \phantomsection
    \label{FA:Web-Interface:Anzeigen von Fehlern} 
     \item[F2060] \textbf{Anzeigen von Fehlern} \\
    \begin{FA}
        \textbf{Ziel:} & Es gibt eine Möglichkeit, den \gls{Nutzer} über aufgetretene Fehler zu informieren. \\
        \textbf{Vorbedingung:} & - \\
        \textbf{Nachbedingung (Erfolg):}  & Eine Fehlermeldung wird angezeigt. \\
        \textbf{Nachbedingung (Fehlschlag):} & Es wird keine Fehlermeldung angezeigt. \\
        \textbf{Akteure:} & System \\
        \textbf{Auslösendes Ereignis:} & Ein durch den \gls{Nutzer} verursachter Fehler ist aufgetreten. \\
    \end{FA}
    \textbf{Beschreibung:}
    \begin{FAList} 
        \item[1.] Ein Fehler tritt im System auf.
        \item[2.] Ein Fenster mit einer Beschreibung des Fehlers wird angezeigt.
    \end{FAList}
    
    
   
    
    
    \phantomsection
    \label{FA:Web-Interface:Anzeigen von Warnungen und Fehlermeldungen}
    \item[F2070] \textbf{Anzeigen von Mallob Warnungen} \\
    \begin{FA}
        \textbf{Ziel:} & Der \gls{Administrator} kann die Fehlermeldungen und Warnungen einsehen, die von Mallob ausgegeben werden. \\
        \textbf{Vorbedingung:} & Der \gls{Administrator} muss angemeldet sein. \\
        \textbf{Nachbedingung (Erfolg):} & Es werden die Fehlermeldungen und Warnungen von Mallob angezeigt. \\
        \textbf{Nachbedingung (Fehlschlag):} & Es werde die Fehlermeldungen und Warnungen von Mallob nicht angezeigt. \\
        \textbf{Akteure:} & \gls{Administrator} \\
        \textbf{Auslösendes Ereignis:} & Der \gls{Administrator} öffnet die \hyperref[pages:admin]{Administratoren-Seite}. \\
    \end{FA}
    \textbf{Beschreibung:}
    \begin{FAList} 
        \item[1.] Navigation zur \hyperref[pages:admin]{Administratoren-Seite}.
        \item[2.] Die Fehlermeldungen und Warnungen werden im \gls{Web-Interface} angezeigt.
    \end{FAList}
    
    
    \phantomsection
    \label{FA:Web-Interface:Einsehen von Job-Informationen}
    \item[F2080] \textbf{Einsehen von Job-Information} \\
    \begin{FA}
        \textbf{Ziel:} & Es gibt verschiedene Wege, \hyperref[B:Job-Informationen]{Informationen} von einem \hyperref[B:Jobs]{Job} anzuzeigen. \\
        \textbf{Vorbedingung:} &  Der \gls{Nutzer} ist angemeldet und besitzt mindestens einen Job. \\
        \textbf{Nachbedingung (Erfolg):}  &  Der \gls{Nutzer} kann \hyperref[B:Job-Informationen]{Informationen} zum gewünschten \hyperref[B:Jobs]{Job} einsehen. \\
        \textbf{Nachbedingung (Fehlschlag):} &  Eine Fehlermeldung wird angezeigt. \\
        \textbf{Akteure:} & \gls{Nutzer} \\
        \textbf{Auslösendes Ereignis:} & \gls{Nutzer} möchte \hyperref[B:Job-Informationen]{Informationen} zu einem \hyperref[B:Jobs]{Job} einsehen. \\
    \end{FA}
    \textbf{Beschreibung:}
    \begin{FAList} 
        \item[1.] Navigation zur \hyperref[pages:job-table]{Job-Tabelle}.
        \item[2.] Auswahl der gewünschten Attribute im \gls{Dropdown-Menue} über der Tabelle.
        \item[3.] Die ausgewählten Attribute werden jeweils als eigene Spalte in der Tabelle angezeigt. 
    \end{FAList}
    \textbf{Alternative 1:}
    \begin{FAList}
        \item[1.] Navigation zur \hyperref[pages:job-table]{Job-Tabelle}.
        \item[2.] Anklicken des entsprechenden \hyperref[B:Jobs]{Jobs} in der Tabelle.
        \item[3.] Die \hyperref[B:Job-Informationen]{Job-Informationen} werden im nebenstehenden Fenster angezeigt.
    \end{FAList}
    \textbf{(gewünscht) Erweiterung zu Alternative 1:}
    \begin{FAList}
        \item[1.a.)] Wechseln der Ansicht der Job-Tabelle zur \hyperref[pages:job-table-alt]{alternativen Ansicht}.
    \end{FAList}
    \textbf{Alternative 2:}
    \begin{FAList}
        \item[1.] Navigation zur \hyperref[pages:job-page]{Job-Seite} des Jobs. 
        \item[2.] Die \hyperref[B:Job-Informationen]{Job-Informationen} werden auf der \hyperref[pages:job-page]{Job-Seite} angezeigt.
    \end{FAList}
    

 
   % \phantomsection
   % \label{FA:Web-Interface:Einsehen von Job-Informationen}
   % \item[F2080] \textbf{Einsehen von Job-Information} \\
   % \begin{FA}
   %     \textbf{Ziel:} & Verschiedene Wege, Informationen zu einem \hyperref[B:Jobs]{Job} anzuzeigen. \\
   %     \textbf{Vorbedingung:} &  Der \gls{Nutzer} ist angemeldet und besitzt mindestens einen Job. \\
   %     \textbf{Nachbedingung (Erfolg):}  &  Der \gls{Nutzer} kann Informationen zum gewünschten \hyperref[B:Jobs]{Job} einsehen. \\
   %     \textbf{Nachbedingung (Fehlschlag):} &  Eine Fehlermeldung wird angezeigt. \\
   %     \textbf{Akteure:} & \gls{Nutzer} \\
   %     \textbf{Auslösendes Ereignis:} & \gls{Nutzer} möchte Informationen zu einem \hyperref[B:Jobs]{Job} einsehen. \\
   % \end{FA}
   % \textbf{Beschreibung:}
   % \begin{FAList} 
   %     \item[1.] Navigation zur Job-Tabelle.
   %     \item[2.] Auswahl der gewünschten Attribute im \gls{Dropdown-Menue} über der Liste.
   %     \item[3.] Die ausgewählten Attribute werden jeweils als eigene Spalte in der Tabelle angezeigt. 
   % \end{FAList}
   % \textbf{Alternative 1:}
   % \begin{FAList}
   %     \item[1.] Navigation zur Job-Tabelle.
   %     \item[2.] Anklicken des entsprechenden \hyperref[B:Jobs]{Jobs} in der Tabelle.
   %     \item[3.] Die Job-Informationen werden im nebenstehenden Fenster angezeigt.
   % \end{FAList}
   % \textbf{Erweiterung der Alternative 1:}
   % \begin{FAList}
   %     \item[4.] Anklicken der Schaltfläche \enquote{open job page}.
   %     \item[5.] Weiterleitung zur \hyperref[pages:job-page]{Job-Seite}.
   %     \item[6.] Die Job-Informationen werden auf der \hyperref[pages:job-page]{Job-Seite} angezeigt.
   % \end{FAList}
   % \textbf{Alternative 2:}
   % \begin{FAList}
   %     \item[1.] Direkter Aufruf der \hyperref[pages:job-page]{Job-Seite} über die entsprechende \gls{URL}.
   %     \item[2.] Die Job-Informationen werden auf der \hyperref[pages:job-page]{Job-Seite} angezeigt.
   % \end{FAList}
    
    
    \phantomsection
    \label{FA:Web-Interface:Hinzufügen von Spalten}
    \item[F2090] \textbf{Hinzufügen von Spalten in der \hyperref[pages:job-table]{Job-Tabelle}} \\
    \begin{FA}
        \textbf{Ziel:} & Es können Spalten zur \hyperref[pages:job-table]{Job-Tabelle} hinzugefügt werden, die jeweils ein Attribut der \hyperref[B:Job-Informationen]{Job-Informationen} beinhalten. \\
        \textbf{Vorbedingung:} & Der \gls{Nutzer} muss im \gls{Web-Interface} angemeldet sein. \\
        \textbf{Nachbedingung (Erfolg):} & Die gewünschte Spalte wurde zur \hyperref[pages:job-table]{Job-Tabelle} hinzugefügt. \\
        \textbf{Nachbedingung (Fehlschlag):} & Die gewünschte Spalte wurde nicht hinzugefügt.\\
        \textbf{Akteure:} & \gls{Nutzer} \\
        \textbf{Auslösendes Ereignis:} & Der \gls{Nutzer} möchte eine Spalte zur \hyperref[pages:job-table]{Job-Tabelle} hinzufügen. \\
    \end{FA}
    \textbf{Beschreibung:}
    \begin{FAList} 
        \item[1.] Navigation zur \hyperref[pages:job-table]{Job-Tabelle}.
        \item[2.] Anklicken des \glslink{Dropdown-Menue}{Dropdown-Menüs}.
        \item[3.] Durch Anklicken auswählen, welches Attribut als Spalte hinzugefügt werden soll. Im \glslink{Dropdown-Menue}{Dropdown-Menü} werden immer nur Attribute angezeigt, die noch nicht als Spalte in der Tabelle aufgeführt sind.
    \end{FAList}
    
    
    \phantomsection
    \label{FA:Web-Interface:Entfernen von Spalten}
    \item[F2100] \textbf{Entfernen von Spalten in der \hyperref[pages:job-table]{Job-Tabelle}} \\
    \begin{FA}
        \textbf{Ziel:} & Es können Spalten aus der \hyperref[pages:job-table]{Job-Tabelle} entfernt werden.\\
        \textbf{Vorbedingung:} & Der \gls{Nutzer} muss im \gls{Web-Interface} angemeldet sein und die es wurde mindestens eine Spalte mittels \hyperref[FA:Web-Interface:Hinzufügen von Spalten]{F2090} hinzugefügt. \\
        \textbf{Nachbedingung (Erfolg):} & Die gewünschte Spalte wurde aus der \hyperref[pages:job-table]{Job-Tabelle} entfernt. \\
        \textbf{Nachbedingung (Fehlschlag):} & Die gewünschte Spalte wurde nicht aus der \hyperref[pages:job-table]{Job-Tabelle} entfernt. \\
        \textbf{Akteure:} & \gls{Nutzer} \\
        \textbf{Auslösendes Ereignis:} & Der \gls{Nutzer} möchte eine angezeigte Spalte aus der Tabelle löschen.\\
    \end{FA}
    \textbf{Beschreibung:}
    \begin{FAList} 
        \item[1.] Navigation zur \hyperref[pages:job-table]{Job-Tabelle}
        \item[2.] Anklicken des \enquote{x}-Symbols in der entsprechenden Spalte
    \end{FAList}
    
    \phantomsection
    \label{FA:Web-Interface:Aktualisieren}
    \item[F2105] \textbf{Aktualisieren der \hyperref[pages:job-table]{Job-Tabelle}} \\
    \begin{FA}
        \textbf{Ziel:} & Die \hyperref[pages:job-table]{Job-Tabelle} kann aktualisiert werden, ohne dass die Seite neu geladen werden muss. \\
        \textbf{Vorbedingung:} & Der \gls{Nutzer} ist angemeldet. \\
        \textbf{Nachbedingung (Erfolg):} & Die \hyperref[pages:job-table]{Job-Tabelle}  und eventuelle im nebenstehenden Fenster angezeigte \hyperref[B:Job-Informationen]{Informationen} wurde aktualisiert.\\
        \textbf{Nachbedingung (Fehlschlag):} & Die \hyperref[pages:job-table]{Job-Tabelle} wurde nicht aktualisiert und eine Fehlermeldung wird angezeigt. \\
        \textbf{Akteure:} & \gls{Nutzer} \\
        \textbf{Auslösendes Ereignis:} & Der \gls{Nutzer} möchte die \hyperref[pages:job-table]{Job-Tabelle} aktualisieren.\\
    \end{FA}
    \textbf{Beschreibung:}
    \begin{FAList} 
        \item[1.] Navigation zur \hyperref[pages:job-table]{Job-Tabelle}.
        \item[2.] Anklicken der entsprechenden Schaltfläche.
    \end{FAList}
    \textbf{Alternative (gewünscht)}
    \begin{FAList}
        \item[1.] Die Tabelle wird automatische aktuell gehalten, die  entsprechende Schaltfläche entfällt.
    \end{FAList}
    
 


    \phantomsection
    \label{FA:Web-Interface:Abmelden} 
    \item[F2110] \textbf{Abmelden} \\
    \begin{FA}
        \textbf{Ziel:} & Der \gls{Nutzer} kann sich wieder abmelden. \\
        \textbf{Vorbedingung:} & Der \gls{Nutzer} ist angemeldet. \\
        \textbf{Nachbedingung (Erfolg):}  & Der \gls{Nutzer} ist abgemeldet und wird zur Login-Seite weitergeleitet. \\
        \textbf{Nachbedingung (Fehlschlag):} & Der \gls{Nutzer} ist weiterhin angemeldet und eine Fehlermeldung wird angezeigt. \\
        \textbf{Akteure:} & \gls{Nutzer}\\
        \textbf{Auslösendes Ereignis:} & \gls{Nutzer} möchte sich abmelden. \\
    \end{FA}
    \textbf{Beschreibung:}
    \begin{FAList} 
        \item[1.] Betätigung der entsprechenden Schaltfläche im Navigations-Menü.
    \end{FAList}




\phantomsection
    \label{FA:Web-Interface:Registrierung von Nutzern} 
    \item[F2120] (Wunschkriterium) \textbf{Registrierung von Nutzern} \\
    \begin{FA}
        \textbf{Ziel:} & Ein \gls{Nutzer} ist in der Lage, ein neues \gls{Nutzerkonto} zu erstellen.\\
        \textbf{Vorbedingung:} &  Der \gls{Nutzer} ist nicht angemeldet. \\
        \textbf{Nachbedingung (Erfolg):}  &  Ein \gls{Vorlaeufiges Nutzerkonto} wird für den \gls{Nutzer} erstellt und er wird zur \hyperref[pages:job-table]{Job-Tabelle} weitergeleitet. \\
        \textbf{Nachbedingung (Fehlschlag):} &  Das \gls{Nutzerkonto} kann nicht erstellt werden und es wird eine Fehlermeldung angezeigt. \\
        \textbf{Akteure:} & Person, welche ein neues \gls{Nutzerkonto} erstellen möchte. \\
        \textbf{Auslösendes Ereignis:} &  Die Person möchte ein neues \gls{Nutzerkonto} erstellen. \\
    \end{FA}
    \textbf{Beschreibung:}
    \begin{FAList}
        \item[1.] Aufrufen des Web-Interfaces.
        \item[2.] Auswählen der Schaltfläche \enquote{register}.
        \item[3.] Weiterleitung zur \hyperref[pages:register]{Registerung}.
        \item[2.] Eingabe des gewünschten Nutzernames.
        \item[3.] Eingabe des gewünschten Passwortes.
        \item[4.] Eingabe der Wiederholung des Passwortes.
        \item[5.] Bestätigung der Eingabe mittels der Schaltfläche \enquote{register}.
        \item[6.] \glslink{Vorlaeufiges Nutzerkonto}{Vorläufiges Konto} wird erstellt.
        \item[7.] \glslink{Vorlaeufiges Nutzerkonto}{Vorläufiges Nutzerkonto} wird durch \gls{System-Administrator} verifiziert.
    \end{FAList}
    
    
    
    \phantomsection
    \label{FA:Web-Interface:Neustart} 
    \item[F2130] (Wunschkriterium) \textbf{Neustart eines abgebrochenen Jobs} \\
    \begin{FA}
        \textbf{Ziel:} & Ein \gls{Nutzer} kann einen abgebrochenen \hyperref[B:Jobs]{Job} neustarten.\\
        \textbf{Vorbedingung:} & Der \gls{Nutzer} hat einen abgebrochenen Job. \\
        \textbf{Nachbedingung (Erfolg):}  &  Der abgebrochene \hyperref[B:Jobs]{Job} wird wieder bearbeitet .\\
        \textbf{Nachbedingung (Fehlschlag):} &  Der abgebrochene \hyperref[B:Jobs]{Job} wird nicht bearbeiten. \\
        \textbf{Akteure:} & \gls{Nutzer}\\
        \textbf{Auslösendes Ereignis:} & Der \gls{Nutzer} möchte einen abgebrochenen \hyperref[B:Jobs]{Job} wieder starten. \\
    \end{FA}
    \textbf{Beschreibung:}
    \begin{FAList} 
        \item[1.] Navigation zur \hyperref[pages:job-table]{Job-Tabelle}.
        \item[2.] Anklicken des neuzustartenden, abgebrochenen Jobs.
        \item[3.] Auswählen der entsprechenden Schaltfläche im der nebenstehenden Fenster.
        \item[4.] Weiterleitung zur \hyperref[pages:submit-job]{Seite zum Einreichen von Jobs}, wobei hier die entsprechenden Optionen bereits wieder ausgefüllt sind.
        \item[5.] Verändern der \hyperref[B:Job-Beschreibung]{Job-Beschreibung} oder der \hyperref[B:Job-Konfiguration]{Job-Konfiguration}.
        \item[6.] Einreichen des Jobs.
    \end{FAList} 
    \textbf{Alternative der Schritte 1-3:}
    \begin{FAList}
        \item[1.] Navigation zur \hyperref[pages:job-page]{Job-Seite} des neuzustartenden Jobs.
        \item[2.] Auswählen der entsprechenden Schaltfläche.
    \end{FAList}
    \textbf{Alternative von Schritt 5.}
    \begin{FAList}
        \item[5.] Nichts verändern.
    \end{FAList}
    
    
    \phantomsection
    \label{FA:Web-Interface:Verwalten von Malllob}
    \item[F2140] (Wunschkriterium) \textbf{Verwalten von Mallob} \\
    \begin{FA}
        \textbf{Ziel:} & Mallob kann vom \gls{Administrator} gestartet, gestoppt und neu gestartet werden. \\
        \textbf{Vorbedingung:} & Der \gls{Administrator} muss im \gls{Web-Interface} angemeldet sein. \\
        \textbf{Nachbedingung (Erfolg):} & Mallob ist gestartet, gestoppt oder neu gestartet. \\
        \textbf{Nachbedingung (Fehlschlag):} & Mallob ist nicht gestartet, gestoppt oder neu gestartet. \\
        \textbf{Akteure:} & \gls{Administrator} \\
        \textbf{Auslösendes Ereignis:} & Der \gls{Administrator} will Mallob starten, stoppen oder neu starten. \\
    \end{FA}
    \textbf{Beschreibung:}
    \begin{FAList} 
        \item[1.] Navigation zur \hyperref[pages:admin]{Administratoren-Seite}.
        \item[2.] Anklicken des Knopfes \enquote{start mallob} .
        \item[3.] Mallob wird gestartet.
    \end{FAList}
    \textbf{Alternative 1 zu den Schritten 2 bis 3:}
    \begin{FAList}
        \item[2.] Anklicken des Knopfes \enquote{stop mallob}.
        \item[3.] Mallob wird gestoppt.
    \end{FAList}
    \textbf{Alternative 2 zu den Schritten 2 bis 3:}
    \begin{FAList}
        \item[2.] Anklicken des Knopfes \enquote{restart mallob}.
        \item[3.] Mallob wird neu gestartet.
    \end{FAList}
    
    
    \phantomsection
    \label{FA:Web-Interface:Sortieren der Tabelle}
    \item[F2150] (Wunschkriterium) \textbf{Sortieren der \hyperref[pages:job-table]{Job-Tabelle} nach Attributen} \\
    \begin{FA}
        \textbf{Ziel:} & Die Einträge der \hyperref[pages:job-table]{Job-Tabelle} können nach den Attributen der verschiedenen Spalten sortiert werden. \\
        \textbf{Vorbedingung:} & Der \gls{Nutzer} muss im \gls{Web-Interface} angemeldet sein. \\
        \textbf{Nachbedingung (Erfolg):} & Die Einträge der \hyperref[pages:job-table]{Job-Tabelle} sind nach dem Wunsch des \glslink{Nutzer}{Nutzers} sortiert. \\
        \textbf{Nachbedingung (Fehlschlag):} & Die Einträge der \hyperref[pages:job-table]{Job-Tabelle} sind nicht nach dem Wunsch des \glslink{Nutzer}{Nutzers} sortiert. \\
        \textbf{Akteure:} & \gls{Nutzer} \\
        \textbf{Auslösendes Ereignis:} & Der \gls{Nutzer} möchte die Einträge der \hyperref[pages:job-table]{Job-Tabelle} sortieren. \\
    \end{FA}
    \textbf{Beschreibung:}
    \begin{FAList} 
        \item[1.] Navigation zur \hyperref[pages:job-table]{Job-Tabelle}.
        \item[2.] Anklicken der Spalte, nach deren Attribut die Tabelle sortiert werden soll.
        \item[3.a.] Zeigt der Pfeil neben dem Attribut nach unten, werden die Einträge absteigend, bzw. alphabetisch sortiert.
        \item[3.b.] Zeigt der Pfeil neben dem Attribut nach oben, werden die Einträge aufsteigend, bzw. umgekehrt alphabetisch sortiert.
    \end{FAList}
    
       \phantomsection
    \label{FA:Web-Interface:Filtern für Admins}
    \item[F2160] (Wunschkriterium) \textbf{Filtern der \hyperref[pages:job-table]{Job-Tabelle} für Administratoren} \\
    \begin{FA}
        \textbf{Ziel:} & Administratoren können wahlweise die \hyperref[B:Jobs]{Jobs} aller \gls{Nutzer} oder nur die eigenen in der Tabelle sehen. \\
        \textbf{Vorbedingung:} & Ein \gls{Administrator} ist angemeldet. \\
        \textbf{Nachbedingung (Erfolg):} & Der Filter wurde korrekt angewandt, es sind nur die gewünschten \hyperref[B:Jobs]{Jobs} zu sehen. \\
        \textbf{Nachbedingung (Fehlschlag):} & Der Filter wurde nicht angewandt, eine Fehlermeldung wird ausgegeben. \\
        \textbf{Akteure:} & \gls{Nutzer} \\
        \textbf{Auslösendes Ereignis:} & Ein \gls{Administrator} möchte ändern, welche \hyperref[B:Jobs]{Jobs} er in der \hyperref[pages:job-table]{Job-Tabelle} sieht.\\
    \end{FA}
    \textbf{Beschreibung:}
    \begin{FAList} 
        \item[1.] Navigation zur \hyperref[pages:job-table]{Job-Tabelle}.
        \item[2.] Setzen der \gls{Checkbox} \enquote{see all jobs}.
    \end{FAList}
    

       \label{FA:Web-Interface:Anzeigen von Diagnosedaten}
    \item[F2165] (Wunschkriterium) \textbf{Anzeigen von Diagnosedaten} \\
    \begin{FA}
        \textbf{Ziel:} & Adminstratoren können Diagnosedaten von Mallob einsehen. \\
        \textbf{Vorbedingung:} & Ein \gls{Administrator} ist angemeldet. \\
        \textbf{Nachbedingung (Erfolg):} & Diagnosedaten werden angezeigt. \\
        \textbf{Nachbedingung (Fehlschlag):} & Diagnosedaten werden nicht angezeigt. \\
        \textbf{Akteure:} & \gls{Nutzer} \\
        \textbf{Auslösendes Ereignis:} & Der \gls{Administrator} öffnet die \hyperref[pages:admin]{Administratoren-Seite}.\\
    \end{FA}
    \textbf{Beschreibung:}
    \begin{FAList} 
        \item[1.] Navigation zur \hyperref[pages:admin]{Administratoren-Seite}.
        \item[2.] Die Diagnosedaten werden angezeigt.
    \end{FAList}
    
    \phantomsection
    \label{FA:Web-Interface:Anzeigen von Plugins}
    \item[F2170] (Wunschkriterium) \textbf{Anzeigen von Plugins} \\
    \begin{FA}
        \textbf{Ziel:} & Das \gls{Web-Interface} kann extern erstelle Plugins anzeigen. \\
        \textbf{Vorbedingung:} & - \\
        \textbf{Nachbedingung (Erfolg):}  &  Plugins werden als Dropdown-Menü in der Navigations-Leiste angezeigt und können ausgewählt werden.\\
        \textbf{Nachbedingung (Fehlschlag):} & Plugins werden nicht angezeigt. \\
        \textbf{Akteure:} & \gls{Nutzer} \\
        \textbf{Auslösendes Ereignis:} &  Mindestens ein Plugin wurde durch \hyperref[FA:System:Einlesen von Plugins bei Systemstart]{F4010} eingelesen\\
    \end{FA}
    \textbf{Beschreibung:}
    \begin{FAList} 
        \item[1.] Einlesen von Plugins durch \hyperref[FA:System:Einlesen von Plugins bei Systemstart]{F4010}
        \item[2.] Anzeigen eines Eintrags für Plugins in der Navigations-Leiste, welcher es ermöglicht, die \hyperref[pages:plugin]{Seiten der Plugins} aufzurufen.
    \end{FAList}
    
    
  \end{itemize}
%-------------------------------------------------------------------
%--------------------Visualisierung---------------------------------
%-------------------------------------------------------------------
\pagebreak

\subsection{Visualisierung}
Die Visualisierung findet \gls{Web-Interface} statt und ist nur dort einsehbar.


\begin{itemize}
    \setlength\itemsep{4em}



    %----------------------Visualisierung - Anzeigen des Systemzustandsja lles 
    
    \phantomsection
    \label{FA:Visualisierung:Anzeigen des Systemzustandes}
    \item[F3000] \textbf{Anzeigen des aktuellen \hyperref[B:Systemzustand]{Systemzustands}} \\
    \begin{FA}
        \textbf{Ziel:} & Das \gls{Web-Interface} bietet eine Visualisierung des aktuellen \hyperref[B:Systemzustand]{Systemzustands} von Mallob. \\
        \textbf{Vorbedingung:} & Der \gls{Nutzer} muss im \gls{Web-Interface} angemeldet sein. \\
        \textbf{Nachbedingung (Erfolg):} & Der \hyperref[B:Systemzustand]{Systemzustand} von Mallob wird im \gls{Web-Interface} angezeigt. \\
        \textbf{Nachbedingung (Fehlschlag):} &  Es wird eine Fehlermeldung im \gls{Web-Interface} angezeigt.\\
        \textbf{Akteure:} & \gls{Nutzer} \\
        \textbf{Auslösendes Ereignis:} & Der \gls{Nutzer} möchte die Visualisierung des \hyperref[B:Systemzustand]{Systemzustands} ansehen. \\
    \end{FA}
    \textbf{Beschreibung:}
    \begin{FAList} 
        \item[1.] Navigation zur Visualisierung im \gls{Web-Interface}.
        \item[2.a.] Wenn die Visualisierung erfolgreich geladen wurde, wird diese im \gls{Web-Interface} angezeigt  .
        \item[3.b.] Wenn die Visualisierung nicht erfolgreich geladen wurde, wird eine Fehlermeldung im \gls{Web-Interface} angezeigt.
    \end{FAList}
    
    
    
     %-----------------Visualisierung - Ansehen von Details
    \phantomsection
    \label{FA:Visualisierung:Anzeigen von Details} 
    \item[F3010] \textbf{Anzeigen von \hyperref[B:Job-Details]{Job-Details}} \\
    \begin{FA}
        \textbf{Ziel:} & Durch Anklicken eines \hyperref[B:Jobs]{Jobs} in der Visualisierung können \hyperref[B:Job-Details]{Details} zum \hyperref[B:Jobs]{Job} eingesehen werden. \\
        \textbf{Vorbedingung:} & Der \gls{Nutzer} ist im \gls{Web-Interface} angemeldet. \\
        \textbf{Nachbedingung (Erfolg):} & Es werden \hyperref[B:Job-Details]{Details} zum ausgewählten \hyperref[B:Jobs]{Jobs} angezeigt. \\
        \textbf{Nachbedingung (Fehlschlag):} & Es wird eine Fehlermeldung angezeigt und es werden keine \hyperref[B:Job-Details]{Details} angezeigt.\\
        \textbf{Akteure:} & \gls{Nutzer} \\
        \textbf{Auslösendes Ereignis:} & Der \gls{Nutzer} möchte die \hyperref[B:Job-Details]{Details} eines \hyperref[B:Jobs]{Job} in der Visualisierung einsehen. \\
    \end{FA}
    \textbf{Beschreibung:}
    \begin{FAList} 
        \item[1.] Navigation zur Visualisierung im \gls{Web-Interface}.
        \item[2.] Anklicken eines \hyperref[B:Jobs]{Jobs} im linken Fenster der Visualisierung.
        \item[3] Die \hyperref[B:Job-Details]{Details} werden im rechten Fenster angezeigt. Ist der \hyperref[B:Jobs]{Job} keiner der eigenen und der \gls{Nutzer} kein \gls{Administrator}, so werden die Datails pseudonymisiert dargestellt. Ebenso werden beiden Schaltflächen nicht angezeigt.
    \end{FAList}

    
    
    %-----------------Visualisierung - Pausieren der Visualisierung
    \phantomsection
    \label{FA:Visualisierung:Pausieren der Visualisierung} 
    \item[F3020] \textbf{Pausieren der Visualisierung} \\
    \begin{FA}
        \textbf{Ziel:} & Die Visualisierung kann pausiert werden. \\
        \textbf{Vorbedingung:} & Der \gls{Nutzer} muss im \gls{Web-Interface} angemeldet sein und die ist  die Visualisierung läuft bereits. \\
        \textbf{Nachbedingung (Erfolg):} & Die Visualisierung ist pausiert und wird nicht mehr aktualisiert \\
        \textbf{Nachbedingung (Fehlschlag):} & Die Visualisierung ist nicht pausiert und läuft weiter. \\
        \textbf{Akteure:} & \gls{Nutzer} \\
        \textbf{Auslösendes Ereignis:} & Der \gls{Nutzer} möchte die Visualisierung pausieren. \\
    \end{FA}
    \textbf{Beschreibung:}
    \begin{FAList} 
        \item[1.] Navigation zur Visualisierung im \gls{Web-Interface}.
        \item[2.] Anklicken der Pause-Taste.
        \item[3.] Die Visualisierung pausiert und wird nicht mehr aktualisiert.
    \end{FAList}
    
    
    %--------------------Visualisierung - Starten der Visualisierung
    \phantomsection
    \label{FA:Visualisierung:Starten der Visualisierung} 
    \item[F3030] \textbf{Starten der Visualisierung} \\
    \begin{FA}
        \textbf{Ziel:} & Die Visualisierung kann nach dem pausieren wieder gestartet werden. \\
        \textbf{Vorbedingung:} & Der \gls{Nutzer} ist im \gls{Web-Interface} angemeldet und die Visualisierung ist bereits pausiert. \\
        \textbf{Nachbedingung (Erfolg):} & Die Visualisierung läuft wieder und wird aktualisiert. \\
        \textbf{Nachbedingung (Fehlschlag):} & Die Visualisierung ist weiterhin pausiert. \\
        \textbf{Akteure:} & \gls{Nutzer} \\
        \textbf{Auslösendes Ereignis:} & Der \gls{Nutzer} möchte die Visualisierung wieder starten.\\
    \end{FA}
    \textbf{Beschreibung:}
    \begin{FAList} 
        \item[1.] Navigation zur Visualisierung im \gls{Web-Interface}.
        \item[2.] Anklicken der Wiedergabe-Taste.
        \item[3.] Die Visualisierung startet und wird wieder aktualisiert.
    \end{FAList}
    
    
    
    
    
    %-------------Visualisierung - Springen zu einem bestimmten Zeitpunkt
    \phantomsection
    \label{FA:Visualisierung:Springen} 
    \item[F3040] \textbf{Springen zu einem bestimmten Zeitpunkt der Visualisierung} \\
    \begin{FA}
        \textbf{Ziel:} & Der \gls{Nutzer} kann zu einem beliebigen Zeitpunkt in der Visualisierung vor- oder zurückspringen. \\
        \textbf{Vorbedingung:} & Der \gls{Nutzer} muss im \gls{Web-Interface} angemeldet sein und Mallob muss gestartet sein. \\
        \textbf{Nachbedingung (Erfolg):} & Die Visualisierung wird ab dem gewählten Zeitpunkt abgespielt. \\
        \textbf{Nachbedingung (Fehlschlag):} & Die Visualisierung wird nicht ab dem gewählten Zeitpunkt abgespielt. \\
        \textbf{Akteure:} & \gls{Nutzer} \\
        \textbf{Auslösendes Ereignis:} & Der \gls{Nutzer} möchte die Visualisierung an einem bestimmten Zeitpunkt ansehen. \\
    \end{FA}
    \textbf{Beschreibung:}
    \begin{FAList} 
        \item[1.] Navigation zur Visualisierung.
        \item[2.] Wählen des gewünschte Zeitpunktes durch Anklicken oder Ziehen des Sliders zur gewünschten Position.
        \item[3.] Die Visualisierung wird ab dem gewählten Zeitpunkt abgespielt.
    \end{FAList}
    
    
    
    
    %----------------Visualisierung - Ändern der Wiedergabegeschwindigkeit
    \phantomsection
    \label{FA:Visualisierung:Aendern der Wiedergabegeschwindigkeit} 
    \item[F3050] (Wunschkriterium) \textbf{Ändern der Wiedergabegeschwindigkeit} \\
    \begin{FA}
        \textbf{Ziel:} & Die Geschwindigkeit, mit der die Visualisierung abgespielt wird kann geändert werden. \\
        \textbf{Vorbedingung:} & Der \gls{Nutzer} ist im \gls{Web-Interface} angemeldet. \\
        \textbf{Nachbedingung (Erfolg):} & Die Visualisierung wird in der gewünschten Geschwindigkeit abgespielt. \\
        \textbf{Nachbedingung (Fehlschlag):} & Die Visualisierung wird weiterhin in der alten Geschwindigkeit abgespielt. \\
        \textbf{Akteure:} & \gls{Nutzer} \\
        \textbf{Auslösendes Ereignis:} & Der \gls{Nutzer} möchte die Wiedergabegeschwindigkeit der Visualisierung ändern. \\
    \end{FA}
    \textbf{Beschreibung:}
    \begin{FAList} 
        \item[1.] Navigation zur Visualisierung im \gls{Web-Interface}.
        \item[2.] Eingabe der gewünschten Wiedergabegeschwindigkeit in das Feld \enquote{replay speed}.
        \item[3.a.] Die Visualisierung wird in der gewünschten Wiedergabegeschwindigkeit abgespielt.
        \item[3.b.] Wenn die gewählte Wiedergabegeschwindigkeit schneller als Echtzeit ist und sich die Visualisierung am aktuellsten Punkt befindet, wird die Wiedergabegeschwindigkeit auf Echtzeit gesetzt.
        \item[3.c.] Wenn die gewählte Wiedergabegeschwindigkeit negativ ist, wird die Visualisierung rückwärts in der gewünschten Wiedergabegeschwindigkeit abgespielt.
    \end{FAList}
    
    
      %---------------Visualisierung - Anzeigen des Binärbaumes für einen Job
    
    \phantomsection
    \label{FA:Visualisierung:Anzeigen des Binaerbaumes für einen Job}
    \item[F3060] (Wunschkriterium) \textbf{Anzeigen des \glslink{Binaerbaum}{Binärbaumes} für einen Job} \\
    \begin{FA}
        \textbf{Ziel:} & Visualisierung des zu einem \hyperref[B:Jobs]{Job} gehörenden \glslink{Binaerbaum}{Binärbaumes}. \\
        \textbf{Vorbedingung:} & Der \gls{Nutzer} muss im \gls{Web-Interface} angemeldet sein und muss einen \hyperref[B:Jobs]{Job} eingereicht oder abgeschlossen haben. \\
        \textbf{Nachbedingung (Erfolg):} & Der \gls{Binaerbaum} zu dem gewünschten \hyperref[B:Jobs]{Job} wird im \gls{Web-Interface} angezeigt. \\
        \textbf{Nachbedingung (Fehlschlag):} & Der \gls{Binaerbaum} wird nicht im \gls{Web-Interface} angezeigt.  \\
        \textbf{Akteure:} & \gls{Nutzer} \\
        \textbf{Auslösendes Ereignis:} & Der \gls{Nutzer} klickt einen \hyperref[B:Jobs]{Job} in der Visualisierung an. \\
    \end{FA}
    \textbf{Beschreibung:}
    \begin{FAList} 
        \item[1.] Navigation zur Visualisierung im \gls{Web-Interface}.
        \item[2.] Anklicken eines \hyperref[B:Jobs]{Jobs} im linken Fenster der Visualisierung.
        \item[3.] Der \gls{Binaerbaum} zu dem ausgewählten \hyperref[B:Jobs]{Job} wird im \gls{Web-Interface} angezeigt.
    \end{FAList}
    
\end{itemize}


%-------------------------------------------------------------------
%----------------------------System---------------------------------
%-------------------------------------------------------------------
\pagebreak

\subsection{System}
    \setlength\itemsep{4em}




\begin{itemize}
    \phantomsection
    \label{FA:System:Einstellungen festlegen}
    \item[F4000] \textbf{Festlegen von Einstellungen mittels einer \gls{Konfigurationsdatei} bei Systemstart} \\
    \begin{FA}
        \textbf{Ziel:} & Bestimmte Einstellungen können mit einer \gls{Konfigurationsdatei} konfiguriert werden, welche bei Systemstart eingelesen werden. \\
        \textbf{Vorbedingung:} & Eine korrekt formatierte \gls{Konfigurationsdatei} existiert an einem fest vorgeschriebenen Ort im Dateisystem.\\
        \textbf{Nachbedingung (Erfolg):}  & Die Einstellungen der \gls{Konfigurationsdatei} sind angewandt. \\
        \textbf{Nachbedingung (Fehlschlag):} & Die \gls{Konfigurationsdatei} kann nicht geladen werden, das Programm beendet sich mit einer entsprechenden Fehlermeldung.\\
        
        \textbf{Akteure:} & System\\
        \textbf{Auslösendes Ereignis:} & Starten des Systems.
    \end{FA}
    \textbf{Beschreibung:}
    \begin{FAList} 
        \item[1.] Starten des Systems.
        \item[2.] \gls{Konfigurationsdatei} wird beim Start automatisch eingelesen.
    \end{FAList} 

  
    \phantomsection
    \label{FA:System:Einlesen von Plugins bei Systemstart}
    \item[F4010] (Wunschkriterium) \textbf{Einlesen von Plugins bei Systemstart} \\
    \begin{FA}
        \textbf{Ziel:} & Das System kann extern erstellte Plugins einlesen. \\
        \textbf{Vorbedingung:} & Es liegen ein einzulesende Plugins vor. \\
        \textbf{Nachbedingung (Erfolg):}  & Die Plugins wurden eingelesen. \\
        \textbf{Nachbedingung (Fehlschlag):} & Ein oder mehrere Plugins konnten nicht eingelesen werden und auf der  \hyperref[pages:admin]{Administratoren-Seite} werden mittels \hyperref[FA:Web-Interface:Anzeigen von Warnungen und Fehlermeldungen]{F2110} angezeigt. \\
        \textbf{Akteure:} & System \\
        \textbf{Auslösendes Ereignis:} & Starten des Systems .\\
    \end{FA}
    \textbf{Beschreibung:}
    \begin{FAList} 
        \item[1.] Starten des Systems.
        \item[2.] Plugins werden beim Start eingelesen.
    \end{FAList} 


\end{itemize}

